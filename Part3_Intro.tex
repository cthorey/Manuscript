\thispagestyle{plain}
\begin{flushleft}
 \Large \vspace{.5cm} \textbf{Introduction - Résumé}
\end{flushleft}

La Lune possède a priori une croûte très poreuse, sans doute du fait
des impacts, et très peu dense, de par son mode de formation (Chapitre
\ref{chap1}). La formation d’intrusions magmatiques y serait donc
favorisée. Nous avons déjà montré dans la partie précédente, à l’aide
d’un modèle de mise en place d’intrusion sous une couche élastique
d’épaisseur constante, que les dômes à faible sont probablement le
résultat d’intrusions magmatiques au sein de la croûte lunaire.

D’autres sites témoignant potentiellement de la présence d’intrusions
magmatiques sur la Lune sont les cratères au sol fracturé.
\citet{Schultz:1976kt} a reporté la présence de $200$ de ces cratères,
principalement situés sur le pourtour des mers lunaires. Ces cratères
montrent des signes évidents de déformations postérieures à leur
formation (Section \ref{C1-sec:moon}). Leur faible profondeur ainsi
que les importants réseaux de fractures en leur sein sous-tendent un
mécanisme capable de soulever leur sol, parfois sur quelques centaines
de mètres. Les deux scénarios proposés pour expliquer ces déformations
sont une intrusion magmatique centrée sous le cratère et la relaxation
visqueuse de la topographie du cratère après l’impact
\citep{Wichman:1996bj}. Cependant, \citet{Dombard:2001gs} ont montré
que, sur la Lune, la relaxation visqueuse des cratères est
probablement trop faible pour générer des déformations telles que
celles observées au sein de ces cratères. De plus, bien que la
relaxation soit cohérente avec la forme convexe du sol de certains de
ces cratères, elle ne permet pas d’expliquer le sol plat, séparé des
murs du cratère par un fossé circulaire, aussi observé pour de
nombreux cratères au sol fracturé.

Le modèle statique de \citet{Pollard:1973ho} a précédemment été
utilisé pour modéliser les déformations engendrées par une intrusion
centrée sous un cratère. Cependant, comme nous l’avons déjà montré
dans le Chapitre \ref{chap1}, ces modèles ne prennent en compte ni le
poids du magma ni la dynamique de l’écoulement et sont donc incapables
de faire des prédictions réalistes. En place de ces modèles statiques,
nous proposons dans le Chapitre \ref{C5-chap6} de modifier cette fois
le modèle de \citet{Michaut:2011kg} pour étudier l’influence de la
dépression liée au cratère sur l’étalement d’une intrusion. Bien que
le refroidissement influence sûrement la dynamique des intrusions
lunaires, nous avons montré dans la partie précédente que celui-ci ne
conduit probablement pas directement à limiter leurs étalements. De
plus, nous nous intéressons ici à décrire la déformation engendrée par
l’intrusion de manière qualitative. Nous nous contentons donc de
décrire l’écoulement pour un fluide isovisqueux.

Ce modèle montre que l’augmentation de la pression lithostatique en
bordure du cratère empêche l’étalement horizontal de l’intrusion, ce
qui conduit donc naturellement à son épaississement et donc au
soulèvement du sol du cratère. La morphologie finale du cratère dépend
du rapport entre le rayon du cratère et le rayon qui conduit à la
transition de l’intrusion dans le régime gravitaire, i.e
$R\sim 4\Lambda$. Ainsi, si le rayon du cratère, noté $C$, est
inférieur à $4\Lambda$, comme c’est vraisemblablement le cas pour de
petits cratères où de profondes intrusions, l’intrusion est dans le
régime élastique quand elle atteint le bord du cratère. Elle a donc
une forme de cloche ; le sol se soulève donc avec une forme convexe.
Si au contraire $C>4\Lambda$, i.e. l’intrusion est peu profonde où le
cratère est grand, l’étalement est contrôlé par le poids du magma
quand il atteint le bord du cratère. Sa forme tabulaire engendre cette
fois un soulèvement en bloc du sol du cratère. De plus, dans ce cas,
le modèle prédit aussi la présence du fossé circulaire observé à la
bordure du cratère pour de nombreux cratères au sol fracturé à sol
plat. En effet, celui-ci, qui n’avait jamais été expliqué
précédemment, résulte de la flexion de la couche élastique sur les
pourtours de l’intrusion dans le régime gravitaire ; la profondeur et
la taille de ce fossé circulaire sont donc d’autant plus importantes
que la longueur d’onde de flexure $\Lambda$ est importante, i.e. que
l’intrusion est profonde. Ce modèle est donc capable de reproduire les
déformations observées au sein de ces cratères et soutient ainsi
l’hypothèse d’intrusion magmatique.

De plus, en fonction de l’apparence du sol et/ou de la taille des
fossés bordant le cratère, le modèle permet de contraindre l’épaisseur
élastique et donc accéder à une estimation de la profondeur de
l’intrusion. Finalement, les déformations étant cantonnées à
l’intérieur du cratère, ce modèle suggère que la décompression
engendrée par la formation du cratère a probablement provoqué la
formation de ces intrusions magmatiques.

La densité du magma augmente quand il se solidifie. En supposant que
ces intrusions se soient mises en place au niveau de leurs zones de
flottabilité neutre, elles devraient avoir laissé une signature dans
le champ de gravité lunaire. Dans le Chapitre \ref{chap7}, nous
proposons donc d’étudier la signature gravitaire des cratères au sol
fracturés. La résolution du champ de gravité obtenu par la mission
GRAIL de la NASA a permis de construire une carte du champ de gravité
lunaire d’une précision sans précédente, i.e $\sim 6$ km à la
surface \citep{Zuber:2013cp}. Ces donnés, associées avec les donnés
topographiques obtenus par l’instrument LOLA de la sonde spatiale LRO
(Lunar Reconnaissance Orbiter) permettent ainsi d’étudier les
anomalies de densité de faible amplitude au sein de la croûte lunaire.

La signature des cratères au sol fracturés est analysée en parallèle
de celles de nombreux cratères « normaux » pour comparaison. Nous
montrons que la signature gravitaire moyenne est négative pour les
cratères « normaux » et positive pour les cratères au sol fracturé. En
particulier, la différence de $\sim 3$ mGal est statistiquement
significative et en accord avec les signaux attendus. Ces résultats
supportent donc aussi la mise en place d’intrusions magmatiques sous
le sol de ces cratères. La différence moyenne de densité entre le
magma et la croûte, obtenue en comparant la signature observée et
celle calculée à l’aide des profils d’épaisseur obtenue dans le
Chapitre \ref{C5-chap6}, est proche de $900$ kg m$^{-3}$. Étant donné
la porosité importante de la croûte lunaire, i.e. $\sim 12\%$, une
telle différence de densité implique des intrusions magmatiques
relativement peu poreuses. En particulier, ceci suggère que ces
intrusions sont suffisamment jeunes pour avoir échappé aux périodes
d’intenses bombardements météoriques prévalents peu après la formation
de la Lune.


%%% Local Variables:
%%% mode: latex
%%% TeX-master: "main"
%%% End:
