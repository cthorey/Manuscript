\thispagestyle{plain}
\begin{center}
 \Large \vspace{.5cm} \textbf{Conclusion}
\end{center}

Intrusive  magmatism  plays a  fundamental  role  in the  accretionary
processes of  terrestrial crust.  On Earth,  tectonic displacements as
well as erosion have exposed numerous intrusions; their morphology and
volume can be constrained.  However,  such observations must be linked
to models of  magma intrusion dynamics to provide  insights into magma
physical properties, injection rate and the intrusion process itself.

In the  first part  of this thesis,  we have developed  a model  for a
cooling magma spreading  beneath an elastic layer.   This model, which
accounts for a realistic magma rheology  as well as the heating of the
wall rocks,  is able to reproduce  the geometry and the  dimensions of
several magmatic  intrusions to a  first order.  Especially,  it shows
the formation of a highly viscous region at the front of the intrusion
which  rapidly constrains  the flow.   Nevertheless, the  formation of
this  region does  not  seem  to coincide  with  the  arrest of  these
intrusions.  A more  rigorous treatment of the dynamics at  the tip of
the  intrusion, developed  in combination  with analogue  experiments,
might  shed   light  on   the  mechanism  at   the  origin   of  their
solidification.

On  other  terrestrial planets,  the  model  allows the  detection  of
intrusions at  depth within  the crust.  For  instance, we  have shown
that  low-slope  lunar  domes   have  probably  formed  following  the
emplacement   of   shallow   magmatic  intrusions.    On   the   Moon,
floor-fractured craters have also been  proposed as resulting from the
emplacement  of  magma at  depth  below  their  floor.  To  test  this
hypothesis, we develop a dynamical model of crater-centered intrusion.
We have  shown that  floor-fractured crater  deformations, as  well as
their  gravitational  signatures,  are indeed  consistent  with  their
intrusive origin.  While  the total volume of  these intrusions should
not exceed $1\%$ of the volume  of the maria, it confirms the presence
of  numerous shallow  magmatic  intrusions in  the  lunar crust  whose
origin have yet to be discovered.

In conclusion, we  have shown in this thesis that  coupling models and
observations can provide a robust framework for probing the importance
of  intrusive magmatism  on  terrestrial planets.   In particular,  it
allows not only to get insights  into the flow physical properties but
also on the intrusion process itself as well as the geological history
of the region surrounding the intrusion.


\clearpage

%%% Local Variables:
%%% mode: latex
%%% TeX-master: "main"
%%% End:
