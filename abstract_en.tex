\thispagestyle{fancy}
\begin{center}
    \Large
    \textbf{Abstract}
\end{center}

\textbf{Key words}: Magmatic intrusions, Elastic-plated gravity
current, Thermal processes, Rheology, Temperature-dependent viscosity,
Elastic  sheet,   Laccolith,  Sill,  Earth,  Moon,   Low-slope  domes,
Floor-fractured craters, Elastic-sheet thickness, Crater depression,
Gravitational anomalies.\\

Intrusive  magmatism  plays a  fundamental  role  in the  accretionary
processes of terrestrial  crust.  Indeed, when magma is  forced to the
surface, only a small amount of  it actually reaches that level.  Most
of the  magma is intruded  into the crust  where it solidifies  into a
wide range of  features, from the small scale sills  and laccoliths to
large  scale  batholiths (several  hundred  kilometers  in size).  The
topographic deformation that could be caused by shallow intrusions can
be constrained by observations of planetary surfaces; that is, volume,
shape and other dimensions of  intrusions can be quantified.  However,
such  observations  must be  linked  to  dynamic  models of  of  magma
emplacement at depth in order  to provide insights into magma physical
properties,  injection  rate,  emplacement  depth  and  the  intrusion
process itself.

In  this  thesis,  we  first  investigate  the  relation  between  the
morphology  of shallow  intermediate-scale magmatic  intrusions (sills
and  laccoliths)  and their  cooling.   We  propose  a model  for  the
spreading   of    an   elastic-plated    gravity   current    with   a
temperature-dependent viscosity  that accounts  for a  realistic magma
rheology, melt crystallization and  heating of the surrounding medium.
The  mechanisms that  drive cooling  of the  intrusions vary  from the
Earth to the Moon and the ability  of the model to reproduce the final
morphologies (aspect  ratio) of  terrestrial laccoliths  and low-slope
lunar domes is examined.

On the  Moon, emplacement  of magmatic intrusions  into the  crust has
also  been proposed  as  a  possible mechanism  for  the formation  of
floor-fractured  craters.  We  propose a  model for  an elastic-plated
gravity current  spreading beneath  an elastic overburden  of variable
thickness.  We  find that  several characteristics  of floor-fractured
craters are indeed  consistent with the emplacement of  a large volume
of magma  beneath their floor.   In addition, using  the unprecedented
resolution  of the  NASA's  Gravity Recovery  and Interior  Laboratory
(GRAIL) mission,  in combination  with topographic data  obtained from
the  Lunar Orbiter  Laser Altimeter  (LOLA) instrument,  we show  that
lunar   floor-fractured   craters  present   gravitational   anomalies
consistent with magmatic intrusions intruding a crust characterized by
a $12\%$  porosity. The implications  in terms of lunar  evolution are
examined.

%%% Local Variables:
%%% mode: latex
%%% TeX-master: "main"
%%% End:
