\thispagestyle{plain}
\begin{flushleft}
  \Large
  \vspace{.5cm}
  \textbf{Remerciement}
\end{flushleft}

Cette thèse prend sûrement racine quelques années avant son début, sur
les flancs  chauds et humides  du volcan  ``El Fugeo'' au  Mexique. En
effet, c'est sans doute ce volcan qui a piquer en premier ma curiosité
pour les sciences de la Terre. Je commencerais donc par remercier Nick
Varley:   sa  passion   sans  limites   ainsi  qu'une   légère  touche
d'insouciance nous  auront toutes  deux permis  d’ effleurait  au plus
près la beauté  et la puissance de ce volcan  capricieux. À colima, je
ne puis aussi oublier Yannes, Irving  et toutes les personnes que j'ai
pu  croisé  sur  mon  chemin.  À  tous  merci  pour  cette  expérience
inoubliable.

Durant  cette période,  je remercie  aussi mes  parents de  ne m'avoir
renier  après ces  longs mois  sans nouvelles.  Merci à  vous pour  le
soutient inconditionnelle  que vous m'avait apporté  durant toutes mes
années  d'études. Même  si  le fond  vous restera  sans  doute un  peu
nébuleux, sachez que cette thèse vous doit beaucoup.

Ces travaux  n'aurait pas non plus  vu le jour sans  Chloé Michaut, ma
directrice de thèse.  Je ne peux  que la remercier pour sa patience et
son accompagnement tout au long de ces trois ans, pour tout ce qu'elle
m'a  apporté  scientifiquement  et   personnellement.   Une  liste  de
remerciement non exhaustive contiendrais sûrement la faculté de suivre
l'évolution  de  ma pensé  souvent  embrumé,  de décrypter  mes  notes
interminables,  d'apprécier   mes  figures  colorés,  de   relire  mes
manuscrits pas finis  et surtout, de m'avoir toujours  encouragé et de
m'avoir  laissé  flâner  à  mon  rythme sans  me  laissé  pour  autant
m'égarer.

Ce travail,  et notamment la  digression champ de gravité,  doit aussi
beaucoup à  Mark Wieczorek.   Merci pour  m'avoir laissé  triturer les
dernières données de  la mission GRAIL, de m'avoir guidé  tout au long
de cette  étude ainsi que de  m'avoir permis de présenter  mes travaux
dans le colorado.











\vspace{2cm}

%%% Local Variables:
%%% mode: latex
%%% TeX-master: "main"
%%% End:
