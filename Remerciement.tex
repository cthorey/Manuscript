\thispagestyle{plain}
\begin{flushleft}
 \Large
 \vspace{.5cm}
 \textbf{Remerciements}
\end{flushleft}

Cette thèse prend sûrement racine quelques années avant son début, sur
les flancs chauds et humides du volcan ``El Fuego'' au Mexique. En
effet, c’est sans doute ce volcan qui a piqué en premier ma curiosité
pour les sciences de la Terre. Je commencerais donc par remercier Nick
Varley : sa passion sans limites ainsi qu’une légère touche
d’insouciance nous auront toutes deux permis d’effleurer au plus près
la beauté et la puissance de ce volcan capricieux. À Colima, je ne
puis aussi oublier Jannes, Irving, Pilar, Ana et toutes les personnes
que j’ai pu croiser sur mon chemin. À tous, merci pour cette
expérience inoubliable.

Durant cette période, je remercie aussi mes parents d’avoir supporté
ces longs mois sans nouvelles. Merci à vous pour le soutien
inconditionnel que vous m’avez apporté durant toutes mes années
d’études. Même si le fond vous restera sans doute un peu nébuleux,
sachez que cette thèse vous doit beaucoup.

Ces travaux n’auraient pas non plus vu le jour sans Chloé Michaut, ma
directrice de thèse. Je ne peux que la remercier pour sa patience et
son accompagnement tout au long de ces trois ans, pour tout ce qu’elle
m’a apporté scientifiquement et personnellement. Une liste de
remerciements non exhaustive contiendrait sûrement la faculté de suivre
l’évolution de ma pensée souvent embrumée, de décrypter mes notes
interminables, d’apprécier mes figures colorées, de relire mes
manuscrits pas finis et surtout, de m’avoir toujours encouragé et de
m’avoir laissé explorer à mon rythme sans me laisser pour autant
m’égarer.

Ce travail,  et notamment la  digression champ de gravité,  doit aussi
beaucoup  à  Mark Wieczorek.  Merci  de  m’avoir laissé  triturer  les
dernières données de  la mission GRAIL, de m’avoir guidé  tout au long
de cette  étude ainsi que de  m’avoir permis de présenter  mes travaux
dans le Colorado. Je remercie les  membres de mon jury d’avoir accepté
d’évaluer  ce  travail,  car  j’imagine  qu’il  existe  lectures  plus
agréables  pour la  rentrée  : Jerome  Neufeld,  Virginie Pinel,  Oded
Aharonson et Edouard Kaminksi.

Sous les auspices  de Lamarck, je remercie aussi  toutes les personnes
qui ont contribué de  près ou de loin au bon  déroulement de ces trois
années.   Tout  d’abord  Mathieu  pour ses  conseils,  les  bavardages
lunaires  et  les sessions  d’escalade  qui  ont accompagné  toute  ma
première année, Sebastiano  et Karine pour leur énergie  et leur bonne
humeur qui se  chargèrent de la deuxième et enfin,  Alicia, Shang Xia,
Claudine, Mélanie, Jean-François, Lucile, Sébastien, Yasuhiro, Foivos,
Virgile,   Joana  et   toutes  les   personnes  du   laboratoire  de
planétologie  et sciences  spatiales de  l’IPGP et  d'AIM que  j’ai pu
oublier devant  l’explosion démographique qui  a eu lieu  durant cette
dernière année.

Sous la  bienveillance de  Jussieu, je remercie  tout particulièrement
Adrien et Kenny, avec qui j’ai  passé beaucoup de temps à procrastiner
au Linnée.   Je remercie aussi  l'intégralité de  la TP team  avec qui
j'ai pris plaisir  à enseigner la physique.  Je ne  peux enfin oublier
Malbec,  le cluster  qui, heure  après heure,  jour après  jour, s’est
affairé aux  tâches que je  lui avais  donné sans rechigner.   Merci à
Alexandre de nous avoir présenté  et au service informatique de l’IPGP
qui le bichonne et l’entretient depuis sa création.

Enfin,   dans  cette   épopée  parisienne,   je  remercie   aussi  mes
indénombrables colocs de la rue  Tolbiac, avec une pensée particulière
pour M.   Nicolas, et les  gens du swing  et de l’escalade.   Un merci
tout particulier  à Valentin et CamCam  qui ont aussi participé  à mon
équilibre  parisien. Merci  aussi  à  mon voisin  du  dessus, rue  des
Gobelins, de m’avoir réveillé  ``délicatement'' tous les matins, c’est
peut-être grâce à  lui que j’ai pu  finir ce travail à  temps. Dans le
camp des  anonymes, je remercie l’homme  au marcel bleu avec  qui j’ai
partagé mes  séances de courses à  pied. Bien qu’elles ne  m’aient pas
apporté d’idées révolutionnaires, ces séances m’auront au moins permis
de me vider la tête.

Je remercie également Clémentine,  Fabian (Pura Vida), Amélie, Adrien,
Lucie, Simon, Mylène et compagnie qui m’accompagnent depuis nos écoles
lyonnaises, avec une  pensée toute particulière pour  Florian, qui m’a
guidé jusque là-bas.

Enfin, je remercie Marion, pour tout !











\vspace{2cm}

%%% Local Variables:
%%% mode: latex
%%% TeX-master: "main"
%%% End:
