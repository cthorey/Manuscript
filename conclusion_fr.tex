\thispagestyle{plain}
\begin{center}
 \Large \vspace{.5cm} \textbf{Conclusion}
\end{center}

Le  magmatisme intrusif  représente  une source  cachée, mais  souvent
importante  du  magmatisme  planétaire.   Sur  Terre,  les  mouvements
tectoniques  ainsi que  l’érosion ont  permis d’exposer  de nombreuses
intrusions  à  la  surface.  De plus,  les  techniques  d'explorations
géophysiques peuvent  fournir des  informations sur  leur morphologie,
leur taille ainsi que leur  croissance même en profondeur.  Cependant,
de telles informations ne  peuvent être interprétées qu’en association
avec  un  modèle  qui  relie  la  déformation  finale  à  l’écoulement
lui-même.

Ainsi, dans la première partie de cette thèse, nous avons développé un
modèle d’étalement du magma sous  une couche élastique. Ce modèle, qui
prend en  compte à la  fois la rhéologie du  magma et le  chauffage de
l’encaissant,  est   capable  de   reproduire  au  premier   ordre  la
morphologie  et les  dimensions de  nombreuses intrusions.   Il prédit
notamment  la  formation d’une  région  très  visqueuse au  front  qui
rapidement contrôle  l’écoulement.  Cependant,  la formation  de cette
région  ne semble  pas  coïncider avec  l’arrêt  des intrusions.   Une
description  plus  fine du  front,  couplée  avec des  expériences  de
laboratoire,  pourrait  sûrement  permettre de  mieux  comprendre  les
mécanismes à l’origine de leur solidification.

Sur  les autres  corps telluriques,  ce modèle  permet de  détecter la
présence d’intrusions en profondeur au  sein de la croûte. Ainsi, nous
avons montré que les dômes à faible pente lunaire avaient probablement
une origine intrusive.  Sur la Lune,  les cratères au sol fracturé ont
aussi été proposés  comme résultant de la mise  en place d’intrusions.
Pour  tester cette  hypothèse,  nous avons  adapté  notre modèle  pour
prendre en compte la dépression  engendrée par le cratère.  Nous avons
montré que les  déformations au sein de ces cratères,  ainsi que leurs
signatures  gravitaires,  soutiennent  toutes  les  deux  une  origine
intrusive.   Bien que  le volume  réuni de  ces intrusions  soit assez
faible en comparaison  du volume des laves au sein  des mers lunaires,
il confirme la présence de nombreuses  intrusions au sein de la croûte
lunaire. Des études plus précises du  champ de contrainte associé à la
dépression pourront  sûrement apporter de précieuses  informations sur
l’origine de ces magmas.

En  conclusion, l’approche  développée dans  cette thèse,  qui associe
modèles et observations, donne non  seulement des informations sur les
propriétés physiques  des intrusions et leurs  profondeurs, mais aussi
sur leurs conditions de mise en  place et sur l’histoire géologique de
la région.





%%% Local Variables:
%%% mode: latex
%%% TeX-master: "main"
%%% End:
