\thispagestyle{plain}
\begin{flushleft}
 \Large \vspace{.5cm} \textbf{Résumé partie II}
\end{flushleft}

\citet{Michaut:2011kg} a développé un modèle d’écoulement gravitaire
sous une couche élastique et sur un plan rigide qui permet de relier
la morphologie finale des intrusions magmatiques aux propriétés de
l’écoulement. En particulier, l’écoulement, contraint par la réponse
élastique sus-jacente, montre deux différents régimes. Dans un premier
régime, la flexion de la couche élastique contrôle l’écoulement ;
l’intrusion est convexe, l’épaisseur évolue comme le rayon à la
puissance $8/7$. Quand l’intrusion devient grande par rapport à la
longueur d’onde de flexure naturelle de la couche élastique,
l’écoulement entre dans un régime gravitaire dans lequel le poids du
magma contrôle l’écoulement ; l’intrusion devient tabulaire, son rayon
augmente comme le temps à la puissance $1/2$ et son épaisseur tend
vers une constante.

L’application  de  ce  modèle  à   la  morphologie  d’une  dizaine  de
laccolites sur  l’île d’Elbe ainsi  que certains dômes à  faible pente
sur la Lune  est très encourageante. En  particulier, leur morphologie
est cohérente avec leur arrêt  dans le régime élastique. Cependant, ce
modèle sous-estime les  dimensions de ces laccolites et  n’est pas non
plus capable d’expliquer l’augmentation  de l’épaisseur des sills avec
leur taille.  En outre,  il n’offre pas  un cadre  théorique suffisant
permettant d’expliquer pourquoi ces laccolites se sont arrêtés dans le
régime élastique sans entrer dans le régime gravitaire.

L’une des hypothèses du modèle de \citet{Michaut:2011kg} est que
l’écoulement est suffisamment rapide pour négliger le refroidissement
de l’intrusion. Cependant, les magmas sont des fluides dont les
propriétés dépendent considérablement de la température. Lorsqu’un
magma refroidit, sa composition ainsi que son taux de cristaux
évoluent, ce qui en retour, modifie la viscosité et la dynamique de
l’écoulement. De nombreuses études ont montré que dans le cas d’un
écoulement gravitaire, la prise en compte de la rhéologie du magma, et
donc du refroidissement de l’écoulement, exerce une influence
importante sur la dynamique de l’écoulement lui-même.

Ainsi, dans le chapitre \ref{C3-JFM}, nous complexifions le modèle de
\citet{Michaut:2011kg} pour prendre en compte le refroidissement de
l’intrusion. Nous proposons un modèle de refroidissement basé sur la
croissance de couches limites thermiques au sein du fluide et au
contact avec l'encaissant. Dans un premier temps, la température de la
roche est maintenue constante et égale au solidus. Ce modèle intègre
aussi une rhéologie simplifiée pour le magma, qui permet notamment de
coupler l’écoulement et le champ de température, ainsi que la chaleur
produite lors de la cristallisation. Nous étudions la dynamique qui
résulte de ce couplage dans chaque régime séparément pour ensuite
étudier l’évolution globale.

Dans le régime élastique, l’anomalie thermique croît en même temps que
l’intrusion, mais un peu moins vite. Ceci entraîne la
formation d’une région « froide » au niveau du front de
l’intrusion. Dans le régime gravitaire, l’épaisseur de l’écoulement
est constante, les pertes de chaleur par conduction compensent
rapidement l’injection de fluide chaud et l’anomalie thermique atteint
un régime stationnaire. Dans chaque cas, une étude quantitative du
transport de la chaleur au sein du fluide nous permet de prédire
l’évolution de l’anomalie thermique en fonction des paramètres de
l’écoulement.

Le couplage entre l'anomalie thermique et l'écoulement entraîne la
ramification de chaque régime en trois phases distinctes. Dans une
première phase, le refroidissement n'a pas encore d'effet sur
l'écoulement et la dynamique est celle d'un fluide isovisqueux chaud.
Dans une deuxième phase, la viscosité effective de l'écoulement
augmente, l’écoulement ralentit et s’épaissit. Finalement, la
dynamique redevient comparable à celle d'un fluide isovisqueux, mais
cette fois, un fluide entièrement froid. Bien que ces phases soient
communes aux deux régimes, nous montrons que la viscosité effective,
qui contrôle le passage d'une phase à l'autre, n’est pas la même dans
les deux régimes. Elle est la viscosité moyenne d’une petite région
au front dans le régime élastique et la viscosité moyenne de
l’écoulement dans le régime gravitaire. Dans le chapitre
\ref{Heating}, nous montrons que cette dynamique n’est que peu
affectée par (1) le chauffage de l’encaissant et (2) une rhéologie
plus réaliste pour le magma. Cependant, ceci nous permet de quantifier
au premier ordre sur quelle distance l'intrusion chauffe, et donc
potentiellement induit du métamorphisme, dans la roche encaissante.

Le  refroidissement  permet  d'expliquer  l'importante  épaisseur  des
laccolites. En effet, le modèle prédit que leur épaisseur augmente non
seulement avec leur  rayon, mais aussi avec le  contraste de viscosité
entre le magma « chaud » et  le magma « froid ». Ainsi, les dimensions
des laccolites  de l’île d’Elbe  sont en accord avec  leur composition
felsique   et  leur   arrêt  dans   la  troisième   phase  du   régime
élastique. Cependant, ni sur Terre ni sur la Lune, l'entrée dans cette
phase de l'écoulement,  qui correspond à la formation  d’une région de
viscosité    importante    au    front,   n’entraîne    l'arrêt    des
laccolites.  Ceux-ci se  solidifient  plus tard  dans  cette phase  de
l'écoulement. Les sills se sont  quant à eux probablement arrêtés dans
le second régime gravitaire.

En conclusion,  bien que le refroidissement  nous permette d'expliquer
au premier ordre  la dimension de ces intrusions,  il n’a probablement
pas directement provoqué leur arrêt. Celui-ci est peut-être simplement
lié au tarissement  de la source de magma en  profondeur. En effet, le
temps  pour atteindre  la  transition augmente  avec  le contraste  de
viscosité.   Si celui-ci  est faible  et l'injection  suffisante pour
atteindre le régime gravitaire, l'intrusion se solidifie sous forme de
sill.  Dans  le  cas  contraire,  elle  se  solidifie  sous  forme  de
laccolite.  Ceci pourrait  expliquer  la  prédominance des  laccolites
felsiques dans la  nature, leur contraste de  viscosité plus important
les rendant  moins à  même d'atteindre le  régime gravitaire  avant un
éventuel tarissement de la source.

%%% Local Variables:
%%% mode: latex
%%% TeX-master: "main"
%%% End:
