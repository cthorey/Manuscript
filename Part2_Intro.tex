\thispagestyle{plain}
\begin{flushleft}
 \Large \vspace{.5cm} \textbf{Introduction - Résumé}
\end{flushleft}
\citet{Michaut:2011kg} a  développé un modèle  d’écoulement gravitaire
sous une couche  élastique et sur un plan rigide  qui permet de relier
la  morphologie finale  des intrusions  magmatiques aux  propriétés de
l’écoulement. En  particulier, l’écoulement  contraint par  la réponse
élastique sus-jacente montre deux  différents régimes. Dans un premier
régime,  la flexion  de  la couche  élastique contrôle  l’écoulement ;
l’intrusion  est  convexe, l’épaisseur  évolue  comme  le rayon  à  la
puissance $8/7$.  Quand l’intrusion  devient grande devant la longueur
d’onde de flexure naturelle de la couche élastique, l’écoulement entre
dans un régime gravitaire où le poids du magma contrôle l’écoulement ;
l’intrusion  est tabulaire,  son rayon  augmente comme  le temps  à la
puissance $1/2$ et son épaisseur tend vers une constante.

L’application  de  ce  modèle  à   la  morphologie  d’une  dizaine  de
laccolites sur  l’île d’Elbe ainsi  que certains dômes à  faible pente
sur la Lune est très encourageante. En particulier, leurs morphologies
sont cohérentes avec leurs arrêts dans le régime élastique. Cependant,
ce modèle  sous-estime les dimensions  de ces laccolites et  n’est pas
non plus  capable d’expliquer l’augmentation de  l’épaisseur des sills
avec  leurs  tailles. De  plus,  il  n’offre  pas un  cadre  théorique
suffisant  permettant  d’expliquer  pourquoi ces  laccolites  se  sont
arrêtés  dans  le  régime  élastique   sans  rentrer  dans  le  régime
gravitaire.

L’une  des  hypothèses du  modèle  de  \citet{Michaut:2011kg} est  que
l’écoulement est suffisamment rapide  pour négliger le refroidissement
de  l’intrusion.  Cependant, les  magmas  sont  des fluides  dont  les
propriétés dépendent considérablement de  la température ;  lorsqu’un magma
refroidit, sa composition ainsi que  son taux de cristaux évoluent, ce
qui en retour,  modifie la viscosité et la  dynamique de l’écoulement.
De nombreuses études ont montré que dans le cas d’un écoulement
gravitaire, la  prise en compte de  la rhéologie du magma,  et donc du
refroidissement de l’écoulement exerce une influence significative sur
la dynamique de l’écoulement lui-même.

Dans le  chapitre \ref{C3-JFM}, nous  complexifions donc le  modèle de
\citet{Michaut:2011kg} pour  prendre en  compte le  refroidissement de
l’intrusion. Nous proposons  un modèle de refroidissement  basé sur la
croissance de  couches limites  thermiques au  contacte avec  la roche
alentour  qui est-elle,  maintenue  à la  température  du solidus.  Ce
modèle intègre aussi une rhéologie simplifiée pour le magma qui permet
de  coupler l’écoulement  et  le  champ de  température  ainsi que  la
chaleur produite lors de la  cristallisation. Nous étudions la réponse
de  la dynamique  à ce  couplage  dans chaque  régime séparément  pour
ensuite étudier l’évolution globale.

Dans le régime élastique, l’anomalie thermique croit en même temps que
l’intrusion elle-même, mais un peu moins vite. Ceci entraîne la
formation d’une région « froide » au niveau du front de
l’intrusion. Dans le régime gravitaire, l’épaisseur de l’écoulement
est constante, les pertes de chaleur par conduction rapidement
contrebalancent l’injection de fluide chaud et l’anomalie thermique
atteint un régime stationnaire. Dans chaque cas, une étude
quantitative du transport de chaleur au sein du fluide nous permet de
prédire l’évolution de l’anomalie thermique en fonction des paramètres
de l’écoulement.

Les  régions froides  s’écoulent  plus difficilement  que les  régions
chaudes et au finale, chaque régime  peut être découpé en trois phases
différentes.   Dans une  première phase,  l’écoulement ne  ressent pas
encore le refroidissement  et se comporte comme  un fluide isovisqueux
chaud.    Dans   une   deuxième  phase,   l’écoulement   ralentit   et
s’épaissit. Finalement, l’écoulement retourne dans une phase similaire
au cas isovisqueux,  mais où cette fois, le  fluide serait entièrement
froid.  Bien  que ces  phases soient communes  aux deux  régimes, nous
montrons que  la viscosité  effective, qui  contrôle le  passage d'une
phase à l'autre, n’est pas la même  dans les deux régimes; elle est la
viscosité  moyenne  d’une  petite  région  au  front  dans  le  régime
élastique  et la  viscosité  moyenne de  l’écoulement  dans le  régime
gravitaire.  Dans  le chapitre  \ref{Heating},  nous  montrons que  la
dynamique  n’est que  peu affectée  par le  chauffage de  l’encaissant
combiné avec  une rhéologie  plus réaliste  pour le  magma. Cependant,
ceci nous  permet de quantifier  au premier ordre sur  quelle distance
l'intrusion chauffe, et donc  potentiellement induit du métamorphisme,
dans la roche encaissante.

Pour conclure cette étude, nous confrontons les prédictions faites par
le nouveau modèle aux différentes observations présentées dans le
Chapitre \ref{chap2}. En particulier, le modèle prédit que l’épaisseur
de l’intrusion augmente comme le rayon à la puissance $8/7$, mais
aussi avec le contraste de viscosité entre le magma « chaud » et le
magma « froid ». Ainsi les laccolites se sont probablement arrêtés
dans la troisième phase du régime élastique; par exemple, si tel est
le cas, la composition felsique des laccolites de l’île d’Elbe permet
d’expliquer leurs épaisseurs et leurs rayons. Cependant, ni sur la
Terre ni sur la Lune, la formation de la région de viscosité
importante au niveau du front n’entraîne l'arrêt
des laccolites qui se solidifient plus tard dans cette phase de
l'écoulement. Les sills se sont quant à eux probablement arrêtés dans
le second régime gravitaire.

En conclusion,  le refroidissement n’a donc  probablement pas provoqué
l’arrêt de  ces intrusions. Celui-ci  est peut-être simplement  lié au
tarissement de  la source de magma  en profondeur. En effet,  le temps
pour atteindre la transition augmente  avec le contraste de viscosité.
Si celui-ci  est faible est  l'injection suffisante pour  atteindre le
régime gravitaire, l'intrusion  se solidifie sous forme  de sill. Dans
le  cas contraire,  elle se  solidifie sous  forme de  laccolite. Ceci
pourrait expliquer  la prédominance  des laccolites felsiques  dans la
nature, leur contraste de viscosité plus important les rendant moins à
même  d'atteindre le  régime  gravitaire avant  le  tarissement de  la
source.

%%% Local Variables:
%%% mode: latex
%%% TeX-master: "main"
%%% End:
