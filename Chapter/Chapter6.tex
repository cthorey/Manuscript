\chapter{Gravitationnal signature of lunar floor-fractured craters}
\label{chap7}

\textit{This chapter is a reproduction of the paper published in Earth
  and Planetary Science Letters (EPSL) untitled: \textbf{Gravitational
    signatures      of      lunar      floor-fractured      craters}}
\citep{Thorey:2015hw}.

\minitoc

\begin{abstract}
  %% Text of abstract

  Lunar floor-fractured  craters are  impact craters  characterized by
  distinctive  shallow   floors  crossed  by  important   networks  of
  fractures. Different  scenarios have been proposed  to explain their
  formations but recent studies showed  that the intrusion of magma at
  depth below the crater floor is the most plausible explanation.  The
  intrusion of  dense magma  within the light  upper-most part  of the
  lunar crust  should have  left a positive  signature in  the gravity
  field.  This  study takes advantage of  the unprecedented resolution
  of the lunar gravity field obtained from the NASA's Gravity Recovery
  and  Interior  Laboratory  (GRAIL)   mission,  in  combination  with
  topographic  data obtained  from the  Lunar Orbiter  Laser Altimeter
  (LOLA) instrument,  to investigate  the gravitational  signatures of
  both  normal   and  floor-fractured  craters.   Despite   the  large
  variability in  their gravitational signatures,  the floor-fractured
  and normal  craters in  the Highlands show  significant differences:
  the   gravitational   anomalies    are   significantly   larger   at
  floor-fractured craters. The  anomaly amplitudes for floor-fractured
  craters are in  agreement with synthetic gravity  anomalies based on
  the predicted intrusion  shapes from a theoretical  flow model.  Our
  results are  consistent with  magmatic intrusions intruding  a crust
  characterized by  a $12\%$ porosity  and where the intrusion  has no
  porosity. Similar studies  have been carried out in  the lunar maria
  and  South Pole  Aikten basin.   Although the  average gravitational
  signature  of  floor-fractured  craters  is larger  than  at  normal
  craters in these regions, they cannot be distinguished statistically
  due to the small number of  craters and the large variability of the
  anomalies.  In general, a better  characterization of the signal due
  solely  to  the initial  impact  crater  is  needed to  isolate  the
  magmatic  intrusion signal  and  characterize  the density  contrast
  between the magma and crust.
\end{abstract}

\section{Introduction}
\label{sec:introduction}
  
There are a class of impact craters on the Moon that are distinguished
by  having  uplifted  floors and  radially/concentric  floor-fractured
networks.  About 200 of these floor-fractured craters (FFCs) have been
identified  by \citet{Schultz:1976kt}  and  these  impact craters  are
interpreted  to have  undergone  endogenous  deformations after  their
formation.   The  most striking  feature  of  these craters  is  their
shallow floors compared  to normal craters of the same  size, with the
uplift  reaching $50\%$  of the  initial  crater depth  in some  cases
\citep{Schultz:1976kt}.  Due  to this  deformation, their  floors show
large  networks  of  radial,   concentric  and  pentagonal  fractures.
Additionally,  depending on  local conditions,  the uplift  results in
either a  convex floor or  a flat  plate-like floor, sometimes  with a
wide     circular     moat     just    interior     to     the     rim
\citep{Schultz:1976kt,Jozwiak:2012dq}.

Intrusion of magma beneath the  crater floor and viscous relaxation of
the crater topography  after the impact are two  proposed scenarios to
explain                       these                       deformations
\citep{Schultz:1976kt,Hall:1981kl,Wichman:1995ju,Dombard:2001gs}.  The
recent  theoretical   model  for   the  dynamics   of  crater-centered
intrusions  of  \citet{Thorey:2014cv}  and  recent  morphological  and
geological studies by \citet{Jozwiak:2012dq}  showed that intrusion of
magma  beneath the  crater floor  is  the most  plausible scenario  to
produce  the   morphological  features  observed   at  floor-fractured
craters.
  
Magmatic  intrusions should  be  emplaced at  their  level of  neutral
buoyancy   \citep{Walker:1989jq,Taisne:2009kj,Wichman:1995ju}.    Upon
cooling and  solidification, however,  their densities will  be larger
than  the surrounding  crustal material  and hence,  leave a  positive
signature in the gravity  field.  \citet{Schultz:1976kt} looked at the
gravitational signature  of some floor-fractured craters  with gravity
derived from radio  tracking data acquired during the  Apollo $15$ and
$16$  missions \citep{Sjogren:1972kk,Sjogren:1974ij}.   Except at  the
floor-fractured crater  Taruntius, where a strong  gravity anomaly was
detected,  no  pronounced gravity  anomalies  were  observed at  other
floor-fractured crater  sites.  However,  the gravity data  from these
two experiments only covered a narrow  swath along the equator and the
resolution of the  data used in these studies were  not able to detect
gravity anomalies for objects smaller  than about $100$ km in diameter
\citep{Schultz:1976kt,Sjogren:1974ij}, which includes  about $88\%$ of
the floor-fractured crater population.

Data from  the NASA Gravity  Recovery and Interior  Laboratory (GRAIL)
NASA mission  have provided a global  map of the Moon's  gravity field
with  an  unprecedented resolution.   These  data  have been  used  to
construct a  model of the  gravity field to spherical  harmonic degree
and order  900, which corresponds  to a half-wavelength  resolution of
$\sim 6$ km at the lunar surface \citep{Zuber:2013cp,Konopliv:2014gm}.
These data,  used in  combination with  the topographic  data obtained
from  the Lunar  Orbiter Laser  Altimeter (LOLA)  instrument allow  investigating mass anomalies located in the lunar crust.  In particular,
these  data allow  resolving  small-scale density  variations in  the
shallow crust \citep{Besserer:2014jr,Wieczorek:2013ipa}  and they have
been     used     to     detect     ancient     igneous     intrusions
\citep{AndrewsHanna:2013ft}.

In this paper,  after some theoretical considerations  on the expected
gravitational   signal   at   floor-fractured   craters   in   section
\ref{sec:theor-cons}, we use  GRAIL gravity to detect  the presence of
magmatic  intrusions  at  floor-fractured   crater  sites  in  section
\ref{sec:grav-sign-lunar}.  Then,  we develop  a method to  derive the
density   contrast   between   the   magma  and   crust   in   section
\ref{sec:magm-intr-char}.  We  discuss its geological  implications in
section     \ref{sec:discussion}    and     conclude    in     section
\ref{sec:conclusion}.

\section{Theoretical considerations}
\label{sec:theor-cons}

The Bouguer  anomaly associated  with a  magmatic intrusion  beneath a
crater depends upon the  intrusion characteristics, namely its density
and shape.  Recently, we showed that the morphology of crater-centered
intrusions depends mainly upon the  thickness of the overlying elastic
layer and  on the  crater size  \citep{Thorey:2014cv}.  Guided  by the
results and predictions of our model that is briefly summarized below,
we here  calculate and  discuss the  expected gravitational  signal at
floor-fractured crater sites.

\subsection{Constitutive equations}
\label{sec:const-equat-1}

In  our model,  the intrusion  is  fed at  a constant  rate through  a
cylindrical conduit  located below  the center of  a crater  floor and
spreads   horizontally   along   a    thin   bedding   plane   (Figure
\ref{Figure2-1}).
%% FIGURE 1
\begin{figure}[h!]
  \graphicspath{ {/Users/thorey/Documents/These/Projet/FFC/Gravi_GRAIL/Article/Papier/SOUMISSION_2_EPSL/} }
  \begin{center}
    \includegraphics[scale=0.60]{Figure_2-1.pdf}
    \caption{Model geometry and parameters.}
    \label{Figure2-1}
  \end{center}
\end{figure}
The  magma makes  rooms for  itself by  lifting the  overlying assumed
elastic  layer, which  is characterized  by a  Young's modulus  $E$, a
Poisson's ratio $\nu$ and an elastic thickness $T_e(r)$ given by
\begin{equation}
  T_e(r)=T_e^0+d_c\xi(r)\hspace{.2cm}\text{, with }\hspace{.2cm} \xi(r)=\frac{1}{1+e^{-\frac{2\alpha(2r-D)}{D}}}-\frac{1}{1+e^{2\alpha}}\label{topo}\\
\end{equation}
where $T_e^0$ is  the overlying layer thickness at  the crater center,
$d_c$  the crater  depth with  respect to  the pre-impact  surface and
$\xi(r)$  a normalized  sigmoid  function which  reproduces a  typical
complex crater depression in terms of the crater diameter $D$ and wall
slope $\alpha$ (Figure \ref{Figure2-1}, top).

The thickness  evolution equation  in cylindrical coordinates  for the
flow of a newtonian fluid is given by \citep{Thorey:2014cv}
\begin{eqnarray}
  \frac{\partial         h}{\partial        t}
  &=&\frac{\rho_{m}g}{12  \mu} \frac{1}{r}  \frac{\partial}{\partial
      r}\left   (r  h^{3}   \frac{\partial  h}{\partial   r}  \right)+
      \frac{\rho_{c}g\Psi       T_{e}^0}{12       \mu}       \frac{1}{r}
      \frac{\partial}{\partial r}\left ( r
      h^{3}  \frac{\partial  \xi(r)}{\partial  r}\right )  \nonumber  \\
  &&+ \frac{E T_{e}^{0^{3}}}{144\mu
     (1-\nu^{2})}\frac{1}{r}\frac{\partial}{\partial  r}\left  (  r
     h^{3}    \frac{\partial}{\partial   r}    \left(\nabla^{2}_{r}
     ((1+\Psi \xi(r))^{3}\nabla^{2}_{r}h )\right)\right )\nonumber\\
  && + w(r,t)
     \label{eq16}\\
  w(r,t)&=&
            \begin{cases} \frac{ \Delta P}{4 \mu Z_{c}} (\frac{a^{2}}{4}-r^{2})&
              r \le \frac{a}{2}\\ 0 & r > \frac{a}{2}
            \end{cases}
                                      \label{eq12}
\end{eqnarray}
where  $h(r,t)$  is  the  intrusion   thickness,  $r$  is  the  radial
coordinate, $t$ is time, $\rho_m$ and $\rho_c$ are the magma and crust
density, $\mu$ is the viscosity,  $\Psi = d_c/T_e^0$ is the thickening
of the overlying layer at the wall, $w(r,t)$ is the injection velocity
and $\Delta P/Z_c$  is the overpressure gradient  driving magma ascent
in the feeder dyke.

The terms  on the right  side of (\ref{eq16})  respectively represent,
from left  to right,  spreading due to  magma weight,  the lithostatic
barrier the  magma has to  face at the  crater wall, squeezing  of the
flow in  response to  elastic deformation of  the overlying  layer and
injection  rate.  Equation  (\ref{eq16}) is  non-dimensionalized using
the crater radius $D/2$ as a  horizontal scale, a height scale $H$ and
a time scale $\tau$ given by
\begin{equation}
  \label{eq18} H= \left (\frac{12\mu Q_{0}}{\rho_{m}g \pi}\right )
  ^{\frac{1}{4}}
\end{equation}
\begin{equation}
  \tau=\frac{\pi     D^{2}      H}{4Q_{0}}\label{eq19}
\end{equation}
where  $H$ is  the characteristic  height scale  of a  gravity current
\citep{Huppert:1982a} and $\tau$ is the characteristic time to fill up
the crater depression for a constant injection rate
\begin{equation} Q_{0}=\frac{\pi \Delta P a^{4}}{128 \mu Z_c} .
  \label{eq11}
\end{equation}
Another useful lengthscale  in the problem is  the flexural wavelength
$\Lambda$ \citep{Michaut:2011kg}
\begin{equation}
  \Lambda &=& \left( \frac{E
      T_{e}^{0^{3}}}{12                        (1-\nu^{2})\rho_{m}g}
  \right )^{\frac{1}{4}}
\end{equation}
which represents the wavelength of deformation of the elastic layer.

Equation (\ref{eq16}) made dimensionless becomes
\begin{eqnarray}    \frac{\partial     h}{\partial    t}&=&\frac{1}{r}
  \frac{\partial}{\partial r}\left  (rh^{3} \frac{\partial h}{\partial
      r} \right)+ \Xi \frac{1}{r} \frac{\partial}{\partial r}\left
                                                            (   rh^{3}\frac{\partial    \xi(r)}{\partial   r}\right   )\nonumber\\
                                                        &+&\Theta
                                                            \frac{1}{r}\frac{\partial}{\partial
                                                            r}\left
                                                            (   rh^{3}
                                                            \frac{\partial}{\partial
                                                            r}
                                                            \nabla^{2}_{r}\left
                                                            ((1+\Psi
                                                            \xi(r))^{3}\nabla^{2}_{r}h
                                                            \right )\right)+
                                                            \frac{32}{\gamma^{2}}
                                                            \left(\frac{1}{4}-\frac{r^{2}}{\gamma^{2}}\right)
                                                            \label{eq21}
\end{eqnarray}
where $\xi(r)$ is also made dimensionless
\begin{equation}
  \xi(r)=\frac{1}{1+e^{-\frac{(r-1)}{\zeta}}}-\frac{1}{1+e^{\frac{1}{\zeta}}}\label{eqqqq}
\end{equation}
and where
\begin{eqnarray}
  \label{Dimensionless1} 
  \gamma&=&\frac{2a}{D}= 0.02\label{n1}\\
  \zeta&=&\frac{d_c}{2\alpha D}=0.13
           \label{n2}\\ \Xi&=&
                               \left(\frac{\rho_{c}gd_{c}}{\rho_{m}gH}\right )
                               = 20 \label{n3}\\
  \Psi&=&\frac{d_{c}}{T_{e}^0} = 1\label{n4}. 
\end{eqnarray}
$\gamma$ is the dimensionless source  width, $\zeta$ is four times the
normalized crater  wall width  and characterizes the  crater geometry,
$\Xi$ is  the ratio between  the lithostatic pressure increase  at the
crater wall  and the  hydrostatic pressure  due to  a magma  column of
thickness  $H$, which  quantifies  the importance  of the  lithostatic
barrier at the crater wall, and $\Psi$ is the dimensionless thickening
of the upper elastic layer,  which characterizes the elastic thickness
increase at the crater wall.   These are dimensionless parameters that
do not  significantly affect our  results and are considered  fixed in
our analysis \citep{Thorey:2014cv}.  The dimensionless number $\Theta$
\begin{equation}
  \Theta=\left ( \frac{2\Lambda}{D} \right )^{4}\label{n5}
\end{equation}
is the dimensionless flexural wavelength  of the upper layer raised to
the power $4$ that quantifies the  length scale over which the elastic
deformation  is effective  relative to  the crater  radius. It  varies
between  $10^{-5}$ and  $10^{-1}$ and  has a  strong influence  on the
intrusion shape and final floor appearance.

\subsection{End-member modes of deformation}
\label{sec:end-member-modes-1}

For a constant injection rate and no crater depression, i.e a constant
upper  elastic  layer ($\xi(r)  =  0  $  and  $T_e(r) =  T_e^0$),  the
numerical  resolution of  the  equations shows  two spreading  regimes
\citep{Michaut:2011kg,Michaut:2013dr}.   The flow  is first  driven by
the bending of the upper  elastic layer.  The intrusion is bell-shaped
and both the radius and the  thickness evolve close to $t^{1/3}$. When
the radius becomes larger than $4\Lambda$, the weight of the intrusion
becomes dominant  over the  bending terms and  the intrusion  enters a
gravity current regime.  In this  second regime, the intrusion shows a
flat  top with  bent edges,  the radius  evolve as  $t^{1/2}$ and  the
thickness tends to a constant \citep{Huppert:1982a,Michaut:2011kg}.
 
For a  constant injection  rate and a  crater-like topography  for the
upper  layer,  i.e.  $T_e(r)$  given  by  (\ref{topo}), the  spreading
regimes are perturbed by the presence of the crater wall.  The central
flat floor of the crater first  acts as a constant elastic upper layer
and the intrusion spreads as described above. However, when it reaches
the  crater  wall,  the  important increase  in  lithostatic  pressure
prevents  the  magma  from   spreading  horizontally.   The  intrusion
thickens   in   response   and    the   crater   floor   is   uplifted
\citep{Thorey:2014cv}.   Accordingly, the  intrusion thickness  can be
estimated from the amount of uplift of the crater floor at the center.

%% FIGURE 2
\begin{figure}[h!]
  \graphicspath{ {/Users/thorey/Documents/These/Projet/FFC/Gravi_GRAIL/Article/Papier/SOUMISSION_2_EPSL/} }
  \begin{center}

    \includegraphics[scale=0.80]{Figure_2-2.pdf}
    \caption{Two  end-member   intrusion  shapes  producing   the  two
      end-member  floor deformations  observed  at  FFC sites:  convex
      floor (left) and plate-like floor (right).}
    \label{Figure2-2}
  \end{center}
\end{figure}

The final morphology  of the crater floor depends mainly  on the ratio
between the  flexural wavelength and  the crater radius, i.e.   on the
dimensionless  number  $\Theta$  (\ref{n5}).   For a  large  value  of
$\Theta$, i.e. a  deep intrusion and/or a small  crater, the intrusion
reaches the wall  in the bending regime. The  intrusion is bell-shaped
and the uplift  of the crater floor leads to  a shallowed convex floor
(Figure \ref{Figure2-2},  left).  In  contrast, for  a small  value of
$\Theta$,  i.e.   a  shallow  intrusion and/or  a  large  crater,  the
intrusion is  in a gravity current  regime when it reaches  the crater
wall.   The  thickening of  the  cylinder-like  intrusion leads  to  a
piston-like uplift of the crater floor and to a shallowed central flat
floor for the crater (Figure \ref{Figure2-2}, right).

Accordingly, this model results into two main types of floor-fractured
craters:  craters  with  convex floors  corresponding  to  bell-shaped
intrusions (Figure \ref{Figure2-2}, left)  and craters with plate-like
floors   corresponding    to   cylinder-shaped    intrusions   (Figure
\ref{Figure2-2},  right).  In  the following,  we consider  craters of
classes   $2$  and   $4$   of   \citet{Schultz:1976kt}  to   represent
manifestations of bell-shaped intrusions,  and craters of classes $1$,
$3$, $5$ and $6$ to be manifestations of cylinder-shaped intrusions.
 
\subsection{Gravitational signature of FFCs: two case studies}
\label{sec:grav-sign-ffcs-1}

The Bouguer gravity  is the gravity anomaly that  remains after taking
into account the gravitational signature of surface topography.  Thus,
if there  are no  lateral variations in  crustal density,  the Bouguer
anomaly of a  floor-fractured crater should be entirely  the result of
the  magmatic  intrusion.   The  Bouguer  anomaly  associated  with  a
magmatic  intrusion   depends  upon  the  intrusion   morphology,  the
intrusion depth  and the  density contrast  $\Delta \rho$  between the
intrusion and the surrounding crust.

We investigate  the signal  expected at  floor-fractured craters  as a
function of the  intrusion shape, diameter and  depth.  In particular,
for the bell and cylinder shaped intrusion, we study the signal for an
intrusion diameter  that varies between  $20$ and $180$ km,  i.e.  the
minimum and  maximum diameter  of floor-fractured craters  observed by
\citet{Schultz:1976kt} and  two intrusion depths, i.e.   $T^0_e =0$ km
and a reasonable upper bound  of $5$ km \citep{Thorey:2014cv}.  We set
the  intrusion thickness  $H_0$ to  a value  of $2$  km, which  is the
maximum  uplift observed  by  \citet{Schultz:1976kt}  and the  density
contrast   between  the   magma  and   the   crust  to   a  value   of
$\Delta \rho = 600$ kg  m$^{-3}$, the maximum density contrast between
mafic  and   crustal  lunar   rocks  using   the  bulk   densities  of
\citet{Kiefer:2012kp}.
 
The two end-member intrusion profiles are obtained by solving equation
(\ref{eq21})  for two  different  values of  the dimensionless  number
$\Theta$ (\ref{n5}),  $\Theta= 10^{-2}$ for the  bell-shaped intrusion
and    $\Theta=    10^{-5}$    for   the    cylinder-like    intrusion
\citep{Thorey:2014cv} .   We stop the  simulations when the  uplift at
the center $h_0$ is significant and  such that the pressure due to the
intrusion weight at the center  is about half the lithostatic pressure
due  to  the  crater  wall   ($h_0=10$).   Finally,  each  profile  is
redimensionalized: the axial coordinate of the dimensionless thickness
profile is multiplied by $D/2$ and the dimensionless thickness profile
is multiplied by $H_0/h_0$, where $h_0$ is the dimensionless thickness
of the intrusion at the end of the simulation.

We calculate the synthetic radial gravity anomaly (more precisely, the
gravity  disturbance)  $\delta_g^s$  corresponding to  each  intrusion
profile using the spherical harmonics expansion
\begin{equation}
  \delta_g^s(r,\theta,\phi)&=&\frac{G
    M}{r^2}\sum_{l=0}^{L_{\text{max}}}\sum_{m=-l}^{l}\left(\frac{R_{i}}{r}\right)^l(l+1)C_{lm}Y_{lm}(\theta,\phi)
\end{equation}
where $r,\theta$ and $\phi$ are the coordinates of observation, $G$ is
the  gravitational constant,  $M$  the  mass of  the  Moon, $R_i$  the
reference radius of  the spherical harmonic coefficients  taken as the
mean     radius     at     the     site     of     intrusion,     i.e.
$R_i =  R_0-T_e^0+\overline{h}$ where $R_0$  is the mean  lunar radius
and  $\overline{h}$  the  mean   intrusion  thickness,  $C_{lm}$,  and
$Y_{lm}$ the spherical harmonic functions  of degree $l$ and order $m$
\citep{wieczorek:1998th}.  Gravitational  accelerations are considered
positive when  directed downward  (see Appendix \ref{chap:A5}  for the
expression of the spherical  harmonic coefficients associated with the
intrusion thickness profile and the calculation details).

%% FIGURE 3
\begin{figure}[h!]
  \graphicspath{ {/Users/thorey/Documents/These/Projet/FFC/Gravi_GRAIL/Article/Papier/SOUMISSION_2_EPSL/} }
  \begin{center}
    \includegraphics[scale=0.4]{Figure_2-3.eps}
    \caption{Top  left: Calculated  synthetic  gravity  anomaly for  a
      bell-shaped  intrusion  at  the  surface for  $T_e^0  =  0$  km,
      $H_0 = 2$ km, $D = 180$  km and $\Delta \rho = 600$ kg m$^{-3}$.
      Dotted  red line  represents the  crater rim.   Top right:  Mean
      synthetic gravity anomaly  as a function of  the crater diameter
      $D$  for  a  bell-shaped  intrusion   with  $H_0  =  2$  km  and
      $\Delta \rho = 600$ kg  m$^{-3}$.  Red solid line: the intrusion
      is  at  the  surface,  $T_e^0  =0$ km.   Blue  solid  line:  the
      intrusion is $5$ km below the  surface, $T_e^0 = 5$ km.  Bottom:
      Same as above, but for a cylinder-shaped intrusion.}
    \label{Figure2-3}
  \end{center}
\end{figure}

The two  different intrusion shapes  result in two different  types of
anomaly (Figure \ref{Figure2-3}, left).   For a bell-shaped intrusion,
with a convex  crater floor, the gravity anomaly  is also bell-shaped.
It  decreases gradually  from the  center to  the crater  wall (Figure
\ref{Figure2-3}, top left).   For an intrusion placed  at the surface,
$T_e^0 =0$,  the signal  barely depends on  the crater  diameter.  The
mean gravity  anomaly measured interior  to the crater wall  is almost
constant  and about  $15$  mGal (Figure  \ref{Figure2-3}, top  right).
Although an increased intrusion depth  decreases the mean value of the
anomaly by  a factor that  is less than  two for craters  smaller than
$50$ km, it barely affects the anomaly for craters larger than $50$ km
(Figure \ref{Figure2-3}; top right).

For  a cylinder-like  intrusion  and a  plate-like  crater floor,  the
gravity anomaly  is relatively  uniform and  sharply decreases  at the
crater  wall  (Figure  \ref{Figure2-3}, bottom  left).   Consequently,
although the  maximum amplitude of the  anomaly is similar to  the one
produced  by  a  bell-shaped  intrusion,  the  mean  gravity  anomaly,
measured interior to  the crater wall, is twice larger  and about $30$
mGal. Similar  to bell-shaped intrusion, an  increased intrusion depth
decreases the mean value of the anomaly  by a factor that is less than
two for  craters smaller than $50$  km but barely affects  the anomaly
for  craters  larger  than  $50$ km  (Figure  \ref{Figure2-3},  bottom
right).
  
One important observation of our modeling  is that even for an extreme
case of a $2$ km thick and a $180$ km diameter intrusion placed at the
surface  with a  large  density  contrast of  $600$  kg m$^{-3}$,  the
expected signal is only of a few tens of mGal. Though GRAIL can easily
detect such small  amplitude anomalies, this gravity  signals could be
masked  by  both  large-scale regional  signals  and  short-wavelength
signals  that  are   unrelated  to  our  idealized   model  of  Figure
\ref{Figure2-1}.   Therefore,  we  need  to  filter  out  these  other
contributions to be able to  detect the potential presence of magmatic
intrusions at  floor-fractured craters.  In the  following, we present
the  gravity model  that we  use and  consider the  remaining expected
signal  after  filtering the  synthetics  the  same way  the  observed
gravity are filtered.

\subsection{Filtered GRAIL gravity}
\label{sec:grails-gravity-model-1}

The  observed  gravity field  on  the  Moon  is  a result  of  several
contributions,   including  surface   topography,  relief   along  the
crust-mantle interface and density  heterogeneities in both the mantle
and the crust. In order to  detect the presence of magmatic intrusions
in the  shallow crust, which  have predicted  anomalies of only  a few
tens of mGal,  we first remove all known signatures  from the observed
gravity field in order to highlight those signals that remain.

To construct this model, we  start with JGGRAIL 900C11A gravity field,
which is developed to spherical harmonic degree 900 and which is based
on all  data obtained  during the GRAIL  primary and  extended mission
\citep{Konopliv:2014gm}.  From  the free-air  gravity model,  we first
compute the Bouguer anomaly by removing the gravitational contribution
of surface  topography and  the long-wavelength variations  in crustal
density that are  predicted from remote sensing data,  as described in
\citet{Wieczorek:2013ipa}. The most prominent  signals that remain are
either associated with large impact  basins or are anticorrelated with
long-wavelength topography.  We interpret  the majority of this signal
as  being the  result of  crustal  thickness variations,  and use  the
Bouguer anomaly to invert for the gravitational signal of relief along
the crust-mantle interface, as described in \citet{Wieczorek:2013ipa}.

\begin{figure}[h!]
  \graphicspath{ {/Users/thorey/Documents/These/Projet/FFC/Gravi_GRAIL/Article/Papier/SOUMISSION_2_EPSL/} }
  \begin{center}
    \includegraphics[scale=1]{Figure_2-4.pdf}
    \caption{Power spectra  for various  gravity models.   Black solid
      line:  free-air gravity  from  the GRAIL  gravity model  JGGRAIL
      900C11A.   Red solid  line: Bouguer  gravity anomaly  assuming a
      constant  crustal density  of  $2550$ kg  m$^{-3}$.  Blue  solid
      line:  Crustal  gravity  anomaly  of our  model  CrustAnom  with
      $\lambda=80$  and which  removes  long-wavelength variations  in
      crustal density as predicted by remote sensing data.}
    \label{Figure2-4}
  \end{center}
\end{figure}

Since  the shortest  wavelength  signals in  the  Bouguer anomaly  are
unlikely to be  the result of crustal thickness  variations, and since
short-wavelength  signals become  highly  amplified when  extrapolated
with  depth  below  the  surface,  we apply  the  low-pass  filter  of
\citet{wieczorek:1998th} to  the Bouguer anomaly before  inverting for
crustal thickness variations.  This  filter is parameterized by having
a value of $0.5$ at spherical harmonic degree $\lambda$. The choice of
$\lambda$  is  subjective,  and  $\lambda$ is  chosen  such  that  the
obtained crustal thickness map does not contain excessive power at the
shortest  wavelengths.   In \citet{Wieczorek:2013ipa},  $\lambda$  was
chosen to  have a  value of  $80$.  Here, we  test several  values for
$\lambda$ and find that $\lambda=80$  is also a good trade-off between
the removal of  regional trends and the removal of  signals due to the
magmatic  intrusion itself  (see Appendix  \ref{chap:A5} Figure  1 for
details on the effects of $\lambda$).

\begin{figure}[h!]
  \graphicspath{ {/Users/thorey/Documents/These/Projet/FFC/Gravi_GRAIL/Article/Papier/SOUMISSION_2_EPSL/} }
  \begin{center}
    \includegraphics[scale=0.4]{Figure_2-5.eps}
    \caption{Top left:  Calculated synthetic gravity  anomaly filtered
      the same way as the  model CrustAnom for a bell-shaped intrusion
      at the surface with  $T_e^0 = 0$ km, $H_0 = 2$ km,  $D = 180$ km
      and $\Delta \rho = 600$ kg m$^{-3}$.  Dotted red line represents
      the  crater  rim. Top  right:  Mean  filtered synthetic  gravity
      anomaly  as  a  function  of  the  crater  diameter  $D$  for  a
      bell-shaped intrusion with $H_0 = 2$  km and $\Delta \rho = 600$
      kg m$^{-3}$.  Red  solid line: $T_e^0 =0$ km.   Blue solid line:
      $T_e^0   =  5$   km.   Bottom:   Same  as   above,  but   for  a
      cylinder-shaped intrusion.}
    \label{Figure2-5}
  \end{center}
\end{figure}

After  removing   the  gravitational  signal  of   the  crustal-mantle
interface from  the Bouguer  anomaly, the remainder  of the  signal is
attributed to  lateral variations in  density within the crust  of the
Moon. To remove short wavelength  noise in the gravitational field, we
also  apply a  cosine filter  to the  spherical harmonic  coefficients
between degree $550$ and $650$. It  is from this map, here referred to
as CrustAnom,  that we  search for gravitational  anomalies associated
with  floor-fractured   craters.   Our  model  CrustAnom   is  roughly
equivalent to a  band-passed Bouguer anomaly, where  both the shortest
and longest wavelength signals are removed (Figure \ref{Figure2-4}).

Upon applying the  same filtering to synthetic  gravity anomalies, the
expected signal at floor-fractured craters  is reduced with respect to
those  considered  in   section  (\ref{sec:grav-sign-ffcs-1})  (Figure
\ref{Figure2-5}).  The filtering, which  affects mostly large craters,
leads to a  drop in the amplitude  of the gravity anomaly  by a factor
larger than two for craters larger than about $80$ km.  As an example,
the mean anomaly for  the extreme case of a $2$ km  thick and $180$ km
large intrusion  should only be of  a few mGal in  the model CrustAnom
(Figure \ref{Figure2-5}).


\section{Gravitational signature of lunar craters}
\label{sec:grav-sign-lunar}

The  gravitational  signal  associated  with  magmatic  intrusions  at
floor-fractured craters will be superimposed on the signal of a normal
impact  crater.  We  use the  model  CrustAnom to  first quantify  the
gravity signal at normal impact craters and then compare to the signal
at found floor-fractured craters.
 
\subsection{Normal and floor-fractured crater populations}
\label{sec:unmod-floor-fract}
 
We use the  dataset of \citet{Head:2010fy} as a  reference catalog for
normal  craters  and  the   dataset  of  \citet{Jozwiak:2012dq}  as  a
reference  catalog  for  floor-fractured craters.   We  consider  only
complex craters  and thus use  a minimum  crater diameter of  $20$ km,
which is the  transitional crater diameter between  simple and complex
lunar  craters  \citep{Pike:1974ux,Pike:1980eh}.   We  use  a  maximum
crater diameter  of $180$ km,  because for larger craters,  the mantle
uplift associated with basin formation becomes apparent in the gravity
data \citep{Melosh:2013cz}.  These criteria  result in a population of
$116$  floor-fractured and  $5101$ normal  craters covering  the whole
lunar surface.
	
\begin{figure}[h!]
  \graphicspath{ {/Users/thorey/Documents/These/Projet/FFC/Gravi_GRAIL/Article/Papier/SOUMISSION_2_EPSL/} }
  \begin{center}

    \includegraphics[scale=0.3]{Figure_3-1.pdf}
    \caption{\textbf{Top  row)} Left:  All  FFCs  (red triangles)  and
      normal craters (light blue circles) in the highlands. Right: All
      FFCs  (red triangles)  and  normal craters  that  have the  same
      spatial    distribution     as    the    FFCs     (light    blue
      circles). \textbf{Middle row)} Same plots but for craters in the
      maria.  \textbf{Bottom row)} Same plots but for craters in South
      Pole Aikten basin (SPA).}
    \label{Figure3-1}
  \end{center}
\end{figure}

The observed  gravity field of an  impact crater will depend  upon the
density of the crust. GRAIL gravity  data show that crustal density is
not constant, and that regional  variations of $\pm$ $250$ kg m$^{-3}$
exist, primarily between the highlands and the South Pole Aitken basin
(SPA). In  addition, surface densities  in the maria  are considerably
higher  than in  the highlands  \citep{Besserer:2014jr}.  To  minimize
potential biases that might arise  from regional variations in crustal
density  or  geologic  evolution,  we divide  each  crater  population
(normal and floor-fractured craters)  into three sub-populations.  The
first is constituted by craters that lie within the highlands, outside
of  both the  maria  and South  Pole-Aitken basin.   We  use the  USGS
geological  maps to  define  the mare  borders and  the  SPA basin  is
defined      using     the      best-fit     outer      ellipse     of
\citet{GarrickBethell:2009dx}.   In  the  highlands,  there  are  $80$
floor-fractured and $4054$ normal craters (Figure \ref{Figure3-1}, top
left).  The second  sub-population is constituted by  craters that lie
within  the maria  and outside  of  SPA basin  of which  there are  22
floor-fractured and 306 normal craters (Figure \ref{Figure3-1}, middle
left).  The last  sub-population is constituted by  craters within the
SPA  of which  there are  14  floor-fractured and  603 normal  craters
(Figure \ref{Figure3-1}, bottom left).
	 
In   each  region   defined   above,  the   spatial  distribution   of
floor-fractured    and   normal    craters   is    different   (Figure
\ref{Figure3-1}, left).  To minimize any biases that might result from
different regional characteristics, we also consider, for each region,
a second sub-population of normal craters that shares the same spatial
distribution as  the floor-fractured craters  (Figure \ref{Figure3-1},
right).  In this  sub-population, we consider all  normal craters that
are less than $150$ km away from a floor fractured crater.

\subsection{Crater gravitational signatures}
\label{sec:crat-grav-sign}
  
In analyzing the  gravitational signature of lunar  impact craters, we
make  use of  a single  number, the  crater gravity  anomaly, that  is
defined  as the  average  gravitational anomaly  with  respect to  the
regional value.   In calculating this  number, we first  calculate the
average gravitational anomaly from our model CrustAnom within the main
crater rim, i.e.  within a circular region defined by its radius $D/2$
where $D$ is  the crater diameter reported  by \citet{Head:2010fy} and
\citet{Jozwiak:2012dq}.  We then subtract  from this value the average
value of the gravity field in an annulus extending from the crater rim
to a radius of one crater diameter $D$ (Appendix \ref{chap:A5}, Figure
2).   Both  gravitational  anomalies  are calculated  at  the  average
elevation of the crater.

\subsubsection{Highlands}
\label{sec:highlands-1}
  
The magnitude of the gravity anomalies at normal crater sites shows an
important  variability  (Figure  \ref{Figure3-2}).   On  average,  the
anomaly is  positive at  the smallest  craters, slowly  decreases with
increasing diameters, and approaches a  constant negative value near a
diameter  of about  100 km.   For crater  diameters between  $100$ and
$180$ km, the  mean magnitude of the gravity  anomalies is independent
of  the diameter  and close  to $-5$  mGal.  The  mean of  the gravity
anomalies for  the whole  population $\mu_{\delta_g}$ is  negative and
equal to  $-0.71$ mGal.  This  number is  well constrained due  to the
large number of  craters.  In particular, the uncertainty  in the mean
(the standard  error, which is  the standard deviation divided  by the
square root of  the number of observations) is equal  to $ 0.12$ mGal.
The population that shares the spatial distribution of floor-fractured
craters   shows   similar    trends   (Figure   \ref{Figure3-2}).    A
Kolmogorov-Smirnov (KS) test was  conducted to compare this population
to the whole  population of normal craters.  The test  reports a value
of $p$,  which is the  probability that  the two population  are drawn
from the same distribution, larger than $10\%$, which confirms that no
significant differences  exist between  the two  populations (Appendix
\ref{chap:A5}, Table 1).
  
%% FIGURE 8
\begin{figure}[h!]
  \graphicspath{ {/Users/thorey/Documents/These/Projet/FFC/Gravi_GRAIL/Article/Papier/SOUMISSION_2_EPSL/} }
  \begin{center}

    \includegraphics[scale=0.3]{Figure_3-2.eps}
    \caption{Magnitude  of  the   gravity  anomaly  $\delta_g$  versus
      diameter $D$ for  the normal crater population  (blue dots), the
      normal   crater  population   that   shares   the  FFC   spatial
      distribution (light  blue dots)  and the  floor-fractured crater
      population (red triangles) in the highlands. Solid line: Mean of
      the  gravity anomalies  binned  in $15$  km diameter  intervals.
      Error bars correspond to the standard error for each bin.  Right
      plot:  Corresponding gravity  anomaly  density distribution  for
      each population in frequency ($\%$).}
    \label{Figure3-2}
  \end{center}
\end{figure}

The  magnitude of  the  gravity anomalies  at floor-fractured  craters
shows  a different  dependence with  diameter than  at normal  craters
(Figure  \ref{Figure3-2}).  Although  the  variance of  the data  with
respect to the average is of  the same order, the gravity anomalies at
floor-fractured craters  are larger.  In  particular, the mean  of the
floor-fractured crater gravity anomalies is positive and approximately
$2.7$  mGal  larger  than  the  mean  of  the  normal  crater  gravity
anomalies.  We made use of a t-test to determine the robustness of the
difference between the mean magnitude  of the gravity anomalies of the
two  populations.    This  test  quantifies  the   significance  of  a
difference  between the  means  of two  populations  assuming the  two
populations have the same variance.  We found that there was only less
than  a $5\%$  chance  that the  difference  in the  mean  of the  two
populations  could have  occurred by  chance (Appendix  \ref{chap:A5},
Table 1).   The same result holds  for the comparison with  the normal
crater   population   that   shares  the   spatial   distribution   of
floor-fractured craters (Appendix \ref{chap:A5}, Table 1).

\subsubsection{Lunar maria and SPA}
\label{sec:lunar-maria-spa}
  
The magnitude of the gravity anomalies at the sites of complex craters
in the lunar  maria shows a variability that is  similar to craters in
the  highlands (Figure  \ref{Figure3-3},  top).   The gravity  anomaly
remains close  to $0$ mGal and  is independent of the  crater diameter
(Figure  \ref{Figure3-3},  top).  The  mean  of  the whole  population
$\mu_{\delta_g}$ is positive and equal to  $1.51 \pm 0.68$ mGal.  A KS
test shows that there is  no significant difference between the entire
normal  crater  population  and  the   one  that  shares  the  spatial
distribution of floor-fractured  craters (Figure \ref{Figure3-3}, top,
Appendix \ref{chap:A5}, Table 1).
%% FIGURE 9
\begin{figure}[h!]
  \graphicspath{ {/Users/thorey/Documents/These/Projet/FFC/Gravi_GRAIL/Article/Papier/SOUMISSION_2_EPSL/} }
  \begin{center}
    \includegraphics[scale=0.3]{Figure_3-3.eps}
    \caption{\textbf{Top}: Magnitude of the gravity anomaly $\delta_g$
      versus  diameter  $D$ for  the  normal  crater population  (blue
      dots), the normal crater population  that shares the FFC spatial
      distribution (light  blue dots)  and the  floor-fractured crater
      population (red triangles) in the maria. Solid line: Mean of the
      gravity anomalies  binned in $15$ km  diameter intervals.  Error
      bars correspond to the standard error for each bin.  Right plot:
      Corresponding  gravity  anomaly  density distribution  for  each
      population in frequency ($\%$).  \textbf{Bottom}: Same plots but
      in South Pole-Aikten basin.}
    \label{Figure3-3}
  \end{center}
\end{figure}

The  normal  craters in  the  South  Pole  Aikten basin  show  gravity
anomalies  that  are somewhat  more  negative  than in  the  highlands
(Figure \ref{Figure3-3},  bottom).  On  average, the  signal decreases
with  increasing   diameter  up  to   $D  \sim  100-120$   km  (Figure
\ref{Figure3-3},  bottom)  and  increases somewhat  again  for  crater
diameters between $120$ and $180$ km (Figure \ref{Figure3-3}, bottom).
A KS  test shows that there  is no significant difference  between the
two  populations of  normal craters  (Figure \ref{Figure3-3},  bottom,
Appendix \ref{chap:A5}, Table 1).

Although the mean  magnitude of the gravity anomalies at  FFC sites is
about $3$ mGal  larger than the mean value observed  at normal craters
in the maria and SPA, the variability in the signal is large and there
is  no significant  statistical  difference between  the  mean of  the
gravity anomalies of  normal and floor-fractured craters  in those two
different regions.  Indeed, a t-test, realized for both regions, shows
that there was more than a $10\%$ chance that these differences in the
mean of  the two populations  could have occurred by  chance (Appendix
\ref{chap:A5}, Table  1).  Nevertheless, the  small number of  FFCs in
the  maria and  in the  SPA makes  difficult to  obtain a  significant
statistic due to the low accuracy in measuring the FFC population mean
($\mu_{\delta_g}   =4.43  \pm   3.52   $  mGal   in   the  maria   and
$\mu_{\delta_g} =-0.25\pm 2.52 $ mGal in SPA).

\section{Magmatic intrusion characteristics}
\label{sec:magm-intr-char}
  
Our  results  show that,  on  average,  crustal gravity  anomalies  at
floor-fractured craters  are larger than  at normal craters,  but also
that   this  result   is  statistically   significant  only   for  the
highlands. This  is in agreement  with the presence of  dense magmatic
intrusions at depth below floor-fractured crater floors.  In addition,
the  amplitudes of  the  gravitational  signatures at  floor-fractured
craters are only of a few mGal, comparable to the predictions based on
the    theoretical    model     of    \citet{Thorey:2014cv}    (Figure
\ref{Figure2-5}).  In  the following, we compare  the observed gravity
signals at each floor-fractured crater  to a synthetic gravity anomaly
constructed based on the theoretical model of \citet{Thorey:2014cv} in
order to  derive the mean  density contrast between the  intrusion and
the  surrounding  crust.   To  that purpose,  the  thickness  of  each
intrusion is needed and we use  LOLA topographic data to estimate this
quantity for each floor-fractured crater.

\subsection{Intrusion thickness}
\label{sec:intrusion-thickness}

The intrusion thickness $H_0$ at the  center of the crater is taken as
the  amount of  shallowing of  the crater  floor with  respect to  the
expected   depth  \citep{Schultz:1976kt,Jozwiak:2012dq}.    Given  the
observed crater depth, the problem  is to estimate the original crater
depth   before    the   intrusion    formed.    In   the    study   of
\citet{Jozwiak:2012dq}, the scaling law which gives the depth $d_c$ as
a function of the crater diameter $D$, derived by \citet{Pike:1974ux},
was used as  an estimate for the initial crater  depth.  However, this
scaling law was calculated using only recent, Erastosthenian, and well
preserved   craters,   and   it   is   generally   acknowledged   that
floor-fractured  craters are  generally older  and more  degraded than
this  population.  In  the  absence  of information  on  the state  of
degradation of lunar craters in the dataset of \citet{Head:2010fy}, we
thus  use the  characteristics of  the normal  craters that  share the
spatial distribution  of floor-fractured craters described  in section
\ref{sec:unmod-floor-fract} as a reference.

\begin{figure}[h!]
  \graphicspath{ {/Users/thorey/Documents/These/Projet/FFC/Gravi_GRAIL/Article/Papier/SOUMISSION_2_EPSL/} }
  \begin{center}

    \includegraphics[scale=0.4]{Figure_4-1.pdf}

    \caption{Crater  depth $d_c$  (km)  versus diameter  $D$ (km)  for
      floor-fractured craters  (red) and normal craters  (blue) in the
      highlands,  SPA  and  the  maria.  The  normal  craters  in  the
      highlands reference to the  populations close to floor-fractured
      craters.  Error bars are the uncertainties in the measurement of
      the  crater depth  and for  clarity, the  uncertainties are  not
      shown  for normal  craters.  Dashed  lines: best  fit using  the
      equation $d_c =  AD^B$ for the floor-fractured  crater (red) and
      the normal  crater (blue) populations in  log-log space.  Values
      of the  coefficient A, B  as well  as the dispersion  around the
      best   fit   line   $\sigma_{fit}$   are   given   in   Appendix
      \ref{chap:A5}, Table 2.}
    \label{Figure4-1}
  \end{center}
\end{figure}


We characterize the depths of  both normal and floor-fractured craters
using  the 64  ppd ($\sim450$  m/pixel) LOLA  gridded topography  data
\citep{Zuber:2009bq}   obtained  from   the   planetary  data   system
geosciences   node.     We   followed   the   method    described   by
\citet{Kalynn:2013fg}  to  derive  the  crater  depth  $d_c$  and  its
uncertainty  $\sigma_{d}$ (see  Appendix  \ref{chap:A5} for  details).
Our  $d_c$-$D$  results  for  normal  craters  show  trends  that  are
consistent with previous  works in the highlands, the  lunar maria and
the SPA (Figure \ref{Figure4-1}).  Indeed,  the crater depth of normal
craters increases  with increasing diameter  and craters in  the maria
are  on  average  shallower  than  in the  highlands  or  SPA  (Figure
\ref{Figure4-1})        \citep{Pike:1974ux,Pike:1980eh,Kalynn:2013fg}.
Nevertheless, the variability in the  degradation state of each crater
results in  an important  variance in the  crater depth  with diameter
with respect to the mean trend.

The same  variability holds for the  floor-fractured crater population
depth   (Figure   \ref{Figure4-1}).    This  variability   makes   the
identification   of   the    uplift   difficult   at   floor-fractured
craters. Although the mean crater  depth of floor-fractured craters in
the highlands and SPA is slightly  smaller than the mean of the normal
crater  population,  the   means  of  the  two   populations  are  not
significantly  different  in  the  three regions  (t-test:  $p>  0.5$,
Appendix \ref{chap:A5}, Table 2).  A detailed geological study at each
crater   would  be   necessary  to   precisely  identify   the  crater
morphological  structures and  decrease the  uncertainty in  the depth
estimation, but such a study is out  of the scope of this article.  We
decided to estimate the intrusion thickness at floor-fractured craters
only to a first order by considering the difference in the mean trends
between normal and floor-fractured craters.

To characterize  the mean trend  of normal craters,  we make use  of a
linear least-squares regression in log-log space to obtain a power law
relationship of the form $d_c=AD^B$ \citep{Pike:1974ux,Kalynn:2013fg}.
We  use  the  same  method  to characterize  the  mean  dependence  of
floor-fractured  crater depth  with crater  diameter.  In  addition of
determining  the  constant  $A$  and   $B$,  we  also  calculated  the
root-mean-squared dispersion $\sigma_{\text{fit}}$ around the best fit
(Appendix  \ref{chap:A5},  Table 2).   Subtracting  the  two best  fit
lines,  we finally  obtain a  first-order estimate  for the  intrusion
thickness at the center $H_0$ for each floor-fractured crater of given
diameter     $D$      to     which     we     assign      an     error
$\sigma_{H_0}
=(\sigma_{\text{fit-FFC}}^2+\sigma_{\text{fit-Unmod. Crater}}^2)^{1/2}$.

\subsection{Density contrast $\Delta \rho$ of the intrusion}
\label{sec:intr-dens-contr}

We  consider   two  different   shapes  for  the   intrusions  beneath
floor-fractured craters: a bell-shaped intrusion for craters that show
a   convex  floor   (class  2   and   4  in   the  classification   of
\citet{Schultz:1976kt}) and a cylindrical-shaped intrusion for craters
that  show  a   plate-like  floor  (class  1,  3,  5   and  6  in  the
classification  of  \citet{Schultz:1976kt}).   The  two  dimensionless
profiles  are described  in  section \ref{sec:end-member-modes-1}  and
each profile is redimensionalized using the thickness of the intrusion
$H_0$      and      its       radius      $D/2$      (see      section
\ref{sec:intrusion-thickness}).   The   method  used  to   derive  the
synthetic  gravity anomaly  from  the intrusion  thickness profile  is
detailed is section \ref{sec:grav-sign-ffcs-1}.  We use a unit density
contrast, i.e.   $\Delta \rho  = 1$  kg m$^{-3}$  and then  filter the
predicted gravity  anomaly in exactly  the same way than  the observed
gravity is  filtered (section  \ref{sec:grails-gravity-model-1}).  The
synthetic   gravity   anomaly   $\delta_g^s$  associated   with   each
floor-fractured crater is defined as the mean of the synthetic gravity
anomaly measured interior to the crater rim.
  
Finally, the  density contrast between  the magma  and the crust  at a
specific floor-fractured crater location is given by the difference of
the observed  gravity anomaly  $\delta_g^{obs}$ and  the value  of the
gravity anomaly for normal craters  $\delta_g^c$ of the same diameter,
divided by the  synthetic gravity anomaly for a  unit density contrast
$\delta_g^s$
\begin{equation}
  \Delta \rho = \frac{\delta_{g}^{obs}-\delta_{g}^c}{\delta_{g}^{s}}
\end{equation}
where $\Delta \rho$ is in kg m$^{-3}$.

We   find   that  the   corrected   gravity   anomalies  observed   at
floor-fractured crater  sites in the  highlands are consistent  with a
mean   density  contrast   between  the   magma  and   the  crust   of
$\mu_{\Delta   \rho}  =   913  \pm   269  $   kg  m$^{-3}$   (Appendix
\ref{chap:A5},  Table  4).   In   the  maria,  the  corrected  gravity
anomalies observed in the  $22$ floor-fractured craters are consistent
with       a      mean       density      contrast       equal      to
$\mu_{\Delta \rho} = 484 \pm 669$ kg m$^{-3}$ (Appendix \ref{chap:A5},
Table 4).  However,  the difference between the  mean density contrast
in the  highlands and the maria  is not significant (a  t-test gives a
probability  greater than  $10\%$  that this  could  occur by  chance,
$p >  0.1$).  In the  South Pole  Aikten basin, the  corrected gravity
anomalies observed in the  $14$ floor-fractured craters are consistent
with       a      mean       density      contrast       equal      to
$\mu_{\Delta   \rho}  =   974  \pm   846  $   kg  m$^{-3}$   (Appendix
\ref{chap:A5}, Table 4).  But, the difference between the mean density
contrast in  the highlands and SPA  is also not significant  (a t test
gives  a probability  greater than  $50\%$  that this  could occur  by
chance, $p > 0.5$).

\section{Discussion}
\label{sec:discussion}

In  this study,  we  used  the gravity  field  obtained  by the  GRAIL
mission, in combination with the topographic data obtained by the LOLA
instrument, to  resolve mass anomalies below  floor-fractured craters.
We  studied separately  the craters  in the  farside highlands,  South
Pole-Aikten basin  and maria to  prevent potential bias  from regional
effects.

We show that the average  gravitational signature of normal craters in
the highlands is negative, whereas the average gravitational signature
of floor-fractured craters is  positive.  Although a large variability
characterizes the magnitude of  gravity anomalies in both populations,
the  difference between  the mean  of  the two  populations, equal  to
$\sim  3$  mGal,  is  statistically  significant.   In  addition,  the
floor-fractured  crater  gravity  anomalies  do not  follow  the  same
dependence  with   diameter  as  normal  craters.    Our  results  are
consistent  with   the  emplacement   of  magmatic   intrusions  below
floor-fractured     craters     as      originally     proposed     by
\citet{Schultz:1976kt}.  Furthermore,  the observed  gravity anomalies
(after  filtering) of  a few  mGal are  in agreement  with the  values
expected   from   the   model    of   crater-centered   intrusion   of
\citet{Thorey:2014cv}.  In  particular, measured gravity  anomalies at
floor-fractured craters imply an  average density contrast between the
magma      and      the      surrounding      crust      equal      to
$\mu_{\Delta \rho} = 913 \pm 269$ kg m$^{-3}$. Thermal annealing could
also participate  to the  measured gravity  anomaly, which  would then
decrease the estimated density contrast,  though this effect should be
limited to a few percents \citep{Michaut:2011dt,Kiefer:2013hr}.

The grain density of lunar basalt can vary from $3270$ kg m$^{-3}$ for
low   Ti  basalt   to  $3450$   kg   m$^{-3}$  for   high  Ti   basalt
\citep{Kiefer:2012kp}.  In contrast, the  lunar crust, which is mainly
anorthositic, shows grain densities that  vary from $2800$ kg m$^{-3}$
to $ 2900$  kg m$^{-3}$. The grain density contrast  between the magma
and the crust should thus be between $370$ and $650$ kg m$^{-3}$, with
an average  of $510$ kg  m$^{-3}$. Impacts have induced  fractures and
created pore space in the lunar rocks decreasing their bulk densities.
GRAIL data are consistent with an  average porosity of about $12\%$ in
the  crust  \citep{Wieczorek:2013ipa},  and  this  porosity  could  be
present in  either, or both, of  the two units (the  surrounding crust
and the magmatic intrusion).

First,  the observed  density contrast  could  be due  to a  pore-free
magmatic intrusion  and a  pore free highland  crust. From  the sample
densities, this would  give rise to a density contrast  of about $510$
kg m$^{-3}$,  which is  smaller than  the observed  $1$-$\sigma$ lower
bound, and thus is probably too small to account for the observations.
Second,  the density  contrast  could  be the  result  of a  fractured
intrusions and  a surrounding fractured  highland crust.  If  each had
the same  level of porosity, this  would give rise to  an even smaller
density contrast.  Lastly, if the  intrusion were unfractured, but the
surrounding  highland  crust  had  a porosity  of  $12\%$,  a  density
contrast of about $852$ kg m$^{-3}$  could be achieved, which is close
to the observed value.

The best scenario  that can account for the  observed density contrast
at floor-fractured craters in the highlands is an unfractured basaltic
intrusion that forms within a fractured highland crust.  Overall, this
implies that the intrusion is sufficiently young to have escaped being
fractured  by  subsequent impact  events.   Given  that most  basaltic
eruptions  occurred  between  $3$  and $4$  billion  years  ago,  this
suggests that  the majority  of the lunar  crust was  fractured before
this date.

Our analyses of  the gravity anomalies in the  South Pole-Aikten basin
and in  the maria  are less  conclusive and  are associated  with much
larger uncertainties.  In  regard to the SPA basin,  although the mean
magnitude  of the  gravity  anomalies for  floor-fractured craters  is
larger than  for normal craters,  we show that the  difference between
the two  populations is not  significant.  We note, however,  that the
average density  contrast associated  with floor-fractured  craters in
the South Pole-Aitken  basin is nearly identical to  that obtained for
the highlands.

Concerning  the  mare regions,  the  corrected  gravity anomalies  for
floor-fractured  craters are  consistent  with a  mean  value for  the
density contrast that  is considerably smaller than  in the highlands.
This is in fact consistent with expectations.  If the density contrast
were the  result of  an unfractured intrusion  forming in  a fractured
basaltic  crust (both  having  the same  grain  density), the  density
contrast  would be  only  about  $403$ kg  m$^{-3}$,  which is  nearly
identical to the mean value found from our analysis.

\section{Conclusion}
\label{sec:conclusion}

The gravitational signature of  the floor-fractured crater population,
first observed by \citet{Schultz:1976kt},  has been investigated using
the unprecedented resolution  of the global gravity  model provided by
the  GRAIL's mission.   We  show that  the  signal at  floor-fractured
craters in the highlands is consistent with the presence of a magmatic
intrusion at depth below the  crater floor.  Derived synthetic gravity
anomalies at each  FFC compared to observations show  that on average,
the density contrast between the magma and the crust is about $913$ kg
m$^{-3}$. This value is in agreement with the intrusion being composed
of  unfractured  basaltic  material  and  forming  in  a  pre-existing
fractured crust \citep{Wieczorek:2013ipa}

Similar  studies have  been  carried out  for floor-fractured  craters
located  in the  South  Pole  Aikten basin  and  in  the lunar  maria.
However, the small number of craters  as well as the large variability
in these two  regions prevent from clearly  differentiating the signal
due to magmatic intrusions from the background.  In general, two major
questions   need  to   be  addressed   before  carrying   out  further
investigations at floor-fractured crater sites:  1) What is the origin
of the  large variability in the  magnitude of the gravity  anomaly at
normal  crater? And  2),  how  can we  better  quantify the  intrusion
thickness at floor-fractured craters?

Indeed,  our results  suggest that  the impact  itself, combined  with
preexisting density  variations within  the crust,  results in  a wide
range of  Bouguer anomalies  at normal  impact craters.   Such density
structures should  also preexist below floor-fractured  craters before
magma emplacement.   To enhance the  intrusion signal, an  estimate of
the expected initial gravity signature is desirable.

Additionally,  we  show  that  the large  variability  in  the  crater
depth-diameter relationship  makes difficult the determination  of the
crater depth itself, and by consequence, the thickness of the magmatic
intrusion.  This variability  comes from the degradation  state of the
crater.   A quantification  of  the degradation  state  of normal  and
floor-fractured craters  might help to  reduce the uncertainty  in the
determination of the initial and current floor-fractured crater depths
and would result in a more accurate derivation of the density contrast
between the magmatic intrusion and the surrounding crust.


%%% Local Variables:
%%% mode: latex
%%% TeX-master: "../main"
%%% End:
