\chapter{Shallow intrusive magmatism} 
\label{chap1} 
\minitoc
 
\section{Eruption  versus intrusion:  review  of  the magma  transport
  processes within the crust}
\label{sec:magma-transp-with}

The transport  of magma from  deep regions  in the Earth  to shallower
layer  occurs through  several mechanisms.   Hereafter, we  review the
important processes  that control how  melts are extracted  from their
source regions,  transported through the  crust and intruded  into the
crust based on the general review made by \citep{Petford:2000cc}.

\subsection{Partial melting and melt ascent}

In  the upper  mantle, melt  and  magma (melt  plus suspended  solids)
formed by partial melting of the  solid rocks. The melt, which is less
dense  than the  surrounding  material, first  percolates through  the
matrix       of       unmolten        material       by       buoyancy
\citep{McKenzy:1984bo,McKenzie:1985jq}. However, percolation acts over
small length  scale (centimeter to  decimeters) and long  range ascent
mechanism (kilometer-scale) must exist to  transport the melt from the
source region to the shallow layer of a planet.


Low inertia diapiric ascent has been invoked traditionally as a common
ascent   mechanism  for   the  melt   \citep{Miller:1999km}.  However,
geological  observations show  that the  melt are  rapidly transported
within the  crust via thin  conduit called  dykes.  Indeed, it  is now
well  accepted   that  large  magma  reservoirs   formed  by  repeated


Although low inertia diapiric ascent is still invoked  in the ductile
lower  crust, it  is now  generally accepted  that magmas  are rapidly
transported within  conduit or  dyke through the  layer of  the crust.
Indeed, while  the ascent velocity  of a buoyant diapir  is controlled
mainly by the viscosity of the host rocks, the average velocity of the
dyke is controlled by the viscosity  of the melt itself resulting in a
much faster ascent.



This melting is  less dense and rises by  buoyancy, through compaction
and   percolation,   across   the   matrix   of   unmolten   materials
\citep{McKenzy:1984bo,McKenzie:1985jq}.  Within the curst


Several and it is now well accepted that magma collect into conduit or
dyke. Indeed,  it is now well  accepted that the formation  of plutons
and magmatic reservoir are the result of sills and dyke and not due to
diapirism .  How,  lack of direct observation made  this process still
unconstrains.

\subsection{Emplacement}

\section{Shallow magmatic intrusions on the Earth}
\label{sec:shall-magm-intr}

\subsection{Magma budget}
\label{sec:magma-budget}

\subsection{Laccolith from Corry et al}
\label{sec:laccolith-from-corry}

\subsection{Laccolith at Elba island}
\label{sec:laccolith-at-elba}

\section{Shallow magmatic intrusion on the Moon}
\label{sec:shall-magm-intr-1}

\subsection{Intrusive magmatism on the Moon}
\label{sec:intr-magm-moon}

\subsection{Low-slopes domes}
\label{sec:low-slopes-domes}

\subsection{Floor-fractured craters}
\label{sec:floor-fract-crat}





%%% Local Variables:
%%% mode: latex
%%% TeX-master: "../main"
%%% End:
