\chapter{Intrusive magmatism on terrestrial planets}
\label{chap8}

The  topographic   deformations  that  could  be   caused  by  shallow
intrusions can  be constrained by observations  of planetary surfaces;
that  is, volume,  shape and  other  dimensions of  intrusions can  be
quantified. In this  thesis, we show that these  observations, used in
combination  with a  model  for  the dynamics  of  the intrusion,  can
provide the  basis of  an effective framework  to study  and constrain
intrusive magmatism on terrestrial planets.

\section{Lunar intrusive magmatism}
\label{sec:terr-intr-magm}

\subsection{Summary}
\label{sec:summary}

While it provides for important  constraints on the Moon's thermal and
petrogenetic evolution,  the total  volume of  melt produced  into the
Moon interior is  poorly known.  Indeed, although  the total extrusive
volume is quantified  through analyses of the lunar  maria, the volume
of intrusive  magma, which should be  large due to the  low density of
the lunar crust,  remains unknown. A first step  in the quantification
of the  intrusive activity  on the  Moon is  the detection  of shallow
magmatic systems.

In this thesis, we focus on  two proposed candidates for shallow lunar
magmatic intrusions:  low-slope domes and floor-fractured  craters. In
Chapter \ref{Heating}, we first show  that the morphology of low-slope
lunar  domes  is  indeed   consistent  with  their  intrusive  origin.
Adapting the model of elastic-plated  gravity current model to account
for the crater depression, we then show in Chapter \ref{C5-chap6} that
the  deformations   observed  at  floor-fractured  craters   are  also
consistent  with the  emplacement of  magmatic intrusions  below their
floor.

Upon cooling and solidification,  crater-centered intrusions should be
denser than the surrounding medium and leave a positive anomaly in the
lunar gravity field.  Using the  intrusion morphology deduced from the
model, the lunar gravity field  obtained from the NASA’s GRAIL mission
and topographic  data obtained  from the LOLA  instrument, we  show in
Chapter  \ref{chap7} that  their  gravitational  signatures is  indeed
larger that the one of  normal impact craters. In particular, measured
gravity anomalies at floor-fractured  craters imply an average density
contrast between the magma and the surrounding crust close to $900$ kg
m$^{-3}$. Given  the $12\%$  porosity of the  lunar crust  revealed by
GRAIL, such  density contrast implies relatively  unfractured magmatic
intrusions.   In particular,  it  suggests that  these intrusions  are
sufficiently young to  have escaped the period  of intense bombardment
following the Moon formation.

Around  $10$ low-slope  lunar  domes and  about $200$  floor-fractured
craters have been detected at the  lunar surface, most of them located
close  or within  the lunar  maria. While  the total  volume of  these
magmatic intrusions should not exceed $1\%$ of the lunar maria volume,
it advocates the  presence of numerous shallow  magmatic intrusions in
the lunar crust.

\subsection{Origin for the magma}
\label{sec:crust-magm-intr}

In Chapter  \ref{C5-chap6}, we claim  that the absence  of deformation
surrounding floor-fractured craters suggests  that the unload pressure
associated with  the crater depression  might have drive  magma ascent
below these craters. However, the  unload pressure associated with the
depression should decrease rapidly with  depth on a length scale equal
to   the   crater   diameter,    i.e.    some   tens   of   kilometers
\citep{Pinel:2000wa}. Hence,  the depression caused by  the impact can
drive magma flow  if the magma is already present  on a similar length
scale, i.e. within the crust. The presence of numerous floor-fractured
craters  thus  raises  the  question of  deeper  and  larger  magmatic
reservoirs within the lunar crust at the time of their formation.

This idea is also supported by recent works on rift volcanism on Earth
showing that a depression can play a crucial role in the trajectory of
magma  on the  local scale  \citep{Maccaferri:2014ft}. In  particular,
\citet{Maccaferri:2014ft} show  that the  graben depression  favor the
formation  of a  stress  barrier  at depth  which  might prevent  dyke
propagation depending  on its  nucleation depth. In  particular, dykes
nucleated deep  below the graben will  tend to be deviated  from their
vertical   trajectory  and   produce   off-rift  volcanism.    Similar
investigation in  a lunar  setting might  also demonstrate  that dykes
initiated deep within  the lunar mantle should have  been deviated off
the crater depression sustaining the idea of shallower magma reservoir
feeding these intrusions.

Such magmatic reservoirs might also have  let a signature in the lunar
gravity field which might be detectable through GRAIL.

In the end,  a detailed analysis of the stress  field below the crater
depression  might  be the  natural  extension  of  our work  on  lunar
intrusive magmatism; in addition, it  might explain why these craters,
apart from  the underlying  low density  breccia, provide  a favorable
environment for magmatic intrusions.

At a regional  scale, it could also prove fruitful  to investigate the
link  between the  load of  the lunar  maria and  the distribution  of
floor-fractured craters, which  are mainly located at the  edge of the
maria itself.

\subsection{Constraining the thickness of the lunar maria}
\label{sec:thickn-lunar-maria}

Approximately $16\%$ of the Moon's surface is covered by basaltic lava
flows  that comprise  the lunar  maria. Although  the total  extent of
these lava  flows is  known, their thicknesses  are more  difficult to
constrain \citep{Thomson:2009eo}. Many  approaches, including indirect
techniques  such  as  gravity,  seismic   or  radar  data,  or  direct
measurements,  through   analyses  of  impact  that   have  completely
penetrated   the  maria,   have  been   proposed  to   estimate  their
thicknesses. Numerous low-slope domes  and floor-fractured craters are
located in the  lunar maria. The underlying  magmatic intrusions might
have intruded  the base of the  basalt layer, which is  more likely to
behave as coherent elastic layer. In such case, the model developed in
this thesis can  help put constrains on the thickness  of the maria in
these locations as shown in  Section \ref{C5-Depth}.  If the intrusion
has intruded  the maria, it would  at least provide for  a lower bound
estimation.

\section{Probing intrusive magmatism on other terrestrial planets}
\label{sec:other-terr-plan}

\citet{Michaut:2013dr}  have already  used the  elastic-plated gravity
current  model  to assess  the  intrusive  origin of  several  Martian
domes. As proven on the Moon,  floor-fractured craters are also a good
first basis to  study intrusive processes. While they  have first been
observed  and described  on the  Moon,  many evidences  show now  that
floor-fractured  craters might  be a  common landscape  on terrestrial
planets.

\begin{figure}[htpb]
  \begin{center}
    \graphicspath{ {/Users/thorey/Documents/These/Manuscript/Figure/Chapter7/} }
    \includegraphics[scale=0.9]{FFCOther.eps}
    \caption{a),  b) and  c) Sample  from the  Martian FFC  population
      located  respectively  at ($0.0^{\circ}$N,$337.3  ^{\circ}  $E),
      ($5.5^{\circ}$S,$322.6         ^{\circ}          $E)         and
      ($6.7^{\circ}$S,$333.4^{\circ}$E).   All are  THEMIS daytime  IR
      image taken modified from  \citet{Sato:2010ex}. d) Potential FFC
      on Mercury reproduced  from \citet{Schultz:1977ec}. e) Barrymore
      crater, $50$ km diameter, located  near Imdr Regio. f) Mona lisa
      Crater,  $85$ km  in diameter,  located  on the  edge of  Eistla
      Regio.   Both  are potential  FFCs  on  Venus.  Reproduced  from
      \citet{Wichman:1995ju}.}
    \label{C7-FFCOther}
  \end{center}
\end{figure}

\begin{itemize}
\item  \textbf{Mars}: On  mars, almost  $200$ floor-fractured  craters
  ,located mostly along a narrow  band south of the dichotomy boundary
  in Arabia Terra, have also been reported \citep{Bamberg:2014hb}. The
  observed deformations  within these craters  is very similar  to the
  one observed on the Moon, though Martian floor-fracture craters tend
  to  exhibit a  more  extensive and  wider  fracture network  (Figure
  \ref{C7-FFCOther}  a,  b  and  c). This  is  attributed  to  complex
  interactions of  the magmatic intrusion with  potential ice/water in
  the  subsurface  \citep{Sato:2010ex,Bamberg:2014hb}. In  particular,
  the  melting  of the  water  (or  possibly  CO$_2$) trapped  in  the
  subsurface  would enhance  erosion of  the floor-fractured  which is
  consistent  with   some  small  and  medium   size  fluvial  outlets
  \citep{Sato:2010ex}.

  Interestingly,  deformations on  Martian floor-fractured  craters is
  not localized within the crater wall but can also extend further the
  crater rim (Figure \ref{C7-FFCOther} b,c).  In contrast to the Moon,
  the overpressure driving  the intrusion might have  been larger than
  the unloading pressure associated  with the depression. In addition,
  Martian magma,  at the  difference of  their lunar  counterpart, are
  most  likely to  be  buoyant  until the  surface  and the  mechanism
  favorable to intrusion below Martian  crater is still debated. Again
  on  Mars,  studying the  stress  field  associated with  the  crater
  depression  could  provide  a  viable  mechanism  to  trigger  magma
  spreading at depth below these craters.

\item   \textbf{Mercury}    \citet{Schultz:1977ec}   propose   several
  candidates  searching for  intra-crater dark  haloes or  other color
  variations  indicating post-impact  emplacement  of mafic  materials
  onto the floor. They did find several crater floors with contrasting
  deposits, and additionally a few rimmed moat-like depression (Figure
  \ref{C7-FFCOther} d).

\item \textbf{Venus}: Venus geologic record  have been largely cut off
  by  resurfacing events  constantly reworking  the Venusian  surface.
  Nevertheless, several candidates have also been proposed on Venus by
  \citet{Wichman:1995ju}.
\end{itemize}

Though most of these observations,  except for Martian FFCs, have been
made in  the late nineties, they  provide an extremely good  basis for
work using new data and new methods at hand today.

%%% Local Variables:
%%% mode: latex
%%% TeX-master: "../main"
%%% End:
