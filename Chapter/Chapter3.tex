\chapter{Model  for  the study  of  a  cooling elastic-plated  gravity
  current}
\label{chap3}
\minitoc

We present here  a general model for the cooling  of an elastic-plated
gravity current. 

\section{Theory}
\label{sec:theory}

\subsection{Formulation}
\label{sec:formulation}

We model an axisymmetric fluid  blister of thickness $h(r,t)$ below an
elastic layer  of constant thickness  $d_c$ and above a  semi infinite
rigid  layer  \citep{Michaut:2011kg}  (Figure  \ref{Figure2-1}).   The
assumption  that the  thickness  of the  fluid tends  to  zero at  the
contact  line leads  to  divergent viscous  stresses,  and hence,  the
theoretical immobility of the blister. To avoid problem at the contact
line,  we  consider  a  thin   pre-wetting  film  of  thickness  $h_f$
\citep{Flitton:1999iv,Lister:2013ia} (Figure \ref{Figure2-1}).

The  fluid is  injected continuously  at the  base and  center of  the
blister at a rate $Q_0(t)$ through a conduit of diameter
$a$. The hot fluid is intruded  at temperature $T_i$ and cools through
the top and the bottom by  conduction in the surrounding medium, whose
temperature $T_s$ is allowed to increase with time.

%% FIGURE 2-1
\begin{figure}[htbp]
  \begin{center}
    \graphicspath{ {/Users/thorey/Documents/These/Projet/Refroidissement/Skin_Model/Paper/Figure/Draft_2/} }
    \includegraphics[scale=0.40]{Figure_2-1.pdf}
    \caption{Model geometry and parameters.}
    \label{Figure2-1}
  \end{center}
\end{figure}

As  it  cools,  the  viscosity  of the  fluid  increases  following  a
prescribed  rheology   $\eta(T)$  bounded  between  two   values:  the
viscosity of the  hottest fluid $\eta_h$ at temperature  $T_i$ and the
viscosity of the coldest fluid $\eta_c$ at temperature $T_0$.

\subsection{Velocity field}
\label{sec:Velocity field}

The  intrusion develops  over a  length scale  $\Lambda$ that  is much
larger than its thickness $H$ ($\Lambda >> H$).  In the laminar regime
and  in   axisymmetrical  coordinates  ($r$,$z$),   the  Navier-stokes
equations within the lubrication assumption are
\begin{eqnarray}
  -\frac{\partial P}{\partial r}  +  \frac{\partial}{\partial z}\left(\eta(T) \frac{\partial u}{\partial z}\right) &=&0\label{V1} \\
  -\frac{\partial P}{\partial z}  - \rho_{m}g&  =&0\label{Npressure}
\end{eqnarray}
where $u(r,z,t)$ the radial velocity,  $\rho_m$ the fluid density, $g$
the standard acceleration  due to gravity and  $P(r,z,t)$ the pressure
within  the fluid.   Integration of  (\ref{Npressure}) thus  gives the
total  pressure  $P(r,z,t)$  within   the  flow.   When  the  vertical
deflection deflection  $h(r,t)$ of  the upper  elastic layer  is small
compared  to  its  thickness  $d_c$,  i.e  $h<<d_c$,  we  can  neglect
stretching   of   the   upper   layer  and   only   consider   bending
stresses. Therefore, the  total pressure $P(r,z,t)$ at a  level $z$ in
the intrusion  is the sum  of three  contributions: the weight  of the
magma and of the upper layer and the bending pressure
\begin{equation}
  P = \rho_m g (h-z)+\rho_rgd_c+D\nabla^4h
\end{equation}
where $h(r,t)$ is the intrusion thickness, $\rho_r$ the density of the
surrounding rocks and $D$ is the flexural rigidity of the thin elastic
layer,  that  depends on  the  Young's  modulus $E$,  Poisson's  ratio
$\nu^*$   and    on   the    elastic   layer   thickness    $d_c$   as
$D   =   Ed_c^3/12(1-\nu^*)$.    Integration   of   (\ref{V1})   using
$\left.\frac{\partial  u}{\partial  z}\right|_{z=h/2}=0$  by  symmetry
around $h/2$, gives:
\begin{equation}
  u(r,z,t)=\frac{1}{2}\left(\rho_m g \frac{\partial h}{\partial      r}+D\frac{\partial}{\partial      r}\left(\nabla^{2}_{r}\left( \nabla^{2}_{r}h(r) \right)\right)\right)\int_0^z
  \frac{1}{\eta(T)}\left( 2 z -h\right)dz
  \label{V2}
\end{equation}

\subsection{Injection rate}

Assuming a Poiseuille flow within the cylindrical feeding conduit, the
vertical  injection velocity  $w(r,t)$  and injection  rate $Q_0$  are
given by:
\begin{equation}
  w(r,t)=
  \begin{cases}
    \frac{ \Delta P}{4 \mu Z_{c}} (\frac{a^{2}}{4}-r^{2})& r \le \frac{a}{2}\\
    0 & r > \frac{a}{2}
  \end{cases}
  \label{eq12}
\end{equation}
\begin{equation}
  Q_{0}=\frac{\pi \Delta P a^{4}}{128 \mu Z_c}
  \label{eq11}
\end{equation}
where  $\Delta P$  is  the  initial overpressure  within  the melt  at
$z=Z_{c}$. The fluid is injected at the liquidus temperature $T_i$.

\subsection{Heat transport equation}
\subsubsection{Local energy conservation}

In the laminar regime and in axisymmetrical coordinates ($r$,$z$), the
local energy  conservation equation within the  lubrication assumption
is written as
\begin{eqnarray}
  \frac{D}{D t}\left(\rho_m C_{p,m} T+\rho_mL(1-\phi)\right)&=& k_m  \frac{\partial^2
                                                                T}{\partial               z^2}\label{EnergyCons}
\end{eqnarray}

where  $T(r,z,t)$  is  the  fluid temperature,  $\phi(r,z,t)$  is  the
crystal fraction  in the melt  and $\rho_m$, $k_m$, $C_{p,m}$  and $L$
are the density,  thermal conductivity, specific heat  and latent heat
of the  fluid.  In this model,  the crystals are considered  only as a
source/sink of energy  as they melt/form during  the flow emplacement.
In particular, they share the same properties that the fluid itself.

Following a common approximation, we  assume that the crystal fraction
is a linear function of temperature over the melting interval
\begin{equation}
  \phi = \frac{T_L-T}{T_L-T_s}
  \label{meltfraction}
\end{equation}
where $T_S$ and $T_L$ are the solidus and liquidus temperatures of the
magma  \citep{Hort:1997hk,Michaut:2006di}.  With  this approximation,
the local energy equation (\ref{EnergyCons}) resumes to
\begin{eqnarray}
  \frac{\partial T}{\partial t}+ u\frac{\partial T}{\partial r}
  + w\frac{\partial T}{\partial z}  &=& \frac{ \kappa_m}{1+St^{-1}}  \frac{\partial^2
                                        T}{\partial               z^2}
                                        \label{EnergyCons2}
\end{eqnarray}
where  $u(r,z,t)$ and  $w(r,z,t)$ are  the radial  and vertical  fluid
velocities,  $St  =\left(C_{p,m}(T_L-T_S)\right)/L$   is  the  Stephan
number   and    $\kappa_m$   is   the   fluid    thermal   diffusivity
$\kappa_m = k_m/\rho_m  C_{p,m}$.  Following \citet{BALMFORTH:2004fm},
we use an integral balance method to solve the heat transport equation
(\ref{EnergyCons2}).   This theory  is based  on the  integral-balance
method of heat-transfer theory of \citet{Goodman:1958ue}, in which the
vertical structure of the temperature  field is represented by a known
function of depth that approximates the expected solution.

\subsubsection{Integral   balance   solution   for   the   temperature
  $T(r,z,t)$}

We model  the cooling of the  fluid blister through the  growth of two
thermal  boundary layers:  one growing  downward  from the  top and  a
second  growing upward  from  the base.   As  we consider  homogeneous
thermal properties for  the surrounding rocks, we assume  that the two
thermal boundary layers grow symmetrically and have the same thickness
$\delta(r,t)$.   In agreement,  the integral-balance  approximation we
use for the vertical temperature profile $T(r,z,t)$ is
\begin{equation}
  T=
  \begin{cases}
    T_b - (T_b-T_s)(1-\frac{z}{\delta})^2 & 0 \le z\le \delta \\
    T_b & \delta \le z\le h-\delta \\
    T_b - (T_b-T_s)(1-\frac{h-z}{\delta})^2 & h-\delta \le z\le h\\
  \end{cases}
  \label{Temperature}
\end{equation}
where $T_b(r,t)$  is the  temperature at  the center  of the  flow and
$T_s(r,t)$ the temperature at the contact with the surrounding rocks (Figure
\ref{Figure2-1}).      The     integral    balance     solution     in
(\ref{Temperature}) assumes  a symmetry around $z=h/2$  and a decrease
of the  temperature in  the two  thermal boundary  layers down  to the
surrounding  rock  temperature   $T_s$  \citep{BALMFORTH:2004fm}.   In
addition,  it  assumes a  uniform  temperature  $T_b$ in  between  the
thermal  boundary  layers.   Then,  as  the fluid  is  injected  at  a
temperature  $T_i$,   we  have  $T_b(r,t)  =T_i$   when  $\delta<h/2$.
However,   if  the   two   thermal  boundary   layers  connect,   then
$\delta  =  h/2$ and  $T_b$  decreases  such  that $T_b\le  T_i$. 

\subsubsection{Integral balance equation}
\label{sec:integr-balance-equat}

We   begin    by   integrating    the   local    energy   conservation
(\ref{EnergyCons2})  over  the  two   thermal  boundary  layers.   The
integration over the  bottom thermal layer, i.e. from  the base, $z=0$
to a level $z = \delta$ gives
\begin{eqnarray}
  &&\frac{\partial}{\partial t}\left( \delta( \bar{T}-T_b)\right)+\frac{1}{r}\frac{\partial}{\partial r} \left( r\delta(\overline{uT}-\bar{u}T_b)\right) + \delta\left( \frac{\partial T_b}{\partial t}+ \overline{u}\frac{\partial T_b}{\partial r}\right)\nonumber\\
  &=&-\frac{\kappa_m}{1+St^{-1}}\left. \frac{\partial T}{\partial z}\right|_{z=0}+w_{inj}(T_{i}-T_b)
      \label{Local1}
\end{eqnarray}
where the  bar indicate the  vertical average over the  bottom thermal
boundary layer
\begin{equation}
  \overline{f} = \frac{1}{\delta}\int_0^{\delta}f dz\nonumber,
\end{equation}
$T_b(r,t)$  is  the temperature  at  $z=\delta$,  $w_{inj}(r)$ is  the
vertical  injection velocity  and  we  have used  the  nullity of  the
thermal gradient at $z=\delta$ and the local mass conservation
\begin{equation}
  \frac{1}{r}\frac{\partial ru}{\partial r} +\frac{\partial w}{\partial z}=0.
  \label{MassConservation}
\end{equation}
The  integration over  the top  thermal layer,  i.e., from  the level,
$z=h-\delta$ to the top $z=h$ gives:
\begin{eqnarray}
  &&\frac{\partial}{\partial t}\left( \delta( \bar{T}-T_b)\right)+\frac{1}{r}\frac{\partial}{\partial r} \left( r\delta(\overline{uT}-\bar{u}T_b)\right) + \delta\left(\frac{\partial T_b}{\partial t}+ \overline{u}\frac{\partial T_b}{\partial r}\right)\nonumber\\
  &=&\frac{\kappa_m}{1+St^{-1}}\left. \frac{\partial T}{\partial z}\right|_{z=h}.
      \label{Local2}
\end{eqnarray}
where,    in    addition    to    the    local    mass    conservation
(\ref{MassConservation})  and   the  fact  the  thermal   gradient  at
$z=h-\delta$ is  equal to  zero, we have  used the  kinematic boundary
condition in $z=h(r,t)$
\begin{equation}
  \frac{\partial h}{\partial t} +u\frac{\partial h}{\partial
    r} = w
\end{equation}
The heat balance equation can then be written by adding (\ref{Local1})
and (\ref{Local2}) and introducing (\ref{Temperature})
\begin{eqnarray}
  &&\frac{\partial}{\partial t}\left( \delta( \bar{T}-T_b)\right)+\frac{1}{r}\frac{\partial}{\partial r} \left( r\delta(\overline{uT}-\bar{u}T_b)\right) + \delta\left( \frac{\partial T_b}{\partial t}+ \overline{u}\frac{\partial T_b}{\partial r}\right)\nonumber\\
  &=&\frac{\kappa_m}{2(1+St^{-1})}\left(\left. \frac{\partial T}{\partial z}\right|_{z=h}-\left. \frac{\partial T}{\partial z}\right|_{z=0}\right)+\frac{w_{inj}}{2}(T_{i}-T_b)
      \label{LocalHeat3}
\end{eqnarray}

\subsubsection{Thermal boundary conditions}
\label{sec:thermal-boundary-condition}

At  the  contact with  the  surrounding  rock,  the  heat is  lost  by
conduction:
\begin{equation}
  k_m\left.\frac{\partial                                    T}{\partial
      z}\right|_{z=0}=k_r\left.\frac{\partial              T_r}{\partial
      z}\right|_{z=0}
  \label{Flux1}
\end{equation}
\begin{equation}
  k_m\left.\frac{\partial                                  T}{\partial
      z}\right|_{z=h}=k_r\left.\frac{\partial            T_r}{\partial
      z}\right|_{z=h}
  \label{Flux2}
\end{equation}

where  $T_r(r,z)$  is  the  temperature  in  the  surrounding  medium.
Assuming  a  semi  infinite  layer  for  the  rigid  layer  below  the
intrusion, \citet{Carslaw:1959wf}  show that the temperature  $T_r$ in
the surrounding  rocks can  be approximated  to a
first order by
\begin{equation}
  T_r(r,z,t)-T_0=(T_{s}-T_0)\operatorname{erfc}{\left(\frac{-z}{2\sqrt{\kappa_r t}}\right)}.
  \label{eq22}
\end{equation}
The  thickness of  the upper  layer is  equal to  the intrusion  depth
$d_c$. However,  we assume that the  depth $d_c$ is large  compared to
the characteristic  length scale for  conduction $L_c$ and we  use the
same approximation to derive $T_r$ above the intrusion
\begin{equation}
  T_r(r,z,t)-T_0=(T_{s}-T_0)\operatorname{erfc}{\left(\frac{z-h}{2\sqrt{\kappa_r t}}\right)}.
  \label{eq11}
\end{equation}

Therefore,  the  two  thermal boundary  conditions  (\ref{Flux1})  and
(\ref{Flux2}) become:

\begin{equation}
  k_m\left.\frac{\partial                                    T}{\partial
      z}\right|_{z=0}= k_r
  \frac{T_{s}-T_{0}}{\sqrt{\pi \kappa_r t}}
  \label{Flux_1}
\end{equation}
\begin{equation}
  k_m\left.\frac{\partial                                    T}{\partial
      z}\right|_{z=h}= -k_r
  \frac{T_{s}-T_{0}}{\sqrt{\pi \kappa_r t}}
  \label{Flux_2}
\end{equation}


\subsection{Dimensionless equations}
\label{sec:dimens-equat}

A global  statement of  mass conservation gives  the evolution  of the
thickness \citep{Huppert:1982a,Michaut:2011kg}
\begin{equation}
  \frac{\partial         h}{\partial        t}         +\frac{1}{r}
  \frac{\partial}{\partial r} \left( r\int_0^hudz\right)&=&
  w_{inj}.
  \label{Mass}
\end{equation}
This   equation  is   then  coupled   to  the   heat-balance  equation
(\ref{LocalHeat3})  through  the  viscosity dependence,  which  itself
depends on temperature, of the  velocity (\ref{V2}).  We first rewrite
the  different  temperatures  such  that  $T=T_0+\left(T_i-T_0)\theta$
  where $\theta(r,z,t)$  is the equivalent  dimensionless temperature.
  In   term   of   $\theta$,  the   integral   balance   approximation
  (\ref{Temperature}) rewrites
  \begin{equation}
    \theta(z)=
    \begin{cases}
      \Theta_b -\left(\Theta_b-\Theta_s\right)(1-\frac{z}{\delta})^2& 0 \le z\le \delta \\
      \Theta_b & \delta \le z\le h-\delta \\
      \Theta_b -\left(\Theta_b-\Theta_s\right)(1-\frac{h-z}{\delta})^2
      & h-\delta \le z\le h
    \end{cases}
    \label{Temperature2}
  \end{equation}
  where            $\Theta_b=\frac{T_b-T_0}{T_{i}-T_0}$            and
  $\Theta_s   =   \frac{T_s-T_0}{T_i-T_0}$.    Equations   (\ref{V2}),
  (\ref{LocalHeat3}) and  (\ref{Mass}) are nondimensionalized  using a
  horizontal scale  $\Lambda$, a vertical  scale $H$ and a  time scale
  $\tau$                            given                           by
  \citep{Michaut:2011kg,Michaut:2013dr,Michaut:2014eq,Thorey:2014cv}
  \begin{eqnarray}
    \Lambda &=& \left(\frac{D}{\rho_m g}\right)^{1/4}\label{L1}\\
    H&=&\left       (\frac{12\eta_h      Q_{0}}{\rho_{m}g       \pi}\right      )
         ^{\frac{1}{4}} w\label{H1}\\
    \tau&=&\frac{\pi \Lambda^{2} H}{Q_{0}}\label{T1}
  \end{eqnarray}
  where  $\Lambda$ represents  the  flexural wavelength  of the  upper
  elastic layer \citep{Michaut:2011kg}, $H$ the thickness of a typical
  gravity current \citep{Huppert:1982wr} and $\tau$ the characteristic
  time to fill up a cylindrical flow of radius $\Lambda$ and thickness
  $H$ at constant rate $Q_0$.  In addition, we can define a horizontal
  velocity                                                       scale
  $U=\Lambda/\tau=\left(\rho_m          g          H^3\right)/\left(12
    \eta_h\Lambda\right)$.

  The dimensionless equations are
  \begin{eqnarray}
    \frac{\partial         h}{\partial        t} &=&- \frac{1}{r}
                                                     \frac{\partial}{\partial r} \left( r\int_0^hudz\right) +w_{inj}\label{EqFinal1}\\
    \frac{\partial}{\partial
    t}\left( \delta( \bar{\theta}-\Theta_b)\right)&=&-\frac{1}{r}\frac{\partial}{\partial
                                                      r}  \left(   r\delta(\overline{u\theta}-\bar{u}\Theta_b)\right)  -
                                                      \delta\left(      \frac{\partial       \Theta_b}{\partial      t}+
                                                      \overline{u}\frac{\partial     \Theta_b}{\partial    r}\right)-
                                                      2Pe^{-1}St_m\frac{\left(\Theta_b-\Theta_s\right)}{\delta}\nonumber\\
                                                 &+&\frac{w_{inj}}{2}(1-\Theta_b)\label{HeatDimensionLess}\\
    u&=&6\left( \frac{\partial h}{\partial      r}+\frac{\partial}{\partial      r}\nabla^2h\right)\int_0^z
         \frac{1}{\eta^*(\theta,\nu)}\left( 2 z -h\right)dz \label{V3}\\
    w_{inj}&=&
               \frac{32}{\gamma^{2}}\left(\frac{1}{4}-\frac{r^{2}}{\gamma^{2}}\right)\hspace{.2cm}
               \text{if} \hspace{.2cm} r < \gamma/2
  \end{eqnarray}
  where $\eta^*(\theta,\nu)$ is the dimensionless rheology   $\eta/\eta_h$
  and the thermal boundary conditions (\ref{Flux_1}) and (\ref{Flux_2}) resume to
  \begin{equation}
    2\frac{\Theta_b-\Theta_s}{\delta}               =               \Omega
    Pe^{1/2}\frac{\Theta_s}{\sqrt{\pi t}}.
    \label{Boundary-Condi}
  \end{equation}
  $\gamma$, $Pe$, $St_m$, $\nu$ and $\Omega$ are the five
  dimensionless numbers that control the dynamics of the flow
  \begin{eqnarray}
    \gamma&=&\frac{a}{\Lambda} \label{gamma}\\
    Pe&=& \frac{H^2}{\kappa_m \tau}\label{Pe}\\
    St_m &=& \frac{C_{p,m}\left(T_L-T_S\right)}{C_{p,m}\left(T_L-T_S\right)+L} \label{St}\\
    \nu&=& \frac{\eta_h}{\eta_c}\label{nu}\\
   \Omega&=&\frac{k_r}{k_m}\left(\frac{\kappa_m}{\kappa_r}\right)^{1/2}\label{omega}
  \end{eqnarray}
  $\gamma$ is  the dimensionless  radius of the  conduit, it  does not
  significantly influence the flow and is  set to $0.02$ in this study
  \citep{Michaut:2009jx,Michaut:2011kg}, $Pe$ is the Peclet number, it
  compares the vertical diffusion of  heat to the horizontal advection
  in the intrusion  interior, $St_m$ is a modified  Stephan number, it
  is the  ratio of sensible heat  between solidus and liquidus  to the
  total energy of the fluid at liquidus temperature, $\nu$ is the
  maximum viscosity contrast, i.e.  the  ratio between the hottest and
  coldest viscosity and and  $\Omega$ is the  ratio between heat conduction  at the
 contact with the encasing rocks and heat diffusion within the fluid.

 \subsection{Further simplifications and final equations}
 \label{sec:furth-simpl}

 The heat balance equations (\ref{HeatDimensionLess}) can reduce to
 \begin{eqnarray}
   \frac{\partial}{\partial
   t}\left( \delta( \bar{\theta}-1)\right)+\frac{1}{r}\frac{\partial}{\partial
   r}
   \left( r\delta(\overline{u\theta}-\bar{u})\right)&=&- 2Pe^{-1}St_m\frac{\left(\Theta_b-\Theta_s\right)}{\delta} 
                                                        \label{HeatD_a}
 \end{eqnarray}

 Indeed, if the thermal  boundary layers exist, $\Theta_b=1$, $\delta$
 is    the     variable    and     the    heat     balance    equation
 (\ref{HeatDimensionLess})  reduces to  the equation  (\ref{HeatD_a}).
 In contrast, if  the thermal boundary layers  merge, $\delta=h/2$ and
 the variable is $\Theta_b$. In  this case, the heat balance equations
 (\ref{HeatDimensionLess}) reduces to:
 \begin{eqnarray}
   \frac{\partial h\bar{\theta}}{\partial t}+\frac{1}{r}\frac{\partial}{\partial
   r} \left( rh\overline{u\theta}\right)-\Theta_b\left(\frac{\partial h}{\partial t}+\frac{1}{r}\frac{\partial}{\partial
   r} \left( rh\bar{u}\right)\right)&=& - 8St_mPe^{-1}\frac{\left(\Theta_b-\Theta_s\right)}{h}+w_{inj}(1-\Theta_b)\nonumber
 \end{eqnarray}
 which we can rewrite using (\ref{EqFinal1}) as
 \begin{equation}
   \frac{\partial h\bar{\theta}}{\partial t}+\frac{1}{r}\frac{\partial}{\partial
     r} \left( rh\overline{u\theta}\right) &=& w_i
   - 8St_mPe^{-1}\frac{\left(\Theta_b-\Theta_s\right)}{h}\nonumber\\
 \end{equation}
 which  also   corresponds  to  (\ref{HeatD_a})  in   the  case  where
 $\delta=h/2$.

 Following \citet{BALMFORTH:2004fm}, we  rewrite (\ref{HeatD_a}) using
 a new variable $\xi = \delta(1-\overline{\theta})$
 \begin{equation}
   \frac{\partial \xi}{\partial t}+\frac{1}{r}\frac{\partial}{\partial r} \left( r\bar{u}\xi\right)-\frac{1}{r}\frac{\partial}{\partial r} \left( r\delta(\overline{u\theta}-\bar{u}\bar{\theta})\right)&=&2Pe^{-1}St_m\frac{\left(\Theta_b-\Theta_s\right)}{\delta}
   \label{EqFinal2}
 \end{equation}
 This equation contains advection  by the vertically integrated radial
 velocity, with a correction accounting  for the vertical structure of
 the temperature field and conduction  cooling. The system composed by
 (\ref{EqFinal1}) and  (\ref{EqFinal2}), whose  main variable  are $h$
 and $\xi$ is  complete.  Indeed, the temperature at  the contact with
 the  surrounding $\Theta_s$  is built  from the  variable $\xi$  
 \begin{equation}
   \Theta_s(r,t)=
   \begin{cases}
     \frac{3 \beta}{4} \xi - \frac{\sqrt{3}}{4} \sqrt{\beta \xi \left(3 \beta \xi + 8\right)} + 1 & \text{if} \hspace{1cm} \xi\leq \xi_t \\
     \frac{- 12 \xi + 6 h{\left (r,t \right )}}{\left(\beta h{\left (r,t \right )} + 6\right) h{\left (r,t \right )}} & \text{if} \hspace{1cm} \xi > \xi_t\\
   \end{cases}
 \end{equation}
 where
 \begin{eqnarray}
   \xi_t(t)&=&\frac{\beta(t) h^{2}{\left (r,t \right )}}{6 \beta(t) h{\left (r,t \right )}
               + 24}\\
   \beta(t) &=& \Omega Pe^{1/2}\frac{1}{\sqrt{\pi t}}
 \end{eqnarray}
 which leads to the expression of $\Theta_b$ and $\delta$
 \begin{equation}
   \Theta_b(r)=
   \begin{cases}
     1 &\text{if } \hspace{1cm} \xi\leq \xi_t \\
     \frac{\Theta_{s}}{4} \left(\beta(t) h{\left (r,t \right )} +
       4\right) & \text{if} \hspace{1cm} \xi > \xi_t\\
   \end{cases}
 \end{equation}
 \begin{equation}
   \delta(r)=
   \begin{cases}
     \frac{1}{\Theta_{s} \beta(t)} \left(- 2 \Theta_{s} + 2\right) &\text{if } \hspace{1cm} \xi\leq \xi_t \\
     h(r,t)/2 & \text{if} \hspace{1cm} \xi > \xi_t\\
   \end{cases}
 \end{equation}


 Equations   (\ref{EqFinal1})   and    (\ref{EqFinal2})   are   solved
 numerically  using  the  Newton-Raphson   method  which  leads  to  a
 second-order scheme in time and  space. In all solutions, we computed
 the mass  and energy conservation as  a test for the  accuracy of the
 convergence.

  \section{Numerical approach}
  \label{sec:numerical-approach}

  \subsection{Equation on the thickness}
  \label{sec:equation-thickness}

  \subsection{Heat transport equation}
  \label{sec:heat-transp-equat}

  \subsection{Convergence}
  \label{sec:convergence}




%%% Local Variables:
%%% mode: latex
%%% TeX-master: "../main"
%%% End:
