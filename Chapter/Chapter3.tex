\chapter{Elastic-plated gravity current with temperature-dependent viscosity} 
\label{C3-JFM} 
\minitoc

\begin{abstract}
  Temperature-dependent  elastic-plated gravity  current has  numerous
  applications in the  nature science, from the intrusion  of magma is
  the shallow  layer of the crust  to the flowing of  melt-water below
  ice sheet.  We develop the  general equations for the elastic-plated
  gravity  current with  temperature-dependent viscosity  for constant
  influx conditions.  Crystallization is also  taken into account as a
  source/sink  of  heat  when   the  fluid  crystallizes/melts  during
  emplacement. We show that the coupling between the thermal structure
  and  the  flow  itself  results in  important  deviations  from  the
  isoviscous case.  In particular,  both regimes, taken individually,
  split in three  phases. In the first phase, the  thermal anomaly has
  the size of  the current itself, the effective  viscosity is minimal
  and the current spreads as in the isoviscous case. The second phase
  is triggered by  the detachment of the thermal  anomaly and followed
  by an important increase in the effective viscosity of the flow. The
  current  slows down  and  thickens.  Finally,  when  the cold  front
  region  becomes  about $10\%$  of  the  flow itself,  the  effective
  viscosity stabilizes to its maximum  value and the current return in
  an isoviscous  dynamics, but with cold  viscosity. Further analyses
  show  that the  effective viscosity  is the  average viscosity  of a
  small region  a the front of  the current in the  bending regime and
  the  average  viscosity  of  the  current  in  the  gravity  regime.
  Therefore,  the  evolution  of  an  elastic-plated  gravity  current
  depends on the relative phase changes  within the two regime and the
  transition   between  the   two  regime   itself.   Application   to
  terrestrial  laccolith,  which  is  a major  process  in  the  crust
  formation,  show that  they  should preferentially  solidify in  the
  third phase of the bending regime.

\end{abstract}

\newpage

\section{Introduction}

Elastic-plated  gravity  currents  involve the  spreading  of  viscous
material beneath  an elastic  sheet. The  applications range  from the
emplacement      of      lava      in      the      shallow      crust
\citep{Michaut:2011kg,Bunger:2011cb} and melt-water drainage below ice
sheet \citep{Das:2008in,Tsai:2010ev} in geological  setting to the the
manufacture of flexible electronics and microelectromechanical systems
(MEMS) in engineering \citep{Hosoi:2004dn}.

When  the thickness  of  the flow  is small  compared  to its  extent,
lubrication  approximation applies  and  the  study of  elastic-plated
gravity currents  resumes to  the study of  a sixth  order, non-linear
partial                      differential                     equation
\citep{Michaut:2011kg,Lister:2013ia,Anonymous:QWXp_4JV}   .   However,
the assumption  that the thickness of  the fluid tends to  zero at the
contact line leads to divergent  viscous stresses, and hence, the need
of      a     regularization      condition      at     the      front
\citep{Flitton:1999iv,Lister:2013ia,Anonymous:QWXp_4JV}.   One  common
approach is to add a thin  pre-wetted film of fluid, thus avoiding the
requirement  for any  boundary conditions  at a  genuine contact  line
\citep{Lister:2013ia,Anonymous:QWXp_4JV}.

The dynamics  of the spreading  has been described in  an axisymmetric
geometry   for    a   Newtonian   fluid   with    constant   viscosity
\citep{Michaut:2011kg,Lister:2013ia,Thorey:2014cv}.    In  particular,
they show the  presence of two distinct regimes  of evolution.  First,
gravity  is negligible  and  the peeling  of the  front  is driven  by
bending; the interior is bell-shaped, the radius evolves as $t^{8/22}$
and  the thickness  evolves as  $t^{7/22}$.  When  the radius  becomes
larger $4\Lambda$, where  $\Lambda$ is the flexural  wavelength of the
upper  layer, the  weight of  the  current becomes  dominant over  the
bending  terms  and the  evolution  enters  a gravity  current  regime
\citep{Huppert:1982a}. In  this second  regime, the  thickness profile
shows a flat top with bent  edges, the radius evolves as $t^{1/2}$ and
the thickness tends to a  constant.  Different analogue experiments of
isoviscous     flows     confirm      these     theoretical     results
\citep{Dixon:1987js,Lister:2013ia}.

However, in  many real geological settings,  the isothermal/isoviscous
assumption  are not  valid.   Indeed,  many geological  elastic-plated
gravity currents  are comprised of  fluid whose viscosity can  vary by
several orders of magnitude depending on its temperature.  This is the
case for  magmas produced by partial  melting of the upper  mantle and
intruding      the      shallow      layers     of      the      crust
\citep{Anonymous:CZVBrBvv,Lejeune:1995fc}.  Therefore, as the fluid is
cooling, its composition  and crystal content changes  which, in turn,
modifies the  viscosity and  the dynamics  of the  flow.  Several
studies have shown that this coupling between the cooling and the flow
itself in a  gravity current results in important  deviations from the
isoviscous                                                        case
\citep{Bercovici:2007vc,BALMFORTH:1999ey,Garel:2014era}.

In  this paper,  we examine  how  the spreading  of an  elastic-plated
gravity current is affected by  the cooling itself.  In particular, we
consider  the  problem  of  an elastic-plated  gravity  current  whose
viscosity depends  on temperature  according to a  prescribed rheology
$\eta(T)$.   This  gives rise  to  a  set  of two  coupled  non-linear
equations  that  we solve  numerically.   We  study the  flow  thermal
structure and its effect on the  dynamics through the rheology in each
regime separately.  In both regimes, we identify different ``thermal''
phases  of  propagation  that  we characterize  by  different  scaling
laws.  We   then  discuss  our  result   implications  concerning  the
emplacement of terrestrial laccoliths.


\section{Theory}
\label{C3-sec:theory}

\subsection{Formulation}
\label{sec:formulation}

We model  the axisymmetric  flow of  fluid below  an elastic  layer of
constant  thickness  $d_c$  and  above a  semi  infinite  rigid  layer
\citep{Michaut:2011kg} (Figure \ref{Figure2-1}).   The assumption that
the thickness of the fluid $h(r,t)$  tends to zero at the contact line
leads to divergent viscous stresses  and to the theoretical immobility
of the current \citep{Flitton:1999iv}. To avoid problem at the contact
line,  we  consider  a  thin   pre-wetting  film  of  thickness  $h_f$
\citep{Lister:2013ia} (Figure \ref{Figure2-1}).

The  fluid is  injected continuously  at the  base and  center of  the
current  at  a constant  rate  $Q_0$  through  a conduit  of  diameter
$a$. The hot fluid is intruded  at temperature $T_i$ and cools through
the  top and  bottom by  conduction in  the surrounding  medium, whose
temperature is  considered constant  and equal to  $T_0$.  In  using a
fixed  temperature at  the flow  boundary, we  essentially assume  the
fluid is bounded by a medium with infinite thermal conductivity.

%% FIGURE 2-1
\begin{figure}
  \begin{center}
    \graphicspath{ {/Users/thorey/Documents/These/Projet/Refroidissement/Skin_Model/Figure/JFM_V13/} }
    \includegraphics[scale=0.40]{Sketch.eps}
    \caption{Model geometry and parameters. The vertical scale is exaggerated.}
    \label{Figure2-1}
  \end{center}
\end{figure}

As  it  cools,  the  viscosity  of the  fluid  increases  following  a
prescribed rheology $\eta(T)$ given by
\begin{equation}
  \eta(T)=\frac{\eta_h
    \eta_c(T_i-T_0)}{\eta_h(T_i-T_0)+(\eta_c-\eta_h)(T-T_0)}
  \label{rheology}
\end{equation}
where $\eta_h$  and $\eta_c$  are the viscosities  of the  hottest and
coldest  fluid  at  the   temperature  $T_i$  and  $T_0$  respectively
\citep{Bercovici:2007vc}.    Although   this   rheology   is   largely
simplified,  the  inverse  dependence   of  viscosity  on  temperature
captures  the  essential  behavior  of  a  viscous  fluid,  i.e.   the
viscosity  variations are  the largest  where the  temperature is  the
coldest
\citep{Anonymous:CZVBrBvv,Marsh:1981dc,Lejeune:1995fc,Giordano:2008em}.

\subsection{Pressure}
\label{sec:Pressure}

The  intrusion develops  over a  length scale  $\Lambda$ that  is much
larger than its thickness $H$ ($\Lambda >> H$).  In the laminar regime
and  in   axisymmetrical  coordinates  ($r$,$z$),   the  Navier-Stokes
equations under the lubrication assumption are
\begin{eqnarray}
  -\frac{\partial P}{\partial r}  +  \frac{\partial}{\partial z}\left(\eta(T) \frac{\partial u}{\partial z}\right) &=&0\label{V1} \\
  -\frac{\partial P}{\partial z}  - \rho_{m}g&  =&0\label{Npressure}
\end{eqnarray}
where $u(r,z,t)$ is the radial velocity, $\rho_m$ the fluid density, $g$
the standard acceleration  due to gravity and  $P(r,z,t)$ the pressure
within  the fluid.   Integration of  (\ref{Npressure}) gives the
total  pressure  $P(r,z,t)$  within   the  flow.   When  the  vertical
deflection $h(r,t)$  of the upper  elastic layer is small  compared to
its thickness  $d_c$, i.e $h<<d_c$,  we can neglect stretching  of the
upper layer and only consider  bending stresses.  Therefore, the total
pressure $P(r,z,t)$ at a level $z$ in  the current is the sum of three
contributions: the weight of the magma  and of the upper layer and the
bending pressure
\begin{equation}
  P = \rho_m g (h-z)+\rho_rgd_c+D\nabla_r^4h
\end{equation}
where  $h(r,t)$ is  the flow  thickness, $\rho_r$  the density  of the
surrounding rocks and $D$ is the flexural rigidity of the thin elastic
layer, that  depends on Young's  modulus $E$, Poisson's  ratio $\nu^*$
and     on     the     elastic    layer     thickness     $d_c$     as
$D  = Ed_c^3/\left(12(1-\nu^*)\right)$.

\subsection{Injection rate}

Assuming a Poiseuille flow within the cylindrical feeding conduit, the
vertical  injection velocity  $w_i(r,t)$  and injection  rate $Q_0$  are
given by
\begin{equation}
  w_i(r,t)=
  \begin{cases}
    \frac{ \Delta P}{4 \eta_h Z_{c}} (\frac{a^{2}}{4}-r^{2})& r \le \frac{a}{2}\\
    0 & r > \frac{a}{2}
  \end{cases}
  \label{eq12}
\end{equation}
\begin{equation}
  Q_{0}=\frac{\pi \Delta P a^{4}}{128 \eta_h Z_c}
  \label{eq11}
\end{equation}
where  $\Delta P$  is  the  initial overpressure  within  the melt  at
$z=Z_{c}$. 

\subsection{Heat transport equation}
\subsubsection{Local energy conservation}

In the laminar regime and in axisymmetrical coordinates ($r$,$z$), the
local energy  conservation equation within the  lubrication assumption
is written as
\begin{eqnarray}
  \frac{D}{D t}\left(\rho_m C_{p,m} T+\rho_mL(1-\phi)\right)&=& k_m  \frac{\partial^2
                                                                T}{\partial               z^2}\label{EnergyCons}
\end{eqnarray}
where  $T(r,z,t)$ is  the fluid  temperature and  $\rho_m$, $k_m$  and
$C_{p,m}$ are the  density, thermal conductivity and  specific heat of
the fluid.  Here, we also account for energy release by crystallization of
the  fluid, which  is  a non  negligible source  of  heat for  magmas;
$\phi(r,z,t)$ is  the crystal fraction in  the melt and $L$  the latent
heat. In this model, the crystals are considered only as a source/sink
of energy as  they melt/form during flow  emplacement.  In particular,
the physical properties of the fluid  are not modified by the presence
of crystals.

Following a common approximation, we  assume that the crystal fraction
is a linear function of temperature over the melting interval
\begin{equation}
  \phi = \frac{T_L-T}{T_L-T_s}
  \label{meltfraction}
\end{equation}
where $T_S$ and $T_L$ are the solidus and liquidus temperatures of the
magma \citep{Hort:1997hk,Michaut:2006di}. In addition, we assume that
the fluid is  injected as its liquidus temperature ,  i.e. $T_L = T_i$
and, for simplicity, that the solidus  temperature  is  equal  to  the
surrounding rock  temperature $T_S  =T_0$. With  these approximations,
the local energy equation (\ref{EnergyCons}) resumes to
\begin{eqnarray}
  \frac{\partial T}{\partial t}+ u\frac{\partial T}{\partial r}
  + w\frac{\partial T}{\partial z}  &=& \frac{ St}{St+1}\kappa_m  \frac{\partial^2
                                        T}{\partial               z^2}
                                        \label{EnergyCons2}
\end{eqnarray}
where  $u(r,z,t)$ and  $w(r,z,t)$ are  the radial  and vertical  fluid
velocities,  $St  =\left(C_{p,m}(T_i-T_0)\right)/L$   is  the  Stephan
number   and    $\kappa_m$   is   the   fluid    thermal   diffusivity
$\kappa_m = k_m/(\rho_m C_{p,m})$.  We  use an integral balance method
to solve the heat transport equation (\ref{EnergyCons2}).  This theory
is based  on the  integral-balance method  of heat-transfer  theory of
\citet{Goodman:1958ue},  in  which  the   vertical  structure  of  the
temperature field  is represented  by a known  function of  depth that
approximates the expected solution.

\subsubsection{Integral   balance   solution   for   the   temperature
  $T(r,z,t)$}

Following \citet{BALMFORTH:1999ey},  we model the cooling  of the flow
through  the  growth  of  two thermal  boundary  layers:  one  growing
downward from the  top and a second growing upward  from the base.  As
we consider homogeneous thermal  properties for the surrounding rocks,
we assume that the two  thermal boundary layers grow symmetrically and
have  the   same  thickness  $\delta(r,t)$.   We   use  the  following
approximation for the vertical temperature profile $T(r,z,t)$
\begin{equation}
  T=
  \begin{cases}
    T_b - (T_b-T_0)(1-\frac{z}{\delta})^2 & 0 \le z\le \delta \\
    T_b & \delta \le z\le h-\delta \\
    T_b - (T_b-T_0)(1-\frac{h-z}{\delta})^2 & h-\delta \le z\le h\\
  \end{cases}
  \label{Temperature}
\end{equation}
where $T_b(r,t)$  is the temperature at  the center of the  flow.  The
integral balance  solution in  (\ref{Temperature}) assumes  a symmetry
around $z=h/2$  and a decrease of  the temperature in the  two thermal
boundary  layers  down  to  the  surrounding  rock  temperature  $T_0$
\citep{BALMFORTH:1999ey}.    In  addition,   it   assumes  a   uniform
temperature $T_b$  in between the  thermal boundary layers.   As
the fluid is injected at  temperature $T_i$, we have $T_b(r,t) =T_i$
as long as $\delta<h/2$.  However, if the two thermal boundary layers
connect,  then   $\delta  =  h/2$   and  $T_b$  becomes   such  that
$T_b\le T_i$.  This profile assures  the continuity of  the temperature
and heat flux within the flow.


\subsubsection{Integral balance equation}
\label{sec:integr-balance-equat}

We begin by integrating the local energy conservation equation
(\ref{EnergyCons2}) separately  over the two thermal  boundary layers.
The integration  over the  bottom thermal layer,  i.e. from  the base,
$z=0$ to a level $z = \delta$ gives
\begin{eqnarray}
  &&\frac{\partial}{\partial t}\left( \delta( \bar{T}-T_b)\right)+\frac{1}{r}\frac{\partial}{\partial r} \left( r\delta(\overline{uT}-\bar{u}T_b)\right) + \delta\left( \frac{\partial T_b}{\partial t}+ \overline{u}\frac{\partial T_b}{\partial r}\right)\nonumber\\
  &=&-\frac{\kappa_m}{1+St}\left. \frac{\partial T}{\partial z}\right|_{z=0}+w_{i}(T_{i}-T_b)
      \label{Local1}
\end{eqnarray}
where the  bars indicate the  vertical average over the  bottom thermal
boundary layer
\begin{equation}
  \overline{f} = \frac{1}{\delta}\int_0^{\delta}f dz\nonumber,
\end{equation}
$T_b(r,t)$  is  the temperature  at  $z=\delta$,  $w_{i}(r)$ is  the
vertical  injection velocity  and  we  have used  the  nullity of  the
thermal gradient at $z=\delta$ and the local mass conservation
\begin{equation}
  \frac{1}{r}\frac{\partial ru}{\partial r} +\frac{\partial w}{\partial z}=0.
  \label{MassConservation}
\end{equation}
The  integration over  the top  thermal layer,  i.e., from  the level,
$z=h-\delta$ to the top $z=h$ gives
\begin{eqnarray}
  &&\frac{\partial}{\partial t}\left( \delta( \bar{T}-T_b)\right)+\frac{1}{r}\frac{\partial}{\partial r} \left( r\delta(\overline{uT}-\bar{u}T_b)\right) + \delta\left(\frac{\partial T_b}{\partial t}+ \overline{u}\frac{\partial T_b}{\partial r}\right)\nonumber\\
  &=&\frac{\kappa_m}{1+St^{-1}}\left. \frac{\partial T}{\partial z}\right|_{z=h}.
      \label{Local2}
\end{eqnarray}
where,    in    addition    to    the    local    mass    conservation
(\ref{MassConservation})  and the  fact that  the thermal  gradient at
$z=h-\delta$ is  equal to  zero, we have  used the  kinematic boundary
condition in $z=h(r,t)$
\begin{equation}
  \frac{\partial h}{\partial t} +u\frac{\partial h}{\partial
    r} = w
\end{equation}

Therefore,  the  heat  balance   equation,  i.e.   the  heat  equation
(\ref{EnergyCons2})  integrated  over  the  flow  thickness,  is
obtained  by adding  (\ref{Local1})  and (\ref{Local2}).   Introducing
(\ref{Temperature}) to derive the conductive fluxes, we finally obtain
\begin{eqnarray}
  &&\frac{\partial}{\partial t}\left( \delta( \bar{T}-T_b)\right)+\frac{1}{r}\frac{\partial}{\partial r} \left( r\delta(\overline{uT}-\bar{u}T_b)\right) + \delta\left( \frac{\partial T_b}{\partial t}+ \overline{u}\frac{\partial T_b}{\partial r}\right)\nonumber\\
  &=&-\frac{2\kappa_m}{(1+St^{-1})}\frac{\left( T_b - T_0\right)}{\delta}+\frac{w_{i}}{2}(T_{i}-T_b)
      \label{LocalHeat3}
\end{eqnarray}

\subsection{Equation of motion}
\label{sec:equation-motion}

A global statement of mass conservation gives
\begin{eqnarray}
  \frac{\partial h}{\partial t}+ \frac{1}{r}
  \frac{\partial}{\partial
  r} \left( r\int_0^hudz\right)= w_i.
  \label{C3}
\end{eqnarray}
To obtain an  equation for the flow thickness, we  first note that the
chosen vertical structure of the temperature field (\ref{Temperature})
is symmetric  around $h/2$, and  thus, the viscosity and  velocity $u$
possess  the same  symmetry.   Taking advantage  of  this symmetry,  we
integrate              once              (\ref{V1})              using
$\left.\frac{\partial u}{\partial z}\right|_{z=h/2}=0$ to get
\begin{equation}
  \frac{\partial   u}{\partial   z}   =   \frac{1}{\eta}\frac{\partial
    P}{\partial r}\left(z-\frac{h}{2}\right).
  \label{C3-deriv}
\end{equation}
Using no-slip  boundary conditions at  the top  and the bottom  of the
flow,  i.e.  $u(r,z=0,t)=u(r,z=h,t)=0$  equation  (\ref{C3})  can  be
rewritten as
\begin{eqnarray}
  \frac{\partial h}{\partial t} = \frac{1}{r}
  \frac{\partial}{\partial
  r} \left( r\int_0^h\frac{\partial u}{\partial z}zdz\right) + w_i
  \label{C3-Mass}
\end{eqnarray}
and injecting (\ref{C3-deriv}) into  (\ref{C3-Mass}) finally gives the
equation for the flow thickness evolution in axisymmetric coordinates
\begin{eqnarray}
  \frac{\partial h}{\partial t} = \frac{1}{r}
  \frac{\partial}{\partial r} \left( r\left(\rho_m g \frac{\partial h}{\partial      r}+D\frac{\partial}{\partial      r}\left(\nabla_r^4h\right)\right)\left(\int_0^h\frac{1}{\eta(y)}\left(y-\frac{h}{2}\right)ydy\right)\right)
  + w_i.
  \label{C3-Mass-2}
\end{eqnarray}
In  addition,  integration  of   (\ref{C3-deriv})  using  the  no-slip
boundary condition at the base of the flow gives
\begin{equation}
  u(r,z,t) = \frac{\partial P}{\partial r} \int_0^z\frac{1}{\eta(y)}\left(y-\frac{h}{2}\right)dy.
\end{equation}

\subsection{Dimensionless equations}
\label{sec:dimens-equat}

We use the characteristic temperature interval $\Delta T = T_i-T_0$ to
nondimensionalize temperatures.  The dimensionless integral balance
approximation (\ref{Temperature}) becomes
\begin{equation}
  \theta(z)=
  \begin{cases}
    \Theta_b\left(1 -(1-\frac{z}{\delta})^2\right)& 0 \le z\le \delta \\
    \Theta_b & \delta \le z\le h-\delta \\
    \Theta_b\left(1-(1-\frac{h-z}{\delta})^2\right)  &   h-\delta  \le
    z\le h
  \end{cases}
  \label{Temperature2}
\end{equation}
where   $\theta(r,z,t)$   is   the   dimensionless   temperature   and
$\Theta_b=\frac{T_b-T_0}{T_{i}-T_0}$.        Finally,        equations
(\ref{LocalHeat3}) and (\ref{C3-Mass-2})  are nondimensionalized using
a horizontal  scale $\Lambda$, a vertical  scale $H$ and a  time scale
$\tau$ given by
\begin{eqnarray}
  \Lambda &=& \left(\frac{D}{\rho_m g}\right)^{1/4}\label{L1}\\
  H&=&\left       (\frac{12\eta_h      Q_{0}}{\rho_{m}g       \pi}\right      )
       ^{1/4} \label{H1}\\
  \tau&=&\frac{\pi \Lambda^{2} H}{Q_{0}}\label{T1}
\end{eqnarray}
where  $\Lambda$  represents  the  flexural wavelength  of  the  upper
elastic layer \citep{Michaut:2011kg}, $H$ the characteristic thickness
of an isoviscous constant flux gravity current with viscosity $\eta_h$
\citep{Huppert:1982wr} and $\tau$ the characteristic time to fill up a
cylindrical flow of  radius $\Lambda$ and thickness $H$  at a constant
rate $Q_0$.   In addition, we  can define a horizontal  velocity scale
$U=\Lambda/\tau=\left(\rho_m           g           H^3\right)/\left(12
  \eta_h\Lambda\right)$.

The dimensionless equations are
\begin{eqnarray}
  \frac{\partial h}{\partial t}& =& \frac{12}{r}
                                    \frac{\partial}{\partial r} \left( r\left( \frac{\partial h}{\partial      r}+\frac{\partial}{\partial      r}\left(\nabla_r^4h\right)\right)I_1(h)\right)
                                    + w_i\label{EqFinal1}\\
  \frac{\partial}{\partial
  t}\left( \delta( \bar{\theta}-\Theta_b)\right)&=&-\frac{1}{r}\frac{\partial}{\partial
                                                    r}  \left(   r\delta(\overline{u\theta}-\bar{u}\Theta_b)\right)  -
                                                    \delta\left(      \frac{\partial       \Theta_b}{\partial      t}+
                                                    \overline{u}\frac{\partial     \Theta_b}{\partial    r}\right)\nonumber\\
                               &-&
                                   2Pe^{-1}St_m\frac{\Theta_b}{\delta}+\frac{w_{i}}{2}(1-\Theta_b)\label{HeatDimensionLess}\\
  w_{i}&=&
           \frac{32}{\gamma^{2}}\left(\frac{1}{4}-\frac{r^{2}}{\gamma^{2}}\right)\hspace{.2cm}
           \text{if} \hspace{.2cm} r < \gamma/2,\hspace{.2cm} w_i=0 \hspace{.2cm}
           \text{if} \hspace{.2cm} r \ge \gamma/2\\
  u(r,z,t)&   =&   12\left(   \frac{\partial   h}{\partial
                 r}+\frac{\partial}{\partial
                 r}\left(\nabla_r^4h\right)\right)I_0(z)\label{C3-Veloc}
\end{eqnarray}
with
\begin{eqnarray}
  I_0(z)&=&\int_0^z \left(\nu+(1-\nu)\theta(y)\right)\left(y-\frac{h}{2}\right)
            dy \label{I_1}\\
  I_1(z) &=& \int_0^z \left(\nu+(1-\nu)\theta(y)\right)\left(y-\frac{h}{2}\right)y dy\label{I_2}
\end{eqnarray}
and where $\gamma$, $Pe$, $St_m$  and $\nu$ are the four dimensionless
numbers that control the dynamics of the flow
\begin{eqnarray}
  \gamma&=&\frac{a}{\Lambda} \label{gamma}\\
  Pe&=&            \frac{H^2}{\kappa_m            \tau}\label{Pe}\\
  St_m &=& \frac{C_{p,m}\left(T_i-T_0\right)}{C_{p,m}\left(T_i-T_0\right)+L} \label{St}\\
  \nu&=& \frac{\eta_h}{\eta_c}\label{nu}.
\end{eqnarray}
$\gamma$  is the  dimensionless radius  of  the conduit,  it does  not
significantly influence  the flow and is  set to $0.02$ in  this study
\citep{Michaut:2009jx,Michaut:2011kg}; $Pe$ is the Peclet number which
compares the vertical diffusion of heat to the horizontal advection in
the interior; $St_m$ is a modified Stephan number which represents the ratio of
sensible heat between solidus and liquidus  to the total energy of the
fluid  at liquidus  temperature  and $\nu$  is  the maximum  viscosity
contrast, i.e.  the ratio between the hottest and coldest viscosity.

  \subsection{Further simplifications}
  \label{sec:furth-simpl}

  \subsubsection{Heat equation}
  \label{sec:heat-equation}

  The heat balance equations (\ref{HeatDimensionLess}) can reduce to
  \begin{eqnarray}
    \frac{\partial}{\partial
    t}\left( \delta( \bar{\theta}-1)\right)+\frac{1}{r}\frac{\partial}{\partial
    r}
    \left( r\delta(\overline{u\theta}-\bar{u})\right)&=&- 2Pe^{-1}St_m\frac{\Theta_b}{\delta} 
                                                         \label{HeatD_a}
  \end{eqnarray}
  Indeed, if the thermal boundary layers exist, $\Theta_b=1$, $\delta$
  is the  variable and  (\ref{HeatDimensionLess}) directly  reduces to
  (\ref{HeatD_a}).  In contrast, if the thermal boundary layers merge,
  $\delta=h/2$ and the variable is  $\Theta_b$. In this case, the heat
  balance equation (\ref{HeatDimensionLess}) reduces to
  \begin{eqnarray}
    \frac{\partial h\bar{\theta}}{\partial t}+\frac{1}{r}\frac{\partial}{\partial
    r} \left( rh\overline{u\theta}\right)-\Theta_b\left(\frac{\partial h}{\partial t}+\frac{1}{r}\frac{\partial}{\partial
    r} \left( rh\bar{u}\right)\right)&=& - 8St_mPe^{-1}\frac{\Theta_b}{h}+w_{i}(1-\Theta_b)
  \end{eqnarray}
  which, by using (\ref{C3}), rewrites
  \begin{equation}
    \frac{\partial h\bar{\theta}}{\partial t}+\frac{1}{r}\frac{\partial}{\partial
      r} \left( rh\overline{u\theta}\right) &=& w_i
    - 8St_mPe^{-1}\frac{\Theta_b}{h}.
\label{eqHS2}
  \end{equation}
  Equation (\ref{eqHS2}) also corresponds to (\ref{HeatD_a}) when
  $\delta=h/2$.

  Following \citet{BALMFORTH:1999ey}, we rewrite (\ref{HeatD_a}) using
  a new variable $\xi = \delta(1-\overline{\theta})$
  \begin{equation}
    \frac{\partial \xi}{\partial t}+\frac{1}{r}\frac{\partial}{\partial r} \left( r\bar{u}\xi\right)-\frac{1}{r}\frac{\partial}{\partial r} \left( r\delta(\overline{u\theta}-\bar{u}\bar{\theta})\right)&=&2Pe^{-1}St_m\frac{\Theta_b}{\delta}.
    \label{EqFinal2}
  \end{equation}
  where $\Theta_b$  and $\delta$ can  be calculated directly  from the
  expression of $\xi$ such that

\begin{tabular}{p{6cm}p{6cm}}
{
\begin{equation}
    \Theta_b(r)=
    \begin{cases}
      1 &\text{if } \hspace{.5cm} \xi\leq \xi_t \nonumber\\
      \frac{3}{2}-\frac{3\xi}{h} & \text{if} \hspace{.5cm} \xi > \xi_t\nonumber
    \end{cases}
  \end{equation}
                                   }
&
{
  \begin{equation}
    \delta(r)=
    \begin{cases}
      3\xi &\text{if } \hspace{.5cm} \xi\leq \xi_t \nonumber\\
      h(r,t)/2 & \text{if} \hspace{.5cm} \xi > \xi_t\nonumber\\
    \end{cases}
  \end{equation}
  }
\end{tabular}

with $\xi_t = h/6$.

The second term on the left hand side of (\ref{EqFinal2}) contains advection by
the  vertically  integrated  radial  velocity  while  the  third  term
contains a  correction accounting  for the  vertical structure  of the
temperature  field. The  term on  the  right is  the loss  of heat  by
conduction in the surrounding medium.

  \subsubsection{Average quantities}
  The average  velocity over  a thermal boundary  layer $\overline{u}$
  reads
  \begin{eqnarray}
    \overline{u}        =\frac{1}{\delta}\int_0^{\delta}udz        &=&
                                                                       u(r,\delta,t) - \frac{1}{\delta}\int_0^{\delta}\frac{\partial
                                                                       u}{\partial
                                                                       z}
                                                                       zdz\label{eqHello}\\
                                                                   &=&\frac{12}{\delta}
                                                                       \frac{\partial
                                                                       P}{\partial
                                                                       r}\left(\delta
                                                                       I_0(\delta)-I_1(\delta)\right)
  \end{eqnarray}
  where $P(r,z,t) = h+\nabla_r^4h$ is  the dimensionless pressure and we
  have used (\ref{C3-deriv}) in  (\ref{eqHello}).  The average rate of
  heat advected 
  $\overline{u\theta}$ over a thermal boundary layer reads
  \begin{eqnarray}
    \overline{u\theta}=\frac{1}{\delta}\int_0^{\delta}u\theta dz &=& \frac{1}{\delta}\left( [ uG(z) ]_{0}^{\delta} -\int_0^\delta
                                                                     G(z)\frac{\partial
                                                                     u}{\partial
                                                                     z}
                                                                     dz\right)\nonumber\\
                                                                 &=&\frac{12}{\delta} \frac{\partial P}{\partial r}\left(G(\delta)I_0(\delta)-I_2(\delta)\right)
  \end{eqnarray}
  where
  \begin{equation}
    G(z)                  =                 \Theta_b\left(                  z
      +\frac{\delta}{3}\left(1-\frac{z}{\delta}\right)^3\right)
  \end{equation}
  denotes a primitive of $\theta$ when $z<\delta$ and
  \begin{equation}
    I_2(z)=\int_0^y\left(\nu+(1-\nu)\theta(y)\right)G(y)
    \left(y-\frac{h}{2}\right)dy.
    \label{I_3}
  \end{equation}
  Therefore, we have
  \begin{equation}
    \overline{u\theta}-\overline{u}\overline{\theta}= \frac{12}{\delta} \frac{\partial P}{\partial r}\left(I_0(\delta)\left(G(\delta)-\delta\overline{\theta}\right)+\overline{\theta}I_1(\delta)-I_2(\delta)\right)
  \end{equation} 
  where  the average  temperature  over a  thermal  boundary layer  is
  $ \overline{\theta} = 2\Theta_{b}/3$

  \subsection{Summary of the equations}
  \label{sec:summary-equations}

  The  coupled equations  governing the  cooling of  an elastic-plated
  gravity current are summarized in term of the integrals (\ref{I_1}),
  (\ref{I_2}) and (\ref{I_3}) as follow
  \begin{eqnarray}
    \frac{\partial h}{\partial t}-\frac{12}{r}
    \frac{\partial}{\partial      r}
    \left( r I_1(h) \frac{\partial P}{\partial
    r}\right)
    \label{C3-HF}
    & =& \mathcal{H}(\frac{\gamma}{2}-r)\frac{32}{\gamma^{2}}\left(\frac{1}{4}-\frac{r^{2}}{\gamma^{2}}\right)\\
    \frac{\partial                                       \xi}{\partial
    t}+\frac{1}{r}\frac{\partial}{\partial                          r}
    \left( r\left(\bar{u}\xi-\Sigma\right)\right)&=&2Pe^{-1}St_m\frac{\Theta_b}{\delta}\label{C3-TF}
  \end{eqnarray}
  with

\begin{tabular}{p{6cm}p{6cm}}
{
\begin{equation}
    \Theta_b(r)=
    \begin{cases}
      1 &\text{if } \hspace{.5cm} \xi\leq \xi_t \nonumber\\
      \frac{3}{2}-\frac{3\xi}{h} & \text{if} \hspace{.5cm} \xi > \xi_t\nonumber
    \end{cases}
  \end{equation}
                                   }
&
{
  \begin{equation}
    \delta(r)=
    \begin{cases}
      3\xi &\text{if } \hspace{.5cm} \xi\leq \xi_t\\
      h(r,t)/2 & \text{if} \hspace{.5cm} \xi > \xi_t\nonumber\\
    \end{cases}
  \end{equation}
  }
\end{tabular}
\begin{eqnarray}
  \overline{u}&=& \frac{12}{\delta}\frac{\partial P}{\partial r}\left(\delta
                  I_0(\delta)-I_1(\delta)\right) \label{ubarF}\\
  \Sigma     &=& \frac{\partial     P}{\partial
                 r}\left(8I_1(\delta)\Theta_b-12I_2(\delta)\right)\label{SigmaF}
\end{eqnarray}
$P =  h+\nabla_r^4h$ is  the dimensionless pressure  and $\mathcal{H}$
the Heaviside function. The expression of
$I_0(\delta)$, $I_1(h)$,  $I_1(\delta)$ and  $I_2(\delta)$ as  well as
the numerical scheme are given in appendix \ref{Numeric}.

\subsection{Preliminary results for an isothermal flow}
\label{sec:prel-results-isoth}

For a  constant injection  rate, a  small pre-wetting  film thickness,
i.e.   $h_f<<1$ and  a viscosity  contrast $\nu$  set to  1, numerical
resolution  of (\ref{C3-HF})  shows two  asymptotic spreading  regimes
\citep{Michaut:2011kg,Lister:2013ia}.
\begin{figure}
  \begin{center}
    \graphicspath{ {/Users/thorey/Documents/These/Projet/Refroidissement/Skin_Model/Figure/JFM_V13/} }
    \includegraphics[scale=0.45]{Scaling_HR_ELASGRAV_Simple.eps}
    \caption{Left: Dimensionless thickness at  the center $h_0$ versus
      dimensionless  time  $t$.   Dotted-lines: scaling  laws  in  the
      bending  regime $h_0=  0.7h_f^{-1/11}t^{8/22}$  and the  gravity
      regime $h_0$  tends to a constant.   Right: Dimensionless radius
      $R$ versus  dimensionless time $t$.  Dotted-lines:  scaling laws
      in  the bending  regime  $R= 2.2h_f^{1/22}t^{7/22}$  and in  the
      gravity current regime $R\propto t^{1/2}$.}
    \label{Scaling_HR_ELASGRAV_Simple}
  \end{center}
\end{figure}

At  early times,  when  $R<<\Lambda$, gravity  is  negligible and  the
spreading dynamics is governed by the bending of the upper layer.  The
spreading  is  very  slow  and   the  interior  has  uniform  pressure
$P =\nabla_r^4h$.  The flow is  bell-shaped and its thickness is given
by
\begin{equation}
  h(r,t) = h_0(t)\left(1-\frac{r^2}{R^2(t)}\right)^2
  \label{IntrusionShape}
\end{equation}
with   $h_0(t)$  the   thickness  of   the  current   at  the   center
\citep{Michaut:2011kg,Lister:2013ia}.       In       this      regime,
\citet{Lister:2013ia} have  shown that the spreading  is controlled by
the propagation  of a peeling by  bending wave at the  flow front with
dimensionless velocity $c$
\begin{equation}
  c=    \frac{\partial             R}{\partial            t}             =h_f^{1/2}
  \left(\frac{\kappa}{1.35}\right)^{5/2}
  \label{WaveVelocity}
\end{equation}
where  $\kappa  =  \partial^2  h/\partial r^2$  is  the  dimensionless
curvature  of  the  interior  solution.   Using  the  propagation  law
(\ref{WaveVelocity})   and  the   form   of   the  interior   solution
(\ref{IntrusionShape}), they find that the  flow radius and height are
given by the following solutions
\begin{eqnarray}
  h_0(t)&=& 0.7 h_f^{-1/11}t^{8/22}\label{ScalingH}\\
  R(t) &=& 2.2h_f^{1/22}t^{7/22}\label{ScalingR}.
\end{eqnarray}
where the numerical pre-factor obtained in our simulations match those
of \citet{Lister:2013ia} (Figure \ref{Scaling_HR_ELASGRAV_Simple}).

In  contrast,  when  the  radius  R  becomes  larger  than  $4\Lambda$
($R>>\Lambda$), the  weight of the  current becomes dominant  over the
bending  terms.  The  pressure is  given by  the hydrostatic  pressure
$P =  h$ and  the current  enters a  classical gravity  current regime
where bending  terms only  affect the  solution near  the edge  of the
current  \citep{Huppert:1982a,Michaut:2011kg,Lister:2013ia}.  In  this
second regime, the radius evolves as $t^{1/2}$ and the thickness tends
to a constant (Figure \ref{Scaling_HR_ELASGRAV_Simple}).

In  the following,  we study  the effect  of the  cooling on  the flow
dynamics in both regimes separately. We first describe the thermal
structure  for an  isoviscous flow,  i.e. $\nu=1$  and then  study the
effect  of the  temperature-dependent viscosity  on the  flow dynamics
without crystallization, i.e $St_m =1$. Finally, we look at the effect
of crystallization  by setting  $St_m<1$.  For simplicity,  we present
the results for a  given film thickness ($h_f=5\cdot10^{-3}$). Results
for different film thicknesses are shown in Appendix \ref{FilmThickness}.

\section{Numerical approach}
\label{C3-sec:numerical-approach}

\subsection{General procedure}
\label{sec:general-procedure}

The coupled nonlinear partial differential equations (\ref{C3-HF}) and
(\ref{C3-TF}) are solved on a grid of size $M$ defined by the relation
$r_i = (i-0.5)\Delta r$ for $i=1,..,M$. The grid is shifted at the center to avoid
problem arising  from the axisymmetrical  geometry. We index  the grid
point by the indice $i$ and denote the solution on this grid $h_i$ and
$\xi_i$ and the secondary variables $\Theta_{b,i}$, $\Theta_{s,i}$ and
$\delta_i$. Both equations can be expressed on the convenient form
\begin{equation}
  \frac{\partial u}{\partial t} - f = 0
\end{equation}
where $u$  is the function we  want to integrate and  $f$ a non-linear
function  that depends  on $u$.   We  solve these  equations by  first
discretizing all the spatial  derivatives using Finite Difference. The
accuracy of the  scheme is determined by the  higher order derivatives
since  their numerical  approximation requires  the largest  number of
sample points. We then get  two systems of $M$ ordinary differential
equations with the form
\begin{equation}
  \frac{\partial u_i}{\partial t} - f_i = 0 \hspace{1cm} i = 1,...,M
\end{equation}
The time derivatives are first  order and, since explicit schemes tend
to be  very sensitive and unstable,  we use a fully  implicit backward
Euler scheme to get
\begin{equation}
  \frac{u_i^{n+1}-u_i^n}{\Delta t} - f_i(u_i^{n+1}) = 0 \hspace{1cm} i
  = 1,...,M
\label{C3-Num-1}
\end{equation}
Since  $f_i(u_i^{n+1})$ is  not a  linear function,  the system  above
cannot be re-arranged to solve $u_i^{n+1}$ in term of $u_i^{n}$ and an
iterative method  has to  be employed  instead. Fixed  point iteration
method have shown  poor results in converging toward  the solution and
we finally apply  second order Newton's method to  obtain the solution
at each time step.  In particular, we first linearize $u^{n+1}$ around
a guess  of the solution  by assuming $u^{n+1}=u^*+\delta  u^n$, where
$u^*$ is a guess and $\delta u^n$ is the error and we drop the $i$ for
clarity.   Then, we  expressed the  non-linear part  using a  Taylor's
expansion
\begin{equation}
  f^{n+1}=f(u^{n+1})=f(u^*+\delta
  u^n)=f(u^*)+J^h_{f}(u^*)\delta u^n\nonumber
\end{equation}
where  $J^u_{f}(u^*)$ is  the  jacobian matrix  for  the function  $f$
evaluated  in $h^*$.   Injecting the  expansion into  (\ref{C3-Num-1})
finally gives a  system of M linear equations for  the correction term
$\delta_h^n$ which can be expressed as
\begin{equation}
  (I-\Delta tJ^u_{f}(u^*))\delta u^n=u^n-u^*+\Delta t f(u^*)
\end{equation}
where $I$ is the identity matrix. Therefore,  each iteration solves  for $\delta u^n$  and we
use $u_n+\delta u^n$  as a new guess $u^*$ in  each iteration. This is
repeated  until $\delta  u^n$  becomes  sufficiently small.   Finally,
since the equations are coupled, we use a fixed-point iteration method
to  converge  toward  the  solution   $(h,\xi)$  at  each  time  step.
Therefore, the algorithm is the following at each time step
\begin{itemize}
\item Start with a guess for the values of all variables.
\item Solve the thickness equation (\ref{C3-HF}) for $h^{n+1}$ using Newton-Rhapsod method.
\item Solve the heat equation (\ref{C3-TF}) for $\xi^{n+1}$ using $h^{n+1}$ as a new guess for $h^*$
  and Newton-Rhapsod method.
\item Repeat  step one until  further iterations cease to  produce any
  significant changes in the values of both $h^{n+1}$ and $\xi^{n+1}$.
\end{itemize}
The computational scheme is summarized in the following.

\subsection{Thickness equation}

The thickness equation (\ref{C3-HF}) is written as
\begin{eqnarray}
  \frac{\partial h}{\partial t}-f(h,\xi)&=&0
\end{eqnarray}
with
\begin{eqnarray}
  f& =& \frac{1}{r}
        \frac{\partial}{\partial      r}
        \left(      r  \phi\left(     \frac{\partial      }{\partial
        r}\left(h+P\right)\right)\right)+w_i\\
  \phi &=& 12I_1(h)
\end{eqnarray}
and where $P$ is the dimensionless bending pressure $P = \nabla^4h$.

\vspace{.5cm} \textbf{Spatial discretization of f} \vspace{.5cm}

The  spatial discretization  is  obtained using  a central  difference
scheme  over  a  sub-grid  shifted  by $0.5\Delta  r$  from  the  main
grid. Therefore, we have
\begin{eqnarray}
  f_i&=&\frac{1}{r_i \Delta_r}\left(r_{i+1/2}\phi_{i+1/2}\left.\left(\frac{\partial h}{\partial r}+\frac{\partial P}{\partial r}\right)\right|_{i+1/2}-r_{i-1/2}\phi_{i-1/2}\left.\left(\frac{\partial h}{\partial r}+\frac{\partial P}{\partial r}\right)\right|_{i-1/2}\right)\nonumber\\
     &=&A_i\phi_{i+1/2}\left(h_{i+1}-h_i\right)-B_i\phi_{i-1/2}\left(h_{i}-h_{i-1}\right)\nonumber\\
     &+&A_i\phi_{i+1/2}\left(P_{i+1}-P_i\right)-B_i\phi_{i-1/2}\left(P_{i}-P_{i-1}\right)\nonumber\\
     &+&w_i\label{C3-Num-3}
\end{eqnarray}
where                $A_i=r_{i+1/2}/(r_i\Delta_r^2)$               and
$B_i=r_{i-1/2}/(r_i\Delta_r^2)$.   The bending  pressure  term $P$  is
very stiff and  needs a careful treatment.  In  particular, the fourth
order derivative requires a fourth order central difference scheme and
therefore, $P_i$ is  expressed over a seven point stencil  on the main
grid such that
\begin{equation}
  P_{i}=   \alpha_{i}h_{i-3}  +   \beta_{i}h_{i-2}+\gamma_{i}  h_{i-1}
  +\lambda_{i}h_{i}+\kappa_{i}h_{i+1}+\delta_ih_{i+2}+\epsilon_ih_{i+3}
  \label{C3-Num-4}
\end{equation}
with
\begin{eqnarray}
  &\alpha_{i}&=\frac{1}{24\Delta r^{4}}\left(-4+3p_3\Delta_r \right)\nonumber \\
  &\beta_{i}&=\frac{1}{24\Delta r^{4}}\left(48-24p_3\Delta_r-2p_2\Delta_r^2+2p_1\Delta_r^3\right) \nonumber\\
  &\gamma_{i}&=\frac{1}{24\Delta r^{4}}\left(-156+39p_3\Delta_r+32p_2\Delta_r^2-16p_1\Delta_r^3\right)\nonumber\\
  &\lambda_{i}&=\frac{1}{24\Delta r^{4}}\left(224-60p_2\Delta r^{2}\right) \nonumber\\
  &\kappa_{i}&=\frac{1}{24\Delta r^{4}}\left( -156-39p_3\Delta_r+32p_2\Delta_r^2+16p_1\Delta_r^3\right)\nonumber\\
  &\delta_{i}&=\frac{1}{24\Delta r^{4}}\left( 48+24p_3\Delta_r-2p_2\Delta_r^2-2p_1\Delta_r^3\right) \nonumber\\
  &\epsilon_{i}&=\frac{1}{24\Delta r^{4}}\left(-4-3p_3\Delta_r \right)\nonumber
\end{eqnarray}
and where $p_1=1/r_i^3$, $p_2=1/r_i^2$ and $p_3 = 2/r_i$. Finally, the
term $\phi_{i-1/2}$  and $\phi_{i-1/2}$, which depend  on the variable
$\Theta_b$, $\delta$ as well as  different power of $h$, are evaluated
in $i-1/2$ and  $i+1/2$ respectively. Different choices  for the value
of the variable at the mid-cell grid point do not show any significant
difference  and a  simple  average  is taken  such  that the  variable
$u_{i+1/2}$ is taken as $0.5(u_i+u_{i+1})$.

\vspace{.5cm}    \textbf{Expression   of    the   jacobian    $J_f^h$}
\vspace{.5cm}

The discretized  function $f_i$ can be  break down in three  part, the
gravitational part $f_i^{g}$  which is expressed in term  of the value
of $h$ on three  grid points $\left\{{i-1,i,i+1}\right\}$, the bending
part $f_i^{b}$ which is expressed in term  of the value of $h$ on nine
grid points  $\left\{{i-4,i-3,...,i+3,i+4}\right\}$ and  the injection
term which depends only on the grid point $i$ such that
\begin{equation}
  f_i = f_i^g+f_i^b+w_i
\end{equation}
Therefore, the jacobian is  nona-diagonal and its coefficient $J_{il}$
are
\begin{equation}
  J_{il}=
  \begin{cases}
    \frac{\partial f^{b}_i}{\partial h_{l}} &
    l = \left\{{i-4,i-3,i-2,i+2,i+3,i+4}\right\}\\
    \frac{\partial       f^{g}_i}{\partial       h_{l}}+\frac{\partial
      f^{b}_i}{\partial h_{l}} & l =
    \left\{{i-1,i,i+1}\right\}\\
    0 & \text{otherwise}
  \end{cases}
  \label{C2-eq12}
\end{equation}
The different  terms can be  easily derived from  (\ref{C3-Num-3}) and
(\ref{C3-Num-4}) with just slight  adjustment coming from the boundary
conditions.

\vspace{.5cm} \textbf{Boundary condition} \vspace{.5cm}

 We begin with
$h_i=h_f$ for  $i=1,..,M$.  Since the  flow is symmetric in  $r=0$, we
require that
\begin{equation}
  \left.\frac{\partial h}{\partial r}\right|_{r=0} =\left.\frac{\partial P}{\partial r}\right|_{r=0} =0
\end{equation}
and therefore for $i=1$, we have
\begin{eqnarray}
  f_i     &=&A_1\phi_{i+1/2}\left(h_{i+1}-h_i\right)\nonumber\\
          &+&A_i\phi_{i+1/2}\left(P_{i+1}-P_i\right)\nonumber\\
          &+&w_i\label{C3-Num-5}
\end{eqnarray}
The expression  of the  bending pressure, evaluated  over a  $7$ point
stencils, is problematic close to the boundary and reflection formulae
will  be  used  in  order   to  accommodate  the  boundary  conditions
\citet{Patankar:1980vu}.   In   particular,  we  have  $h_0   =  h_1$,
$h_{-1}=h_2$ and  $h_{-2}=h_3$.  Similarly, boundary condition  at the
end of the mesh is accounted by using a grid much larger than the flow
itself and requiring
\begin{equation}
  \left.\frac{\partial h}{\partial r}\right|_{r=r_M} =\left.\frac{\partial P}{\partial r}\right|_{r=r_M} =0
\end{equation}
which gives for $i=M$
\begin{eqnarray}
  f_i     &=&B_i\phi_{i-1/2}\left(h_{i}-h_{i-1}\right)\nonumber\\
          &+&B_i\phi_{i-1/2}\left(P_{i}-P_{i-1}\right)\nonumber\\
          &+&w_i\label{C3-Num-5}
\end{eqnarray}
with $h_{i>=M}=h_f$.


\vspace{.5cm} \textbf{Newton-Rhapsod method} \vspace{.5cm}

The Newton-Rhapsod method reads
\begin{equation}
  (I-\Delta tJ^h_{f}(h_k^*))\delta h_k^n=h^n-h_k^*+\Delta t f(h_k^*)
\end{equation}
where the  $k$ refers  to the $k$  iterations, $I$ is  a $M  \times M$
diagonal  matrix and  $J_f^h(h^*)$  is a  $M  \times M$  nona-diagonal
matrix.  This  system  of  linear  equations can  be  solved  using  a
nona-diagonal algorithm. At the first  iteration, we use $h^*_1 = h^n$
as     a    first     guess    and     then    we     iterate    using
$h^*_k  = h^n+\delta  h_{k-1}^n$ as  a new  guess for  each iterations
until $\delta h^n_{k}$ becomes  sufficiently small.  In particular, we
require that
\begin{equation}
  \delta h^n_k/h^*_{k}<\epsilon
\end{equation}
with $\epsilon = 10^{-4}$. 

\subsection{Heat equation}

The heat equation (\ref{C3-TF}) is written as
\begin{eqnarray}
  \frac{\partial \xi}{\partial t}-g(h,\xi)&=&0
\end{eqnarray}
with
\begin{eqnarray}
  g& =& \frac{1}{r}\frac{\partial}{\partial                          r}
        \left( r\Gamma\xi\right) +\frac{1}{r}\frac{\partial}{\partial                          r}
        \left(r\Sigma\right)+2Pe^{-1}St_m\frac{\left(\Theta_b-\Theta_s\right)}{\delta}\\
  \Gamma&=& -\overline{u}
\end{eqnarray}

\vspace{.5cm} \textbf{Spatial discretization of g} \vspace{.5cm}

As for the thickness equation,  the spatial discretization is obtained
using  a  central  difference  scheme   over  a  sub-grid  shifted  by
$0.5\Delta r$ from the main grid. Therefore, we have
\begin{eqnarray}
  g_i &=& \left(C_i\Gamma_{i+1/2}\xi_{i+1/2}-D_i\Gamma_{i-1/2}\xi_{i-1/2}\right)\\
      &+&\left(C_i\Sigma_{i+1/2}-D_i\Sigma_{i-1/2}\right)\\
      &+&2Pe^{-1}St_m\frac{\Theta_{b,i}-\Theta_{s,i}}{\delta_i}
\end{eqnarray}
with         $C_i         =r_{i+1/2}/(r_i\Delta        r)$         and
$D_i =r_{i-1/2}/(r_i\Delta r)$.   We use the average  between the grid
point $i$ and $i-1$ (resp. $i+1$) to evaluate the quantity in $\Gamma$
and  $\Sigma$ at  $i-1/2$ (resp.   $i+1/2$).   In addition,  we use  a
classical upwind  scheme to handle $\xi$  at the mid grid  point which
requires
\begin{eqnarray}
  \xi_{i+1/2} &=& \xi_i\\
  \xi_{i-1/2} &=& \xi_{i-1}
\end{eqnarray}

\vspace{.5cm}  \textbf{Expression   of  the   Jacobian  $J_{g}^{\xi}$}
\vspace{.5cm}

The expression  of the Jacobian  is much straightforward in  that case
and its coefficient $J_{il}$ are
\begin{equation}
  J_{il}=
  \begin{cases}
    -D_i\Gamma_{i-1/2}&
    l = i-1\\
    C_i\Gamma_{i+1/2} & l = i \\
    0 & \text{otherwise}
  \end{cases}
  \label{C2-eq12}
\end{equation}
with only slight adjustment coming from the boundary conditions.

\vspace{.5cm} \textbf{Boundary conditions} \vspace{.5cm}

We  consider $\Theta_b  =1$  and $\delta  = 10^{-4}$  in  the film  at
$t=0$. In this way, we ensure  that the average temperature across the
film at $t=0$ is close to $1$. By construction, $D_1=0$ and therefore,
for $i=1$ we have
\begin{eqnarray}
  g_i &=& C_i\Gamma_{i+1/2}\xi_{i}+ C_i\Sigma_{i+1/2} +2Pe^{-1}St_m\frac{\Theta_{b,i}-\Theta_{s,i}}{\delta_i}
\end{eqnarray}
For   $i=M$,   we    consider   that   $\Gamma_{i+1/2}=\Gamma_i$   and
$\Sigma_{i=1/2}=\Sigma_i$.   However,  the  choice  for  the  boundary
condition at the border of the grid $i=M$ is not important as we solve
the problem over a grid much larger than the flow itself.

\vspace{.5cm} \textbf{Newton-Rhapsod method} \vspace{.5cm}

The Newton-Rhapsod method reads
\begin{equation}
  (I-\Delta tJ^{\xi}_{g}(\xi_k^*))\delta \xi_k^n=\xi^n-\xi_k^*+\Delta t f(\xi_k^*)
\end{equation}
where the  $k$ refers  to the $k$  iterations, $I$ is  a $M  \times M$
diagonal  matrix and  $J_f^h(\xi^*)$ is  a $M  \times M$  tri-diagonal
matrix.   This  system of  linear  equations  can  be solved  using  a
tri-diagonal algorithm.  As for the  thickness equation, at  the first
iteration,  we use  $\xi^*_1 =  \xi^n$ as  a first  guess and  then we
iterate using $\xi^*_k = \xi^n+\delta  \xi_{k-1}^n$ as a new guess for
each iterations  until $\delta \xi^n_{k}$ becomes  sufficiently small.
In particular, we require that
\begin{equation}
  \delta \xi^n_k/\xi^*_{k}<\epsilon
\end{equation}
with $\epsilon = 10^{-4}$. In addition, at each iteration the quantity
$\Theta^*_{s,k}$, $\Theta^*_{b,k}$ and $\delta^*_k$, that are needed to evaluate $\Gamma$ and
$\Sigma$,  are  derived from  the value of  $\xi^*_{k}$  using
(\ref{C3-TS}), (\ref{C3-TB}) and (\ref{C3-DELTA}) respectively.

\section{Evolution in the bending regime}
\label{sec:evol-bend-regime}

We first concentrate on the case  in which only bending contributes to
the dynamics pressure.  The governing equations are thus (\ref{C3-HF}) and
(\ref{C3-TF}) where  $P=\nabla_r^4h$.  

\subsection{Thermal structure for an isoviscous flow, effect of $Pe$}
\label{sec:thermal-structure-an}

The current  cools by conduction  and thermal boundary layers  form at
the contact with the surrounding  medium.  These boundary layers first
connect at the  tip of the flow, where the  small thickness induces an
important  cooling (Figure  \ref{Grid_Time_ELAS}).  A  region of  cold
fluid forms at the front.

As the  current thickens  with time, a  balance between  advection and
diffusion of heat is never reached in the interior of the current. The
hot thermal  anomaly grows in extent  with time but it  extends slower
than the  current itself and  the cold fluid  region at the  tip grows
even faster.   For instance, for $Pe  =100$, while the region  of cold
fluid extends over about $10\%$ of  the current at $t=0.5$, it extends
over  about  $20\%$ at  $t  =10$  (Figure \ref{Grid_Time_ELAS}).   The
smaller $Pe$,  the more  important the  conductive
cooling   and  the   larger   the  cold   fluid   region  is   (Figure
\ref{Thickness_Temperature_Profile_ELAS}   and  \ref{Grid_PeNu_ELAS}).
For  instance, at  $t=10$, while  the cold  fluid region  extends over
about $20\%$  of the current for  $Pe=100$, it extends over  more than
$70\%$ for $Pe=1$ (Figure \ref{Grid_PeNu_ELAS}).

\begin{figure}
  \begin{center}
    \graphicspath{ {/Users/thorey/Documents/These/Projet/Refroidissement/Skin_Model/Figure/JFM_V13/} }
    \includegraphics[scale=0.35]{Grid_Time_ELAS_Pe1_Nu1.eps}
    \caption{Snapshots of  the flow thermal  structure $\theta(r,z,t)$
      at  different  times  indicated   on  the  plot.   Dashed  lines
      represent  the thermal  boundary  layers. Solid  grey lines  are
      isotherms for  $\theta =  0.2$, $0.4$,  $0.6$ and  $0.8$.  Here,
      $\nu=1$, $Pe =100$, $St_m = 1$.}
    \label{Grid_Time_ELAS}
  \end{center}
\end{figure}

\begin{figure}
  \begin{center}
    \graphicspath{ {/Users/thorey/Documents/These/Projet/Refroidissement/Skin_Model/Figure/JFM_V13/} }
    \includegraphics[scale=0.4]{Thickness_Temperature_Profile_ELAS.eps}
    \caption{Left: thickness normalized by the thickness at the center
      $h(r,t)/h_0(t)$  versus radial  axis normalized  by the  current
      radius $r/R(t)$  at different  times indicated  on the  plot for
      $Pe=1$   and  $\nu=1.0$.    Solid-lines  represent   the
      thickness profiles.   Dashed-lines represent the thermal
      boundary layers.  Right: Same plot but for $\nu=10^{-3}$.}
    \label{Thickness_Temperature_Profile_ELAS}
  \end{center}
\end{figure}

\begin{figure}
  \begin{center}
    \graphicspath{ {/Users/thorey/Documents/These/Projet/Refroidissement/Skin_Model/Figure/JFM_V13/} }
    \includegraphics[scale=0.48]{Grid_PeNu_ELAS_Final_2.eps}
    \caption{Snapshots of  the flow thermal  structure $\theta(r,z,t)$
      for different set ($\nu$,$Pe$) with  $\nu= 1$, $0.1$, $0.01$ and
      $0.001$  and $Pe=1$,  $10$, $100$  and $1000$  at $t=10$.  While
      $Pe$ controls the  thermal structure of the flow, it  has only a
      small influence on the flow aspect ratio which is controlled by $\nu$.}
    \label{Grid_PeNu_ELAS}
  \end{center}
\end{figure}

\subsection{Thickness and temperature profile, effect of $\nu$}
\label{sec:thickn-temp-prof-1-e}

When accounting for  the temperature dependence of  the viscosity, the
region of cold  fluid at the tip  is marked by a  higher viscosity and
enhances flow thickening at the  expense of spreading.  The larger the
viscosity  contrast,  the  larger  the aspect  ratio  $h_0/R$  (Figure
\ref{Grid_PeNu_ELAS}).  For  instance, for  the same value  of $Pe=1$,
while the aspect ratio is $0.7$ for $\nu=1$ at $t=10$, it is $4.2$ for
the  same   time  and  $\nu=10^{-3}$   (Figure  \ref{Grid_PeNu_ELAS}).
Nevertheless, the  shape of the flow  remains essentially self-similar
(\ref{IntrusionShape}) and cannot be  differentiated from the shape of
an isoviscous current  if the thickness and the  radial coordinates are
rescaled by  the thickness  at the center  $h_0(t)$ and  radius $R(t)$
(Figure \ref{Thickness_Temperature_Profile_ELAS}).

The flow thermal  structure is similar to the  isoviscous case (Figure
\ref{Grid_PeNu_ELAS}), the  thermal anomaly rapidly detaches  from the
tip of the  current and a region  of cold fluid develops  at the front
where  the heat  loss is  largest. However,  the important  thickening
induced by the viscosity increase limits heat loss to the surrounding.
The  larger  the viscosity  contrast  $\nu$,  the more  important  the
thickening and  the larger the thermal  anomaly at a given  time.  For
instance, for  $Pe=1$, while  the thermal  anomaly extends  over about
$30\%$ of  the flow for $\nu=1$  at $t=10$, it extends  over more than
$50\%$ for $\nu=10^{-3}$ (Figure \ref{Grid_PeNu_ELAS}).

As expected, a larger Peclet number  leads to a larger thermal anomaly
(Figure  \ref{Grid_PeNu_ELAS}).   However, although  different  Peclet
numbers cause very  different thermal structures, the  influence of the
Peclet number on  the flow morphology is small, much  smaller than the
effect of the viscosity contrast $\nu$ (Figure \ref{Grid_PeNu_ELAS}).  For instance, for
$\nu=10^{-3}$ at $t=10$, the thermal  anomaly is still attached to the
tip of the current for $Pe = 1000$ whereas it makes about $50\%$ of the
current for $Pe=1$; but, the thickness $h_0$ and the
radius $R$ in both cases differ only by a few percents (Figure
\ref{Grid_PeNu_ELAS}). This suggests that the
spreading of  the flow is  not controlled  by the mean  temperature or
average viscosity of the flow. 
  
\subsection{Evolution of the thickness and the radius}
\label{sec:evol-thickn-radi-e}

The dynamics show three different  spreading phases.  The thickness as
well as the radius first follow  the isoviscous scaling laws for a hot
viscosity   current   $h_0\propto   t^{8/22}$   (\ref{ScalingH})   and
$R\propto  t^{7/22}$ (\ref{ScalingR})  (Figure \ref{Scaling_HR_ELAS}).
In the  second phase,  thickening occurs at  the expense  of spreading
because the thermal  anomaly has detached from the  current radius and
the viscous cold fluid region at the front slows down the spreading.
Finally, the  dynamics enters  a third phase  where the  thickness and
radius  follow the  scaling laws  for the  spreading of  an isoviscous
current characterized by a dimensionless cold viscosity $1/\nu$. These
scaling laws  are obtained from (\ref{ScalingH})  and (\ref{ScalingR})
by rescaling the characteristic thickness and time by $\nu^{1/4}$ and
read
\begin{eqnarray}
  h_{0} & = &0.7 \nu^{-2/11} h_f^{-2/22}t^{8/22}\label{ScalingH-Visco}\\
  R& = & 2.2 \nu^{1/11}h_f^{1/22} t^{7/22}\label{ScalingR-Visco}.
\end{eqnarray}
\begin{figure}
  \begin{center}
    \graphicspath{ {/Users/thorey/Documents/These/Projet/Refroidissement/Skin_Model/Figure/JFM_V13/} }
    \includegraphics[scale=0.45]{Scaling_HR_ELAS.eps}
    \caption{Left: Dimensionless thickness at  the center $h_0$ versus
      dimensionless time  $t$ for different sets  $(\nu,Pe)$ indicated
      on      the      plot.      Dotted-lines:      scaling      laws
      $h_0=  0.7h_f^{-1/11}\nu^{-2/11}t^{8/22}$ for  $\nu  = 1.0$  and
      $0.001$.  Right:  Dimensionless radius $R$  versus dimensionless
      time  $t$  for  the  same sets  $(\nu,Pe)$.   Dotted-lines:  the
      scaling    laws    $R=   2.2h_f^{1/22}\nu^{1/11}t^{7/22}$    for
      $\nu = 1.0$ and $0.001$.}
    \label{Scaling_HR_ELAS}
  \end{center}
\end{figure}
The dependence on  the viscosity contrast $\nu$ indeed  fits very well
the  third phase  of the  flow observed  in the  numerical simulations
(Figure  \ref{Scaling_HR_ELAS}).   These   results  suggest  that  the
effective viscosity $\eta_e$  that governs the flow  dynamics is first
close to the  viscosity of the hot fluid; it  rapidly increases in the
second phase  to asymptotically tend to  the one of the  cold fluid in
the third phase.

The time  the flow spends in  each phase depends on  the Peclet number
$Pe$.  For instance,  for $\nu=10^{-3}$, while the  current leaves the
first phase at $t \sim 10^{-6}$ for $Pe =1.0$, this transition happens
only   after   $t   \sim   10^{-2}$    for   $Pe   =   10^3$   (Figure
\ref{Scaling_HR_ELAS}).  The  larger  the   Peclet  number,  the  less
efficient the  cooling, and thus  the longer  the flow remains  in the
first  phase  and the  later  it  reaches  the  third phase.

\subsection{Characterization of the thermal anomaly}
\label{sec:char-therm-anom-e}

Following \citet{Garel:2012bh},  we quantify  the size of  the thermal
anomaly  through   a  critical  thermal  radius   $R_c(t)$  where  the
temperature  at the  center of  the flow  $\Theta_b$ is  $1\%$ of  the
injection temperature, i.e. $\Theta_b(r=0)-\Theta_b(r=R_c) = 0.99$.

The thermal  anomaly is first advected  at the same velocity as the
current itself,  i.e.  $R(t) = R_c(t)$  (Figure \ref{R_Rc_ELAS} left).
After  a time  that depends  on $Pe$  and $\nu$,  the thermal  anomaly
detaches from  the tip and  $R(t)-R_c(t)$ increases with  time (Figure
\ref{R_Rc_ELAS}) .

In  the bending  regime, the  interior  pressure is  constant and  the
thickness profile  $h(r)$ is  given by  (\ref{IntrusionShape}) (Figure
\ref{Thickness_Temperature_Profile_ELAS}).   The size  of the  thermal
anomaly $R_c(t)$  is given by  the radius  where advection of  heat is
equal to heat loss
\begin{equation}
  \frac{d}{d    t}\left(\theta(r=   R_c,t)\right)    \propto   Pe^{-1}
  \frac{\partial^2}{\partial z^2}\left(\theta(r=R_c,t)\right).
  \label{HeatequationThermal}
\end{equation}
Assuming  that,  at the  edge  of  the  thermal anomaly,  $\theta$  is
constant and  close to  $\Theta_b$, i.e. $\theta\approx  \Theta_b$, we
obtain by  integrating (\ref{HeatequationThermal}) over $h$  and using
(\ref{Temperature2})
\begin{eqnarray}
  \frac{d}{dt}\left(\int_0^h\theta dz\right)&\propto& Pe^{-1} \frac{\Theta_b}{h}\nonumber\\
  \Theta_b\frac{d  h}{d   t}&\propto& Pe^{-1}
                                      \frac{\Theta_b}{h}\nonumber\\
  \frac{d h}{d t}&\propto& \frac{Pe^{-1}}{h}\label{Calcul1}.
\end{eqnarray}
Using  the thickness  profile (\ref{IntrusionShape}),  (\ref{Calcul1})
becomes
\begin{eqnarray}
  \alpha^2\left(1+\frac{R_c}{R}\right)^2\frac{\partial h_0}{\partial
  t}+\frac{4h_0R_c^2}{R^3}\frac{\partial
  R}{\partial
  t}\alpha\left(1+\frac{R_c}{R}\right) =\frac{Pe^{-1}}{\alpha^2\left(1+\frac{R_c}{R}\right)^2h_0}\nonumber
\end{eqnarray}
where we  introduce $\alpha (t)=  \left(R(t)-R_c(t)\right)/R(t)$, i.e.
the normalized  region beyond  $r=R_c(t)$.  In the  limit $\alpha<<1$,
i.e. $R_c/R\sim 1$ and discarding higher-order terms, we finally get
\begin{equation}
  \alpha^3\propto \frac{Pe^{-1}} {h_0^2(t)}\frac{R}{\frac{\partial R}{\partial t}}.
\end{equation}
Substituting  the  thickness  $h_0(t)$  and the  radius
$R(t)$ by  their respective scaling laws  (\ref{ScalingH-Visco}) and
(\ref{ScalingR-Visco}), the relative size of the normalized cold front
region $\alpha$ reads
\begin{equation}
  \alpha(t)&\propto& Pe^{-1/3}\nu^{4/33} h_f^{2/33}t^{1/11}.
  \label{ScalingXi}
\end{equation}
which is equivalent to
\begin{equation}
  R(t)-R_c(t) = 2.1 Pe^{-1/3}\nu^{7/33} h_f^{7/66}t^{9/22}
  \label{ScalingRRc}
\end{equation}
where the  numerically prefactor, which  depends on the  definition of
the thermal anomaly, has been chosen to fit the simulations.
\begin{figure}
  \begin{center}
    \graphicspath{ {/Users/thorey/Documents/These/Projet/Refroidissement/Skin_Model/Figure/JFM_V13/} }
    \includegraphics[scale=0.45]{R_Rc_ELAS.eps}
    \caption{Left:  Extent  of  the cold  fluid  region  $R(t)-R_c(t)$
      versus   dimensionless    time   for    different   combinations
      ($\nu$,$Pe$) indicated on the plot.   Right: Same plot but where
      we   rescale  the   extent   of  the   cold   fluid  region   by
      $Pe^{-1/3}\nu^{7/33}$.       Dotted-line:       scaling      law
      $(R(t)-R_c(t))Pe^{1/3}\nu^{-7/33}= 2.1 h_f^{7/66}t^{9/22}$.}
    \label{R_Rc_ELAS}
  \end{center}
\end{figure}

The predicted  scaling law  for the  extent of  the cold  fluid region
(\ref{ScalingRRc}) indeed  closely fits the numerical  simulations for
$\nu<1$ and the different Peclet numbers (Figure \ref{R_Rc_ELAS}). For
$\nu=1$ and $Pe=1$, the condition  $R-R_c<<R$ is no more respected for
$t>0.1$, the thermal anomaly is much  smaller than the flow itself and
the scaling law (\ref{ScalingRRc}) is no more applicable as expected.

\subsection{Effective viscosity of the current}
\label{sec:effect-visc-blist-e}

We  use  the   predicted  scaling  law  for   the  thickness  $h_0(t)$
(\ref{ScalingH-Visco}) to  infer the  time evolution of  the effective
viscosity    $\eta_e(t)$.      Indeed,    substituting     $\nu$    by
$\eta_h/\eta_e(t)$   in  (\ref{ScalingH-Visco})   and  inverting   for
$\eta_e(t)/\eta_h$, we get
\begin{eqnarray}
  \eta_e(t)/\eta_h&=& \left(\frac{h_0(t)t^{-8/22}}{0.7 h_f^{-2/22}}\right)^{11/2}\label{eff-visco}
\end{eqnarray}
where $h_0(t)$ is given by the simulation.
\begin{figure}
  \begin{center}
    \graphicspath{ {/Users/thorey/Documents/These/Projet/Refroidissement/Skin_Model/Figure/JFM_V13/} }
    \includegraphics[scale=0.45]{Visco_ELAS_Version_2.eps}
    \caption{Top:  Dimensionless   viscosity  $\eta(t)/\eta_h$  versus
      dimensionless time  t for  different combinations  ($\nu$, $Pe$)
      indicated  on  the  plot.    Solid  lines:  effective  viscosity
      $\eta_e/\eta_h$  defined  by  (\ref{eff-visco}).   Dashed-lines:
      average         flow         viscosity        defined         by
      $\overline{\eta_a(t)}/\eta_h                                   =
      \frac{1}{V(t)}\int_0^{R(t)}\int_0^{h(r,t)} r \eta(\theta) dr dz$
      where $V(t)$ is the current volume.  Dotted-lines: average front
      viscosity   $\eta_f/\eta_h$   defined  by   (\ref{front-visco}).
      Bottom left:  dimensionless effective viscosity  $\eta_e$ versus
      time where the  time has been rescaled by the  time for the flow
      to enter the second phase  $t_{b2}$.  Bottom right: Same as left
      but where the time has been rescaled by the time for the flow to
      enter the third phase $t_{b3}$.  }
    \label{Visco_ELAS_Version_2}
  \end{center}
\end{figure}

As suggested  by the results of  section \ref{sec:evol-thickn-radi-e},
the effective viscosity is first  close to the hot viscosity $\eta_h$,
i.e.   $\eta_e/\eta_h \sim  1$.  It  rapidly increases  in the  second
phase of propagation and finally  tends to the cold viscosity $\eta_c$
in  the   third  phase,  i.e.   $\eta_e/\eta_h   \sim  1/\nu$  (Figure
\ref{Visco_ELAS_Version_2} top).  The spreading in the isoviscous case
is controlled by  the propagation of a peeling by  bending wave at the
tip of the current  \citep{Lister:2013ia}.  In agreement, the behavior
of the effective viscosity has to  be linked with the rapid cooling of
the  front.   To  test  this  hypothesis,  we  calculate  the  average
viscosity $\eta_f(t)$ over a fixed front region of size $L$ in between
$R(t)-L$ and $R(t)$ such that
\begin{eqnarray}
  \eta_f/\eta_h =
  \frac{1}{V_f}\int_{R-L}^{R}\int_0^{h}  r  \eta(\theta)
  dr dz \label{front-visco}
\end{eqnarray}
where $V_f(t)$ is the volume of this region.  The numerical evaluation
of $\eta_f(t)$ for a constant size $L \sim 0.1$ gives a good agreement
with the evolution of the  effective viscosity $\eta_e$ for the second
phase of propagation  (Figure \ref{Visco_ELAS_Version_2}).  Therefore,
the effective viscosity, and thus the different phases of propagation,
are controlled by the average viscosity of a small region at the front
of the current.

At the  initiation of  the flow,  the pre-wetted  film is  composed by
fluid at the injection temperature, the thermal anomaly is attached to
the front  and the current spreads  with a hot viscosity  $\eta_h$. As
soon as the film has cooled, the thermal anomaly detaches from the tip
of  the  current and  the  effective  viscosity increases.   The  time
$t_{b2}$ the current enters this second  phase of the flow thus scales
as the time to cool the  pre-wetted film thickness by conduction, i.e.
$t_{b2}=0.1Peh_f^2$ where the numerical  prefactor has been matched to
the simulations.  Indeed, when rescaling the time of the simulations by
$t_{b2}$, the different combinations $(\nu,Pe)$ enter the second phase
simultaneously   (Figure  \ref{Visco_ELAS_Version_2},   bottom  left).
Then, the  size of the cold  fluid region at the  front increases, the
effective viscosity  increases, and when $R(t)-R_c(t)$  becomes larger
than $\sim  0.1$, the  current behaves as  an isoviscous  current with
cold viscosity $\eta_c$.  Therefore, the time $t_{b3}$ for the flow to
enter this third  phase scales as the time for  the cold fluid region,
whose  size  is  given  by   (\ref{ScalingRRc}),  to  be  larger  than
$\sim 0.1$. In  particular, we define the time $t_{b3}$  as the time for
the  effective  to  reach  $90\%$   of  its  maximum  value  $\eta_c$.
Inverting (\ref{ScalingRRc})  and matching the numerical  prefactor to
the                simulation                thus                gives
$t_{b3}\sim  0.01  Pe^{22/27}\nu^{-14/27}h_f^{-7/27}$.   Indeed,  when
rescaling  the time  of  the simulations  by  $t_{b3}$, the  different
combinations $(\nu,Pe)$  enter the third phase  simultaneously (Figure
\ref{Visco_ELAS_Version_2}, bottom right).

\subsection{Note on the effect of crystallization}
\label{sec:note-effect-cryst-1}

Here, we  examine the effect  of crystallization on the  flow dynamics
and use  values of $St_m <  1$.  Crystallization induces a  release of
latent heat in the fluid, increasing the amount of available energy at
a given time.
\begin{figure}
  \begin{center}
    \graphicspath{ {/Users/thorey/Documents/These/Projet/Refroidissement/Skin_Model/Figure/JFM_V13/} }
    \includegraphics[scale=0.45]{Scaling_HR_ELAS_Stm.eps}
    \caption{Left: Dimensionless thickness at  the center $h_0$ versus
      dimensionless time $t$ for  different values of $St_m$ indicated
      on the  plot, $\nu=0.001$ and $Pe  =10.0$.  Dotted-line: scaling
      law $h_0= 0.7h_f^{-1/11}\nu^{-2/11}t^{8/22}$  for $\nu = 0.001$.
      Right: Dimensionless  radius $R$  versus dimensionless  time $t$
      for  the same  combinations  of  dimensionless numbers.   Dotted
      lines:  scaling  law $R=  2.2h_f^{1/22}\nu^{1/11}t^{7/22}$  for
      $\nu = 0.001$.}
    \label{Scaling_HR_ELAS_Stm}
  \end{center}
\end{figure}
When $St_m<1$,  the tip of the  current remains hot for  a longer time
and the  flow transitions to the  second phase later than  in the case
where   $St_m=1$    (Figure   \ref{Scaling_HR_ELAS_Stm}).     As   the
crystallization  acts only  to reduce  the  cooling term  by a  factor
$St_m$ in (\ref{C3-TF}), one  can easily rewrite (\ref{ScalingRRc}) to
acount for the effect of crystallization on the size of the cold fluid
region
\begin{equation}
  R(t)-R_c(t) =2.1Pe^{-1/3}St_m^{1/3}\nu^{7/33}
  h_f^{7/66}t^{9/22}.\label{ScalingRRcFinal}\\
\end{equation}
Indeed, the  dependence with the  dimensionless number $St_m$  is well
described   by  the   scaling   law  (\ref{ScalingRRcFinal})   (Figure
\ref{R_Rc_ELAS_Stm}).  Accordingly, the time $t_{b2}$ and $t_{b3}$ for
the  current  to  enter  the  second  and  third  phase  of  the  flow
respectively are delayed and when accounting for crystallization read
\begin{eqnarray}
  t_{b2}&\sim&0.1Pe St_m^{-1} h_f^2\label{tb2}\\
  t_{b3}&\sim&   10^{-2}  St_m^{-22/27}Pe^{22/27}\nu^{-14/27}h_f^{-7/27}\label{tb3}
\end{eqnarray}
\begin{figure}
  \begin{center}
    \graphicspath{ {/Users/thorey/Documents/These/Projet/Refroidissement/Skin_Model/Figure/JFM_V13/} }
    \includegraphics[scale=0.45]{R_Rc_ELAS_Stm.eps}
    \caption{Left:  Extent  of  the cold  fluid  region  $R(t)-R_c(t)$
      versus   dimensionless    time   for    different   combinations
      ($\nu$,$St_m$) indicated  on the  plot and $Pe=1$.   Right: Same
      plot but  where we have  rescaled the  extent of the  cold fluid
      region  by  $St_m^{1/3}\nu^{7/33}$.   Dotted-line:  scaling  law
      $(R(t)-R_c(t))St_m^{-1/3}\nu^{-7/33}=                  2.1
      h_f^{7/66}t^{9/22}$.}
    \label{R_Rc_ELAS_Stm}
  \end{center}
\end{figure}


\section{Evolution in the gravity current regime}
\label{sec:evol-grav-curr-1}

To study the late time behavior, we concentrate on the case where only
the weight  of the fluid  contributes to the pressure.   The governing
equation  are thus  (\ref{C3-HF}) and  (\ref{C3-TF}) where  $P=h$.  We
follow      the     same      development  as      in     Section
\ref{sec:evol-bend-regime}.   In  particular,  we first  describe  the
thermal structure for an isoviscous flow, i.e. $\nu=1$, then we study
the effect of the temperature-dependent viscosity on the current
dynamics without crystallization, i.e.  $\nu<1$ and $St_m=1$. Finally,
we look at the effect of crystallization by setting $St_m<1$.

\subsection{Thermal structure for an isoviscous flow, effect of $Pe$}
\label{sec:thermal-structure-an-1}
  
As in the bending  regime, the bulk of the fluid  first expands at the
injection temperature and $R_c \sim R$.  As the bottom and the top cool
by conduction,  thermal boundary layers  form at the contact  with the
surrounding medium and connect at the  tip of the current. However, in
the gravity  current regime, the thickness  of the current tends  to a
constant.   Therefore, conduction  in the  surrounding medium  rapidly
balances the input of heat at  the center and when the thermal anomaly
detaches from the tip of the current, its extent reaches a steady state
profile (Figure \ref{Grid_Time_GRAV}).

\begin{figure}
  \begin{center}
    \graphicspath{ {/Users/thorey/Documents/These/Projet/Refroidissement/Skin_Model/Figure/JFM_V13/} }
    \includegraphics[scale=0.35]{Grid_Time_GRAV_Pe1_Nu1.eps}
    \caption{Snapshots of  the flow thermal  structure $\theta(r,z,t)$
      at different times indicated on the plot.  Dashed lines: thermal
      boundary layers.  Here, $\nu=1$, $Pe =100$ and $St_m = 1$.}
    \label{Grid_Time_GRAV}
  \end{center}
\end{figure}

The radius of the steady state  thermal anomaly $R_c$ depends on $Pe$.
In  particular, the  larger the  number  $Pe$, the  larger the  radius
$R_c$ is. For instance, for $\nu=1$, while the thermal anomaly $R_c$ is less than $1$
in the  steady state  regime for  $Pe=1$, it  is about
$12$ for $Pe=10^3$ (Figure \ref{Grid_PeNu_GRAV}).

\begin{figure}
  \begin{center}
    \graphicspath{ {/Users/thorey/Documents/These/Projet/Refroidissement/Skin_Model/Figure/JFM_V13/} }
    \includegraphics[scale=0.45]{Grid_PeNu_GRAV_Final_2.eps}
    \caption{Snapshots of  the flow thermal  structure $\theta(r,z,t)$
      for different sets ($\nu$,$Pe$) with $\nu= 1$ ,$0.1$ ,$0.01$ and
      $0.001$ and  $Pe=1$, $10$,  $100$ and  $1000$ at  $t=200$.}
    \label{Grid_PeNu_GRAV}
  \end{center}
\end{figure}

\subsection{Thickness and temperature profile, effect of $\nu$}
\label{sec:thickn-temp-prof}

For a current with a viscosity that depends on temperature, as soon as
the thermal anomaly  detaches from the current radius,  the cold fluid
at  the  front tends  to  slow  down  the  spreading and  enhance  the
thickening of  the flow (Figure \ref{Grid_PeNu_GRAV}).   For instance,
for $Pe=1$, while the aspect ratio $h_0/R$ is about $0.12$ for $\nu=1$
at   $t=200$,    it   is   $\sim   1$    for   $\nu=10^{-3}$   (Figure
\ref{Grid_PeNu_GRAV}).  The  shape of the current  is not self-similar
and the front  steepens when the viscosity increases  in comparison to
the isoviscous case as  in \citet{Bercovici:2007vc}. However, when the
current becomes much larger than the thermal anomaly, the current side
slumps to become less steep (Figure \ref{Grid_PeNu_GRAV}) and recovers
a shape similar to the isoviscous flow with cold viscosity.

The  thermal  structure  is  similar   to  the  isoviscous  case.   In
particular, after  a time  that depends on  $Pe$, the  thermal anomaly
reaches a  steady-state profile (Figure \ref{Grid_PeNu_GRAV}).   As in
the bending regime,  the thickening at the center limits  heat loss to
the  surrounding for  large values  of the  viscosity contrast  $\nu$.
Therefore, the  extent of the  thermal anomaly in the  steady-state is
slightly larger  for a larger  viscosity contrast.  For  instance, for
$Pe=10$ at $t=200$,  while the thermal anomaly extends  over less than
$2$ for $\nu=1$, it reaches $Rc\sim3$ for $\nu=10^{-3}$.

The  flow morphology  is  more sensitive  to the  $Pe$  number in  the
gravity current regime  than in the bending regime  and different $Pe$
lead  to different  current  morphologies for  a  given $\nu$  (Figure
\ref{Grid_PeNu_GRAV}).   For instance,  for $\nu=10^{-3}$  at $t=200$,
the thermal  anomaly is still attached  to the tip of  the current for
$Pe =  10^3$ and  the aspect  ratio of  the flow  $h_0/R$ is  close to
$0.15$.  In contrast, for $Pe=1$,  the thermal anomaly radius $R_c$ is
less than  $30\%$ of the  current radius and  the aspect ratio  of the
flow is much larger $h_0/R = 1.15$ (Figure \ref{Grid_PeNu_GRAV}).

\subsection{Evolution of the thickness and the radius}
\label{sec:evol-thickn-radi-g}
  
As in the bending regime,  the dynamics show three different spreading
phases.   The  thickness  as  well  as the  radius  first  follow  the
isoviscous  scaling laws  for  a given  hot  viscosity $\eta_h$,  i.e.
$h_0$   tends   to  a   constant   and   $R\propto  t^{1/2}$   (Figure
\ref{Scaling_HR_GRAV}).   In a  second  phase,  the thickness  rapidly
increases and  the spreading slows  down.  Finally, the  thickness and
radius follow  the isoviscous  scaling laws but  for a  cold viscosity
flow.

These dimensionless scaling laws read, as a function of $\nu$
\begin{figure}
  \begin{center}
    \graphicspath{ {/Users/thorey/Documents/These/Projet/Refroidissement/Skin_Model/Figure/JFM_V13/} }
    \includegraphics[scale=0.45]{Scaling_HR_GRAV.eps}
    \caption{Left: Dimensionless thickness at  the center $h_0$ versus
      dimensionless time  $t$ for different sets  $(\nu,Pe)$ indicated
      on   the  plot.    Dotted-lines  represent   the  scaling   laws
      $h_0=  2.1\nu^{-1/4}$ for  $\nu  = 1.0$  and $10^{-2}$.   Right:
      Dimensionless radius  $R$ versus dimensionless time  $t$ for the
      same sets  $(\nu,Pe)$.  Dotted-lines represent the  scaling laws
      $R= 1.1\nu^{1/8}t^{1/2}$ for $\nu = 1.0$ and $10^{-2}$.}
    \label{Scaling_HR_GRAV}
  \end{center}
\end{figure}
\begin{eqnarray}
  h_0 &=& 2.1\nu^{-1/4}\label{scaling-H-gravi-2}\\
  R(t) &=& 1.1\nu^{1/8} t^{1/2}\label{scaling-R-gravi-2}
\end{eqnarray}
They   perfectly    matched   our   numerical    simulations   (Figure
\ref{Scaling_HR_GRAV}).  Therefore,  the effective  viscosity $\eta_e$
that controls the flow dynamics is first close to the viscosity of the
hot fluid $\eta_h$;  it then rapidly increases  to asymptotically tend
to the viscosity of the cold fluid $\eta_c$ in the third phase.

As in  the bending regime, the  time the current spends  in each phase
depends on $Pe$ (Figure \ref{Scaling_HR_GRAV}).  For
instance, for $\nu=10^{-2}$, while the  current leaves the first phase
at  $t\sim10^{-1}$   for  $Pe=  1.0$,  the   transition  occurs  after
$t \sim 10^1$ for $Pe=10^2$.  In  general, the larger $Pe$, the longer
the current  remains in the first  phase and the later  is reached the
third phase.

\subsection{Characterization of the thermal anomaly}
\label{sec:char-therm-anom-g}

The thermal  anomaly is first advected  at the same velocity as the
current  itself, i.e.   $R_c(t)/R(t) \sim  1$ (Figure  \ref{R_Rc_GRAV}
left).   After a  time that  depends on  $Pe$ and  $\nu$, the  thermal
anomaly detaches  from the  front and  reaches a  steady-state profile
(Figure \ref{Grid_PeNu_GRAV} and \ref{R_Rc_GRAV}).
\begin{figure}
  \begin{center}
    \graphicspath{ {/Users/thorey/Documents/These/Projet/Refroidissement/Skin_Model/Figure/JFM_V13/} }
    \includegraphics[scale=0.45]{R_Rc_GRAV.eps}
    \caption{Left:  Normalized  thermal anomaly  radius  $R_c(t)/R(t)$
      versus   dimensionless    time   for    different   combinations
      ($\nu$,$Pe$) indicated on the plot.   Right: Same plot but where
      we rescale  the normalized thermal anomaly  radius $R_c(t)/R(t)$
      by $Pe^{1/2}\nu^{-1/4}$.}
    \label{R_Rc_GRAV}
  \end{center}
\end{figure}

We  propose a  simple  thermal budget  to predict  the  extent of  the
thermal  anomaly in  the steady-state  regime.  When the  size of  the
thermal  anomaly  reaches  a  steady state,  a  balance  between  heat
advection and  diffusion in  the surrounding  medium in  a dimensional
form gives
\begin{equation}
  \rho C_p U_0 \frac{\Delta T}{R_c} = \frac{8 k \Delta T}{h_0^2}
\end{equation}
where  $\Delta T$  hold for  a mean  temperature contrast  between the
fluid and  the surroundings  advected at  a mean  velocity $U_0$.  For a
gravity  current,  and  by  opposition  to  the  bending  regime,  the
thickness  $h_0$  reaches  a  constant. Taking  $U_0$  as  a  horizontal
redistribution of the injection rate, we write
\begin{equation}
  U_0=Q_0/(2\pi R_c h_0)
\end{equation}
which gives
\begin{equation}
  R_c=\frac{1}{4}\sqrt{\frac{h_0 Q_0}{\pi \kappa}}
  \label{RcDimensionn}
\end{equation}
By  non-dimensionalizing  (\ref{RcDimensionn}),  we  finally  get  the
expression for  the thermal anomaly  radius $R_c$ in the  steady state
regime. In  particular, we  have $R_c \propto  Pe^{1/2}\nu^{-1/8}$ and
then
\begin{equation}
  \frac{R_c}{R(t)} = 0.7Pe^{1/2}\nu^{-1/4}t^{-1/2}
  \label{Scaling-Rc-Gravy}
\end{equation}
where  we  have  used (\ref{scaling-R-gravi-2})  and  the  numerical
prefactor, which depends on the definition of the thermal anomaly, has
been chosen to fit the simulations.

The  analytical solution  for  the normalized  thermal anomaly  radius
$R_c/R(t)$   (\ref{Scaling-Rc-Gravy})  closely   fits  the   numerical
simulations  (Figure  \ref{R_Rc_GRAV}).    Indeed,  when  the  thermal
anomaly enters  the steady state,  the thermal anomaly  radius remains
constant  and  the  normalized thermal  anomaly  radius  $R_c(t)/R(t)$
evolves  as the  inverse of  the current  radius, i.e.   as $t^{-1/2}$
(Figure \ref{R_Rc_GRAV}).  Furthermore, both  the dependence with $Pe$
and $\nu$  vanish when rescaling $R_c/R(t)$  by $Pe^{1/2}\nu^{-1/4}$
in the steady state regime (Figure \ref{R_Rc_GRAV}, right).

\subsection{Effective viscosity of the current}
\label{sec:effect-visc-blist-g}

Repeating      the     same      exercise   as      in     section
(\ref{sec:effect-visc-blist-e}), we use the  predicted scaling law for
the  radius $R(t)$  (\ref{scaling-R-gravi-2}) to  infer the  effective
viscosity $\eta_e(t)$ of the current
\begin{eqnarray}
  \eta_e(t)/\eta_h&=& \left(\frac{R(t)t^{-1/2}}{1.1}\right)^{-8}\label{eff-visco-grav}.
\end{eqnarray}


\begin{figure}
  \begin{center}
    \graphicspath{ {/Users/thorey/Documents/These/Projet/Refroidissement/Skin_Model/Figure/JFM_V13/} }
    \includegraphics[scale=0.45]{Visco_GRAV_2.eps}
    \caption{Top:  Dimensionless   viscosity  $\eta(t)/\eta_h$  versus
      dimensionless time  t for  different combinations  ($\nu$, $Pe$)
      indicated  on  the  plot.    Solid  lines:  effective  viscosity
      $\eta_e/\eta_h$  defined  by  (\ref{eff-visco}).   Dashed-lines:
      average         flow         viscosity        defined         by
      $\overline{\eta_a(t)}/\eta_h                                   =
      \frac{1}{V(t)}\int_0^{R(t)}\int_0^{h(r,t)} r \eta(\theta) dr dz$
      where $V(t)$ is the  current volume.  Bottom left: dimensionless
      effective viscosity $\eta_e$ versus time where the time has been
      rescaled by  the time $t_{g2}$ (\ref{tg2}).   Bottom right: Same
      as  left  but where  the  time  has  been rescaled  by  $t_{g3}$
      (\ref{tg3}). }
    \label{Visco_GRAV_2}
  \end{center}
\end{figure}

As expected,  the effective  viscosity in  the gravity  current regime
represents  the average  viscosity of  the current  and the  different
phases of propagation reflect changes  in the average viscosity of the
flow (Figure \ref{Visco_GRAV_2}).

At  flow initiation,  the  thermal  anomaly is  advected  at the  same
velocity as the  current itself  and the  current spreads  with hot
viscosity $\eta_h$. When the thermal anomaly detaches from the tip and
enters a steady state, $\eta_e$ increases.  The time $t_{g2}$ to enter
this  second  phase scales  with  the  time  to  cool the  current  by
conduction,  i.e.  $t_{g2}=10^{-2}Pe$  where the  numerical pre-factor
has been matched to the  simulations.  Indeed, when rescaling the time
by $t_{g2}$,  the different  combinations $(\nu,Pe)$ enter  the second
phase simultaneously (Figure  \ref{Visco_GRAV_2}, bottom left).  Then,
the  size  of the  cold  fluid  region  at  the front  increases,  the
effective viscosity increases and, when  the current is large compared
to  the steady-state  thermal  anomaly radius,  i.e.  $R_c/R<0.6$  the
current behaves as an isoviscous current with cold viscosity $\eta_c$.
Therefore, the  time $t_{g3}$ for the  flow to enter this  third phase
scales as  the time for  the normalized  thermal anomaly radius  to be
smaller than $\sim0.6$.  As in the  bending regime, we define the time
$t_{g3}$ as the time for the  effective to reach $90\%$ of its maximum
value  $\eta_c$.    Inverting  (\ref{ScalingRRc})  and   matching  the
numerical     prefactor    to     the     simulation    thus     gives
$t_{g3}=5 Pe\nu^{-1/2}$. In agreement, when  rescaling the time of the
simulations by  $t_{g3}$, the different combinations  $(\nu,Pe)$ enter
the  third  phase simultaneously  (Figure  \ref{Visco_GRAV_2},
bottom right).

\subsection{Note on the effect of crystallization}
\label{sec:note-effect-cryst-2}

As in the bending regime,  crystallization induces a release of latent
heat to  the fluid,  increasing the  amount of  available energy  at a
given time.
\begin{figure}
  \begin{center}
    \graphicspath{ {/Users/thorey/Documents/These/Projet/Refroidissement/Skin_Model/Figure/JFM_V13/} }
    \includegraphics[scale=0.45]{Scaling_HR_GRAV_Stm.eps}
    \caption{Left: Dimensionless thickness at  the center $h_0$ versus
      dimensionless time $t$ for different sets $(\nu,St_m)$ indicated
      on the plot  and $Pe=1$. Right: Dimensionless  radius $R$ versus
      dimensionless  time  $t$  for  the same  sets  $(\nu,St_m)$  and
      $Pe=1$.}
    \label{Scaling_HR_GRAV_Stm}
  \end{center}
\end{figure}
As a  result, when $St_m<1$, the  current is hotter on  average and it
transitions to the second phase later  than in the case where $St_m=1$
(Figure      \ref{Scaling_HR_GRAV_Stm}).       As      in      section
(\ref{sec:note-effect-cryst-1}),     one     can    easily     rewrite
(\ref{Scaling-Rc-Gravy}) to account for  the effect of crystallization
on the thermal anomaly evolution
\begin{equation}
  \frac{R_c}{R(t)} = 0.7 St_m^{-1/2}Pe^{1/2}\nu^{-1/4}t^{-1/2}
  \label{Scaling-Rc-Gravy_Stm}
\end{equation}
Indeed, the  dependence with the  dimensionless number $St_m$  is well
described  by  the  scaling law  (\ref{Scaling-Rc-Gravy_Stm})  (Figure
\ref{R_Rc_GRAV_Stm}).  Accordingly, the time $t_{g2}$ and $t_{g3}$ for
the  current to  enter the  second  and third  phase of  the flow  are
both delayed and finally read
\begin{eqnarray}
  t_{g2}&\sim&10^{-2}PeSt_m^{-1}\label{tg2}\\
  t_{g3}&\sim& 5Pe St_m^{-1}\nu^{-1/2}\label{tg3}
\end{eqnarray}

\begin{figure}
  \begin{center}
    \graphicspath{ {/Users/thorey/Documents/These/Projet/Refroidissement/Skin_Model/Figure/JFM_V13/} }
    \includegraphics[scale=0.45]{R_Rc_GRAV_Stm.eps}
    \caption{Left:  Normalized  thermal anomaly  radius  $R_c(t)/R(t)$
      versus    dimensionless   time    for   different    combinations
      ($\nu$,$St_m$) indicated  on the  plot and $Pe=1$.   Right: Same
      plot but where  we have rescaled the  normalized thermal anomaly
      radius $R_c(t)/R(t)$ by $St_m^{-1/2}Pe^{1/2}\nu^{-1/4}$.}
    \label{R_Rc_GRAV_Stm}
  \end{center}
\end{figure}

\section{Different evolutions with bending and gravity}
\label{sec:diff-evol-with-1}

For an isoviscous  flow with $h_f<< h<< d_c$, the  flow passes through
three             asymptotic             dynamical             regimes
\citep{Michaut:2011kg,Lister:2013ia}.   When the  radius  $R$ is  much
smaller than a  critical radius $R_c\sim 4$, the  interior solution is
bell-shaped and peeling by bending controls propagation.  In contrast,
when $R>>R_c$, bending stresses can be neglected almost everywhere and
the   flow   enters   a   gravity   current   regime.    In   between,
\citet{Lister:2013ia} also describe a  short intermediate regime where
the peeling by bending continues  to control the propagation but where
the flow  shows an interior  flat-topped region due to  the increasing
effect of gravity. For simplicity, we only consider the two asymptotic
regimes. At the transition, the isoviscous current is characterized by
$R\sim4$, $h_0 \sim 2$ and $t  \sim 10$. In the following, we consider
a modified  Peclet number $Pe_m  = Pe St_m^{-1}$ which  integrates the
effect of crystallization for clarity.

For a current  with a temperature dependent  viscosity, the transition
between the bending regime and the gravity regime also occurs when the
radius   of   the   current   is    close   to   $R\sim   4$   (Figure
\ref{Scaling_HR_ELASGRAV_h}). However,  the time and thickness  of the
current at  the transition depends on  the thermal state of  the flow,
i.e.   on   the  combination  of  ($\nu$,$Pe_m$)   considered  (Figure
\ref{Scaling_HR_ELASGRAV_h}).   For  instance,  for $\nu=0.01$  and  a
small value of  $Pe$, i.e.  $Pe = 1.0$ the  current transitions to the
gravity regime  when it is in  the third thermal phase  of the bending
regime,  i.e. at  $t \sim  50$  with $h_0  \sim 8$,  as an  isoviscous
current with  given cold viscosity  $\eta_c=100$.  In contrast,  for a
larger value of  $Pe$, i.e. $Pe= 10^5$, the current  remains longer in
the  first  phase of  the  bending  regime  and  it spreads  with  hot
viscosity $\eta_h$ for a longer period  of time.  As a consequence, it
reaches  the  transition sooner  at  $t\sim  30$  and with  a  smaller
thickness $h_0\sim 5$ when it is  still in the second thermal phase of
the  bending regime.   For  an  even larger  Peclet  number $Pe$,  the
current  will transition  in the  first thermal  phase of  the bending
regime at $t \sim 10$ and with $h_0 \sim 2$, as in the isoviscous case
with viscosity $\eta_h$.

\begin{figure}
  \begin{center}
    \graphicspath{ {/Users/thorey/Documents/These/Projet/Refroidissement/Skin_Model/Figure/JFM_V13/} }
    \includegraphics[scale=0.45]{Scaling_HR_ELASGRAV_h.eps}
    \caption{Left: Dimensionless thickness at  the center $h_0$ versus
      dimensionless time  for different  sets $(\nu,Pe)$  indicated on
      the plot.  The grey line represents the isoviscous case $\nu=1$.
      Right:  Same   plot  but  for  the   dimensionless  radius  $R$.
      Horizontal  black dotted-line  represents the  transition radius
      between the bending and the gravity regime.}
    \label{Scaling_HR_ELASGRAV_h}
  \end{center}
\end{figure}

Overall, the time for the current to reach the transition $t_t$ is the
time for its radius to be larger than $4$. It can be obtained from the
scaling  law followed  by  the  radius $R(t)$  in  the bending  regime
(\ref{ScalingR-Visco})         and         is         equal         to
$6.5(\eta_e/\eta_h)^{2/7}h_f^{-1/7}$  where  $\eta_e$ is  the  effective
viscosity of the  current (see Section \ref{sec:effect-visc-blist-e}).
In  particular,  it  is  bound  by two  values  corresponding  to  two
end-member  cases:  the case  where  the  current transitions  to  the
gravity regime  while it  is in  the first  bending phase,  i.e.  when
$\eta_e= \eta_h$  and $t_h  \sim 6.5h_f^{-1/7}$ and  the case  where the
current transitions  to the gravity  regime while  it is in  the third
bending       phase,       i.e.       $\eta_e=       \eta_c$       and
$t_c \sim 6.5\nu^{-2/7}h_f^{-1/7}$.  Indeed,  when rescaling the time of
the simulation  by $t_c$,  the different  combinations ($\nu$,$Pe_m$),
for  which the  third thermal  phase of  the bending  regime has  been
reached before the  transition to the gravity regime,  collapse on the
same curve (Figure \ref{Scaling_HR_ELASGRAV_h2}, right).

\begin{table}
  \begin{center}
    \begin{tabular}{ccccc}
      Name&From&To&Expression\\
      $t_t$&Bending&Gravity&$6.5(\eta_e/\eta_h)^{2/7}h_f^{-1/7}$\\
      $t_h$&Bending&Gravity&$6.5h_f^{-1/7}$\\
      $t_c$&Bending&Gravity&$6.5\nu^{-2/7}h_f^{-1/7}$\\
      Bending regime&\multicolumn{3}{c}{} \\
      $t_{b2}$&Phase 1& Phase 2&$0.1Pe St_m^{-1} h_f^2$\\
      $t_{b3}$&Phase 2& Phase 3 &$10^{-2} St_m^{-22/27}Pe^{22/27}\nu^{-14/27}h_f^{-7/27}$\\
      Gravity regime&\multicolumn{3}{c}{} \\
      $t_{g2}$ &Phase 1& Phase 2 &$10^{-2}PeSt_m^{-1}$\\
      $t_{g3}$ &Phase 2& Phase 3 &$ 5Pe St_m^{-1}\nu^{-1/2}$\\
    \end{tabular}
    \caption{Summary of the different  transition times.  $t_t$ is the
      transition time  between bending and  gravity which is  bound by
      $t_h$, when the current transitions in the first bending thermal
      phase,  and $t_c$,  when the  current transitions  in the  third
      bending thermal phase.  $t_{b2}$ (resp. $t_{b3}$) represents the
      time to transition  from phase 1 to phase 2  (resp. from phase 2
      to phase  3) in  the bending  regime. $t_{g2}$  (resp. $t_{g3}$)
      represents  the time  to  transition  from phase  1  to phase  2
      (resp. from phase 2 to phase 3) in the gravity regime. }
    \label{tab:TimeTransition}
  \end{center}
\end{table}
The subsequent  evolution in  the gravity regime  also depends  on the
combinations ($\nu$,$Pe_m$)  considered.  Indeed,  in contrast  to the
bending regime where the effective viscosity is that of a small region
at the tip,  the effective viscosity is the average  flow viscosity in
the gravity  regime.  Therefore, the  effective viscosity of  the flow
can drastically decrease  when entering the gravity regime  and a flow
in the $i$th thermal phase of the bending regime can transition in the
$j$th thermal phase of the gravity  regime with $i\ge j$ which results
in  $6$  possible  scenarios  (see  Appendix  \ref{PhaseDi}  for  more
details).  For instance, a current in  the second thermal phase of the
bending regime can  transition into the first or  second thermal phase
of the  gravity current regime. However,  the case where a  current in
the third thermal phase of the bending regime transitions to the first
thermal  phase of  the  gravity  regime is  not  possible  and in  the
following, we  details the five  remaining scenarios as a  function of
the combination ($\nu$, $Pe_m$) considered.


\begin{figure}
  \begin{center}
    \graphicspath{ {/Users/thorey/Documents/These/Projet/Refroidissement/Skin_Model/Figure/JFM_V13/} }
    \includegraphics[scale=0.45]{Scaling_HR_ELASGRAV_h2.eps}
    \caption{Left: Dimensionless thickness at  the center $h_0$ versus
      time where  the time  has been  rescaled by  the time  $t_c$ the
      current transitions  to the  gravity regime while  it is  in the
      third bending phase (Table \ref{tab:TimeTransition}).  The grey line represents the isoviscous
      case with  given viscosity $\eta_h$.   Right: Same plot  but for
      the  dimensionless  radius  $R$.  Horizontal  black  dotted-line
      represents  the transition  radius between  the bending  and the
      gravity regime.}
    \label{Scaling_HR_ELASGRAV_h2}
  \end{center}
\end{figure}

We  first consider  the  case  where the  current  transitions to  the
gravity regime  in the first thermal  phase of the bending  regime. In
that case, the time  for the transition is $t_h$ and  is less than the
time for the  second bending thermal phase  change $t_{b2}$; comparing
$t_h$    and    $t_{b2}$    gives    $Pe>65    h_f^{-15/7}$    (Figure
\ref{Phase_Diagram_ELASGRAV},  Table   \ref{tab:TimeTransition}).   As
$t_h<t_{g2}$ for $Pe>65 h_f^{-15/7}$,  the current then transitions to
the first  thermal phase  of the gravity  current regime  ($B_1G_1$ in
Figure  \ref{Phase_Diagram_ELASGRAV} and  Figure \ref{PD_ALLpossible},
Table \ref{tab:ParameterAnalysis} in Appendix \ref{PhaseDi}).

At the opposite, if the current  has reached the third thermal bending
phase, the transition  occurs at $t_c$ and is  necessarily larger than
$t_{b3}$;      comparing      $t_c$       and      $t_{b3}$      gives
$\nu   >   8.3   \cdot   10^{-13}   Pe_m^{7/2}   h_f^{-1/2}$   (Figure
\ref{Phase_Diagram_ELASGRAV},  Table   \ref{tab:TimeTransition}).   As
$t_c>t_{g2}$ for $\nu >  8.3\cdot 10^{-13} Pe_m^{7/2} h_f^{-1/2}$, the
current can either transition to the  second or third thermal phase of
the    bending    regime     (Figure    \ref{PD_ALLpossible},    Table
\ref{tab:ParameterAnalysis}   in  Appendix   \ref{PhaseDi}).   If   it
transitions to the second phase  of the gravity regime, then comparing
$t_c$ and  $t_{g3}$ gives $\nu  < 0.3Pe_m^{14/3} h_f^{2/3}$  ($B_3G_2$ on
Figure  \ref{Phase_Diagram_ELASGRAV}) and  if  it  transitions to  the
third phase of the gravity current, then $\nu > 0.3Pe_m^{14/3} h_f^{2/3}$
($B_3G_3$ on Figure \ref{Phase_Diagram_ELASGRAV}).

In  the case  where the  transition occurs  when it  is in  the second
bending  phase, the  time for  the  transition is  not exactly  known.
However, it is bounded by $t_h$ and $t_c$ and we can therefore predict
some evolution scenarios.  Indeed,  the transition time is necessarily
smaller    than    $t_c$.     Therefore,   if    $t_c<t_{g2}$,    i.e.
$\nu>7.0\cdot 10^9Pe_m^{-7/2} h_f^{-1/2}$, the transition time is also
smaller than $t_{g2}$ and the current transitions to the first gravity
thermal  phase  ($B_2G_1$   on  Figure  \ref{Phase_Diagram_ELASGRAV}).
Similarly,  if  $t_h>t_{g2}$,  i.e.   $Pe_m<650h_f^{-1/7}$,  then  the
transition time is larger than $t_{g2}$ and the current transitions to
the    second   gravity    current   phase    ($B_2G_2$   on    Figure
\ref{Phase_Diagram_ELASGRAV}).

\begin{figure}
  \begin{center}
    \graphicspath{ {/Users/thorey/Documents/These/Projet/Refroidissement/Skin_Model/Figure/JFM_V13/} }
    \includegraphics[scale=0.45]{PhaseDiagramJFM.eps}
    \caption{Left: Phase  diagram for  the evolution with  bending and
      gravity for  different combinations  ($\nu$,$Pe_m$) and  a given
      value of $h_f  = 0.005$.  $B_iG_j$ refers to the  case where the
      current transitions from the $i$th  bending thermal phase to the
      $j$th gravity  thermal phase  where $i$  and $j  \in \{1,2,3\}$.
      Right: Application to the spreading of laccoliths on a subset of
      the  parameter  space  relevant  for the  study  of  terrestrial
      laccoliths  described  in  Section  \ref{sec:range-valu-dimens}.
      Rectangle:  subset   of  the  parameter  space   containing  the
      laccoliths from \citet{Rocchi:2002jy}}
    \label{Phase_Diagram_ELASGRAV}
  \end{center}
\end{figure}

\section{Application to the arrest of terrestrial laccoliths}
\label{sec:appl-arrest-terr}

At shallow  depth in  the upper  crust, roof  lifting is  the dominant
process  by which  magma makes  room for  itself, which  leads to  the
formation   of  laccoliths   by  bending   of  the   overlying  strata
\citep{Johnson:1973ho,Pollard:1973ho}.  The  isoviscous elastic-plated
gravity current model has been used  to study their formation and show
that their bell-shaped  morphology is consistent with  their arrest in
the  bending  regime  \citep{Michaut:2011kg,Bunger:2011cb}.   However,
their radius is too small to  be fractured controlled and their arrest
might be better explained by their cooling \citep{Michaut:2011kg}.

\subsection{Range of values for the dimensionless numbers}
\label{sec:range-valu-dimens}
 
For a  lava density  $\rho_m$ of $2500$  kg m$^{-3}$,  Young's modulus
values between $10$  and $100$ GPa and intrusion  depths between $0.5$
and $5$ km,  the characteristic length scale  $\Lambda$ varies between
$1$ and $10$  km.  The overpressure in magma  reservoirs driving dykes
is   typically    between   a   few    to   several   tens    of   MPa
\citep{Tait:1988vn,Marti:2000fe} and  the conduit length  $Z_c$ varies
from a few  to several tens of kilometers. Lava  viscosity at eruption
temperature  $\eta_h$  depends mainly  on  its  composition and  water
content; close to its liquidus  temperature, it can varies from $10^3$
to       $10^{6}$       Pa        s       for       felsic       lavas
\citep{Anonymous:CZVBrBvv,Giordano:2008em,Whittington:2009fv,Chevrel:2013jn}.
Hence, injection  rate $Q_0$ are  in a range $0.01-1$  m$^3$ s$^{-1}$,
the height  scale $H$ varies  between $0.1$ m and  $2$ m and  the time
scale $\tau$ varies from several months to a few years.

For a  latent heat of crystallization  $L = 4.18~10^5$ J  kg$^{-1}$, a
difference between solidus temperature  $T_S$ and liquidus temperature
$T_L$ between $100$ K and $300$ K, the number $St_m$ varies from $0.1$
to  $0.5$.   For  a  thermal   diffusivity  for  the  magma  equal  to
$\kappa_m=  10^{-6}$ m  s$^{-2}$, the  Peclet number  can varies  from
$10^{-4}$ to $10$ and therefore,  $Pe_m$ varies from $0.001$ to $100$.
Finally, the increase in viscosity upon cooling can varies from $4$ to
$10$                orders                 of                magnitude
\citep{Anonymous:CZVBrBvv,Lejeune:1995fc,Giordano:2008em,Diniega:2013eh}.
We  thus consider  values of  $\nu$  that vary  between $10^{-4}$  and
$10^{-10}$.

The model  also considers  a thin pre-wetted  film of  thickness $h_f$
whose  meaning in  the application  to the  spreading of  laccolith is
unclear.  In  particular, the  model shows  no convergence  when $h_f$
tends to zero \citep{Lister:2013ia} and therefore, the thickness $h_f$
might be  linked to some structural  length scale at the  front of the
laccolith or to the natural imperfection of the flow geometry.  For the
purpose  of  the  application,  we  choose a  film  thickness  of  $1$
mm, i.e. the minimum length scale with physical signification for the
spreading of laccoliths  which give a dimensionless  $h_f$ that varies
between $10^{-2}$ and $10^{-4}$.  The limited effect of changing $h_f$
is detailed in  Appendix \ref{FilmThickness} and in  the following, we
set $h_f$ to $10^{-3}$.

It is  generally assume  that magma stops  spreading when  its crystal
content becomes close  to its maximum packing, i.e.   $\phi \sim 60\%$
\citep{Pinkerton:1992fw}.   Beyond  this   point,  crystal  collisions
dominate   and   the   viscosity   jumps   to   much   higher   values
\citep{Lejeune:1995fc,Giordano:2008em}.   We   assume  that   this  is
equivalent to  $\eta_e$ tending to  $\eta_c$ in our model.   With this
assumption, the  model thus predicts  that a magmatic  intrusion would
solidify as a  laccolith upon reaching the third thermal  phase of the
bending   regime.     The   phase   diagram   proposed    in   Section
\ref{sec:diff-evol-with-1}  simplifies and  predicts that  most felsic
magmatic  intrusions should  indeed  solidify as  a laccolith  (Figure
\ref{Phase_Diagram_ELASGRAV}, right).

\subsection{Comparison with observations}
\label{sec:range-valu-dimens}

\citet{Rocchi:2002jy} provide data for the intrusion size and depth of
nine  laccoliths at  Elba  Island, Italy.   The  length and  thickness
scales can be  estimated for each laccolith such that  the data can be
nondimensionalized  and  compare to  the  model.   In particular,  the
thickness $h_0$  as a function  of its radius  $R$ for a  current that
solidifies in  the third phase  of the  bending regime can  be derived
from     the      scaling     laws      (\ref{ScalingH-Visco})     and
(\ref{ScalingR-Visco}) and should follow
\begin{equation}
  h_0 \sim 0.75\nu^{-2/7}R^{8/7}\label{Hr}
\end{equation}

For this example, each laccolith is part of a larger intrusive system,
and  hence variability  of  the model  parameters  should be  limited,
except  for the  overlying elastic  layer thickness,  taken to  be the
intrusion depth,  whose variation between laccoliths  is accounted for
in  the  nondimensionalization.  The  observations  show  a very  good
agreement with the model for a  viscosity contrast close to $8$ orders
of magnitude, which is consistent with the felsic composition of these
laccoliths            (Figure             \ref{Data},            left)
\citep{Marsh:1981dc,Diniega:2013eh}.

\begin{figure}
  \begin{center}
    \graphicspath{ {/Users/thorey/Documents/These/Projet/Refroidissement/Skin_Model/Figure/JFM_V13/} }
    \includegraphics[scale=0.45]{Dimensionlessdata.eps}
    \caption{Left: Dimensionless maximum thickness $h_0$ versus radius
      $R$ for  laccoliths from Elba Island,  Italy.  Thickness, radius
      and    depth    for    each    laccolith    are    taken    from
      \citet{Rocchi:2002jy}.   Parameters  for  calculating  $\Lambda$
      (\ref{L1}) and  $H$ (\ref{H1})  are $E=10^9$  GPa, $\nu^*=0.25$,
      $\rho_m = 2500$ kg m$^{-3}$, $g=9.81$ m s$^{-2}$, $\eta_h =10^6$
      Pa s  and $Q_0 =  0.1$ m$^3$  s$^{-1}$. For depth  of intrusions
      between  $1.9$  and $3.7$  km,  the  length scale  $\Lambda$  is
      between $2490$  m and $3680$  m.  $H$  is constant and  equal to
      $1.98$   m.   Dotted   lines:  scaling   laws  (\ref{Hr})   with
      $h_f =  0.001$ and two  values for the viscosity  contrast $\nu$
      indicated on the  plot. $\nu = 8.2\cdot  10^{-9}$ represents the
      least  square  best  fit  for the  data.   Right:  Dimensionless
      thickness  $\hat{h_0}$ versus  $\hat{R}$  where $\hat{h_0}$  and
      $\hat{R}$   are   given   by   (\ref{Scaling2}).    Substituting
      (\ref{T1})        into        (\ref{Pe}),       we        obtain
      $Pe  =  Q_0  H  /(\pi \kappa  \Lambda^2)$;  the  parameters  for
      calculating $Pe$ for each laccolith are the same than those used
      for the  nondimensionalization, $\kappa=10^{-6}$ m  s$^{-2}$ and
      $St_m$ is  considered constant  and set  to $1$.   The viscosity
      contrast is set  to $\nu =8.2\cdot 10^{-9}$  for all laccoliths.
      Dotted line: scaling law $ \hat{h_0} \sim 0.75\hat{R}^{8/7}$.}
    \label{Data}
  \end{center}
\end{figure}

If the laccoliths  have stopped spreading as soon as  they reached the
third  phase of  the bending  regime,  the variance  in thickness  and
radius between  the different intrusions  should be explained  only by
variations  in the  Peclet number,  most likely  due to  variations in
intrusion   depths    in   this    example.    Indeed,    in   Section
(\ref{sec:effect-visc-blist-g}), we  show that  the time  $t_{b3}$ the
current reaches the  third phase of the bending regime,  and hence its
thickness  and its  radius at  this time,  depends on  the combination
($\nu$,$Pe_m$)  considered.  To  test this  hypothesis, we  estimate for
each laccolith its Peclet number $Pe$ using its corresponding depth of
intrusion and use for the  viscosity contrast $\nu = 8.2\cdot 10^{-9}$
determined   previously   (Figure   \ref{Phase_Diagram_ELASGRAV}   and
\ref{Data}  right). Then,  we  rescale the  variables  using the  time
$t_{b3}$ (\ref{tb3}) as follow
\begin{equation}
  \hat{t}=     Pe_m^{-22/27}\nu^{14/27}t      \hspace{.5cm}     \hat{R}=
  Pe_m^{-7/27}\nu^{2/27}R\hspace{.5cm}                        \hat{h_0}=
  Pe_m^{-8/27}\nu^{-10/27}h_0.
  \label{Scaling2}
\end{equation}
In  term of  $\hat{h_0}$  and $\hat{R}$,  the  scaling law  (\ref{Hr})
rewrites $  \hat{h_0} \sim 0.75\hat{R}^{8/7}$  and does not  depend on
the dimensionless  numbers anymore. However, the  different laccoliths
do not  collapse on the same  dot after rescaling. In  particular, the
dependence of $Pe$  of our scaling, resulting  from different intrusion
depths,  is not  enough  to  explain the  variability  in  the size  of
terrestrial laccoliths. An additional  cooling mechanism, depending on
$Pe$,  is thus  required to  explain the  exact extent  of laccoliths,
which could be  extraction of heat by circulation of  fluid or heating
of the wall rocks during the intrusion.

\section{Summary and conclusion}
\label{sec:conclusion}

Isothermal  elastic-plated   gravity  current  shows   two  asymptotic
regimes.  At early times, the gravity is negligible and the peeling of
the  front is  driven  by  the bending  of  the  overlying layer.   In
contrast,  at late  times, the  own  flow weight  becomes the  driving
pressure  and the  current  evolves  in a  so  called gravity  current
regime.  In this  study, we have developed a theory  for the evolution
of  an elastic-plated  gravity  current with  a temperature  dependent
viscosity. In  particular, we study  the response  of the flow  to its
cooling in each regime separately.

Scaling  analyses  of  the  heat  transport  equation  show  that  the
evolution   of   the  thermal   structure   depends   on  the   regime
considered. In the bending regime, since the flow constantly thickens,
the thermal  anomaly grows with time  but slower than the  flow itself
and a region  of cold fluid rapidly  forms at the front.   The size of
the cold fluid  region depends on the dimensionless  parameters of the
system,  i.e.   the  Peclet  number,  the  viscosity  contrast  and  a
dimensionless number that accounts  for crystallization.  In contrast,
in the  gravity current  regime, since  the flow  tends to  a constant
thickness, the  temperature profile  diffuses to an  almost stationary
profile and the thermal anomaly  reaches a steady-state.  The time for
reaching this steady-state also  scales with the dimensionless numbers
of the system.

Numerical analyses  of the equations  show that the combine  effect of
cooling  and  temperature-dependent   viscosity  result  in  important
deviations from  the isoviscous case.   In particular, each  regime is
split in three different phases. A  first phase where the flow behaves
as an  isoviscous flow with hot  viscosity.  A second phase  where the
flow slows down and drastically thickens.  A last phase where the flow
returns in  an isoviscous flow  but with cold viscosity.   These three
phases are linked to the coupling  between the thermal anomaly and the
flow  itself and  in  particular,  the second  phase  of  the flow  is
triggered by the detachment of  the thermal anomaly.  However, we show
that the effective  viscosity of the flow is  drastically different in
the two regimes.  While the dynamics  is governed by the local thermal
condition  at the  front  in the  bending regime,  it  is the  average
thermal structure of the current that controls the flow in the gravity
regime.

The  final evolution  of an  elastic-plated gravity  current therefore
depends  on the  relative  phase  change within  each  regime and  the
transition  between the  bending and  the gravity  regime itself.   We
provide a general phase diagram  that predicts the different evolution
scenarios as a  function of the dimensionless  parameters.  We finally
apply the  results to  the spreading of  magmatic intrusions  and show
that, if  cooling is indeed an  efficient mechanism for the  arrest of
laccolith  in the  bending regime,  as confirmed  by observations,  an
additional cooling  mechanism is needed  to explain the exact  size of
laccoliths.


\section*{Appendix A: Numerical scheme}\label{Numeric}

The coupled nonlinear partial differential equations (\ref{C3-HF}) and
(\ref{C3-TF}) are  solved on a grid much  larger than the
flow itself and  shifted at the center to avoid  problems arising from
the  axisymmetrical  geometry.   The  procedure  used  to  solve  both
equations,   (\ref{C3-HF})   and   (\ref{C3-TF}),    is   to   use   a
finite-difference scheme  for spatial  discretization coupled  with an
implicit  backward Euler  scheme  in time.   In  addition, since  each
equation is non-linear, we use Newton-Raphson method to iterate towards
the  solution at  each time  step for  both equations.   We begin  the
computation with  $h=h_f$, $\Theta_b =1$  and $\delta =  10^{-4}$ over
the whole domain. In addition, we impose
\begin{equation}
  \left.\frac{\partial h}{\partial r}\right|_{r=0} =\left.\frac{\partial P}{\partial r}\right|_{r=0} =0
\end{equation}
and $h=h_f$ at the end of the grid.

The   expressions  of   $I_0(\delta)$,   $I_1(h)$,  $I_1(\delta)$   and
$I_2(\delta)$ are the following
\begin{eqnarray}
  I_0(\delta)  &=&-  \frac{\delta}{12}  \left(-  6 \delta  \nu  +  (1-\nu)
                   \left(- 5 \Theta_{b} \delta  + 4 \Theta_{b} h\right) +
                   6 h \nu\right)\\
  I_1(h) &=&\frac{1}{60} \left((1-\nu) \left(- 4 \Theta_{b} \delta^{3} +
             10 \Theta_{b} \delta^{2} h - 10 \Theta_{b} \delta h^{2} +
             5 \Theta_{b} h^{3}\right) + 5 h^{3} \nu\right)\\
  I_1(\delta)  &=&- \frac{\delta^{2}}{120}  \left(-  40  \delta \nu  +
                  (1-\nu) \left(-  36 \Theta_{b} \delta +  25 \Theta_{b}
                   h\right) + 30 h \nu\right)\\
I_2(\delta)&=&- \frac{\Theta_{b} \delta^{2}}{2520}  \left(- 882 \delta
               \nu  +  (1-\nu)  \left(-  778  \Theta_{b}  \delta  +  560
               \Theta_{b} h\right) + 735 h \nu\right)
\end{eqnarray}
and therefore, (\ref{ubarF}) and (\ref{SigmaF}) reduce to 
\begin{eqnarray}
\overline{u}&=&-  \frac{\delta}{10}  \left(-  20  \delta   \nu  +  (1-\nu)  \left(-  14
  \Theta_{b} \delta + 15 \Theta_{b} h\right) + 30 h \nu\right)\\
\Sigma &=&- \frac{\Theta_{b} \delta^{2}}{630} \left(- 2058 \delta \nu + (1-\nu) \left(- 1784 \Theta_{b} \delta + 1330 \Theta_{b} h\right) + 1785 h \nu\right)
\end{eqnarray}

\section*{Appendix B: Phase transitions}\label{PhaseDi}

A current in the $ith$ thermal phase can transition in the $j$th phase
of the  gravity regime. In  the following,  we show that  a transition
from $i$  to $j$ where  $i<j$ is  not possible. Indeed,  the effective
viscosity is that of  a small region at the tip  in the bending regime
whereas it is the average viscosity of the flow in the gravity regime.
Therefore,  it  cannot  increase  during the  transition  and  indeed,
$B_1G_2$,    $B_1G_3$    and    $B_2G_3$   are    unfeasible    (Table
\ref{tab:ParameterAnalysis}  and   Figure  \ref{PD_ALLpossible}).   In
addition, the transition  from the third thermal phase  of the bending
regime to the  first thermal phase of the gravity  regime implies that
$t_c<t_{b2}$   and  $t_c>t_{g3}$,   which  is   not  possible   (Table
\ref{tab:ParameterAnalysis}    and    Figure    \ref{PD_ALLpossible}).
Therefore,  the  five possible  sequences  that  remain are  $B_1G_1$,
$B_2G_1$,     $B_2G_2$,      $B_3G_2$     and      $B_3G_3$     (Table
\ref{tab:ParameterAnalysis} and Figure \ref{PD_ALLpossible}).

\begin{figure}
  \begin{center}
    \graphicspath{ {/Users/thorey/Documents/These/Projet/Refroidissement/Skin_Model/Figure/JFM_V13/} }
    \includegraphics[scale=0.7]{PhaseDiagramJFM_Appendix2.eps}
    \caption{Phase       transitions      reported       in      Table
      \ref{tab:ParameterAnalysis}}
    \label{PD_ALLpossible}
  \end{center}
\end{figure}

\begin{table}
  \begin{center}
    \begin{tabular}{cccccc}
      Transition &Condition 1& Condition 2& Condition 3&Output\\
      \multicolumn{5}{c}{Transition  in  the   first  bending  thermal
      phase $B1$} \\
      $t_t=t_h$ & $t_h<t_{b2}$ & $t_h<t_{g2}$ &- &$B_1G_1$\\
                 &$Pe_m>65h_f^{15/7}$&$Pe_m>650 h_f^{-1/7}$&- & Feasible\\
      $t_t=t_h$ & $t_h<t_{b2}$ & $t_h>t_{g2}$ &$t_h<t_{g3}$&$B_1G_2$ \\
                 &$Pe_m>65h_f^{15/7}$&$Pe_m<650 h_f^{-1/7}$&$\nu<0.6Pe_m^2h_f^{2/7}$& Unfeasible\\
      $t_t=t_h$ & $t_h<t_{b2}$ & $t_h>t_{g3}$ &-&$B_1G_3$ \\
                 &$Pe_m>65h_f^{15/7}$&$\nu>0.6Pe_m^2h_f^{2/7}$&-&
                                                               Unfeasible\\
      \multicolumn{5}{c}{Transition  in  the   second  bending  thermal
      phase $B2$} \\
      $t_h<t_t<t_c$ & $t_h>t_{b2}$ & $t_c<t_{b3}$ &$t_c<t_{g2}$& $B_2G_1$\\
                 &$Pe_m<65h_f^{-15/7}$&$\nu<\alpha
                                        Pe_m^{7/2}h_f^{-1/2}$&$\nu>\beta
                                                               Pe_m^{-7/2}h_f^{-1/2}$&Feasible\\
      $t_h<t_t<t_c$ & $t_h>t_{b2}$ &$t_c<t_{b3}$& $t_c<t_{g3}$&$B_2G_2$ or $B_2G_1$\\
                 &$Pe_m<65h_f^{-15/7}$&$\nu<\alpha                    Pe_m^{7/2}h_f^{-1/2}$
                                          &$\nu<0.3Pe_m^{14/3}h_f^{2/3}$&
                                                                       Feasible\\
      $t_h<t_t<t_c$ & $t_h>t_{b2}$ &$t_c<t_{b3}$& $t_h>t_{g2}$&$B_2G_2$\\
                 &$Pe_m<65h_f^{-15/7}$&$\nu<\alpha      Pe_m^{7/2}h_f^{-1/2}$      &$Pe_m<650
                                                                                     h_f^{-1/7}$&
                                                                                                  Feasible\\
      $t_h<t_t<t_c$ & $t_h>t_{b2}$ &$t_c<t_{b3}$& $t_h>t_{g3}$&$B_2G_3$\\
                 &$Pe_m<65h_f^{-15/7}$&$\nu<\alpha
                                        Pe_m^{7/2}h_f^{-1/2}$
                                          &$\nu>0.6Pe_m^2h_f^{2/7}$&
                                                                  Unfeasible\\
      \multicolumn{5}{c}{Transition  in  the   third  bending  thermal
      phase $B3$} \\
      $t_t=t_c$ & $t_c>t_{b3}$ & $t_c<t_{g2}$ &-&$B_3G_1$\\
                 &$\nu> \alpha Pe_m^{7/2}h_f^{-1/2}$&$\nu>\beta Pe_m^{-7/2}h_f^{-1/2}$&-&Unfeasible\\
      $t_t=t_c$ & $t_c<t_{b2}$ & $t_c>t_{g2}$ &$t_c<t_{g3}$& $B_3G_2$\\
                 &$\nu> \alpha Pe_m^{7/2}h_f^{-1/2}$&$\nu<\beta Pe_m^{-7/2}h_f^{-1/2}$&$\nu<0.3Pe_m^{14/3}h_f^{2/3}$& Feasible\\
      $t_t=t_c$ & $t_c<t_{b2}$ & $t_c>t_{g3}$ &-&$B_3G_3$ \\
                 &$\nu>\alpha Pe_m^{7/2}h_f^{-1/2}$&$\nu>0.3Pe_m^{14/3}h_f^{2/3}$&-& Feasible
    \end{tabular}
    \caption{Parameter space  analysis. The coefficients  $\alpha$ and
      $\beta$         are        $\alpha=8.3\cdot10^{-13}$         and
      $\beta=7.0 \cdot 10^{9}$}
    \label{tab:ParameterAnalysis}
  \end{center}
\end{table}

\section*{Appendix C: Effect of the pre-wetted film thickness}\label{FilmThickness}

The divergence of the viscous stresses at the contact line imposes the
need    for    a    regularization     condition    at    the    front
\citep{Lister:2013ia,Flitton:1999iv,Anonymous:QWXp_4JV}.     In   this
study, we show  the results of the simulations for  a thin pre-wettted
film of constant dimensionless thickness $h_f = 5 ~ 10^{-3}$.

\begin{figure}
  \begin{center}
    \graphicspath{ {/Users/thorey/Documents/These/Projet/Refroidissement/Skin_Model/Figure/JFM_V13/} }
    \includegraphics[scale=0.4]{Scaling_HR_ELAS_APPENDIX.eps}
    \caption{Left:    Dimensionless    thickness   at    the    center
      $h_0h_f^{2/22}$ versus dimensionless time $t$ for different sets
      $(\nu,h_f)$ indicated  on the plot.  Dashed-lines  represent the
      scaling   laws   $h_0h_f^{2/22}=   0.7\nu^{-2/11}t^{8/22}$   for
      $\nu = 1.0$ and $0.001$.  Right: Dimensionless radius $R$ versus
      dimensionless   time  $t$   for  the   same  sets   $(\nu,h_f)$.
      Dashed-lines       represent        the       scaling       laws
      $Rh_f^{-1/22}=  2.2\nu^{1/11}t^{7/22}$  for   $\nu  =  1.0$  and
      $0.001$.}
    \label{Scaling_HR_ELAS_APPENDIX}
  \end{center}
\end{figure}
\begin{figure}
  \begin{center}
    \graphicspath{ {/Users/thorey/Documents/These/Projet/Refroidissement/Skin_Model/Figure/JFM_V13/} }
    \includegraphics[scale=0.4]{R_Rc_ELAS_APPENDIX.eps}
    \caption{Left:  Extent  of  the cold  fluid  region  $R(t)-R_c(t)$
      versus    dimensionless   time    for   different    combinations
      ($\nu$,$h_f$) indicated on the plot.  Right: Same plot but where
      we  have  rescaled  the  extent  of the  cold  fluid  region  by
      $h_f^{7/66}$.          Dashed-line:          scaling         law
      $(R(t)-R_c(t))h_f^{-7/66}= 2.1 Pe^{-1/3}\nu^{7/33}t^{9/22}$.}
    \label{R_Rc_ELAS_APPENDIX}
  \end{center}
\end{figure}

However,   the    different   scaling   laws   derived    in   Section
\ref{sec:evol-bend-regime}  depends on  the film  thickness $h_f$,  as
confirmed   numerically  (Figure   \ref{Scaling_HR_ELAS_APPENDIX}  and
\ref{R_Rc_ELAS_APPENDIX})  and   in  particular,  the   phase  diagram
presented in section \ref{sec:diff-evol-with-1} and its application to
the spreading of laccolith thus depends on the chosen value for $h_f$.
The meaning of the pre-wetted film thickness in the application to the
spreading  of laccolith  is  unclear, however,  reasonable values  for
$h_f$  should range  from  a few  centimeters to  no  less than  $0.1$
milimeter $\sim 10^{-4}$. Indeed the height  scale has to be linked to
some  structural  length   scale  at  the  front   of  the  laccolith.
Therefore, as  the dependence with $h_f$  is weak, a variation  of $2$
orders of magnitude does not  change significantly the results (Figure
\ref{PhaseDiagramJFM_Appendix}).

The  same  result hold  when  we  look  at  the relation  between  the
thickness and the radius of the laccolith (\ref{Hr}). Indeed, the best
fit for  the value of  the viscosity contrast scales  as $h_f^{-1/2}$,
i.e.    $\nu_{\text{best}}=h_f^{-1/2}2.59~10^{-10}$   and   therefore,
varying  $h_f$  by  two  orders of  magnitudes  change  the  viscosity
contrast  by  one order  of  magnitude  which  is acceptable  for  our
application.

\begin{figure}
  \begin{center}
    \graphicspath{ {/Users/thorey/Documents/These/Projet/Refroidissement/Skin_Model/Figure/JFM_V13/} }
    \includegraphics[scale=0.36]{PhaseDiagramJFM_Appendix.eps}
    \caption{Phase diagrams for the evolution with bending and gravity
      for different  combinations ($\nu$,$Pe_m$) and  different values
      for  the   film  thickness   $h_f  =  10^{-2}$,   $10^{-3}$  and
      $10^{-4}$.}
    \label{PhaseDiagramJFM_Appendix}
  \end{center}
\end{figure}



\bibliographystyle{agufull08}
\bibliography{/Users/thorey/Dropbox/Library}

%%% Local Variables:
%%% mode: latex
%%% TeX-master: "../main"
%%% End:
