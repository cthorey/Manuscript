
\chapter{Toward a more realistic model- Relaxing the thermal boundary
  condition and changing the rheology}
\label{Heating}

\minitoc
The previous Chapter  was first step toward the  understanding how the
cooling of the  laccolith interacts with its  dynamics.  Hereafter, we
investigate  the  changes  triggered  by   both  the  heating  of  the
surrounding layer and a more realistic rheology.

\section{Introduction}
\label{C4-sec:introduction}
 
Contact metamorphism around sill intrusion is a common process

\citep{Everett:2008tn}

\section{Theory}
\label{C4-sec:theory-1}

We  consider   the  model  of  elastic-plated   gravity  current  with
temperature-dependent viscosity described  in Section \ref{C3-sec:theory}
where we relax the isothermal boundary condition. In the following, we
specify only the change in the  theory that comes with the new thermal
boundary condition  and refer  the reader to  Section \ref{C3-sec:theory}
for more details about the derivation.

\subsection{Thermal boundary condition}
\label{C4-sec:formulation-1}

We now  consider the heating  of the  surrounding medium by  the magma
itself at the contact with the surrounding rock, the continuity of the
temperature imposes to rewrite the vertical temperature profile as
\begin{equation}
  T=
  \begin{cases}
    T_b - (T_b-T_s)(1-\frac{z}{\delta})^2 & 0 \le z\le \delta \\
    T_b & \delta \le z\le h-\delta \\
    T_b - (T_b-T_s)(1-\frac{h-z}{\delta})^2 & h-\delta \le z\le h\\
  \end{cases}
  \label{C4-Temperature}
\end{equation}
where  $\delta(r,t)$   is  the   thermal  boundary   layer  thickness,
$T(r,z,t)$ is temperature of the  fluid, $T_b(r,t)$ is the temperature
at the center of the profile  and $T_s(r,t)$ is the temperature of the
surface,   i.e.   $T(r,z=0,t)=T(r,z=h,t)=T_s(r,t)$.  As   in   Section
\ref{C3-sec:theory},  this   profile  assures   the  continuity   of  the
temperature and heat flux within  the flow. In addition, continuity of
the heat flux across the boundaries reads
\begin{eqnarray}
  k_m\left.\frac{\partial                                    T}{\partial
  z}\right|_{z=0}&=&k_r\left.\frac{\partial              T_r}{\partial
                     z}\right|_{z=0}  \label{C4-Flux1}\\
  k_m\left.\frac{\partial                                  T}{\partial
  z}\right|_{z=h}&=&k_r\left.\frac{\partial            T_r}{\partial
                     z}\right|_{z=h}
                     \label{C4-Flux2}
\end{eqnarray}
where  $T_r(r,z)$ is  the temperature  in the  surrounding medium  and
$k_r$ its  thermal conductivity.  Assuming  a semi infinite  layer for
the rigid layer below  the intrusion, \citet{Carslaw:1959wf} show that
the temperature $T_r$ in the  surrounding rocks can be approximated to
a first order by
\begin{equation}
  T_r(r,z,t)-T_0=(T_{s}-T_0)\operatorname{erfc}{\left(\frac{-z}{2\sqrt{\kappa_r t}}\right)}.
  \label{C4-eq22}
\end{equation}
The  thickness of  the upper  layer is  equal to  the intrusion  depth
$d_c$. However,  we assume that the  depth $d_c$ is large  compared to
the characteristic  length scale for  conduction $L_c$ and we  use the
same approximation to derive $T_r$ above the intrusion
\begin{equation}
  T_r(r,z,t)-T_0=(T_{s}-T_0)\operatorname{erfc}{\left(\frac{z-h}{2\sqrt{\kappa_r t}}\right)}.
  \label{C4-eq11}
\end{equation}
Therefore, the  two thermal  boundary conditions  (\ref{C4-Flux1}) and
(\ref{C4-Flux2}) become
\begin{eqnarray}
  k_m\left.\frac{\partial                                    T}{\partial
  z}\right|_{z=0}&=& k_r
                     \frac{T_{s}-T_{0}}{\sqrt{\pi \kappa_r t}}  \label{C4-2Flux_1}\\
  k_m\left.\frac{\partial                                    T}{\partial
  z}\right|_{z=h}&=& -k_r
                     \frac{T_{s}-T_{0}}{\sqrt{\pi \kappa_r t}}.
                     \label{C4-2Flux_2}
\end{eqnarray}



\subsection{Dimensionless equations}
\label{C4-sec:dimens-equat-1}

Except  for   the  conduction   term,  which   now  account   for  the
dimensionless  surface temperature  $\Theta_s$, the  coupled equations
governing the  cooling the current  are very similar  to (\ref{C3-HF})
and (\ref{C3-TF}) and reads
\begin{eqnarray}
  \frac{\partial h}{\partial t}-\frac{12}{r}
  \frac{\partial}{\partial      r}
  \left( r I_1(h) \frac{\partial P}{\partial
  r}\right)
  \label{C4-HF}
  & =& \mathcal{H}(\frac{\gamma}{2}-r)\frac{32}{\gamma^{2}}\left(\frac{1}{4}-\frac{r^{2}}{\gamma^{2}}\right)\\
  \frac{\partial                                       \xi}{\partial
  t}+\frac{1}{r}\frac{\partial}{\partial                          r}
  \left( r\left(\bar{u}\xi-\Sigma\right)\right)&=&2Pe^{-1}St_m\frac{\Theta_b-\Theta_s}{\delta}\label{C4-TF}
\end{eqnarray}
with
\begin{eqnarray}
  \overline{\theta}&=&\frac{1}{3}\left(2\Theta_b+\Theta_s\right)\label{C4-tbar}\\
  \overline{u}&=&\frac{12}{\delta}
                  \frac{\partial
                  P}{\partial
                  r}\left(\delta
                  I_0(\delta)-I_1(\delta)\right)\\
  \Sigma &=& \frac{12}{\delta} \frac{\partial P}{\partial r}\left(I_0(\delta)\left(G(\delta)-\delta\overline{\theta}\right)+\overline{\theta}I_1(\delta)-I_2(\delta)\right).
\end{eqnarray}
where $G(z)$ denotes  a primitive of $\theta(z)$  when $z<\delta$. The
coupling between equations (\ref{C4-HF}) and (\ref{C4-TF}), i.e. the rheology is
contained in the  three integrals $I_0(z)$, $I_1(z)$  and $I_2(z)$ and
is discussed  in the  next section.   The thermal  boundary conditions
(\ref{C4-2Flux_1})  and (\ref{C4-2Flux_2})  reduce in  a dimensionless
form to
\begin{equation}
  2\frac{\Theta_b-\Theta_s}{\delta}               =               \Omega
  Pe^{1/2}\frac{\Theta_s}{\sqrt{\pi t}}.
  \label{C4-Boundary-Condi}
\end{equation}
where $\Omega$ is a new dimensionless number; it is equal to
\begin{equation}
  \Omega=\frac{k_r}{k_m}\left(\frac{\kappa_m}{\kappa_r}\right)^{1/2}\label{C4-omega}
\end{equation}
and represents the  ratio between heat conduction at  the contact with
the encasing rocks and heat diffusion within the fluid.

Finally,      using      this     thermal      boundary      condition
(\ref{C4-Boundary-Condi}), we  can show  that the  different variables
can   be   expressed   in   term  of   $\xi$   such   that   (Appendix
\ref{C4-Heat:AppendixA})
\begin{equation}
  \Theta_s(r,t)=
  \begin{cases}
    \frac{3 \beta}{4} \xi - \frac{\sqrt{3}}{4} \sqrt{\beta \xi \left(3 \beta \xi + 8\right)} + 1 & \text{if} \hspace{1cm} \xi\leq \xi_t \\
    \frac{- 12  \xi +  6 h{\left  (r,t \right  )}}{\left(\beta h{\left
            (r,t  \right  )} +  6\right)  h{\left  (r,t \right  )}}  &
    \text{if} \hspace{1cm} \xi > \xi_t
  \end{cases}
  \label{C4-TS}
\end{equation}
and
\begin{equation}
  \Theta_b(r)=
  \begin{cases}
    1 &\text{if } \hspace{1cm} \xi\leq \xi_t \\
    \frac{\Theta_{s}}{4}  \left(\beta(t)  h{\left  (r,t  \right  )}  +
      4\right) & \text{if} \hspace{1cm} \xi > \xi_t
  \end{cases}
  \label{C4-TB}
\end{equation}
\begin{equation}
  \delta(r)=
  \begin{cases}
    \frac{1}{\Theta_{s} \beta(t)} \left(- 2 \Theta_{s} + 2\right) &\text{if } \hspace{1cm} \xi\leq \xi_t \\
    h(r,t)/2 & \text{if} \hspace{1cm} \xi > \xi_t
  \end{cases}
  \label{C4-DELTA}
\end{equation}
with
\begin{eqnarray}
  \xi_t(t)&=&\frac{\beta(t) h^{2}{\left (r,t \right )}}{6 \beta(t) h{\left (r,t \right )}
              + 24}\\
  \beta(t) &=& \Omega Pe^{1/2}\frac{1}{\sqrt{\pi t}}
\end{eqnarray}

\subsection{Rheology}
\label{C4-sec:rheology}

The model derived in  Section \ref{C4-sec:dimens-equat-1} does not assume
a specific relation  between viscosity and temperature  and the choice
of the  rheology $\eta(T)$, which  appears in the  integrals $I_0(z)$,
$I_1(z)$   and   $I_2(z)$   remains   to  be   defined.    In   Section
\ref{C3-sec:theory},  we assume  a viscosity  inversely dependent  on the
temperature which reads in a dimension form
\begin{equation}
  \eta(T)=\frac{\eta_h
    \eta_c(T_i-T_0)}{\eta_h(T_i-T_0)+(\eta_c-\eta_h)(T-T_0)}.
\end{equation}
where $\eta_h$  and $\eta_c$  are the viscosities  of the  hottest and
coldest  fluid  at  the   temperature  $T_i$  and  $T_0$  respectively
\citep{Bercovici:2007vc}.   While  this   model  possesses  some  nice
simplification properties, it  restricts the change in  viscosity to a
very narrow  range of  temperature close  to $T=T_0$,  i.e. $\theta=0$
(Figure   \ref{C4-Rheology}).   In   contrast,  the   Arrhenius  model
($\eta \sim  \exp(-k/T)$), which is  a more realistic model  to relate
temperature and viscosity of lavas \citep{Blatt:2ViMWPc0}, describes a
viscosity  that increases  over  a much  larger  range of  temperature
(Figure \ref{C4-Rheology}).  To get some insights into the effect of a
more  realistic temperature-dependent  viscosity, we  thus also  use a
first-order approximation of the Arrhenius  model as a second rheology
$\eta_2(T)$ \citep{Diniega:2013eh}
\begin{eqnarray}
  \eta_2(T)                          =                          \eta_h
  \exp\left(-\log\left(\frac{\eta_h}{\eta_c}\right)\left(1-\frac{T-T_0}{T_i-T_0}\right)\right)
\end{eqnarray}
\begin{figure}[htbp]
  \begin{center}
    \graphicspath{ {/Users/thorey/Documents/These/Projet/Refroidissement/Skin_Model/Figure/Figure_Heating/} }
    \includegraphics[scale=0.8]{Rheology.eps}
    \caption{Dimensionless viscosity  versus dimensionless temperature
      for   both  rheology   $\eta_1$   (\ref{C4-rheo-1})  and   $\eta_2$
      (\ref{C4-rheo-2}).}
    \label{C4-Rheology}
  \end{center}
\end{figure}
In a dimensionless form, they read
\begin{eqnarray}
  \eta_1(\theta)/\eta_h&=&\frac{1}{\nu+(1-\nu)\theta} \label{C4-rheo-1}\\
  \eta_2(\theta)/\eta_h&=&\exp\left(-\log(\nu)\left(1-\theta\right)\right)  \label{C4-rheo-2}
\end{eqnarray}
where   $\nu$  is   the  viscosity   contrast  described   in  Section
\ref{C3-sec:theory} and  represents the  ratio between the  hot viscosity
$\eta_h$  and   the  cold  viscosity  $\eta_c$.    The  expression  of
$I_0(\delta)$, $I_1(\delta)$, $I_1(h)$ and $I_2(\delta)$, necessary to
close the model,  are given in Appendix  \ref{C4-Heat:AppendixB} for both
rheologies.

\subsection{Comparison with the isothermal model}
\label{C4-sec:some-limits}

We showed that relaxing the isothermal boundary condition introduces a
new dimensionless number $\Omega$ which  controls how much heat can be
transferred    to   the    surrounding    rocks.     In   the    limit
$\Omega \rightarrow \infty$, the model should thus reduce to the model
described     in    Section     \ref{C3-sec:theory}.     Indeed,     when
$\Omega\rightarrow \infty$, the  coefficient $\beta\rightarrow \infty$
and  then  $\xi_t\rightarrow  h/6$ (Section  \ref{C3-sec:theory}).   When
$\xi<\xi_t$,  injecting  the  corresponding expression  of  $\Theta_s$
(\ref{C4-TS})   in   the    corresponding   expression   of   $\delta$
(\ref{C4-DELTA}) gives
\begin{equation}
  \delta =\frac{3 \beta \xi +\sqrt{3} \sqrt{\beta \xi (3 \beta \xi +8)}+8}{2 \beta }
\end{equation}
which  tends   to  $3\xi$  when  $\beta   \rightarrow  \infty$.   When
$\xi>\xi_t$,  injecting  the  corresponding expression  of  $\Theta_s$
(\ref{C4-TS})   in   the   corresponding  expression   of   $\Theta_b$
(\ref{C4-TB}) gives
\begin{equation}
  \Theta_b = \frac{3 (\beta  h+4) (h-2 \xi )}{2 h (\beta  h+6)}
\end{equation}
which tends  to $3/2-3\xi/h$ when $\beta  \rightarrow \infty$ (Section
\ref{C3-sec:theory}). Finally,  taking the  limit of $\Theta_s$  for both
$\xi>\xi_t$ and $\xi<\xi_t$ show that  $\Theta_s$ indeed tends to zero
when $\Omega\rightarrow \infty$.

For magmatic  intrusion, the thermal  parameters of the magma  and the
encasing  rocks are  close and  the dimensionless  number $\Omega$  is
close to  $1$. In the following,  we study the effect  of relaxing the
isothermal   boundary  condition   on   the   dynamics  by   comparing
$\Omega=10^5$ and  $\Omega = 1$  in both regimes separately.   We also
investigate  the effect  of  a  more realistic  rheology  on the  flow
dynamics. 

\section{Evolution in the bending regime}
\label{C4-sec:evol-bend-regime-1}

We  follow the  same approach  as in  the previous  Chapter and  first
concentrate  on the  case in  which  only bending  contributes to  the
pressure.   The   governing  equations  are  thus   (\ref{C4-HF})  and
(\ref{C4-TF}) where  $P =  \nabla_r^4h$. In  the previous  Chapter, we
show that  the dynamics in the  bending regime depends on  the average
viscosity of  a small region  at the front of  the current and  can be
divided in three phases. A first phase where the current behaves as an
isoviscous flow  with hot  viscosity.  A second  phase where  the flow
slows down  and thickens.  A last  phase where the flow  returns in an
isoviscous flow but with cold viscosity.  Hereafter, we first describe
how the thermal boundary condition influences the timing for the phase
transition  by looking  at  two values  for  the dimensionless  number
$\Omega$,  i.e.   $\Omega=1$  and   $\Omega=10^5$  using  the  inverse
temperature  dependence  for  the  rheology  $\eta_1(\theta)$,  as  in
Chapter  \ref{C3-JFM}. We  thus investigate  the effect  of changing  the
rheology.

\begin{figure}[htpb]
  \begin{center}
    \graphicspath{ {/Users/thorey/Documents/These/Projet/Refroidissement/Skin_Model/Figure/Figure_Heating/} }
    \includegraphics[scale=0.55]{Grid_PeOmega_ELAS_Berco_3.eps}
    \caption{Snapshots of  the flow thermal  structure $\theta(r,z,t)$
      for  different  sets  ($Pe$,$\Omega$)  with  $Pe=  1.0$  ,$10.0$
      ,$100.0$   and   $\Omega=10^5$   and   $1.0$   at   $t=10$   for
      $\nu=0.001$. The thermal structure  in the surrounding medium is
      given  by  (\ref{C4-eq11})  and   reads  in  a  dimensionless  form
      $\Theta_r(r,z,t)=\Theta_s(r,t)\operatorname{erfc}{\left(Pe^{1/2}\frac{\kappa_m}{\kappa_r}\frac{(z-h)}{2\sqrt{t}}\right)}$
      where the ratio $\kappa_m/\kappa_r$ is set to $1$.}
    \label{C4-Grid_PeOmega_Heating}
  \end{center}
\end{figure}

\begin{figure}[htpb]
  \begin{center}
    \graphicspath{ {/Users/thorey/Documents/These/Projet/Refroidissement/Skin_Model/Figure/Figure_Heating/} }
    \includegraphics[scale=0.45]{Scaling_HR_ELAS_Omega.eps}
    \caption{Left: Dimensionless thickness at  the center $h_0$ versus
      dimensionless time  $t$ for different sets  $(\nu,Pe)$ indicated
      on      the      plot.      Dotted-lines:      scaling      laws
      $h_0=  0.7h_f^{-1/11}\nu^{-2/11}t^{8/22}$ for  $\nu  = 1.0$  and
      $0.001$.  Right:  Dimensionless radius $R$  versus dimensionless
      time  $t$  for  the  same sets  $(\nu,Pe)$.   Dotted-lines:  the
      scaling    laws    $R=   2.2h_f^{1/22}\nu^{1/11}t^{7/22}$    for
      $\nu = 1.0$ and $0.001$.}
    \label{C4-Scaling_HR_ELAS_Omega}
  \end{center}
\end{figure}
\subsection{Relaxing  the   thermal  boundary  condition,   effect  of
  $\Omega$}
\label{C4-sec:infl-therm-bound}

As for the isothermal boundary  condition, the thermal boundary layers
first connect  at the front  and a region of  cold fluid forms  at the
current tip  for $\Omega = 1$  (Figure \ref{C4-Grid_PeOmega_Heating}).
However, in  that case, the  heating of the surrounding  medium limits
heat loss in the central region of the current and the thermal anomaly
extends  further into  the flow.   For instance,  for $\nu=0.001$  and
$Pe=1.0$, while the thermal anomaly extends over $50\%$ of the current
for $\Omega = 10^5$ at $t=10$, it  extends over $75\%$ of the flow for
$\Omega=1$ (Figure \ref{C4-Grid_PeOmega_Heating}).

As for $\Omega=10^5$, the current  first behave has an isoviscous flow
with hot viscosity, it then slows  down and thickens to finally behave
again  as  an   isoviscous  flow  but  with   cold  viscosity  (Figure
\ref{C4-Scaling_HR_ELAS_Omega}).  As the current tip remains hot for a
longueur  period of  time,  the  transition to  the  second and  third
bending regime are however delayed relative to the case $\Omega= 10^5$
(Figure    \ref{C4-Scaling_HR_ELAS_Omega}).     For   instance,    for
$\nu=10^{-3}$  and  $Pe=1.0$,  while  the transitions  to  the  second
bending phase occurs at $t\sim 10^{-5}$ for $\Omega=10^{5}$, it occurs
only    after   $t\sim    10^{-4}$   for    $\Omega=   1.0$    (Figure
\ref{C4-Scaling_HR_ELAS_Omega}).

In addition, the second phase  of thickening show two different stages
for $\Omega =  1.0$ and $Pe=100.0$: a first stage  where the thickness
drastically increases and a second stage where it continues increasing
but   much  slower   (Figure  \ref{C4-Scaling_HR_ELAS_Omega}).    This
transition, enhanced  by the new thermal  boundary condition, reflects
the detachment  of the  thermal anomaly and  is discussed  in Appendix
\ref{C4-Heat:AppendixC}.

\subsection{Considering   a  more   realistic   rheology,  effect   of
  $\eta(\theta)$}
\label{C4-sec:infl-therm-bound}


The first  order Arrhenius rheology $\eta_2(\theta)$  widens the range
of  temperature  over  which significant  viscosity  variation  occurs
(Figure \ref{C4-Rheology}), i.e.  $\sim80\%$  of the temperature range
against $\sim10\%$ for $\eta_1(\theta)$.

\begin{figure}[htpb]
  \begin{center}
    \graphicspath{ {/Users/thorey/Documents/These/Projet/Refroidissement/Skin_Model/Figure/Figure_Heating/} }
    \includegraphics[scale=0.45]{Scaling_HR_ELAS_Rheology.eps}
    \caption{Left: Dimensionless thickness at  the center $h_0$ versus
      dimensionless time  $t$ for different sets  $(\nu,Pe)$ indicated
      on      the      plot.      Dotted-lines:      scaling      laws
      $h_0=  0.7h_f^{-1/11}\nu^{-2/11}t^{8/22}$  for  $\nu  =  0.001$.
      Right: Dimensionless  radius $R$  versus dimensionless  time $t$
      for the  same sets  $(\nu,Pe)$.  Dotted-lines: the  scaling laws
      $R= 2.2h_f^{1/22}\nu^{1/11}t^{7/22}$ for $\nu=0.001$.}
    \label{C4-Scaling_HR_ELAS_Rheology}
  \end{center}
\end{figure}

Therefore, the effective flow viscosity  starts to increase sooner and
the transition to  the second phase of the flow  in the bending regime
occur   sooner   than   for   the   rheology   previously   considered
$\eta_1(\theta)$    (Figure   \ref{C4-Scaling_HR_ELAS_Rheology}).     For
instance, for $\nu=10^{-3}$ and $Pe=100.0$,  while the second phase of
the  flow   starts  around  $t\sim   4~  10^{-3}$  for   the  rheology
$\eta_1(\theta)$, it  starts around  $t\sim 10^{-4}$ for  the rheology
$\eta_2(\theta)$ (Figure \ref{C4-Scaling_HR_ELAS_Rheology}).

\subsection{Characterization of the thermal anomaly}
\label{C4-sec:char-therm-anom}

As  in Chapter  \ref{C3-JFM},  we  quantify the  size  of the  thermal
anomaly  through   a  critical  thermal  radius   $R_c(t)$  where  the
temperature  at the  center of  the flow  $\Theta_b$ is  $1\%$ of  the
injection temperature,  i.e.  $\Theta_b(r=0)-\Theta_b(r=R_c)=0.99$. As
expected,  the thermal  anomaly is  larger when  relaxing the  thermal
boundary condition and changing the rheology $\eta(\theta)$ has almost
no        effect        on         its        evolution        (Figure
\ref{C4-ELAS_RRc_Rheol_Boundary}). In addition, the extent of the cold
fluid region $R(t)-R_c(t)$  is growing slightly slower  with time when
considering $\Omega=1$  in comparison to the  isothermal boundary case
$\Omega=10^5$   (Figure  \ref{C4-ELAS_RRc_Rheol_Boundary}).    In  the
following, we characterize  the thermal anomaly evolution  in the more
realistic case where $\Omega=1$ and $\eta(\theta)=\eta_2(\theta)$.

\begin{figure}[htpb]
  \begin{center}
    \graphicspath{ {/Users/thorey/Documents/These/Projet/Refroidissement/Skin_Model/Figure/Figure_Heating/} }
    \includegraphics[scale=0.4]{ELAS_RRc_Rheol_Boundary.eps}
    \caption{Left:  Extent  of  the cold  fluid  region  $R(t)-R_c(t)$
      versus   dimensionless    time   for    different   combinations
      ($\eta$,$\Omega$)   indicated  on   the  plot,   $\nu=0.01$  and
      $Pe=1.0$. Same plot but for $Pe=100.0$.}
    \label{C4-ELAS_RRc_Rheol_Boundary}
  \end{center}
\end{figure}

The size of the thermal anomaly  $R_c(t)$ is given by the radius where
advection of heat is equal to heat loss
\begin{equation}
  \frac{d}{d    t}\left(\theta(r=   R_c,t)\right)    \propto   Pe^{-1}
  \frac{\partial^2}{\partial z^2}\left(\theta(r=R_c,t)\right).
  \label{C4-HeatequationThermal}
\end{equation}
which, by integration over the thickness of the flow $h$, becomes
\begin{eqnarray}
  \frac{d}{dt}\left(\int_0^h\theta           dz\right)-\Theta_s\frac{d
  h}{dt}&\propto& Pe^{-1} \frac{\Theta_b-\Theta_s}{h}\nonumber\\
  \overline{\theta}\frac{d h}{dt}+h\frac{d \overline{\theta}}{dt}-\Theta_s\frac{d
  h}{dt}&\propto& Pe^{-1}
                  \frac{\Theta_b-\Theta_s}{h}\nonumber\\
  \frac{2}{3}\left(\Theta_b-\Theta_s\right)\frac{d  h}{d   t}&\propto& Pe^{-1}
                                                                       \frac{\Theta_b-\Theta_s}{h}\nonumber\\
  \frac{d h}{d t}&\propto& \frac{Pe^{-1}}{h}\label{C4-Calcul2}
\end{eqnarray}
where $\overline{\theta}$  is equal  to $(\int_0^h \theta  dz)/h$ here
and  we   have  assumed   that  $\overline{\theta}$  is   constant  at
$r=R_c$. However,  while relaxing the thermal  boundary condition does
not   modify   the   balance    (\ref{Calcul1}),   the   scaling   law
(\ref{ScalingRRc})   clearly   not   matches   the   prediction   when
$\Omega=1.0$  (Figure  \ref{C4-ELAS_RRc_RArrhenius_1-0}) and  the  new
thermal  anomaly evolution  must be  linked to  a change  in the  heat
advection  rate,  i.e.   the  left  hand  side  term  in  the  balance
(\ref{C4-Calcul2}).

Using the thickness profile (\ref{IntrusionShape}), the left hand side
term of (\ref{C4-Calcul2}) becomes
\begin{equation}
  \frac{d h}{d t} = \alpha^2\left(1+\frac{R_c}{R}\right)^2\frac{\partial h_0}{\partial
    t}+\frac{4h_0R_c^2}{R^3}\frac{\partial
    R}{\partial
    t}\alpha\left(1+\frac{R_c}{R}\right) 
  \label{C4-balance}
\end{equation}
where $\alpha  (t)$ is the  normalized region beyond  $r=R_c(t)$, i.e.
$\alpha(t)= \left(R(t)-R_c(t)\right)/R(t)$.  Neglecting the convective
term to keep  only the inflation term this  time in (\ref{C4-balance})
reduces the balance (\ref{C4-Calcul2}) to
\begin{equation}
  \alpha^2\left(1+\frac{R_c}{R}\right)^2\frac{\partial h_0}{\partial
    t}\propto \frac{Pe^{-1}}{\alpha^2\left(1+\frac{R_c}{R}\right)^2h_0}
\end{equation}
which, in the limit $\alpha<<1$, becomes
\begin{equation}
  \alpha^4\frac{\partial h_0}{\partial
    t} \propto \frac{Pe^{-1}}{h_0\frac{\partial h_0}{\partial t}}.
\end{equation}

Substituting the  thickness at the  center $h_0(t)$ by  its respective
scaling  law   (\ref{ScalingH-Visco}),  the   relative  size   of  the
normalized cold front region $\alpha$ reads
\begin{equation}
  \alpha(t) \propto h_f^{1/22}\nu^{1/11}Pe^{-1/4}t^{7/44}
\end{equation}
which is equivalent to
\begin{equation}
  R(t)-R_c(t) = 0.8h_f^{1/11}\nu^{2/11}Pe^{-1/4}t^{17/44}
  \label{C4-ScalingRRc-Heating}
\end{equation}
where the numerical prefactor, which  depends on the definition of the
thermal anomaly, has been chosen to fit the simulations.

\begin{figure}
  \begin{center}
    \graphicspath{ {/Users/thorey/Documents/These/Projet/Refroidissement/Skin_Model/Figure/Figure_Heating/} }
    \includegraphics[scale=0.4]{ELAS_RRc_RArrhenius_1-0.eps}
    \caption{a) Extent of the cold fluid region $R(t)-R_c(t)$ rescaled
      by $Pe^{-1/3}\nu^{7/33}$ versus  time for different combinations
      ($\nu$,$Pe$) indicated  on the  plot.  Dotted-line:  scaling law
      $(R(t)-R_c(t))Pe^{1/3}\nu^{-7/33}=  2.1  h_f^{7/66}t^{9/22}$  b)
      Same plot  but where  we rescale  the extent  of the  cold fluid
      region  by  $Pe^{-1/4}\nu^{2/11}$.    Dotted-line:  scaling  law
      $(R(t)-R_c(t))Pe^{1/4}\nu^{-2/11}= 0.7 h_f^{1/11}t^{17/44}$.}
    \label{C4-ELAS_RRc_RArrhenius_1-0}
  \end{center}
\end{figure}

The new predicted  scaling law for the evolution of  the extent of the
cold  fluid region  (\ref{C4-ScalingRRc-Heating}) shows  a good  match
with   the    simulations   (Figure   \ref{C4-ELAS_RRc_RArrhenius_1-0}
b). Therefore, the evolution of the thermal anomaly is governed by the
inflation rate at the intrusion tip when relaxing the thermal boundary
condition.   The cold  fluid  region grows  slightly  slower than  for
isothermal   boundary  condition,   from  $t^{9/22}$   to  $t^{17/44}$
($9/22\sim 0.40$ , $17/44 \sim 0.38$).  In addition, the dependence is
weaker in the Peclet number, from a power $1/3$ to $1/4$ when relaxing
the  isothermal  boundary  condition.   Indeed, for  small  $Pe$,  the
vertical  diffusion is  efficient on  the emplacement  time scale  and
rapidly heat up the surrounding medium,  the heat loss in the interior
are smaller and  the thermal anomaly larger in comparison  to the case
where  $\Omega= 10^5$.   In contrast,  for large  values of  $Pe$, the
advection dominates and  the saving of heat due to  the heating of the
medium  is less  important decreasing  the overall  difference between
small and large values of $Pe$.

\begin{figure}
  \begin{center}
    \graphicspath{ {/Users/thorey/Documents/These/Projet/Refroidissement/Skin_Model/Figure/Figure_Heating/} }
    \includegraphics[scale=0.4]{Visco_ELAS_Heating.eps}
    \caption{a)  Dimensionless effective  viscosity versus  time where
      the time has been rescaled by the time for the flow to enter the
      second phase $t_{b2}$. b) Same as  left but where we rescale the
      viscosity by $\nu$ and the time by $t_{b3}$.}
    \label{C4-Visco_ELAS_Heating}
  \end{center}
\end{figure}

The time $t_{b2}$ the intrusion enters the second phase of the flow is
left    unchanged    and    is   given    by    (\ref{tb2})    (Figure
\ref{C4-Visco_ELAS_Heating} a). In contrast, the time $t_{b3}$ for the
current to enter the third phase of  the flow, which is defined as the
time for the effective viscosity to  reach $90\%$ of its maximum value
$\eta_c$  and  depends on  the  evolution  of  the cold  fluid  region
(Section \ref{sec:effect-visc-blist-e}), is now larger and equal to
\begin{equation}
  t_{b3}=0.4h_f^{-4/17}\nu^{-8/17}Pe^{11/17}St_m^{-11/17}
  \label{C4-tb3}
\end{equation}

\subsection{Size of the thermal aureole}
\label{C4-sec:char-therm-anom}

The size  of the  thermal aureole, the  heated region  surrounding the
current,  scales as  $(\kappa_r  \tau)^{1/2}$,  i.e.  $Pe^{-1/2}$  and
hence is  much larger  for small  values of  $Pe$.  Indeed,  for large
values of $Pe$, advection dominates  on the emplacement time scale and
the thermal aureole  is restricted to a small zone  around the current
(Figure  \ref{C4-Grid_PeOmega_Heating}). For  instance, the  thickness of
the thermal aureole at the center  for $Pe=1.0$ is almost equal to the
current thickness $h_0$ whereas it is  only a few percent of $h_0$ for
$Pe=100.0$ (Figure \ref{C4-Grid_PeOmega_Heating}).

\section{Evolution in the gravity regime}
\label{C4-sec:evol-grav-regime}

As in chapter \ref{C3-JFM}, we now  consider the late time behavior in
which only the weight of the fluid contributes to the dynamic pressure
$P$. The governing equations are (\ref{C4-HF}) and (\ref{C4-TF}) where
$P=h$. We follow the same methodology than in the previous section. We
first consider the  effect of relaxing the  thermal boundary condition
and changing the rheology on  the intrusion dynamics. We then quantify
the  evolution of  the thermal  anomaly when  relaxing the  isothermal
boundary condition  and a more  realistic rheology and  finally derive
the times for the phase transition for this more realistic model.

\subsection{Relaxing  the   thermal  boundary  condition,   effect  of
  $\Omega$}
\label{C4-sec:infl-therm-bound}

As in the bending regime, for  small value of $\Omega$, the heating of
the surrounding medium  limits heat loss in the central  region of the
current and  the thermal  anomaly extends further  into the  flow. For
instance, for $Pe=1$ and $\nu=0.01$  at $t=200$, while $R_c\sim 1$ for
$\Omega=10^5$,  $R_c$ is  larger than  $5$  for $\Omega  =1 $  (Figure
\ref{C4-Grid_PeOmega_Heating_GRAV}).  In  addition, after  it detaches
from  the  current   tip,  the  thermal  anomaly  does   not  reach  a
steady-state  profile  but keeps  growing  with  time instead  (Figure
\ref{C4-Grid_TIME_GRAV}).   Indeed,  in  contrast  to  the  isothermal
boundary  case,  the  constant  increase of  the  surface  temperature
continuously  decreases the  heat loss  in the  central region  of the
current which allows  the propagation of the thermal  anomaly into the
flow.
\begin{figure}[htpb]
  \begin{center}
    \graphicspath{ {/Users/thorey/Documents/These/Projet/Refroidissement/Skin_Model/Figure/Figure_Heating/} }
    \includegraphics[scale=0.55]{Grid_PeOmega_GRAV_Berco_1.eps}
    \caption{Snapshots of  the flow thermal  structure $\theta(r,z,t)$
      for  different  sets  ($Pe$,$\Omega$)  with  $Pe=  1.0$  ,$10.0$
      ,$100.0$   and   $\Omega=10^5$   and  $1.0$   at   $t=200$   for
      $\nu=0.01$. The  thermal structure in the  surrounding medium is
      given  by  (\ref{C4-eq11}) and  reads  in  a dimensionless  form
      $\Theta_r(r,z,t)=\Theta_s(r,t)\operatorname{erfc}{\left(Pe^{1/2}\frac{\kappa_m}{\kappa_r}\frac{(z-h)}{2\sqrt{t}}\right)}$
      where the ratio $\kappa_m/\kappa_r$ is set to $1$.}
    \label{C4-Grid_PeOmega_Heating_GRAV}
  \end{center}
\end{figure}


For small values of $Pe$, the  efficient heat conduction results in an
almost    vertical    isothermal    current   at    $t=200$    (Figure
\ref{C4-Grid_PeOmega_Heating_GRAV}). In contrast,  for large values of
$Pe$, the  vertical diffusion of  heat is less efficient,  the thermal
aureole  is restricted  to a  small region  around the  intrusion, the
thermal anomaly is larger and the temperature gradient within the flow
are stronger (Figure \ref{C4-Grid_PeOmega_Heating_GRAV}).

As we have seen in  chapter \ref{C3-JFM}, the gravity current dynamics
is  governed  by the  average  viscosity,  and therefore  the  average
temperature of  the current.  In  particular, the dynamics  show three
phases: a first phase where  average temperature of the current slowly
decreases  and the  current behave  as  an isothermal  hot current,  a
second phase where  the thermal anomaly detaches from the  tip and the
current rapidly thickens and a last phase where the thermal anomaly is
small enough so that the current behaves as an isothermal cold gravity
current.

\begin{figure}[htpb]
  \begin{center}
    \graphicspath{ {/Users/thorey/Documents/These/Projet/Refroidissement/Skin_Model/Figure/Figure_Heating/} }
    \includegraphics[scale=0.45]{GridTime_GRAV_Bercovici_Pe1_Nu-2.eps}
    \caption{Snapshots of  the flow thermal  structure $\theta(r,z,t)$
      for  different  sets  ($Pe$,$\Omega$)  with  $Pe=  1.0$  ,$10.0$
      ,$100.0$   and   $\Omega=10^5$   and  $1.0$   at   $t=200$   for
      $\nu=0.01$. The  thermal structure in the  surrounding medium is
      given  by  (\ref{C4-eq11}) and  reads  in  a dimensionless  form
      $\Theta_r(r,z,t)=\Theta_s(r,t)\operatorname{erfc}{\left(Pe^{1/2}\frac{\kappa_m}{\kappa_r}\frac{(z-h)}{2\sqrt{t}}\right)}$
      where the ratio $\kappa_m/\kappa_r$ is set to $1$.}
    \label{C4-Grid_TIME_GRAV}
  \end{center}
\end{figure}

While  these  three  phases   also  characterized  the  dynamics  when
$\Omega=1.0$,  their  extent and  duration  are  modified by  the  new
thermal  boundary  condition (Figure  \ref{C4-Scaling_HR_GRAV_Omega}).
In particular, the  current remains hot for a longueur  period of time
and the detachment of the  thermal anomaly, which trigger the entrance
into the  second phase of  the flow, is  delayed in comparison  to the
case where $\Omega =10^5$.  For instance, for $\nu=0.01$ and $Pe=1.0$,
while  this  transition  occurs  around $t=1$  for  $\Omega=10^5$,  it
happens     only    after     $t=10$     for    $\Omega=1$     (Figure
\ref{C4-Scaling_HR_GRAV_Omega}).   As  the  thermal anomaly  does  not
reach a steady state for $\Omega=1$, the cooling of the current in the
second phase  is also  slower than  for $\Omega=10^5$,  the thickening
rate is  smaller and  the current  reaches the  third phase  also much
later for $\Omega=1$  (Figure \ref{C4-Scaling_HR_GRAV_Omega}).  In the
next  section, we  consider  the  effect a  the  first order  Arhenius
rheology of the dynamics for $\Omega=1.0$.

\begin{figure}[htpb]
  \begin{center}
    \graphicspath{ {/Users/thorey/Documents/These/Projet/Refroidissement/Skin_Model/Figure/Figure_Heating/} }
    \includegraphics[scale=0.45]{Scaling_HR_GRAV_Omega.eps}
    \caption{Left: Dimensionless thickness at  the center $h_0$ versus
      dimensionless time  $t$ for different sets  $(\nu,Pe)$ indicated
      on      the      plot.      Dotted-lines:      scaling      laws
      $h_0=  0.7h_f^{-1/11}\nu^{-2/11}t^{8/22}$ for  $\nu  = 1.0$  and
      $0.001$.  Right:  Dimensionless radius $R$  versus dimensionless
      time  $t$  for  the  same sets  $(\nu,Pe)$.   Dotted-lines:  the
      scaling    laws    $R=   2.2h_f^{1/22}\nu^{1/11}t^{7/22}$    for
      $\nu = 1.0$ and $0.001$.}
    \label{C4-Scaling_HR_GRAV_Omega}
  \end{center}
\end{figure}

\subsection{Considering   a  more   realistic   rheology,  effect   of
  $\eta(\theta)$}
\label{C4-sec:cons-more-real-1}
 
The  chosen rheology  $\eta(\theta)$ also  affect the  timing for  the
phase transition,  and, in particular, these  transitions occur sooner
for  the  first order  Arrhenius  rheology  $\eta_2(\theta)$ than  for
$\eta=\eta_1(\theta)$.  Indeed,  the viscosity  increases over  a wide
range of temperature and, in  particular, the transition to the second
regime  occurs before  the  detachment of  the  thermal anomaly.   The
difference in time  is about one order of magnitude  and for instance,
for $\nu=0.01$ and $Pe=1.0$, while  the transition to the second phase
of the  flow occurs at  $t=10^{-1}$ with $\eta=\eta_1$, it  happens at
$t=10^{-2}$  with $\eta=\eta_2$.   In the  following, we  quantify the
evolution of  the thermal  anomaly for a  current with  $\Omega=1$ and
$\eta(\theta)=\eta_2(\theta)$.

\begin{figure}[htpb]
  \begin{center}
    \graphicspath{ {/Users/thorey/Documents/These/Projet/Refroidissement/Skin_Model/Figure/Figure_Heating/} }
    \includegraphics[scale=0.45]{Scaling_HR_GRAV_Rheology.eps}
    \caption{Left: Dimensionless thickness at  the center $h_0$ versus
      dimensionless time  $t$ for different sets  $(\nu,Pe)$ indicated
      on      the      plot.      Dotted-lines:      scaling      laws
      $h_0=  0.7h_f^{-1/11}\nu^{-2/11}t^{8/22}$ for  $\nu  = 1.0$  and
      $0.001$.  Right:  Dimensionless radius $R$  versus dimensionless
      time  $t$  for  the  same sets  $(\nu,Pe)$.   Dotted-lines:  the
      scaling    laws    $R=   2.2h_f^{1/22}\nu^{1/11}t^{7/22}$    for
      $\nu = 1.0$ and $0.001$.}
    \label{C4-HR_GRAV_Rheology}
  \end{center}
\end{figure}

\subsection{Characterization of the thermal anomaly}
\label{C4-sec:char-therm-anom-2}

As in the bending regime, the thermal anomaly is first attached to the
tip of the current, i.e. $R_c(t)/R(t)=1$. After a time that depends on
$Pe$ as well  as $\nu$, the thermal detaches from  the tip and follows
its own  evolution. However, in  contrast to the isoviscous  case, the
thermal anomaly  does not reach  a steady  state and $R_c/R$  does not
evolve as  $t^{-1/2}$ (Figure \ref{C4-GRAV_RRc_RArrhenius_1-0}  a). As
in  chapter \ref{C3-JFM},  we  develop a  simple  thermal budget  that
account  for the  heating of  the surrounding  medium to  quantify the
evolution of the thermal anomaly.

When  the thermal  anomaly has  detached from  the intrusion  front, a
balance between heat advection and diffusion in the surrounding medium
in a dimensional form reads
\begin{equation}
  \rho C_p U_0 \frac{\Delta T}{R_c} \approx k_m \frac{\Delta T}{h_0^2}
  \label{C4-bilan}
\end{equation}
where $\Delta  T$ is the  mean temperature contrast between  the fluid
and the  surrounding and  $U_0$ is  taken as  a redistribution  of the
injection rate at $r=R_c$, i.e. $U_0=Q_0/(2\pi R_c h_0)$. In addition,
the continuity  of the  heat flux  at the  boundary (\ref{C4-2Flux_1})
imposes
\begin{equation}
  k_m\frac{\Delta   T}{h_0}\approx   k_s   \frac{\Delta   T}{\sqrt{\pi
      \kappa_r t}}.
  \label{C4-FluxEstimate}
\end{equation}
Injecting (\ref{C4-FluxEstimate}) and the  expression for the velocity
into (\ref{C4-bilan}) gives
\begin{equation}
  R_c \approx  \left(\frac{Q_0\kappa_r^{1/2}}{\kappa_m k_s}\right)^{1/2}
  t^{1/4}.
  \label{C4-Bilan2}
\end{equation}
By non-dimensionalizing (\ref{C4-Bilan2}), we  obtain the evolution of
the   thermal   anomaly   when   it  has   detached   from   the   tip
$R_c(t)\sim \Omega^{-2}Pe^{1/4}t^{1/4}$ and hence
\begin{equation}
  \frac{R_c(t)}{R(t)} = 1.8\Omega^{-2}Pe^{1/4}\nu^{-1/8}t^{-1/4}
  \label{C4-Rc}
\end{equation}
where  we  have  used   (\ref{scaling-R-gravi-2})  and  the  numerical
prefactor, which depends on the definition of the thermal anomaly, has
been  chosen to  fit the  simulations. The  scaling law  (\ref{C4-Rc})
indeed closely  fit the simulations  and both the dependence  with the
Peclet number $Pe$ and the  viscosity contrast vanishes when rescaling
by $Pe^{1/4}\nu^{-1/8}$ (Figure \ref{C4-GRAV_RRc_RArrhenius_1-0} b).

\begin{figure}
  \begin{center}
    \graphicspath{ {/Users/thorey/Documents/These/Projet/Refroidissement/Skin_Model/Figure/Figure_Heating/} }
    \includegraphics[scale=0.4]{GRAV_RRc_RArrhenius_1-0.eps}
    \caption{a) Normalized thermal anomaly radius $R_c(t)/R(t)$ versus
      time for  different combinations  ($\nu$,$Pe$) indicated  on the
      plot.  Dotted-line:  $R_c(t)/R(t)\sim t^{1/2}$ b) Same  plot but
      where   we   rescale   the   normalized   thermal   anomaly   by
      $Pe^{1/4}\nu^{-1/8}$.        Dotted-line:      scaling       law
      $(R_c(t)/R(t))Pe^{-1/4}\nu^{1/8}= 1.8t^{-1/4}$.}
    \label{C4-GRAV_RRc_RArrhenius_1-0}
  \end{center}
\end{figure}

The time  $t_{g2}$ the intrusion enters  the second phase of  the flow
still scale  as the time to  cool the intrusion by  conduction and the
prefactor  used  in  (\ref{tg2})  also  show a  good  match  with  the
simulations (Figure \ref{C4-Visco_GRAV_Heating}  a).  Indeed, although
this transition is  delayed by the heating of  the surrounding medium,
the  first  order Arrhenius  rheology  also  initiates the  transition
earlier and both effects compensates  for each other. In contrast, the
time $t_{g3}$  for the current to  enter the third phase  of the flow,
which is  defined as  the time  for the  effective viscosity  to reach
$90\%$ of its  maximum value $\eta_c$ and depends on  the evolution of
the  thermal anomaly  (Section \ref{sec:effect-visc-blist-g}),  is now
larger and equal to
\begin{equation}
  t_{g3}= 80 \Omega^{-8}\nu^{-1/2}Pe St_m^{-1}
  \label{C4-tg3}
\end{equation}

\begin{figure}
  \begin{center}
    \graphicspath{ {/Users/thorey/Documents/These/Projet/Refroidissement/Skin_Model/Figure/Figure_Heating/} }
    \includegraphics[scale=0.4]{Visco_GRAV_Heating.eps}
    \caption{a)  Dimensionless effective  viscosity versus  time where
      the time has been rescaled by the time for the flow to enter the
      second phase $t_{g2}$. b) Same as  left but where we rescale the
      viscosity by $\nu$ and the time by $t_{g3}$.}
    \label{C4-Visco_GRAV_Heating}
  \end{center}
\end{figure}

\subsection{Size of the thermal aureole}
\label{C4-sec:char-therm-anom}

The size  of the  thermal aureole, the  heated region  surrounding the
current,  scales as  $(\kappa_r  \tau)^{1/2}$,  i.e.  $Pe^{-1/2}$  and
hence is  much larger  for small  values of  $Pe$.  Indeed,  for large
values of $Pe$, advection dominates  on the emplacement time scale and
the thermal aureole  is restricted to a small zone  around the current
(Figure  \ref{C4-Grid_PeOmega_Heating}). For  instance, the  thickness of
the thermal aureole at the center  for $Pe=1.0$ is almost equal to the
current thickness $h_0$ whereas it is  only a few percent of $h_0$ for
$Pe=100.0$ (Figure \ref{C4-Grid_PeOmega_Heating}).


\section{Evolution with bending and gravity}
\label{sec:evol-with-bend}

As  discussed  in   the  previous  chapter,  for  a   current  with  a
temperature-dependent  viscosity, the  transition between  the bending
regime and  the gravity regime occurs  when the radius of  the current
reaches  $R\sim4$. The  time and  the thickness  of the  flow at  this
transition  depends on  the thermal  state of  the current  which also
depends   on  the   thermal  boundary   condition  and   the  rheology
considered. For the more realistic model discussed previously, we show
that, overall, the  current tends to remain hot for  a longueur period
of time delaying the phase transitions in each regime.


\begin{table}
  \begin{center}
    \begin{tabular}{ccccc}
      Name&From&To&Expression\\
      $t_t$&Bending&Gravity&$6.5(\eta_e/\eta_h)^{2/7}h_f^{-1/7}$\\
      $t_t^h$&Bending&Gravity&$6.5h_f^{-1/7}$\\
      $t_t^c$&Bending&Gravity&$6.5\nu^{-2/7}h_f^{-1/7}$\\
      Bending regime&\multicolumn{3}{c}{} \\
      $t_{b2}$&Phase 1& Phase 2&$0.1 Pe St_m^{-1} h_f^2$\\
      $t_{b3}$&Phase 2& Phase 3 &$0.4 h_f^{-4/17} St_m^{-11/17}Pe^{11/17}\nu^{-8/17}$\\
      Gravity regime&\multicolumn{3}{c}{} \\
      $t_{g2}$ &Phase 1& Phase 2 &$10^{-2}PeSt_m^{-1}$\\
      $t_{g3}$ &Phase 2& Phase 3 &$ 80Pe St_m^{-1}\nu^{-1/2}$\\
    \end{tabular}
    \caption{Summary of the different  transition times.  $t_t$ is the
      transition time  between bending and  gravity which is  bound by
      $t_t^h$,  when  the current  transitions  in  the first  bending
      thermal phase, and $t_t^c$, when  the current transitions in the
      third  bending   thermal  phase.   $t_{b2}$   (resp.   $t_{b3}$)
      represents  the time  to  transition  from phase  1  to phase  2
      (resp. from phase 2 to phase  3) in the bending regime. $t_{g2}$
      (resp. $t_{g3}$) represents the time  to transition from phase 1
      to  phase 2  (resp. from  phase  2 to  phase 3)  in the  gravity
      regime. }
    \label{tab:TimeTransition}
  \end{center}
\end{table}


\newpage
\section*{Appendix A: Main variables}
 \label{C4-Heat:AppendixA}

The variable $\xi$ is the sufficient variable to solve for in the heat
transport equation (\ref{C4-TF}). Indeed,
\begin{equation}
  \xi&=&\frac{\delta}{3} \left(- 2 \Theta_{b} - \Theta_{s} + 3\right)\label{C4-xi}
\end{equation}
where   we    have   used   (\ref{C4-tbar}).    In    addition,   from
(\ref{C4-Boundary-Condi}), we can rewrite
\begin{eqnarray}
  \Theta_s &=& \frac{2 \Theta_{b}}{\beta \delta + 2}\label{C4-Ts},\\
  \delta  &=&   \frac{1}{\Theta_{s}  \beta}   \left(2  \Theta_{b}   -  2
              \Theta_{s}\right)\label{C4-D},\\
  \Theta_b &=& \frac{\Theta_{s}}{2} \left(\beta \delta + 2\right)\label{C4-Tb}
\end{eqnarray}
When  the  thermal  boundary  layer just  merged,  then  $\Theta_b=1$,
$\delta = h/2$ and injecting (\ref{C4-Ts}) into (\ref{C4-xi}) gives
\begin{equation}
  \xi_t(t)=\frac{\beta(t) h^{2}{\left (r,t \right )}}{6 \beta(t) h{\left (r,t \right )}
    + 24}\label{C4-xit}
\end{equation}
Therefore,  when  $\xi<\xi_t$,  the  thermal boundary  layer  are  not
merged, $\Theta_b=1$ and injecting (\ref{C4-D}) into (\ref{C4-xi}) and
solving for $\Theta_s$ gives
\begin{equation}
  \Theta_s = \frac{3 \beta}{4} \xi - \frac{\sqrt{3}}{4} \sqrt{\beta \xi \left(3 \beta \xi + 8\right)} + 1.
\end{equation}
In contrast, when $\xi>\xi_t$, the  thermal boundary layer are merged,
$\delta=h/2$  and  injecting   (\ref{C4-Tb})  into  (\ref{C4-xi})  and
solving for $\Theta_s$ gives
\begin{equation}
  \Theta_s = \frac{- 12 \xi + 6 h}{\left(\beta h + 6\right) h}.
\end{equation}

\section*{Appendix B: Integral expressions}
\label{C4-Heat:AppendixB}

The  model  developed in  Section  \ref{C4-sec:theory-1}  depends on  the
integrals
\begin{eqnarray}
  I_0(z)&=&\int_0^z\frac{1}{\eta(y)}\left(y-\frac{h}{2}\right)
            dy \\
  I_1(z) &=& \int_0^z\frac{1}{\eta(y)}\left(y-\frac{h}{2}\right)y dy\\
  I_2(z)&=&\int_0^y                         \frac{1}{\eta(y)}
            \left(y-\frac{h}{2}\right)G(y)dy
\end{eqnarray}
where $G(z)$  is a  primitive of $\theta(z)$  where $z<\delta$  and is
given by
\begin{equation}
  G(z) = \frac{z \left(3 \delta ^2 \Theta_s+3 \delta z (\Theta_b-\Theta_s)+z^2 (\Theta_s-\Theta_b)\right)}{3 \delta ^2}.
\end{equation}
In  particular, the  model requires  the expression  of $I_0(\delta)$,
$I_1(\delta)$, $I_1(h)$ and $I_2(\delta)$. 

\vspace{.5cm} \textbf{Rheology 1: $\eta(\theta)=\eta_1(\theta)$} \vspace{.5cm}

In that case, the four integrals can be easily derived and read
\begin{eqnarray}
I_0(\delta)&=&\frac{\delta}{12} \left(6 \delta \nu + (1-\nu) \left(- \alpha_1 \delta + 2 \alpha_1 h + 6 \Theta_{b} \delta - 6 \Theta_{b} h\right) - 6 h \nu\right)\nonumber\\
I_1(\delta)&=&\frac{\delta^{2}}{120} \left(40 \delta \nu + (1-\nu) \left(- 4 \alpha_1 \delta + 5 \alpha_1 h + 40 \Theta_{b} \delta - 30 \Theta_{b} h\right) - 30 h \nu\right)\nonumber\\
I_1(h)&=&\frac{1}{60} \left((1-\nu) \left(- 4 \alpha_1 \delta^{3} + 10 \alpha_1 \delta^{2} h - 10 \alpha_1 \delta h^{2} + 5 \Theta_{b} h^{3}\right) + 5 h^{3} \nu\right)\nonumber\\
I_2(\delta)&=&- \frac{\delta^{2}}{2520} \left(378  \alpha_1 \delta \nu -
               315  \alpha_1 h  \nu -  840 \Theta_{b}  \delta \nu  + 630
               \Theta_{b} h \nu \right)\nonumber\\
&&-\frac{\delta^{2}}{2520}(1-\nu)  \left(- 50  \alpha_1^{2} \delta  + 70
    \alpha_1^{2}  h  +  462  \alpha_1   \Theta_{b}  \delta  -  420  \alpha_1
    \Theta_{b}  h -  840  \Theta_{b}^{2} \delta  + 630  \Theta_{b}^{2}
    h\right)\nonumber
\end{eqnarray}
where $\alpha=\Theta_b-\Theta_s$ has been introduced for clarity.

\vspace{.5cm}   \textbf{Rheology   2:   $\eta(\theta)=\eta_2(\theta)$}
\vspace{.5cm}

For cases where $\nu<1$, we have
\begin{eqnarray}
I_0(\delta)&=&-\frac{\delta  \nu ^{1-\Theta_b} \left(\sqrt{\pi } \sqrt{\alpha_1} (2 \delta -h) \sqrt{-\alpha_2}
   \text{erf}\left(\sqrt{\alpha_1} \sqrt{-\alpha_2}\right)+2 \delta  \left(\nu ^{\alpha_1}-1\right)\right)}{4 \alpha_1 \alpha_2}\nonumber\\
I_1(\delta)&=&\frac{\delta ^2 \nu ^{1-\Theta_b} \left(\sqrt{\pi } \erf\left(\sqrt{\alpha_1} \sqrt{-\alpha_2}\right) (\alpha_1 (h-2
   \delta ) \alpha_2+\delta )\right)}{4           \alpha_1^{3/2}          (-\alpha_2)^{3/2}}\nonumber\\
&&+\frac{\delta ^2 \nu ^{1-\Theta_b} \left(\sqrt{\alpha_1} \sqrt{-\alpha_2} \left(2 \delta  \left(\nu ^{\alpha_1}-2\right)-h \nu
   ^{\alpha_1}+h\right)\right)}{4           \alpha_1^{3/2}          (-\alpha_2)^{3/2}}\nonumber\\
I_1(h)&=&\frac{\nu ^{1-\Theta_b} \left(\sqrt{\alpha_1} \sqrt{-\alpha_2} \left(12 \delta ^2 \left(\delta  \left(\nu
   ^{\alpha_1}-2\right)-h \nu ^{\alpha_1}+h\right)+\alpha_1  (2 \delta -h)^3
          \log    (\nu    )\right)\rihgt)}{12   \alpha_1^{3/2}    (-\alpha_2)^{3/2}}\nonumber\\
&&-\frac{\nu ^{1-\Theta_b} \left(3 \sqrt{\pi } \delta 
   \text{erf}\left(\sqrt{\alpha_1} \sqrt{-\alpha_2}\right) \left(\alpha_1 (h-2 \delta )^2 \alpha_2-2 \delta
   ^2\right)\right)}{12 \alpha_1^{3/2} (-\alpha_2)^{3/2}}\nonumber
\end{eqnarray}
\begin{eqnarray}
I_2(\delta)&=&\frac{\delta ^2 \nu ^{1-\Theta_b} \left(\sqrt{\pi } \text{erf}\left(\sqrt{\alpha_1} \sqrt{-\alpha_2}\right) \left(-2
   \alpha_1 (2  \delta -h)  (\alpha_1-3 \Theta_b)  \alpha_2^2-6 \delta
               \Theta_b  \alpha_2-3   \delta  \right)\right)}{24
               \alpha_1^{3/2} (-\alpha_2)^{5/2}}\nonumber\\
&&+\frac{\delta ^2 \nu ^{1-\Theta_b} \left(2
   \sqrt{\alpha_1} \nu ^{\alpha_1} \sqrt{-\alpha_2} \left(\nu ^{-\alpha_1} (\alpha_2 (-2 \delta  (\alpha_1-6
   \Theta_b)-3 h \Theta_b)+2 \delta -h)\right)\right)}{24
               \alpha_1^{3/2} (-\alpha_2)^{5/2}}\nonumber\\
&&+\frac{\delta ^2 \nu ^{1-\Theta_b} \left(2
   \sqrt{\alpha_1} \nu ^{\alpha_1} \sqrt{-\alpha_2} \left(2 \delta  \alpha_1 \alpha_2-6 \delta  \Theta_b \alpha_2+\delta
   -\alpha_1 h \alpha_2+3 h \Theta_b \alpha_2+h\right)\right)}{24 \alpha_1^{3/2} (-\alpha_2)^{5/2}}\nonumber
\end{eqnarray}
where    in   addition    to    $\alpha_1$,    we   also    introduced
$\alpha_2=log(\nu)$  for  clarity.  In  the  case  where $\nu=1$,  the
expression above simplify and read
\begin{eqnarray}
I_0(\delta)&=&\frac{1}{2} \delta  (\delta -h)\nonumber\\
I_1(\delta)&=&\frac{1}{12} \delta ^2 (4 \delta -3 h)\nonumber\\
I_1(h)&=&\frac{h^3}{12}\nonumber\\
I_2(\delta)&=&-\frac{1}{120} \delta ^2 (18 \delta  \alpha_1-40 \delta  \Theta_b-15 \alpha_1 h+30 h \Theta_b)\nonumber
\end{eqnarray}

\section*{Appendix C: Second phase of the bending regime}
\label{C4-Heat:AppendixC}

For some simulations, the second  phase of important thickening in the
bending regime occurs in two stages: a first stage where the thickness
drastically increases and a second stage where it continues increasing
but   much    slower   (Figure    \ref{C4-Scaling_HR_ELAS_Omega}   and
\ref{C4-Scaling_HR_ELAS_Rheology}).   To get  some insights  into this
transition, we  run some  simulations for  $\Omega=1.0$ with  a higher
spatial  resolution, i.e.   $Dr=0.005$  instead  of $Dr=0.01$  (Figure
\ref{C4-Appendix_Phase2_N11_0_Pe100_0_nu0_001}).

The  simulations   show  that  this  transition   corresponds  to  the
detachment       of       the      thermal       anomaly       (Figure
\ref{C4-Appendix_Phase2_N11_0_Pe100_0_nu0_001}).     In    particular,
during the first  stage, the thermal anomaly is still  attached to the
tip  and  the  prewetting  film, located  beyond  $r=R(t)$,  is  still
cooling.  In contrast, during the second stage, which is characterized
by  a decrease  in the  thickening rate,  the prewetting  film located
beyond $r=R(t)$  is entirely cold  , i.e.  $\overline{\theta}  =0$ for
$r>R(t)$ and the thermal anomaly slowly gets away from the tip (Figure
\ref{C4-Appendix_Phase2_N11_0_Pe100_0_nu0_001}).   For  instance,  for
$\eta_1(\theta)$, $\nu=0.001$ and  $Pe=100.0$, the transitions between
the two  stages occurs at  $t=1.8~10^{-2}$ and indeed coincide  to the
film          becoming          entirely         cold          (Figure
\ref{C4-Appendix_Phase2_N11_0_Pe100_0_nu0_001}  a,  b,  c).   For  the
rheology  $\eta_2(\theta)$, the  transition  is  smoother because  the
viscosity   increases  on   a  wide   range  of   temperature  (Figure
\ref{C4-Appendix_Phase2_N11_0_Pe100_0_nu0_001} d, e,  f). Even if this
transition  should be  present for  all the  simulations, the  smaller
spatial resolution  used in this  chapter and chapter  \ref{C3-JFM} do
not allow to  resolve this transition for all the  combinations of the
dimensionless numbers.


\begin{figure}[htpb]
  \begin{center}
    \graphicspath{ {/Users/thorey/Documents/These/Projet/Refroidissement/Skin_Model/Figure/Figure_Heating/} }
    \includegraphics[scale=0.42]{Appendix_Phase2_N11_0_Pe100_0_nu0_001.eps}
    \caption{a)  Dimensionless  thickness $h_0$  versus  dimensionless
      time  $t$  for  $Pe=100.0$, $\nu=0.001$,  $\Omega=1.0$  and  the
      rheology  $\eta_1(\theta)$.   Colors  refer  to  the  time  $t$.
      Dotted              line:               Scaling              law
      $h_0= 0.7h_f^{-1/11}\nu^{-2/11}t^{8/22}$. Vertical dashed-lines:
      initial,  intermediate  and  final   times  of  the  temperature
      profiles  plotted in  c).   b) Dimensionless  radius $R$  versus
      dimensionless  time  $t$  for $Pe=100.0$,  $\nu=0.001$  and  the
      rheology  $\eta_1(\theta)$.   Colors  refer  to  the  time  $t$.
      Dotted line:  Scaling law  $R= 2.2h_f^{1/22}\nu^{1/11}t^{7/22}$.
      Vertical dashed-lines: same than in a). c) Dimensionless average
      temperature over  the flow thickness  $\overline{\theta}$ versus
      radial   axis  $r$   for  times   between  $t=3.8~10^{-3}$   and
      $t=7.6~10^{-2}$.   Dashed-line profiles:  profiles at  the three
      different times underlined in a) and b). Colors also refer to
      the time on the same scale than  a) and b).  d), e) and f), same
      plots  than  a),  b)  and  c) but  for  the  Arrhenius  rheology
      $\eta_2$.}
    \label{C4-Appendix_Phase2_N11_0_Pe100_0_nu0_001}
  \end{center}
\end{figure}

\newpage
\bibliographystyle{agufull08}
\bibliography{/Users/thorey/Dropbox/Library}



%%% Local Variables:
%%% mode: latex
%%% TeX-master: "../main"
%%% End:
