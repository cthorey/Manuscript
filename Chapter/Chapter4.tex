
\chapter{Toward a more realistic model- Relaxing the thermal boundary
  condition and changing the rheology}
\label{Heating}

\minitoc
The previous Chapter  was first step toward the  understanding how the
cooling of the  laccolith interacts with its  dynamics.  Hereafter, we
investigate  the  changes  triggered  by   both  the  heating  of  the
surrounding layer and a more realistic rheology.

\section{Introduction}
\label{sec:introduction}
 
Contact metamorphism around sill intrusion is a common process

\citep{Everett:2008tn}

\section{Theory}
\label{sec:theory-1}

We  consider   the  model  of  elastic-plated   gravity  current  with
temperature-dependent viscosity described  in Section \ref{sec:theory}
where we relax the isothermal boundary condition. In the following, we
specify only the change in the  theory that comes with the new thermal
boundary condition  and refer  the reader to  Section \ref{sec:theory}
for more details about the derivation.

\subsection{Thermal boundary condition}
\label{sec:formulation-1}

We now  consider the heating  of the  surrounding medium by  the magma
itself at the contact with the surrounding rock, the continuity of the
temperature imposes to rewrite the vertical temperature profile as
\begin{equation}
  T=
  \begin{cases}
    T_b - (T_b-T_s)(1-\frac{z}{\delta})^2 & 0 \le z\le \delta \\
    T_b & \delta \le z\le h-\delta \\
    T_b - (T_b-T_s)(1-\frac{h-z}{\delta})^2 & h-\delta \le z\le h\\
  \end{cases}
  \label{C4-Temperature}
\end{equation}
where  $\delta(r,t)$   is  the   thermal  boundary   layer  thickness,
$T(r,z,t)$ is temperature of the  fluid, $T_b(r,t)$ is the temperature
at the center of the profile  and $T_s(r,t)$ is the temperature of the
surface,   i.e.   $T(r,z=0,t)=T(r,z=h,t)=T_s(r,t)$.  As   in   Section
\ref{sec:theory},  this   profile  assures   the  continuity   of  the
temperature and heat flux within  the flow. In addition, continuity of
the heat flux across the boundaries reads
\begin{eqnarray}
  k_m\left.\frac{\partial                                    T}{\partial
  z}\right|_{z=0}&=&k_r\left.\frac{\partial              T_r}{\partial
                     z}\right|_{z=0}  \label{C4-Flux1}\\
  k_m\left.\frac{\partial                                  T}{\partial
  z}\right|_{z=h}&=&k_r\left.\frac{\partial            T_r}{\partial
                     z}\right|_{z=h}
                     \label{C4-Flux2}
\end{eqnarray}
where  $T_r(r,z)$ is  the temperature  in the  surrounding medium  and
$k_r$ its  thermal conductivity.  Assuming  a semi infinite  layer for
the rigid layer below  the intrusion, \citet{Carslaw:1959wf} show that
the temperature $T_r$ in the  surrounding rocks can be approximated to
a first order by
\begin{equation}
  T_r(r,z,t)-T_0=(T_{s}-T_0)\operatorname{erfc}{\left(\frac{-z}{2\sqrt{\kappa_r t}}\right)}.
  \label{eq22}
\end{equation}
The  thickness of  the upper  layer is  equal to  the intrusion  depth
$d_c$. However,  we assume that the  depth $d_c$ is large  compared to
the characteristic  length scale for  conduction $L_c$ and we  use the
same approximation to derive $T_r$ above the intrusion
\begin{equation}
  T_r(r,z,t)-T_0=(T_{s}-T_0)\operatorname{erfc}{\left(\frac{z-h}{2\sqrt{\kappa_r t}}\right)}.
  \label{eq11}
\end{equation}
Therefore, the  two thermal  boundary conditions  (\ref{C4-Flux1}) and
(\ref{C4-Flux2}) become
\begin{eqnarray}
  k_m\left.\frac{\partial                                    T}{\partial
  z}\right|_{z=0}&=& k_r
                     \frac{T_{s}-T_{0}}{\sqrt{\pi \kappa_r t}}  \label{C4-2Flux_1}\\
  k_m\left.\frac{\partial                                    T}{\partial
  z}\right|_{z=h}&=& -k_r
                     \frac{T_{s}-T_{0}}{\sqrt{\pi \kappa_r t}}.
                     \label{C4-2Flux_2}
\end{eqnarray}



\subsection{Dimensionless equations}
\label{sec:dimens-equat-1}

Except  for   the  conduction   term,  which   now  account   for  the
dimensionless  surface temperature  $\Theta_s$, the  coupled equations
governing the  cooling the current  are very similar  to (\ref{C3-HF})
and (\ref{C3-TF}) and reads
\begin{eqnarray}
  \frac{\partial h}{\partial t}-\frac{12}{r}
  \frac{\partial}{\partial      r}
  \left( r I_1(h) \frac{\partial P}{\partial
  r}\right)
  \label{C4-HF}
  & =& \mathcal{H}(\frac{\gamma}{2}-r)\frac{32}{\gamma^{2}}\left(\frac{1}{4}-\frac{r^{2}}{\gamma^{2}}\right)\\
  \frac{\partial                                       \xi}{\partial
  t}+\frac{1}{r}\frac{\partial}{\partial                          r}
  \left( r\left(\bar{u}\xi-\Sigma\right)\right)&=&2Pe^{-1}St_m\frac{\Theta_b-\Theta_s}{\delta}\label{C4-TF}
\end{eqnarray}
with
\begin{eqnarray}
  \overline{\theta}&=&\frac{1}{3}\left(2\Theta_b+\Theta_s\right)\label{C4-tbar}\\
  \overline{u}&=&\frac{12}{\delta}
                  \frac{\partial
                  P}{\partial
                  r}\left(\delta
                  I_0(\delta)-I_1(\delta)\right)\\
  \Sigma &=& \frac{12}{\delta} \frac{\partial P}{\partial r}\left(I_0(\delta)\left(G(\delta)-\delta\overline{\theta}\right)+\overline{\theta}I_1(\delta)-I_2(\delta)\right).
\end{eqnarray}
where $G(z)$ denotes  a primitive of $\theta(z)$  when $z<\delta$. The
coupling between equations (\ref{C4-HF}) and (\ref{C4-TF}), i.e. the rheology is
contained in the  three integrals $I_0(z)$, $I_1(z)$  and $I_2(z)$ and
is discussed  in the  next section.   The thermal  boundary conditions
(\ref{C4-2Flux_1})  and (\ref{C4-2Flux_2})  reduce in  a dimensionless
form to
\begin{equation}
  2\frac{\Theta_b-\Theta_s}{\delta}               =               \Omega
  Pe^{1/2}\frac{\Theta_s}{\sqrt{\pi t}}.
  \label{C4-Boundary-Condi}
\end{equation}
where $\Omega$ is a new dimensionless number; it is equal to
\begin{equation}
  \Omega=\frac{k_r}{k_m}\left(\frac{\kappa_m}{\kappa_r}\right)^{1/2}\label{omega}
\end{equation}
and represents the  ratio between heat conduction at  the contact with
the encasing rocks and heat diffusion within the fluid.

Finally,      using      this     thermal      boundary      condition
(\ref{C4-Boundary-Condi}), we  can show  that the  different variables
can   be   expressed   in   term  of   $\xi$   such   that   (Appendix
\ref{Heat:AppendixA})
\begin{equation}
  \Theta_s(r,t)=
  \begin{cases}
    \frac{3 \beta}{4} \xi - \frac{\sqrt{3}}{4} \sqrt{\beta \xi \left(3 \beta \xi + 8\right)} + 1 & \text{if} \hspace{1cm} \xi\leq \xi_t \\
    \frac{- 12  \xi +  6 h{\left  (r,t \right  )}}{\left(\beta h{\left
            (r,t  \right  )} +  6\right)  h{\left  (r,t \right  )}}  &
    \text{if} \hspace{1cm} \xi > \xi_t
  \end{cases}
  \label{C4-TS}
\end{equation}
and
\begin{equation}
  \Theta_b(r)=
  \begin{cases}
    1 &\text{if } \hspace{1cm} \xi\leq \xi_t \\
    \frac{\Theta_{s}}{4}  \left(\beta(t)  h{\left  (r,t  \right  )}  +
      4\right) & \text{if} \hspace{1cm} \xi > \xi_t
  \end{cases}
  \label{C4-TB}
\end{equation}
\begin{equation}
  \delta(r)=
  \begin{cases}
    \frac{1}{\Theta_{s} \beta(t)} \left(- 2 \Theta_{s} + 2\right) &\text{if } \hspace{1cm} \xi\leq \xi_t \\
    h(r,t)/2 & \text{if} \hspace{1cm} \xi > \xi_t
  \end{cases}
  \label{C4-DELTA}
\end{equation}
with
\begin{eqnarray}
  \xi_t(t)&=&\frac{\beta(t) h^{2}{\left (r,t \right )}}{6 \beta(t) h{\left (r,t \right )}
              + 24}\\
  \beta(t) &=& \Omega Pe^{1/2}\frac{1}{\sqrt{\pi t}}
\end{eqnarray}

\subsection{Rheology}
\label{sec:rheology}

The model derived in  Section \ref{sec:dimens-equat-1} does not assume
a specific relation  between viscosity and temperature  and the choice
of the  rheology $\eta(T)$, which  appears in the  integrals $I_0(z)$,
$I_1(z)$   and   $I_2(z)$   remains   to  be   defined.    In   Section
\ref{sec:theory},  we assume  a viscosity  inversely dependent  on the
temperature which reads in a dimension form
\begin{equation}
  \eta(T)=\frac{\eta_h
    \eta_c(T_i-T_0)}{\eta_h(T_i-T_0)+(\eta_c-\eta_h)(T-T_0)}.
\end{equation}
where $\eta_h$  and $\eta_c$  are the viscosities  of the  hottest and
coldest  fluid  at  the   temperature  $T_i$  and  $T_0$  respectively
\citep{Bercovici:2007vc}.   While  this   model  possesses  some  nice
simplification properties, it  restricts the change in  viscosity to a
very narrow  range of  temperature close  to $T=T_0$,  i.e. $\theta=0$
(Figure   \ref{C4-Rheology}).   In   contrast,  the   Arrhenius  model
($\eta \sim  \exp(-k/T)$), which is  a more realistic model  to relate
temperature and viscosity of lavas \citep{Blatt:2ViMWPc0}, describes a
viscosity  that increases  over  a much  larger  range of  temperature
(Figure \ref{C4-Rheology}).  To get some insights into the effect of a
more  realistic temperature-dependent  viscosity, we  thus also  use a
first-order approximation of the Arrhenius  model as a second rheology
$\eta_2(T)$ \citep{Diniega:2013eh}
\begin{eqnarray}
  \eta_2(T)                          =                          \eta_h
  \exp\left(-\log\left(\frac{\eta_h}{\eta_c}\right)\left(1-\frac{T-T_0}{T_i-T_0}\right)\right)
\end{eqnarray}
\begin{figure}[htbp]
  \begin{center}
    \graphicspath{ {/Users/thorey/Documents/These/Projet/Refroidissement/Skin_Model/Figure/Figure_Heating/} }
    \includegraphics[scale=0.8]{Rheology.eps}
    \caption{Dimensionless viscosity  versus dimensionless temperature
      for   both  rheology   $\eta_1$   (\ref{rheo-1})  and   $\eta_2$
      (\ref{rheo-2}).}
    \label{C4-Rheology}
  \end{center}
\end{figure}
In a dimensionless form, they read
\begin{eqnarray}
  \eta_1(\theta)/\eta_h&=&\frac{1}{\nu+(1-\nu)\theta} \label{rheo-1}\\
  \eta_2(\theta)/\eta_h&=&\exp\left(-\log(\nu)\left(1-\theta\right)\right)  \label{rheo-2}
\end{eqnarray}
where   $\nu$  is   the  viscosity   contrast  described   in  Section
\ref{sec:theory} and  represents the  ratio between the  hot viscosity
$\eta_h$  and   the  cold  viscosity  $\eta_c$.    The  expression  of
$I_0(\delta)$, $I_1(\delta)$, $I_1(h)$ and $I_2(\delta)$, necessary to
close the model,  are given in Appendix  \ref{Heat:AppendixB} for both
rheologies.

\subsection{Comparison with the isothermal model}
\label{sec:some-limits}

We showed that relaxing the isothermal boundary condition introduces a
new dimensionless number $\Omega$ which  controls how much heat can be
transferred    to   the    surrounding    rocks.     In   the    limit
$\Omega \rightarrow \infty$, the model should thus reduce to the model
described     in    Section     \ref{sec:theory}.     Indeed,     when
$\Omega\rightarrow \infty$, the  coefficient $\beta\rightarrow \infty$
and  then  $\xi_t\rightarrow  h/6$ (Section  \ref{sec:theory}).   When
$\xi<\xi_t$,  injecting  the  corresponding expression  of  $\Theta_s$
(\ref{C4-TS})   in   the    corresponding   expression   of   $\delta$
(\ref{C4-DELTA}) gives
\begin{equation}
  \delta =\frac{3 \beta \xi +\sqrt{3} \sqrt{\beta \xi (3 \beta \xi +8)}+8}{2 \beta }
\end{equation}
which  tends   to  $3\xi$  when  $\beta   \rightarrow  \infty$.   When
$\xi>\xi_t$,  injecting  the  corresponding expression  of  $\Theta_s$
(\ref{C4-TS})   in   the   corresponding  expression   of   $\Theta_b$
(\ref{C4-TB}) gives
\begin{equation}
  \Theta_b = \frac{3 (\beta  h+4) (h-2 \xi )}{2 h (\beta  h+6)}
\end{equation}
which tends  to $3/2-3\xi/h$ when $\beta  \rightarrow \infty$ (Section
\ref{sec:theory}). Finally,  taking the  limit of $\Theta_s$  for both
$\xi>\xi_t$ and $\xi<\xi_t$ show that  $\Theta_s$ indeed tends to zero
when $\Omega\rightarrow \infty$.

For magmatic  intrusion, the thermal  parameters of the magma  and the
encasing  rocks are  close and  the dimensionless  number $\Omega$  is
close to  $1$. In the following,  we study the effect  of relaxing the
isothermal   boundary  condition   on   the   dynamics  by   comparing
$\Omega=10^5$ and  $\Omega = 1$  in both regimes separately.   We also
investigate  the effect  of  a  more realistic  rheology  on the  flow
dynamics. 

\section{Evolution in the bending regime}
\label{sec:evol-bend-regime-1}

We  follow the  same approach  as in  the previous  Chapter and  first
concentrate  on the  case in  which  only bending  contributes to  the
pressure.   The   governing  equations  are  thus   (\ref{C4-HF})  and
(\ref{C4-TF}) where  $P =  \nabla_r^4h$. In  the previous  Chapter, we
show that  the dynamics in the  bending regime depends on  the average
viscosity of  a small region  at the front of  the current and  can be
divided in three phases. A first phase where the current behaves as an
isoviscous flow  with hot  viscosity.  A second  phase where  the flow
slows down  and thickens.  A last  phase where the flow  returns in an
isoviscous flow but with cold viscosity.  Hereafter, we first describe
how the thermal boundary condition influences the timing for the phase
transition  by looking  at  two values  for  the dimensionless  number
$\Omega$,  i.e.   $\Omega=1$  and   $\Omega=10^5$  using  the  inverse
temperature  dependence  for  the  rheology  $\eta_1(\theta)$,  as  in
Chapter  \ref{JFM}. We  thus investigate  the effect  of changing  the
rheology.

\begin{figure}
  \begin{center}
    \graphicspath{ {/Users/thorey/Documents/These/Projet/Refroidissement/Skin_Model/Figure/Figure_Heating/} }
    \includegraphics[scale=0.55]{Grid_PeOmega_ELAS_Berco_3.eps}
    \caption{Snapshots of  the flow thermal  structure $\theta(r,z,t)$
      for  different  sets  ($Pe$,$\Omega$)  with  $Pe=  1.0$  ,$10.0$
      ,$100.0$   and   $\Omega=10^5$   and   $1.0$   at   $t=10$   for
      $\nu=0.001$. The thermal structure  in the surrounding medium is
      given  by  (\ref{eq11})  and   reads  in  a  dimensionless  form
      $\Theta_r(r,z,t)=\Theta_s(r,t)\operatorname{erfc}{\left(Pe^{1/2}\frac{\kappa_m}{\kappa_r}\frac{(z-h)}{2\sqrt{t}}\right)}$
      where the ratio $\kappa_m/\kappa_r$ is set to $1$.}
    \label{Grid_PeOmega_Heating}
  \end{center}
\end{figure}

\begin{figure}
  \begin{center}
    \graphicspath{ {/Users/thorey/Documents/These/Projet/Refroidissement/Skin_Model/Figure/Figure_Heating/} }
    \includegraphics[scale=0.45]{Scaling_HR_ELAS_Omega.eps}
    \caption{Left: Dimensionless thickness at  the center $h_0$ versus
      dimensionless time  $t$ for different sets  $(\nu,Pe)$ indicated
      on      the      plot.      Dotted-lines:      scaling      laws
      $h_0=  0.7h_f^{-1/11}\nu^{-2/11}t^{8/22}$ for  $\nu  = 1.0$  and
      $0.001$.  Right:  Dimensionless radius $R$  versus dimensionless
      time  $t$  for  the  same sets  $(\nu,Pe)$.   Dotted-lines:  the
      scaling    laws    $R=   2.2h_f^{1/22}\nu^{1/11}t^{7/22}$    for
      $\nu = 1.0$ and $0.001$.}
    \label{Scaling_HR_ELAS_Omega}
  \end{center}
\end{figure}
\subsection{Relaxing  the   thermal  boundary  condition,   effect  of
  $\Omega$}
\label{sec:infl-therm-bound}

As for the isothermal boundary  condition, the thermal boundary layers
first connect  at the front  and a region of  cold fluid forms  at the
current  tip for  $\Omega  =  1$ (Figure  \ref{Grid_PeOmega_Heating}).
However, in  that case, the  heating of the surrounding  medium limits
heat loss in the central region of the current and the thermal anomaly
extends  further into  the flow.   For instance,  for $\nu=0.001$  and
$Pe=1.0$, while the thermal anomaly extends over $50\%$ of the current
for $\Omega = 10^5$ at $t=10$, it  extends over $75\%$ of the flow for
$\Omega=1$ (Figure \ref{Grid_PeOmega_Heating}).

As for $\Omega=10^5$, the current  first behave has an isoviscous flow
with hot viscosity, it then slows  down and thickens to finally behave
again  as  an   isoviscous  flow  but  with   cold  viscosity  (Figure
\ref{Scaling_HR_ELAS_Omega}).  As  the current  tip remains hot  for a
longueur  period of  time,  the  transition to  the  second and  third
bending regime are however delayed relative to the case $\Omega= 10^5$
(Figure \ref{Scaling_HR_ELAS_Omega}).  For instance, for $\nu=10^{-3}$
and $Pe=1.0$, while the transitions to the second bending phase occurs
at  $t\sim   10^{-5}$  for  $\Omega=10^{5}$,  it   occurs  only  after
$t\sim       10^{-4}$       for      $\Omega=       1.0$       (Figure
\ref{Scaling_HR_ELAS_Omega}).

In addition, the second phase of thickening show two different stages:
a first stage  where the thickness drastically increases  and a second
stage  where   it  continues   increasing  but  much   slower  (Figure
\ref{Scaling_HR_ELAS_Omega}). This transition, enhanced by the smaller
spatial resolutions of  the simulations in this  chapter, reflects the
detachment  of  the  thermal  anomaly and  is  discussed  in  Appendix
\ref{Heat:AppendixC}.

\subsection{Characterization of the thermal anomaly}
\label{sec:char-therm-anom}

As in Chapter  \ref{JFM}, we quantify the size of  the thermal anomaly
through a  critical thermal radius  $R_c(t)$ where the  temperature at
the  center  of  the  flow   $\Theta_b$  is  $1\%$  of  the  injection
temperature, i.e.  $\Theta_b(r=0)-\Theta_b(r=R_c)=0.99$.  As expected,
the  thermal anomaly  is larger  when considering  the heating  of the
surrounding medium relative to the  isothermal boundary condition at a
same time (Figure \ref{ELAS_RRc_RArrhenius_1-0}).

The size of the thermal anomaly  $R_c(t)$ is given by the radius where
advection of heat is equal to heat loss (\ref{HeatequationThermal})
\begin{equation}
  \frac{d}{d    t}\left(\theta(r=   R_c,t)\right)    \propto   Pe^{-1}
  \frac{\partial^2}{\partial z^2}\left(\theta(r=R_c,t)\right).
  \label{C4-HeatequationThermal}
\end{equation}
which,  by integration  over the  thickness  of the  flow $h$  becomes
(\ref{Temperature2})
\begin{eqnarray}
  \frac{d}{dt}\left(\int_0^h\theta           dz\right)-\Theta_s\frac{d
  h}{dt}&\propto& Pe^{-1} \frac{\Theta_b-\Theta_s}{h}\nonumber\\
  \overline{\theta}\frac{d h}{dt}+h\frac{d \overline{\theta}}{dt}-\Theta_s\frac{d
  h}{dt}&\propto& Pe^{-1}
                  \frac{\Theta_b-\Theta_s}{h}\nonumber\\
  \frac{2}{3}\left(\Theta_b-\Theta_s\right)\frac{d  h}{d   t}&\propto& Pe^{-1}
                                                                       \frac{\Theta_b-\Theta_s}{h}\nonumber\\
  \frac{d h}{d t}&\propto& \frac{Pe^{-1}}{h}\label{Calcul2}
\end{eqnarray}
where $\overline{\theta}$  is equal  to $(\int_0^h \theta  dz)/h$ here
and  we   have  assumed   that  $\overline{\theta}$  is   constant  at
$r=R_c$. Therefore,  relaxing the thermal boundary  condition does not
modify (\ref{Calcul1}) and the larger size of the thermal anomaly when
$\Omega=1$ must be  linked to the heat advection rate,  i.e.  the left
hand side term in the balance (\ref{Calcul2}).

Using the thickness profile (\ref{IntrusionShape}), the left hand side
term of (\ref{Calcul2}) becomes
\begin{equation}
  \frac{d h}{d t} = \alpha^2\left(1+\frac{R_c}{R}\right)^2\frac{\partial h_0}{\partial
    t}+\frac{4h_0R_c^2}{R^3}\frac{\partial
    R}{\partial
    t}\alpha\left(1+\frac{R_c}{R}\right) 
\end{equation}
where  $\alpha  (t)$  is  the  normalized  region  beyond  $r=R_c(t)$,
i.e. $\alpha(t)= \left(R(t)-R_c(t)\right)/R(t)$. While the second term
of the right hand side is dominant in the isothermal boundary case, it
becomes  weaker than  the  first term  when  $\Omega=1$. Indeed,  when
$\Omega=1$,  the heating  of  the surrounding  medium. Therefore,  the
balance (\ref{Calcul2}) reduces to
\begin{equation}
  \alpha^2\left(1+\frac{R_c}{R}\right)^2\frac{\partial h_0}{\partial
    t}\propto \frac{Pe^{-1}}{\alpha^2\left(1+\frac{R_c}{R}\right)^2h_0}
\end{equation}
which, in the limit $\alpha<<1$, becomes
\begin{equation}
  \alpha^4\frac{\partial h_0}{\partial
    t} \propto \frac{Pe^{-1}}{h_0\frac{\partial h_0}{\partial t}}.
\end{equation}
\begin{figure}
  \begin{center}
    \graphicspath{ {/Users/thorey/Documents/These/Projet/Refroidissement/Skin_Model/Figure/Figure_Heating/} }
    \includegraphics[scale=0.45]{ELAS_RRc_RBercovici_1-0.eps}
    \caption{Left:  Extent  of  the cold  fluid  region  $R(t)-R_c(t)$
      versus   dimensionless    time   for    different   combinations
      ($\nu$,$Pe$) indicated  on the  plot, the first  order Arrhenius
      rheology $\eta_2(\theta)$ and $\Omega=1$.   Right: Same plot but
      where  we  rescale  the  extent  of the  cold  fluid  region  by
      $Pe^{-1/4}\nu^{2/11}$.       Dotted-line:       scaling      law
      $(R(t)-R_c(t))Pe^{1/4}\nu^{-2/11}= 0.9 h_f^{1/11}t^{17/44}$.}
    \label{ELAS_RRc_RBercovici_1-0}
  \end{center}
\end{figure}

Substituting the  thickness at the  center $h_0(t)$ by  its respective
scaling  law   (\ref{ScalingH-Visco}),  the   relative  size   of  the
normalized cold front region $\alpha$ reads
\begin{equation}
  \alpha(t) \propto h_f^{1/22}\nu^{1/11}Pe^{-1/4}t^{7/44}
\end{equation}
which is equivalent to
\begin{equation}
  R(t)-R_c(t) = 0.8h_f^{1/11}\nu^{2/11}Pe^{-1/4}t^{17/44}
  \label{ScalingRRc-Heating}
\end{equation}
where the numerical prefactor, which  depends on the definition of the
thermal anomaly, has been chosen to fit the simulations.

The predicted scaling law for the  evolution of the extent of the cold
fluid  region  (\ref{ScalingRRc-Heating})   indeed  closely  fits  the
numerical  simulations,  even for  the  isoviscous  case $\nu=1$  with
$Pe=1$.  The  cold fluid  region thus cools  slightly slower  than for
isothermal  boundary condition,  from $t^{9/22}$  to $t^{17/44}$.   In
addition, the dependence is weaker in  the Peclet number, from a power
$1/3$    to   $1/4$    when   relaxing    the   isothermal    boundary
condition. Indeed, for small $Pe$, the vertical diffusion is efficient
on  the emplacement  time scale,  the heat  loss in  the interior  are
smaller and the thermal anomaly  larger. In contrast, for large values
of $Pe$,  the advection dominates  and the saving  of heat due  to the
heating  of  the  medium  is less  important  decreasing  the  overall
difference between small and large  values of $Pe$.  In the following,
we  consider  the  more   realistic  first  order  Arrhenius  rheology
$\eta_2(\theta)$  and  we  define  the   final  value  for  the  phase
transitions within the bending regime.

\subsection{Influence of the rheology, effect of $\eta(\theta)$}
\label{sec:infl-therm-bound}

The first  order Arrhenius rheology $\eta_2(\theta)$  widens the range
of  temperature  over  which significant  viscosity  variation  occurs
(Figure \ref{C4-Rheology}), i.e.  $\sim80\%$  of the temperature range
against $\sim10\%$ for $\eta_1(\theta)$. Therefore, the effective flow
viscosity starts to  increase sooner and the  phase transitions within
the  bending regime  occur  sooner than  for  the rheology  previously
considered  $\eta_1(\theta)$ (Figure  \ref{Scaling_HR_ELAS_Rheology}).
For instance, for  $\nu=10^{-3}$ and $Pe=1.0$, while  the second phase
of  the   flow  starts  around   $t\sim  10^{-4}$  for   the  rheology
$\eta_1(\theta)$, it  starts around  $t\sim 10^{-5}$ for  the rheology
$\eta_2(\theta)$ (Figure \ref{Scaling_HR_ELAS_Rheology}).

\begin{figure}
  \begin{center}
    \graphicspath{ {/Users/thorey/Documents/These/Projet/Refroidissement/Skin_Model/Figure/Figure_Heating/} }
    \includegraphics[scale=0.45]{Scaling_HR_ELAS_Rheology.eps}
    \caption{Left: Dimensionless thickness at  the center $h_0$ versus
      dimensionless time  $t$ for different sets  $(\nu,Pe)$ indicated
      on      the      plot.      Dotted-lines:      scaling      laws
      $h_0=  0.7h_f^{-1/11}\nu^{-2/11}t^{8/22}$  for  $\nu  =  0.001$.
      Right: Dimensionless  radius $R$  versus dimensionless  time $t$
      for the  same sets  $(\nu,Pe)$.  Dotted-lines: the  scaling laws
      $R= 2.2h_f^{1/22}\nu^{1/11}t^{7/22}$ for $\nu=0.001$.}
    \label{Scaling_HR_ELAS_Rheology}
  \end{center}
\end{figure}


\subsection{Size of the thermal aureole}
\label{sec:char-therm-anom}

The size  of the  thermal aureole, the  heated region  surrounding the
current,  scales as  $(\kappa_r  \tau)^{1/2}$,  i.e.  $Pe^{-1/2}$  and
hence is  much larger  for small  values of  $Pe$.  Indeed,  for large
values of $Pe$, advection dominates  on the emplacement time scale and
the thermal aureole  is restricted to a small zone  around the current
(Figure  \ref{Grid_PeOmega_Heating}). For  instance, the  thickness of
the thermal aureole at the center  for $Pe=1.0$ is almost equal to the
current thickness $h_0$ whereas it is  only a few percent of $h_0$ for
$Pe=100.0$ (Figure \ref{Grid_PeOmega_Heating}).

\section{Evolution in the gravity regime}
\label{sec:evol-grav-regime}


\section*{Appendix A: Main variables}
 \label{Heat:AppendixA}

The variable $\xi$ is the sufficient variable to solve for in the heat
transport equation (\ref{EqFinal2}). Indeed,
\begin{equation}
  \xi&=&\frac{\delta}{3} \left(- 2 \Theta_{b} - \Theta_{s} + 3\right)\label{C4-xi}
\end{equation}
where   we    have   used   (\ref{C4-tbar}).    In    addition,   from
(\ref{C4-Boundary-Condi}), we can rewrite
\begin{eqnarray}
  \Theta_s &=& \frac{2 \Theta_{b}}{\beta \delta + 2}\label{C4-Ts},\\
  \delta  &=&   \frac{1}{\Theta_{s}  \beta}   \left(2  \Theta_{b}   -  2
              \Theta_{s}\right)\label{C4-D},\\
  \Theta_b &=& \frac{\Theta_{s}}{2} \left(\beta \delta + 2\right)\label{C4-Tb}
\end{eqnarray}
When  the  thermal  boundary  layer just  merged,  then  $\Theta_b=1$,
$\delta = h/2$ and injecting (\ref{C4-Ts}) into (\ref{C4-xi}) gives
\begin{equation}
  \xi_t(t)=\frac{\beta(t) h^{2}{\left (r,t \right )}}{6 \beta(t) h{\left (r,t \right )}
    + 24}\label{C4-xit}
\end{equation}
Therefore,  when  $\xi<\xi_t$,  the  thermal boundary  layer  are  not
merged, $\Theta_b=1$ and injecting (\ref{C4-D}) into (\ref{C4-xi}) and
solving for $\Theta_s$ gives
\begin{equation}
  \Theta_s = \frac{3 \beta}{4} \xi - \frac{\sqrt{3}}{4} \sqrt{\beta \xi \left(3 \beta \xi + 8\right)} + 1.
\end{equation}
In contrast, when $\xi>\xi_t$, the  thermal boundary layer are merged,
$\delta=h/2$  and  injecting   (\ref{C4-Tb})  into  (\ref{C4-xi})  and
solving for $\Theta_s$ gives
\begin{equation}
  \Theta_s = \frac{- 12 \xi + 6 h}{\left(\beta h + 6\right) h}.
\end{equation}

\section*{Appendix B: Integral expressions}
\label{Heat:AppendixB}

The  model  developed in  Section  \ref{sec:theory-1}  depends on  the
integrals
\begin{eqnarray}
  I_0(z)&=&\int_0^z\frac{1}{\eta(y)}\left(y-\frac{h}{2}\right)
            dy \\
  I_1(z) &=& \int_0^z\frac{1}{\eta(y)}\left(y-\frac{h}{2}\right)y dy\\
  I_2(z)&=&\int_0^y                         \frac{1}{\eta(y)}
            \left(y-\frac{h}{2}\right)G(y)dy
\end{eqnarray}
where $G(z)$  is a  primitive of $\theta(z)$  where $z<\delta$  and is
given by
\begin{equation}
  G(z) = \frac{z \left(3 \delta ^2 \Theta_s+3 \delta z (\Theta_b-\Theta_s)+z^2 (\Theta_s-\Theta_b)\right)}{3 \delta ^2}.
\end{equation}
In  particular, the  model requires  the expression  of $I_0(\delta)$,
$I_1(\delta)$, $I_1(h)$ and $I_2(\delta)$. 

\vspace{.5cm} \textbf{Rheology 1: $\eta(\theta)=\eta_1(\theta)$} \vspace{.5cm}

In that case, the four integrals can be easily derived and read
\begin{eqnarray}
I_0(\delta)&=&\frac{\delta}{12} \left(6 \delta \nu + (1-\nu) \left(- \alpha_1 \delta + 2 \alpha_1 h + 6 \Theta_{b} \delta - 6 \Theta_{b} h\right) - 6 h \nu\right)\nonumber\\
I_1(\delta)&=&\frac{\delta^{2}}{120} \left(40 \delta \nu + (1-\nu) \left(- 4 \alpha_1 \delta + 5 \alpha_1 h + 40 \Theta_{b} \delta - 30 \Theta_{b} h\right) - 30 h \nu\right)\nonumber\\
I_1(h)&=&\frac{1}{60} \left((1-\nu) \left(- 4 \alpha_1 \delta^{3} + 10 \alpha_1 \delta^{2} h - 10 \alpha_1 \delta h^{2} + 5 \Theta_{b} h^{3}\right) + 5 h^{3} \nu\right)\nonumber\\
I_2(\delta)&=&- \frac{\delta^{2}}{2520} \left(378  \alpha_1 \delta \nu -
               315  \alpha_1 h  \nu -  840 \Theta_{b}  \delta \nu  + 630
               \Theta_{b} h \nu \right)\nonumber\\
&&-\frac{\delta^{2}}{2520}(1-\nu)  \left(- 50  \alpha_1^{2} \delta  + 70
    \alpha_1^{2}  h  +  462  \alpha_1   \Theta_{b}  \delta  -  420  \alpha_1
    \Theta_{b}  h -  840  \Theta_{b}^{2} \delta  + 630  \Theta_{b}^{2}
    h\right)\nonumber
\end{eqnarray}
where $\alpha=\Theta_b-\Theta_s$ has been introduced for clarity.

\vspace{.5cm}   \textbf{Rheology   2:   $\eta(\theta)=\eta_2(\theta)$}
\vspace{.5cm}

For cases where $\nu<1$, we have
\begin{eqnarray}
I_0(\delta)&=&-\frac{\delta  \nu ^{1-\Theta_b} \left(\sqrt{\pi } \sqrt{\alpha_1} (2 \delta -h) \sqrt{-\alpha_2}
   \text{erf}\left(\sqrt{\alpha_1} \sqrt{-\alpha_2}\right)+2 \delta  \left(\nu ^{\alpha_1}-1\right)\right)}{4 \alpha_1 \alpha_2}\nonumber\\
I_1(\delta)&=&\frac{\delta ^2 \nu ^{1-\Theta_b} \left(\sqrt{\pi } \erf\left(\sqrt{\alpha_1} \sqrt{-\alpha_2}\right) (\alpha_1 (h-2
   \delta ) \alpha_2+\delta )\right)}{4           \alpha_1^{3/2}          (-\alpha_2)^{3/2}}\nonumber\\
&&+\frac{\delta ^2 \nu ^{1-\Theta_b} \left(\sqrt{\alpha_1} \sqrt{-\alpha_2} \left(2 \delta  \left(\nu ^{\alpha_1}-2\right)-h \nu
   ^{\alpha_1}+h\right)\right)}{4           \alpha_1^{3/2}          (-\alpha_2)^{3/2}}\nonumber\\
I_1(h)&=&\frac{\nu ^{1-\Theta_b} \left(\sqrt{\alpha_1} \sqrt{-\alpha_2} \left(12 \delta ^2 \left(\delta  \left(\nu
   ^{\alpha_1}-2\right)-h \nu ^{\alpha_1}+h\right)+\alpha_1  (2 \delta -h)^3
          \log    (\nu    )\right)\rihgt)}{12   \alpha_1^{3/2}    (-\alpha_2)^{3/2}}\nonumber\\
&&-\frac{\nu ^{1-\Theta_b} \left(3 \sqrt{\pi } \delta 
   \text{erf}\left(\sqrt{\alpha_1} \sqrt{-\alpha_2}\right) \left(\alpha_1 (h-2 \delta )^2 \alpha_2-2 \delta
   ^2\right)\right)}{12 \alpha_1^{3/2} (-\alpha_2)^{3/2}}\nonumber
\end{eqnarray}
\begin{eqnarray}
I_2(\delta)&=&\frac{\delta ^2 \nu ^{1-\Theta_b} \left(\sqrt{\pi } \text{erf}\left(\sqrt{\alpha_1} \sqrt{-\alpha_2}\right) \left(-2
   \alpha_1 (2  \delta -h)  (\alpha_1-3 \Theta_b)  \alpha_2^2-6 \delta
               \Theta_b  \alpha_2-3   \delta  \right)\right)}{24
               \alpha_1^{3/2} (-\alpha_2)^{5/2}}\nonumber\\
&&+\frac{\delta ^2 \nu ^{1-\Theta_b} \left(2
   \sqrt{\alpha_1} \nu ^{\alpha_1} \sqrt{-\alpha_2} \left(\nu ^{-\alpha_1} (\alpha_2 (-2 \delta  (\alpha_1-6
   \Theta_b)-3 h \Theta_b)+2 \delta -h)\right)\right)}{24
               \alpha_1^{3/2} (-\alpha_2)^{5/2}}\nonumber\\
&&+\frac{\delta ^2 \nu ^{1-\Theta_b} \left(2
   \sqrt{\alpha_1} \nu ^{\alpha_1} \sqrt{-\alpha_2} \left(2 \delta  \alpha_1 \alpha_2-6 \delta  \Theta_b \alpha_2+\delta
   -\alpha_1 h \alpha_2+3 h \Theta_b \alpha_2+h\right)\right)}{24 \alpha_1^{3/2} (-\alpha_2)^{5/2}}\nonumber
\end{eqnarray}
where    in   addition    to    $\alpha_1$,    we   also    introduced
$\alpha_2=log(\nu)$  for  clarity.  In  the  case  where $\nu=1$,  the
expression above simplify and read
\begin{eqnarray}
I_0(\delta)&=&\frac{1}{2} \delta  (\delta -h)\nonumber\\
I_1(\delta)&=&\frac{1}{12} \delta ^2 (4 \delta -3 h)\nonumber\\
I_1(h)&=&\frac{h^3}{12}\nonumber\\
I_2(\delta)&=&-\frac{1}{120} \delta ^2 (18 \delta  \alpha_1-40 \delta  \Theta_b-15 \alpha_1 h+30 h \Theta_b)\nonumber
\end{eqnarray}

\section*{Appendix C: Second phase of the bending regime}
\label{Heat:AppendixC}

For all simulations,  the second phase of important  thickening in the
bending regime occurs in two stages: a first stage where the thickness
drastically increases and a second stage where it continues increasing
but      much     slower      (Figure     \ref{Scaling_HR_ELAS_Omega},
\ref{Scaling_HR_ELAS_Rheology}                                     and
\ref{Appendix_Phase2_N11_0_Pe100_0_nu0_001}).    The   transition   is
sharper  for the  rheology  $\eta_1(\theta)$ and  is  enhanced by  the
smaller    spatial   resolution    of   the    numerical   simulations
($\Delta r = 0.005$) in Chapter \ref{Heating} than in previous Chapter
(Figure \ref{Appendix_Phase2_N11_0_Pe100_0_nu0_001}).

In  all cases,  it reflects  the transition  from the  last moment  of
prewetting film cooling to the  detachment of the thermal anomaly. For
instance,  for  $\eta_1(\theta)$,   $\nu=0.001$  and  $Pe=100.0$,  the
transitions  between  the two  stages  occurs  at $t=1.8~10^{-2}$  and
coincide    to   the    film    becoming    entirely   cold    (Figure
\ref{Appendix_Phase2_N11_0_Pe100_0_nu0_001}  a,  b, c).   Before  this
time, the average  temperature in the film is non  zero and after this
time,   the   average  temperature   in   the   film  is   effectively
$\overline{\theta}=0$   and   the   thermal  anomaly   grows   (Figure
\ref{Appendix_Phase2_N11_0_Pe100_0_nu0_001}  c).    For  the  rheology
$\eta_2(\theta)$,  the   viscosity  increases  on  a   wide  range  of
temperature    and     the    transition    is     smoother    (Figure
\ref{Appendix_Phase2_N11_0_Pe100_0_nu0_001} d, e, f).



\begin{figure}
  \begin{center}
    \graphicspath{ {/Users/thorey/Documents/These/Projet/Refroidissement/Skin_Model/Figure/Figure_Heating/} }
    \includegraphics[scale=0.42]{Appendix_Phase2_N11_0_Pe100_0_nu0_001.eps}
    \caption{a)  Dimensionless  thickness $h_0$  versus  dimensionless
      time   $t$  for   $Pe=100.0$,  $\nu=0.001$   and  the   rheology
      $\eta_1(\theta)$.  Colors  refer to the time  $t$.  Dotted line:
      Scaling  law $h_0=  0.7h_f^{-1/11}\nu^{-2/11}t^{8/22}$. Vertical
      dashed-lines:  initial,  intermediate  and final  times  of  the
      temperature profiles plotted in  c). b) Dimensionless radius $R$
      versus dimensionless  time $t$  for $Pe=100.0$,  $\nu=0.001$ and
      the rheology  $\eta_1(\theta)$.  Colors  refer to the  time $t$.
      Dotted line:  Scaling law  $R= 2.2h_f^{1/22}\nu^{1/11}t^{7/22}$.
      Vertical dashed-lines: same than in a). d) Dimensionless average
      temperature over  the flow thickness  $\overline{\theta}$ versus
      radial   axis  $r$   for  times   between  $t=3.8~10^{-3}$   and
      $t=7.6~10^{-2}$.   Dashed-line profiles:  profiles at  the three
      different times  marked in a) and  b). Colors also refer  to the
      time on  the same  scale than a)  and b).  d),  e) and  f), same
      plots  than  a),  b)  and  c) but  for  the  Arrhenius  rheology
      $\eta_2$.}
    \label{Appendix_Phase2_N11_0_Pe100_0_nu0_001}
  \end{center}
\end{figure}


\bibliographystyle{agufull08}
\bibliography{/Users/thorey/Dropbox/Library}



%%% Local Variables:
%%% mode: latex
%%% TeX-master: "../main"
%%% End:
