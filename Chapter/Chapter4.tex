
\chapter{Toward  a more  realistic model  and its  application to  the
  spreading of shallow magmatic intrusions}
\label{Heating}

\minitoc

The previous chapter was a first  step toward the understanding of the
coupling between  the cooling and  the spreading of  an elastic-plated
gravity  current.   Hereafter,  we  investigate  the  changes  in  the
dynamics  caused by  both the  heating of  the wall  rocks and  a more
realistic  rheology  for  the  magma.    We  then  compare  the  model
predictions with the observations presented in Chapter \ref{chap2}.


\section{Motivation}
\label{sec:introduction}

Numerous  geological  studies  demonstrate  that  magmatic  intrusions
affect  the  host  rock  by  developing  contact-metamorphic  aureoles
\citep{Jaeger:1959du,Galushkin:1997dy,Senger:2014tt}.   For  instance,
the  Leadville  Limestone  in  Colorado, USA,  famous  for  preserving
fossils  dating   back  to  the  Carboniferous   period,  was  locally
transformed  into  marble  following  the intrusion  of  the  Treasure
Mountain Dome (Figure \ref{Treasure}).  The increase in the geothermal
gradient in  sedimentary basins also  tends to accelerate  the thermal
maturation of organic matter in the surrounding, promoting hydrocarbon
generation   \citep{Senger:2014tt}.    Release    of   $CO_2$   during
metamorphic processes has also been  proposed to help the formation of
ore    deposits   in    the    vicinity    of   magmatic    intrusions
\citep{SILLITOE:1998bs,Ganino:2008ft,Zhou:2008hc}.

The size of  the contact aureole depends on the  context and can reach
more  than  $100\%$  of  the   intrusion  thickness  in  many  regions
\citep{Galushkin:1997dy}.   This contact  aureole,  by insulating  the
flow, may also  affect the dynamics of the  magmatic intrusion itself.
In the following,  we relax the isothermal boundary  condition used in
Chapter \ref{C3-JFM} to investigate its influence of the dynamics.

\begin{figure}[h!]
  \begin{center}
    \graphicspath{ {/Users/thorey/Documents/These/Manuscript/Figure/Chapter4/} }
    \includegraphics[scale=0.9]{Treasure.eps}
    \caption{a)  Sketch of  the granitic  Treasure Mountain  Laccolith
      intruded roughly  $\sim 20$ Ma  years ago in Colorado,  USA. The
      Leadville Limestone (white layer)  was metamorphosed by the heat
      from the intrusion, and was transformed into marble.  During the
      last $10$ Ma, the area was eroded, and the marble as well as the
      laccolith are today exposed at the surface.  b) Cross section of
      the strata from the West flank of the Treasure Mountain dome. c)
      Marble  vein  visible  from  the  West  flank  of  the  Treasure
      Mountain. The  quality of this  Marble was selected to  clad the
      exterior  of the  Lincoln  Memorial and  a  variety of  building
      throughout the United States.}
    \label{Treasure}
  \end{center}
\end{figure}
\newpage
\section{Theory}
\label{C4-sec:theory-1}

We  consider   the  model  of  elastic-plated   gravity  current  with
temperature-dependent      viscosity     described      in     Section
\ref{C3-sec:theory}  in   which  we  relax  the   isothermal  boundary
condition. In  the following,  we specify only  the changes  in theory
that come  from the new  thermal boundary  condition and we  refer the
reader  to  Section \ref{C3-sec:theory}  for  more  details about  the
derivation.

\subsection{Thermal boundary condition}
\label{C4-sec:formulation-1}

We now consider  the heating of the surrounding medium  by the flowing
magma.  The vertical temperature profile respecting continuity writes
\begin{equation}
  T=
  \begin{cases}
    T_b - (T_b-T_s)(1-\frac{z}{\delta})^2 & 0 \le z\le \delta \\
    T_b & \delta \le z\le h-\delta \\
    T_b - (T_b-T_s)(1-\frac{h-z}{\delta})^2 & h-\delta \le z\le h\\
  \end{cases}
  \label{C4-Temperature}
\end{equation}
where  $\delta(r,t)$   is  the   thermal  boundary   layer  thickness,
$T(r,z,t)$  is  the  temperature  of  the  fluid,  $T_b(r,t)$  is  the
temperature at  the center of  the profile  and $T_s(r,t)$ is  now the
temperature  at the  surface, i.e.   $T(r,z=0,t)=T(r,z=h,t)=T_s(r,t)$.
As in Section \ref{C3-sec:theory}, this profile assures the continuity
of  the temperature  and  heat  flux within  the  flow.  In  addition,
continuity of the heat flux across the flow boundaries reads
\begin{eqnarray}
  k_m\left.\frac{\partial                                    T}{\partial
  z}\right|_{z=0}&=&k_r\left.\frac{\partial              T_r}{\partial
                     z}\right|_{z=0}  \label{C4-Flux1}\\
  k_m\left.\frac{\partial                                  T}{\partial
  z}\right|_{z=h}&=&k_r\left.\frac{\partial            T_r}{\partial
                     z}\right|_{z=h}
                     \label{C4-Flux2}
\end{eqnarray}
where  $T_r(r,z)$ is  the temperature  in the  surrounding medium  and
$k_r$ its  thermal conductivity.  Assuming  a semi infinite  layer for
the rigid layer below  the intrusion, \citet{Carslaw:1959wf} show that
the temperature $T_r$ in the  surrounding rocks can be approximated to
a first order by
\begin{equation}
  T_r(r,z,t)-T_0=(T_{s}-T_0)\operatorname{erfc}{\left(\frac{-z}{2\sqrt{\kappa_r t}}\right)}.
  \label{C4-eq22}
\end{equation}
The  thickness of  the upper  layer is  equal to  the intrusion  depth
$d_c$. However,  we assume that the  depth $d_c$ is large  compared to
the characteristic  length scale for  conduction $L_c$ and we  use the
same approximation to derive $T_r$ above the intrusion
\begin{equation}
  T_r(r,z,t)-T_0=(T_{s}-T_0)\operatorname{erfc}{\left(\frac{z-h}{2\sqrt{\kappa_r t}}\right)}.
  \label{C4-eq11}
\end{equation}
Therefore, the  two thermal  boundary conditions  (\ref{C4-Flux1}) and
(\ref{C4-Flux2}) become
\begin{eqnarray}
  k_m\left.\frac{\partial                                    T}{\partial
  z}\right|_{z=0}&=& k_r
                     \frac{T_{s}-T_{0}}{\sqrt{\pi \kappa_r t}}  \label{C4-2Flux_1}\\
  k_m\left.\frac{\partial                                    T}{\partial
  z}\right|_{z=h}&=& -k_r
                     \frac{T_{s}-T_{0}}{\sqrt{\pi \kappa_r t}}.
                     \label{C4-2Flux_2}
\end{eqnarray}


\subsection{Dimensionless equations}
\label{C4-sec:dimens-equat-1}

Except  for   the  conduction  term,   which  now  accounts   for  the
dimensionless    surface   temperature    $\Theta_s$,   the    coupled
dimensionless equations  governing the  cooling of  the flow  are very
similar to (\ref{C3-HF}) and (\ref{C3-TF}) and read
\begin{eqnarray}
  \frac{\partial h}{\partial t}-\frac{12}{r}
  \frac{\partial}{\partial      r}
  \left( r I_1(h) \frac{\partial P}{\partial
  r}\right)
  \label{C4-HF}
  & =& \mathcal{H}(\frac{\gamma}{2}-r)\frac{32}{\gamma^{2}}\left(\frac{1}{4}-\frac{r^{2}}{\gamma^{2}}\right)\\
  \frac{\partial                                       \xi}{\partial
  t}+\frac{1}{r}\frac{\partial}{\partial                          r}
  \left( r\left(\bar{u}\xi-\Sigma\right)\right)&=&2Pe^{-1}St_m\frac{\Theta_b-\Theta_s}{\delta}\label{C4-TF}
\end{eqnarray}
with
\begin{eqnarray}
  \overline{\theta}&=&\frac{1}{3}\left(2\Theta_b+\Theta_s\right)\label{C4-tbar}\\
  \overline{u}&=&\frac{12}{\delta}
                  \frac{\partial
                  P}{\partial
                  r}\left(\delta
                  I_0(\delta)-I_1(\delta)\right)\\
  \Sigma &=& \frac{12}{\delta} \frac{\partial P}{\partial r}\left(I_0(\delta)\left(G(\delta)-\delta\overline{\theta}\right)+\overline{\theta}I_1(\delta)-I_2(\delta)\right).
\end{eqnarray}
where $G(z)$ denotes  a primitive of $\theta(z)$  when $z<\delta$. The
rheology, which couples equations  (\ref{C4-HF}) and (\ref{C4-TF}), is
contained in the  three integrals $I_0(z)$, $I_1(z)$  and $I_2(z)$ and
is discussed  in the  next section.   The thermal  boundary conditions
(\ref{C4-2Flux_1})  and (\ref{C4-2Flux_2})  reduce in  a dimensionless
form to
\begin{equation}
  2\frac{\Theta_b-\Theta_s}{\delta}               =   \beta(t)\Theta_s
  \label{C4-Boundary-Condi}
\end{equation}
where
\begin{equation}
  \beta(t) = \frac{\Omega_1\Omega_2^{-1/2} Pe^{1/2}}{\sqrt{\pi t}}.\label{C4_beta}
\end{equation}
$\Omega_1$ and $\Omega_2$ are two new dimensionless numbers and read
\begin{eqnarray}
  \Omega_1&=&\frac{\kappa_r}{\kappa_m}\label{C4-Omega1}\\
  \Omega_2&=&\frac{k_r}{k_m}\label{C4-Omega2}
\end{eqnarray}
They represent  the ratio of  wall rocks to magma  thermal diffusivity
(\ref{C4-Omega1})    and   thermal    conductivity   (\ref{C4-Omega2})
respectively.   While the  former should  be close  to $1$,  the later
could show small variations depending on the nature of the surrounding
rocks  as   well  as  its  porosity   content  \citep{Buttner:1998hy}.
However, variations of both dimensionless numbers will only affect the
size of the  thermal aureole. In this Chapter, we  focus mainly on the
dynamics of  the flow itself and  for a sake of  simplicity, we rather
consider only the combination of both which we call $\Omega$ and reads
\begin{eqnarray}
  \Omega &=& \Omega_1\Omega_2^{-1/2}\\
         &=& \frac{\kappa_r}{\kappa_m}\left(\frac{k_m}{k_r}\right)^{1/2}\label{C4-Omega}
\end{eqnarray}
It represents  the ratio between  heat conduction at the  contact with
the encasing  rocks and heat  diffusion within the fluid  and quantify
how much heat is transferred to the wall rocks.

As in the precedent chapter, the heat transport equation (\ref{C4-TF})
is  written  in   terms  of  $\xi$  which   represents  our  ``thermal
variable''.  In that case, $\xi$ writes
\begin{equation}
  \xi&=&\frac{\delta}{3} \left(- 2 \Theta_{b} - \Theta_{s} + 3\right)\label{C4-xi}
\end{equation}
where  we have  used (\ref{C4-tbar})  in (\ref{C4-xi}).   In addition,
from  (\ref{C4-Boundary-Condi}),  we  can  derive  an  expression  for
$\Theta_s$,  $\delta$  and  $\Theta_b$  as a  function  of  the  other
variables
\begin{eqnarray}
  \Theta_s &=& \frac{2 \Theta_{b}}{\beta \delta + 2}\label{C4-Ts},\\
  \delta  &=&   \frac{1}{\Theta_{s}  \beta}   \left(2  \Theta_{b}   -  2
              \Theta_{s}\right)\label{C4-D},\\
  \Theta_b &=& \frac{\Theta_{s}}{2} \left(\beta \delta + 2\right)\label{C4-Tb}.
\end{eqnarray}
When the thermal boundary layers  have just merged, then $\Theta_b=1$,
$\delta = h/2$ and injecting (\ref{C4-Ts}) into (\ref{C4-xi}) gives
\begin{equation}
  \xi_t(t)=\frac{\beta(t) h^{2}{\left (r,t \right )}}{6 \beta(t) h{\left (r,t \right )}
    + 24}.\label{C4-xit}
\end{equation}
Therefore,  when  $\xi<\xi_t$, the  thermal  boundary  layers are  not
merged, $\Theta_b=1$ and injecting (\ref{C4-D}) into (\ref{C4-xi}) and
solving for $\Theta_s$ gives
\begin{equation}
  \Theta_s = \frac{3 \beta}{4} \xi - \frac{\sqrt{3}}{4} \sqrt{\beta \xi \left(3 \beta \xi + 8\right)} + 1.
\end{equation}
In  contrast,  when  $\xi>\xi_t$,  the thermal  boundary  layers  have
merged,  $\delta=h/2$ and  injecting (\ref{C4-Tb})  into (\ref{C4-xi})
and solving for $\Theta_s$ gives
\begin{equation}
  \Theta_s = \frac{- 12 \xi + 6 h}{\left(\beta h + 6\right) h}.
\end{equation}

In the end, we then have
\begin{equation}
  \Theta_s(r,t)=
  \begin{cases}
    \frac{3 \beta}{4} \xi - \frac{\sqrt{3}}{4} \sqrt{\beta \xi \left(3 \beta \xi + 8\right)} + 1 & \text{if} \hspace{1cm} \xi\leq \xi_t \\
    \frac{- 12  \xi +  6 h{\left  (r,t \right  )}}{\left(\beta h{\left
            (r,t  \right  )} +  6\right)  h{\left  (r,t \right  )}}  &
    \text{if} \hspace{1cm} \xi > \xi_t
  \end{cases}
  \label{C4-TS}
\end{equation}
and
\begin{equation}
  \Theta_b(r)=
  \begin{cases}
    1 &\text{if } \hspace{1cm} \xi\leq \xi_t \\
    \frac{\Theta_{s}}{4}  \left(\beta(t)  h{\left  (r,t  \right  )}  +
      4\right) & \text{if} \hspace{1cm} \xi > \xi_t
  \end{cases}
  \label{C4-TB}
\end{equation}
\begin{equation}
  \delta(r)=
  \begin{cases}
    \frac{1}{\Theta_{s} \beta(t)} \left(- 2 \Theta_{s} + 2\right) &\text{if } \hspace{1cm} \xi\leq \xi_t \\
    h(r,t)/2 & \text{if} \hspace{1cm} \xi > \xi_t
  \end{cases}
  \label{C4-DELTA}
\end{equation}
with
\begin{eqnarray}
  \xi_t(t)&=&\frac{\beta(t) h^{2}{\left (r,t \right )}}{6 \beta(t) h{\left (r,t \right )}
              + 24}
\end{eqnarray}

\subsection{Rheology}
\label{C4-sec:rheology}

The model derived in  Section \ref{C4-sec:dimens-equat-1} does not yet
assume a specific  relation between viscosity and  temperature and the
choice of the rheology $\eta(T)$,  which is contained in the integrals
$I_0(z)$, $I_1(z)$  and $I_2(z)$, remains  to be defined.   In Section
\ref{C3-sec:theory}, we assumed a viscosity inversely dependent on the
temperature which reads in a dimensional form
\begin{equation}
  \eta(T)=\frac{\eta_h
    \eta_c(T_i-T_0)}{\eta_h(T_i-T_0)+(\eta_c-\eta_h)(T-T_0)}.
\end{equation}
where $\eta_h$  and $\eta_c$  are the viscosities  of the  hottest and
coldest  fluid  at  the   temperature  $T_i$  and  $T_0$  respectively
\citep{Bercovici:2007vc}.
\begin{figure}[h!]
  \begin{center}
    \graphicspath{ {/Users/thorey/Documents/These/Projet/Refroidissement/Skin_Model/Figure/Figure_Heating/} }
    \includegraphics[scale=0.8]{Rheology.eps}
    \caption{Dimensionless viscosity  versus dimensionless temperature
      for  both  rheologies  $\eta_1$ (\ref{C4-rheo-1})  and  $\eta_2$
      (\ref{C4-rheo-2}) using $\nu=0.001$.}
    \label{C4-Rheology}
  \end{center}
\end{figure}
While  this model  possesses some  nice simplification  properties, it
restricts  the  change  in  viscosity   to  a  very  narrow  range  of
temperatures    close   to    $T=T_0$,   i.e.     $\theta=0$   (Figure
\ref{C4-Rheology}).     In     contrast,    the     Arrhenius    model
($\eta \sim  \exp(-k/T)$), which is  a more realistic model  to relate
temperature and viscosity  for lavas \citep{Blatt:2ViMWPc0}, describes
a viscosity  that increases over  a much larger range  of temperatures
(Figure \ref{C4-Rheology}).  To get some insights into the effect of a
more  realistic temperature-dependent  viscosity, we  thus also  use a
first-order approximation of the Arrhenius  model as a second rheology
$\eta_2(T)$ \citep{Diniega:2013eh}
\begin{eqnarray}
  \eta_2(T)                          =                          \eta_h
  \exp\left(-\log\left(\frac{\eta_h}{\eta_c}\right)\left(1-\frac{T-T_0}{T_i-T_0}\right)\right)
\end{eqnarray}
In a dimensionless form, they read
\begin{eqnarray}
  \eta_1(\theta)/\eta_h&=&\frac{1}{\nu+(1-\nu)\theta} \label{C4-rheo-1}\\
  \eta_2(\theta)/\eta_h&=&\exp\left(-\log(\nu)\left(1-\theta\right)\right)  \label{C4-rheo-2}
\end{eqnarray}
where  $\nu$ is  the  viscosity contrast  which  described in  Section
\ref{C3-sec:theory} and represents the ratio between the hot viscosity
$\eta_h$  and   the  cold  viscosity  $\eta_c$.    The  expression  of
$I_0(\delta)$, $I_1(\delta)$, $I_1(h)$ and $I_2(\delta)$, necessary to
close  the  model,  are  given  in  Appendix  \ref{chap:A1}  for  both
rheologies.

\subsection{Comparison with the isothermal model}
\label{C4-sec:some-limits}

We showed that relaxing the isothermal boundary condition introduces a
new dimensionless number $\Omega$ which  controls how much heat can be
transferred    to   the    surrounding    rocks.     In   the    limit
$\Omega \rightarrow \infty$, the model should thus reduce to the model
described    in    Section    \ref{C3-sec:theory}.     Indeed,    when
$\Omega\rightarrow \infty$, the  coefficient $\beta\rightarrow \infty$
and then $\xi_t\rightarrow  h/6$ (Section \ref{C3-sec:heat-equation}).
When $\xi<\xi_t$, injecting the corresponding expression of $\Theta_s$
(\ref{C4-TS})   in   the    corresponding   expression   of   $\delta$
(\ref{C4-DELTA}) gives
\begin{equation}
  \delta =\frac{3 \beta \xi +\sqrt{3} \sqrt{\beta \xi (3 \beta \xi +8)}+8}{2 \beta }
\end{equation}
which indeed  tends to  $3\xi$ when $\beta  \rightarrow \infty$  as in
Section \ref{C3-sec:heat-equation}.   When $\xi>\xi_t$,  injecting the
corresponding   expression   of   $\Theta_s$  (\ref{C4-TS})   in   the
corresponding expression of $\Theta_b$ (\ref{C4-TB}) gives
\begin{equation}
  \Theta_b = \frac{3 (\beta  h+4) (h-2 \xi )}{2 h (\beta  h+6)}
\end{equation}
which  indeed tends  to $3/2-3\xi/h$  when $\beta  \rightarrow \infty$
(Section  \ref{C3-sec:heat-equation}). Finally,  taking  the limit  of
$\Theta_s$ for  both $\xi>\xi_t$ and $\xi<\xi_t$  show that $\Theta_s$
indeed tends to zero when $\Omega\rightarrow \infty$.

For magmatic intrusions,  the thermal parameters of the  magma and the
encasing rocks are  close and the dimensionless  number $\Omega$ would
be close to $1$. In the following, we study the effect of relaxing the
isothermal   boundary  condition   on   the   dynamics  by   comparing
$\Omega=10^5$ and  $\Omega = 1$  in both regimes separately.   We also
investigate  the effect  of  a  more realistic  rheology  on the  flow
dynamics.

\begin{figure}[h!]
  \begin{center}
    \graphicspath{ {/Users/thorey/Documents/These/Projet/Refroidissement/Skin_Model/Figure/Figure_Heating/} }
    \includegraphics[scale=0.55]{Grid_PeOmega_ELAS_Berco_3.eps}
    \caption{Snapshots of  the flow thermal  structure $\theta(r,z,t)$
      for  different  sets  ($Pe$,$\Omega$)  with  $Pe=  1.0$  ,$10.0$
      ,$100.0$, $\Omega=10^5$ and $1.0$ at $t=10$ for $\nu=0.001$. The
      thermal   structure   above   the    intrusion   is   given   by
      (\ref{C4-eq11})    and   reads    in   a    dimensionless   form
      $\Theta_r(r,z,t)=\Theta_s(r,t)\operatorname{erfc}{\left(Pe^{1/2}\Omega_1\frac{(z-h)}{2\sqrt{t}}\right)}$
      where $\Omega_1$ is set to  $1$. The thermal structure below the
      intrusion is similar and not shown for clarity.}
    \label{C4-Grid_PeOmega_Heating}
  \end{center}
\end{figure}


\section{Evolution in the bending regime}
\label{C4-sec:evol-bend-regime-1}

We  follow the  same approach  as in  the previous  Chapter and  first
concentrate  on the  case in  which  only bending  contributes to  the
dynamic pressure.  The governing  equations are thus (\ref{C4-HF}) and
(\ref{C4-TF})  where  $P  =  \nabla_r^4h$.   For  isothermal  boundary
conditions, we show that the dynamics in the bending regime depends on
the average  viscosity of a small  region at the front  of the current
and can be divided into three phases. Hereafter, we first describe how
the thermal  boundary condition  influences the  timing for  the phase
transition  by looking  at  two values  for  the dimensionless  number
$\Omega$,     i.e.       $\Omega=1$     and      $\Omega=10^5$     and
$\eta(\theta)=\eta_1(\theta)$.   We  then  investigate the  effect  of
changing the rheology.

\subsection{Relaxing  the   thermal  boundary  condition,   effect  of
  $\Omega$}
\label{C4-sec:infl-therm-bound-el}

Heating of the surrounding medium acts  to limit heat loss in the flow
central   region   and  the   thermal   anomaly   is  larger   (Figure
\ref{C4-Grid_PeOmega_Heating}).   For  instance, for  $\nu=0.001$  and
$Pe=1.0$, while the thermal anomaly extends over $50\%$ of the current
for $\Omega  = 10^5$ at  $t=10$, it makes up  more than $75\%$  of the
flow    for    $\Omega=1$   (Figure    \ref{C4-Grid_PeOmega_Heating}).
Nevertheless, the thin tip of the current is still rapidly cooling and
the radius  of the flow  still grows  faster than the  thermal anomaly
extent when relaxing the thermal boundary condition.

\begin{figure}[h!]
  \begin{center}
    \graphicspath{ {/Users/thorey/Documents/These/Projet/Refroidissement/Skin_Model/Figure/Figure_Heating/} }
    \includegraphics[scale=0.45]{Scaling_HR_ELAS_Omega.eps}
    \caption{Left: Dimensionless thickness at  the center $h_0$ versus
      dimensionless   time  $t$   for  different   sets  $(\Omega,Pe)$
      indicated    on   the    plot.     Dotted-line:   scaling    law
      $h_0=   0.7h_f^{-1/11}\nu^{-2/11}t^{8/22}$    for   $\nu=0.001$.
      Right: Dimensionless  radius $R$  versus dimensionless  time $t$
      for  the  same  sets $(\Omega,Pe)$.   Dotted-line:  scaling  law
      $R= 2.2h_f^{1/22}\nu^{1/11}t^{7/22}$ for $\nu  = 0.001$.  In all
      simulations, $\nu=0.001$ and $\eta(\theta)=\eta_1(\theta)$.}
    \label{C4-Scaling_HR_ELAS_Omega}
  \end{center}
\end{figure}


Hence, the dynamics for $\Omega=1$ also passes through three different
phases.   The current  first behaves  as an  isoviscous flow  with hot
viscosity, it then slows down and  thickens to finally behave again as
an   isoviscous    flow   but   with   a    cold   viscosity   (Figure
\ref{C4-Scaling_HR_ELAS_Omega}).  As the current tip remains hot for a
longer period of time, the transitions to the second and third bending
regime are however delayed relatively to the case where $\Omega= 10^5$
(Figure    \ref{C4-Scaling_HR_ELAS_Omega}).     For   instance,    for
$\nu=10^{-3}$ and $Pe=1.0$, while the transition to the second bending
phase already begins at $t\sim 10^{-6}$ for $\Omega=10^{5}$, it occurs
only    after   $t\sim    10^{-5}$   for    $\Omega=   1.0$    (Figure
\ref{C4-Scaling_HR_ELAS_Omega}).

In addition, the second phase of thickening shows two different stages
for $\Omega =  1.0$ and $Pe=100.0$, a first stage  where the thickness
drastically  increases  and  a  second stage  where  it  continues  to
increase     but     at     a     much     slower     rate     (Figure
\ref{C4-Scaling_HR_ELAS_Omega}).  This transition, enhanced by the new
thermal  boundary condition,  reflects the  detachment of  the thermal
anomaly  in the  second bending  phase  and is  discussed in  Appendix
\ref{C4-Heat:AppendixC}.

\subsection{Considering   a  more   realistic   rheology,  effect   of
  $\eta(\theta)$}
\label{C4-sec:infl-therm-bound-el}


The first  order Arrhenius rheology $\eta_2(\theta)$  widens the range
of temperature over which significant viscosity variation occurs, i.e.
$\sim80\%$   of  the   temperature   range   against  $\sim10\%$   for
$\eta_1(\theta)$ (Figure \ref{C4-Rheology}).

\begin{figure}[htpb]
  \begin{center}
    \graphicspath{ {/Users/thorey/Documents/These/Projet/Refroidissement/Skin_Model/Figure/Figure_Heating/} }
    \includegraphics[scale=0.45]{Scaling_HR_ELAS_Rheology.eps}
    \caption{Left: Dimensionless thickness at  the center $h_0$ versus
      dimensionless time $t$ for  different sets $(\eta,Pe)$ indicated
      on      the      plot.        Dotted-line:      scaling      law
      $h_0=  0.7h_f^{-1/11}\nu^{-2/11}t^{8/22}$  for  $\nu  =  0.001$.
      Right: Dimensionless  radius $R$  versus dimensionless  time $t$
      for  the  same  sets   $(\eta,Pe)$.   Dotted-line:  scaling  law
      $R=  2.2h_f^{1/22}\nu^{1/11}t^{7/22}$ for  $\nu=0.001$.  In  all
      simulations, $\nu=0.001$ and $\Omega=1.0$.}
    \label{C4-Scaling_HR_ELAS_Rheology}
  \end{center}
\end{figure}

Therefore, the effective flow viscosity  starts to increase sooner and
the transition to the second bending  phase occurs sooner than for the
rheology     previously     considered    $\eta_1(\theta)$     (Figure
\ref{C4-Scaling_HR_ELAS_Rheology}).   For instance,  for $\nu=10^{-3}$
and  $Pe=1.0$,  while the  second  phase  of  the flow  starts  around
$t\sim 10^{-5}$  for the  rheology $\eta_1(\theta)$, it  starts around
$t\sim   10^{-6}$   for    the   rheology   $\eta_2(\theta)$   (Figure
\ref{C4-Scaling_HR_ELAS_Rheology}).   In  particular,  the  change  in
rheology almost compensates for the delay caused by the heating of the
surrounding medium. For  instance, the transition time  for the second
bending phase  for a  flow characterized by  $\eta=\eta_1(\theta)$ and
$\Omega=10^5$  is almost  the same  than for  a flow  characterized by
$\eta=\eta_2(\theta)$          and         $\Omega=1$          (Figure
\ref{C4-Scaling_HR_ELAS_Omega} and \ref{C4-Scaling_HR_ELAS_Rheology}).

\subsection{Characterization of the thermal anomaly}
\label{C4-sec:char-therm-anom}

As  in Chapter  \ref{C3-JFM},  we  quantify the  size  of the  thermal
anomaly  through   a  critical  thermal  radius   $R_c(t)$  where  the
temperature  at the  center of  the flow  $\Theta_b$ is  $1\%$ of  the
injection temperature,  i.e.  $\Theta_b(r=0)-\Theta_b(r=R_c)=0.99$. As
expected,  the thermal  anomaly is  larger when  relaxing the  thermal
boundary condition and changing the rheology $\eta(\theta)$ has almost
no        effect        on         its        evolution        (Figure
\ref{C4-ELAS_RRc_Rheol_Boundary}). In addition, the extent of the cold
fluid region $R(t)-R_c(t)$  is growing slightly slower  with time when
considering $\Omega=1$  in comparison to the  isothermal boundary case
$\Omega=10^5$   (Figure  \ref{C4-ELAS_RRc_Rheol_Boundary}).    In  the
following, we characterize  the thermal anomaly evolution  in the more
realistic case where $\Omega=1$ and $\eta(\theta)=\eta_2(\theta)$.

\begin{figure}[h!]
  \begin{center}
    \graphicspath{ {/Users/thorey/Documents/These/Projet/Refroidissement/Skin_Model/Figure/Figure_Heating/} }
    \includegraphics[scale=0.4]{ELAS_RRc_Rheol_Boundary.eps}
    \caption{Left:  Extent  of  the cold  fluid  region  $R(t)-R_c(t)$
      versus   dimensionless    time   for    different   combinations
      ($\eta$,$\Omega$)   indicated  on   the  plot,   $\nu=0.01$  and
      $Pe=1.0$. Same plot but for $Pe=100.0$.}
    \label{C4-ELAS_RRc_Rheol_Boundary}
  \end{center}
\end{figure}

As in Section \ref{C3-sec:char-therm-anom-e},  the size of the thermal
anomaly $R_c(t)$  is given by  the radius  where advection of  heat is
equal to heat loss
\begin{equation}
  \frac{d}{d    t}\left(\theta(r=   R_c,t)\right)    \approx   Pe^{-1}
  \frac{\partial^2}{\partial z^2}\left(\theta(r=R_c,t)\right)
  \label{C4-HeatequationThermal}
\end{equation}
which, by integration over the thickness of the flow $h$, becomes
\begin{eqnarray}
  \frac{d}{dt}\left(\int_0^h\theta           dz\right)-\Theta_s\frac{d
  h}{dt}&\approx& Pe^{-1} \frac{\Theta_b-\Theta_s}{h}\nonumber\\
  \overline{\theta}\frac{d h}{dt}+h\frac{d \overline{\theta}}{dt}-\Theta_s\frac{d
  h}{dt}&\approx& Pe^{-1}
                  \frac{\Theta_b-\Theta_s}{h}\nonumber\\
  \frac{2}{3}\left(\Theta_b-\Theta_s\right)\frac{d h}{d t} +h\frac{d\overline{\theta}}{dt}&\approx& Pe^{-1}
                                                                                                    \frac{\Theta_b-\Theta_s}{h}\label{C4-Calcul2}
\end{eqnarray}
where $\overline{\theta}$  is equal to $(\int_0^h  \theta dz)/h$ here.
Using     the     thickness     profile     (\ref{C3-IntrusionShape}),
(\ref{C4-Calcul2}) becomes
\begin{eqnarray}
  \alpha^2\left(1+\frac{R_c}{R}\right)^2\left(\frac{2}{3}\left(\Theta_b-\Theta_s\right)\frac{d h_0}{d
  t}+h_0\frac{d \overline{\theta}}{d
  t}\right)+&\nonumber\\
  \frac{8h_0R_c^2\left(\Theta_b-\Theta_s\right)}{3R^3}\frac{d
  R}{d
  t}\alpha\left(1+\frac{R_c}{R}\right)                        &\approx
                                                                \frac{Pe^{-1}\left(\Theta_b-\Theta_s\right)}{\alpha^2\left(1+\frac{R_c}{R}\right)^2h_0}
                                                                \label{C4-bill}
\end{eqnarray}
where $\alpha  (t)$ is the  normalized region beyond  $r=R_c(t)$, i.e.
$\alpha(t)= \left(R(t)-R_c(t)\right)/R(t)$. In  the limit $\alpha<<1$,
i.e.   $R_c/R\sim  1$,  and  neglecting  the  higher  order  terms  in
(\ref{C4-bill}) ($\propto  \alpha^2$), we obtain the  same scaling law
for the size of the normalized cold front region $\alpha$ than the one
found in Section  \ref{C3-sec:char-therm-anom-e}.  However, it clearly
does   not   match   the    prediction   when   $\Omega=1.0$   (Figure
\ref{C4-ELAS_RRc_RArrhenius_1-0})   and   the  new   thermal   anomaly
evolution must be linked to a  change in the heat advection rate, i.e.
the left  hand side term  in the balance  (\ref{C4-bill}).  Neglecting
the  advection  term  to  keep  only the  inflation  term  instead  in
(\ref{C4-bill}) leads to
\begin{equation}
  \alpha^2\left(1+\frac{R_c}{R}\right)^2\frac{d h_0}{d
    t}\approx \frac{Pe^{-1}}{\alpha^2\left(1+\frac{R_c}{R}\right)^2h_0}
\end{equation}
which, in the limit $\alpha<<1$, becomes
\begin{equation}
  \alpha^4\frac{\partial h_0}{\partial
    t} \approx \frac{Pe^{-1}}{h_0\frac{\partial h_0}{\partial t}}.
\end{equation}
Substituting    $h_0(t)$    by     its    respective    scaling    law
(\ref{C3-ScalingH-Visco}), the  relative size  of the  normalized cold
front region $\alpha$ reads
\begin{equation}
  \alpha(t) \propto h_f^{1/22}\nu^{1/11}Pe^{-1/4}t^{7/44}
\end{equation}
which is equivalent to
\begin{equation}
  R(t)-R_c(t) = 0.8h_f^{1/11}\nu^{2/11}Pe^{-1/4}t^{17/44}
  \label{C4-ScalingRRc-Heating}
\end{equation}
where the numerical prefactor, which  depends on the definition of the
thermal anomaly, has been chosen to fit the simulations.

\begin{figure}[h!]
  \begin{center}
    \graphicspath{ {/Users/thorey/Documents/These/Projet/Refroidissement/Skin_Model/Figure/Figure_Heating/} }
    \includegraphics[scale=0.4]{ELAS_RRc_RArrhenius_1-0.eps}
    \caption{a) Extent of the cold fluid region $R(t)-R_c(t)$ rescaled
      by $Pe^{-1/3}\nu^{7/33}$ versus  time for different combinations
      ($\nu$,$Pe$) indicated  on the  plot.  Dotted-line:  scaling law
      (\ref{C3-ScalingRRc})
      $(R(t)-R_c(t))Pe^{1/3}\nu^{-7/33}=    2.1    h_f^{7/66}t^{9/22}$
      derived in Section \ref{C3-sec:effect-visc-blist-e} b) Same plot
      but where  we rescale  the extent  of the  cold fluid  region by
      $Pe^{-1/4}\nu^{2/11}$.       Dotted-line:       scaling      law
      $(R(t)-R_c(t))Pe^{1/4}\nu^{-2/11}= 0.7  h_f^{1/11}t^{17/44}$. In
      all           simulations,            $\Omega=1.0$           and
      $\eta(\theta)=\eta_2(\theta)$.}
    \label{C4-ELAS_RRc_RArrhenius_1-0}
  \end{center}
\end{figure}

This new scaling law for the evolution of the extent of the cold fluid
region (\ref{C4-ScalingRRc-Heating}) shows a  much better fit with the
simulations  (Figure \ref{C4-ELAS_RRc_RArrhenius_1-0}  b). Indeed,  in
that case, the  constant heating of the surrounding  medium limits the
expansion of  the cold  fluid region in  comparison to  the isothermal
case. In  particular, the  thermal anomaly  reaches the  third bending
phase much  later and  therefore, the scaling  law corresponds  to the
evolution of the thermal anomaly in  the second bending phase; a phase
dominated  by  inflation.  Therefore,  the  evolution  of the  thermal
anomaly is  governed by the inflation  rate at the intrusion  tip when
relaxing the thermal boundary condition.

The cold fluid  region grows effectively slightly slower,  with a time
exponent  equal  to $17/44$  instead  of  $9/22$ ($17/44  \sim  0.38$,
$9/22\sim  0.40$) and  the dependence  in  the Peclet  number $Pe$  is
weaker, i.e.   it changes from  a power  $1/3$ to $1/4$.   Indeed, for
small $Pe$,  vertical diffusion is  efficient on the  emplacement time
scale and  rapidly heats up the  surrounding medium. The heat  loss in
the interior is  smaller and the thermal anomaly  larger in comparison
to the  case where $\Omega= 10^5$.   In contrast, for large  values of
$Pe$, advection dominates and the saving of heat by insulation is less
important decreasing  the overall  difference between small  and large
values of $Pe$.

\begin{figure}[h!]
  \begin{center}
    \graphicspath{ {/Users/thorey/Documents/These/Projet/Refroidissement/Skin_Model/Figure/Figure_Heating/} }
    \includegraphics[scale=0.4]{Visco_ELAS_Heating.eps}
    \caption{a)  Dimensionless effective  viscosity versus  time where
      the time has been rescaled by the time for the flow to enter the
      second phase $t_{b2}$. b) Same as  left but where we rescale the
      viscosity by $\nu$ and the time by $t_{b3}$. In all simulations,
      $\Omega=1.0$ and $\eta(\theta)=\eta_2$.}
    \label{C4-Visco_ELAS_Heating}
  \end{center}
\end{figure}

As  we  show  in Section  \ref{C4-sec:infl-therm-bound-el},  the  time
$t_{b2}$ for  the current to enter  the second bending phase  does not
change much  as the delay  induced by  the heating of  the surrounding
medium is offset  by the change in rheology.  Accordingly,  we use the
time      $t_{b2}$     (\ref{C3-tb2})      defined     in      section
\ref{C3-sec:effect-visc-blist-e}  as   the  time  to  cool   the  thin
prewetting film  to characterize the first  bending transition (Figure
\ref{C4-Visco_ELAS_Heating} a). In contrast, the time $t_{b3}$ for the
current to enter the third phase of the flow is now larger. Processing
as    in    Section   \ref{C3-sec:effect-visc-blist-e}    but    using
(\ref{C4-ScalingRRc-Heating})  for the  evolution  of  the cold  fluid
region $R(t)-R_c(t)$  instead of  (\ref{C3-ScalingRRc}), we  find that
$t_{b3}$ is given by
\begin{equation}
  t_{b3}=0.4h_f^{-4/17}\nu^{-8/17}Pe^{11/17}St_m^{-11/17}
  \label{C4-tb3}
\end{equation}

\section{Evolution in the gravity regime}
\label{C4-sec:evol-grav-regime}

As in chapter \ref{C3-JFM}, we now  consider the late time behavior in
which only the weight of the fluid contributes to the dynamic pressure
$P$. The governing equations are (\ref{C4-HF}) and (\ref{C4-TF}) where
$P=h$. We follow the same framework as in the previous section.

\begin{figure}[h!]
  \begin{center}
    \graphicspath{ {/Users/thorey/Documents/These/Projet/Refroidissement/Skin_Model/Figure/Figure_Heating/} }
    \includegraphics[scale=0.55]{Grid_PeOmega_GRAV_Berco_100.eps}
    \caption{Snapshots of  the flow thermal  structure $\theta(r,z,t)$
      for  different  sets  ($Pe$,$\Omega$)  with  $Pe=  1.0$  ,$10.0$
      ,$100.0$   and   $\Omega=10^5$   and  $1.0$   at   $t=100$   for
      $\nu=0.01$. The  thermal structure above the  intrusion is given
      by   (\ref{C4-eq11})  and   reads   in   a  dimensionless   form
      $\Theta_r(r,z,t)=\Theta_s(r,t)\operatorname{erfc}{\left(Pe^{1/2}\Omega_1\frac{(z-h)}{2\sqrt{t}}\right)}$
      where $\Omega_1$ is set to  $1$. The thermal structure below the
      intrusion is similar and not shown for clarity.}
    \label{C4-Grid_PeOmega_Heating_GRAV}
  \end{center}
\end{figure}

\subsection{Relaxing  the   thermal  boundary  condition,   effect  of
  $\Omega$}
\label{C4-sec:infl-therm-bound}

As in the  bending regime, for a small value  of $\Omega$, the heating
of  the  surrounding medium  insulates  the  current and  the  thermal
anomaly is larger. For instance, for $Pe=1$ and $\nu=0.01$ at $t=100$,
while  $R_c\sim 1$  for $\Omega=10^5$,  $R_c$ is  larger than  $5$ for
$\Omega   =1   $  (Figure   \ref{C4-Grid_PeOmega_Heating_GRAV}).    In
addition, after it detaches from  the current tip, the thermal anomaly
does  not reach  a steady-state  profile but  keeps growing  with time
instead (Figure \ref{C4-Grid_TIME_GRAV}).  Indeed,  in contrast to the
isothermal  boundary  case,  the  constant  increase  of  the  surface
temperature continuously decreases the heat loss in the central region
of the current which allows an expansion of the thermal anomaly.
\begin{figure}[h!]
  \begin{center}
    \graphicspath{ {/Users/thorey/Documents/These/Projet/Refroidissement/Skin_Model/Figure/Figure_Heating/} }
    \includegraphics[scale=0.45]{GridTime_GRAV_Bercovici_Pe1_Nu-2.eps}
    \caption{Snapshots of  the flow thermal  structure $\theta(r,z,t)$
      at  different  times  indicated   on  the  plot.   Dashed  lines
      represent  the thermal  boundary  layers. Solid  grey lines  are
      isotherms for  $\theta =  0.2$, $0.4$,  $0.6$ and  $0.8$.  Here,
      $\nu=0.01$,  $Pe =1.0$  and $St_m  = 1$.  The thermal  structure
      above the intrusion  is given by (\ref{C4-eq11}) and  reads in a
      dimensionless                                               form
      $\Theta_r(r,z,t)=\Theta_s(r,t)\operatorname{erfc}{\left(Pe^{1/2}\Omega_1\frac{(z-h)}{2\sqrt{t}}\right)}$
      where $\Omega_1$ is set to  $1$. The thermal structure below the
      intrusion is similar and not shown for clarity.}
    \label{C4-Grid_TIME_GRAV}
  \end{center}
\end{figure}

For small  values of $Pe$,  the important heating of  the surroundings
medium results in an flow that is almost vertically isothermal (Figure
\ref{C4-Grid_PeOmega_Heating_GRAV}  and \ref{C4-Grid_TIME_GRAV}).   In
contrast, for large values of $Pe$,  the vertical diffusion of heat is
less efficient,  the thermal aureole  is restricted to a  small region
around  the  intrusion,   the  thermal  anomaly  is   larger  and  the
temperature   gradient   within   the  flow   are   stronger   (Figure
\ref{C4-Grid_PeOmega_Heating_GRAV}).
\begin{figure}[h!]
  \begin{center}
    \graphicspath{ {/Users/thorey/Documents/These/Projet/Refroidissement/Skin_Model/Figure/Figure_Heating/} }
    \includegraphics[scale=0.45]{Scaling_HR_GRAV_Omega.eps}
    \caption{Left: Dimensionless thickness at  the center $h_0$ versus
      dimensionless   time  $t$   for  different   sets  $(\Omega,Pe)$
      indicated    on   the    plot.     Dotted-line:   scaling    law
      $h_0=  0.7h_f^{-1/11}\nu^{-2/11}t^{8/22}$  for   $\nu  =  0.01$.
      Right: Dimensionless  radius $R$  versus dimensionless  time $t$
      for  the  same  sets $(\Omega,Pe)$.   Dotted-line:  scaling  law
      $R= 2.2h_f^{1/22}\nu^{1/11}t^{7/22}$  for $\nu = 0.01$.   In all
      simulations, $\nu=0.01$ and $\eta(\theta)=\eta_1$.}
    \label{C4-Scaling_HR_GRAV_Omega}
  \end{center}
\end{figure}

While three  phases also characterize the  dynamics when $\Omega=1.0$,
their durations  are modified  by the  new thermal  boundary condition
(Figure \ref{C4-Scaling_HR_GRAV_Omega}).   In particular,  the current
remains  hot for  a longer  period  of time  and the  second phase  is
delayed in comparison to the case where $\Omega =10^5$.  For instance,
for  $\nu=0.01$ and  $Pe=1.0$, while  the first  transition occurs  at
$t \approx 0.1$ for $\Omega=10^5$,  it happens only after $t\approx 1$
for  $\Omega=1$   (Figure  \ref{C4-Scaling_HR_GRAV_Omega}).    As  the
thermal  anomaly does  not reach  a steady  state for  $\Omega=1$, the
cooling of the current in the second gravity phase is also slower than
for $\Omega=10^5$  and the current  reaches the third phase  also much
later for $\Omega=1$  (Figure \ref{C4-Scaling_HR_GRAV_Omega}).  In the
next  section, we  consider the  effect  of the  first order  Arhenius
rheology on the dynamics for $\Omega=1.0$.

\subsection{Considering   a  more   realistic   rheology,  effect   of
  $\eta(\theta)$}
\label{C4-sec:cons-more-real-1}
 
\begin{figure}[h!]
  \begin{center}
    \graphicspath{ {/Users/thorey/Documents/These/Projet/Refroidissement/Skin_Model/Figure/Figure_Heating/} }
    \includegraphics[scale=0.45]{Scaling_HR_GRAV_Rheology.eps}
    \caption{Left: Dimensionless thickness at  the center $h_0$ versus
      dimensionless time $t$ for  different sets $(\eta,Pe)$ indicated
      on      the      plot.        Dotted-line:      scaling      law
      $h_0= 0.7h_f^{-1/11}\nu^{-2/11}t^{8/22}$ for $\nu=0.01$.  Right:
      Dimensionless radius  $R$ versus dimensionless time  $t$ for the
      same    sets    $(\eta,Pe)$.      Dotted-line:    scaling    law
      $R=  2.2h_f^{1/22}\nu^{1/11}t^{7/22}$ for  $\nu  =0.01$. In  all
      simulations, $\nu=0.01$ and $\Omega=1$.}
    \label{C4-HR_GRAV_Rheology}
  \end{center}
\end{figure}

As  in the  bending regime,  the chosen  rheology $\eta(\theta)$  also
affects the timing for the phase transition, and, in particular, these
transitions  occur  sooner  for  the first  order  Arrhenius  rheology
$\eta_2(\theta)$ than  for $\eta=\eta_1(\theta)$.  In  particular, the
delay  induced  in  the  phase  transitions  by  the  heating  of  the
surrounding  medium is  almost  offset by  the  first order  Arrhenius
rheology.  For  instance, the transition  to the second  gravity phase
occurs  around   the  same  time   for  a  current   characterized  by
$\eta(\theta)   =\eta_1$   and   $\Omega=10^5$  as   for   a   current
characterized  by  $\eta(\theta)  =\eta_2$ and  $\Omega=1.0$  (Compare
Figure \ref{C4-Scaling_HR_GRAV_Omega} and \ref{C4-HR_GRAV_Rheology}).

\subsection{Characterization of the thermal anomaly}
\label{C4-sec:char-therm-anom-2}

As in the bending regime, the thermal anomaly is first attached to the
tip of the current, i.e. $R_c(t)/R(t)=1$. After a time that depends on
$Pe$ as well  as $\nu$, the thermal anomaly detaches  from the tip and
follows  its own  evolution. However,  in contrast  to the  isoviscous
case, the  thermal anomaly does not  reach a steady state  and $R_c/R$
does     not     evolve      as     $t^{-1/2}$     anymore     (Figure
\ref{C4-GRAV_RRc_RArrhenius_1-0} a).  We modify  the thermal budget of
Section \ref{C3-sec:char-therm-anom-g}  to account for the  heating of
the surrounding medium.

\begin{figure}[h!]
  \begin{center}
    \graphicspath{ {/Users/thorey/Documents/These/Projet/Refroidissement/Skin_Model/Figure/Figure_Heating/} }
    \includegraphics[scale=0.4]{GRAV_RRc_1-0.eps}
    \caption{a) Normalized thermal anomaly radius $R_c(t)/R(t)$ versus
      time for  different combinations  ($\nu$,$Pe$) indicated  on the
      plot.     Dotted-line:   Scaling    law    found   in    Section
      \ref{C3-sec:char-therm-anom-g}  $R_c(t)/R(t)\sim  t^{-1/2}$  for
      comparison.  b)  Same plot but  where we rescale  the normalized
      thermal anomaly  by $Pe^{1/4}\nu^{-1/8}$.   Dotted-line: scaling
      law  $(R_c(t)/R(t))Pe^{-1/4}\nu^{1/8}=   1.8t^{-1/4}$.   In  all
      simulations, $\Omega=1.0$ and $\eta(\theta)=\eta_1$.}
    \label{C4-GRAV_RRc_RArrhenius_1-0}
  \end{center}
\end{figure}

When  the thermal  anomaly has  detached from  the intrusion  front, a
balance between heat advection and diffusion in the surrounding medium
in a dimensional form reads
\begin{equation}
  \rho C_p U_0 \frac{\Delta T}{R_c} \approx k_m \frac{\Delta T}{h_0^2}
  \label{C4-bilan}
\end{equation}
where $\Delta  T$ is the  mean temperature contrast between  the fluid
and the  surroundings and $U_0$  is a redistribution of  the injection
rate at  $r=R_c$, i.e.   $U_0=Q_0/(2\pi R_c  h_0)$.  In  addition, the
continuity of the heat flux at the boundary (\ref{C4-2Flux_1}) imposes
\begin{equation}
  k_m\frac{\Delta   T}{h_0}\approx   k_s   \frac{\Delta   T}{\sqrt{\pi
      \kappa_r t}}.
  \label{C4-FluxEstimate}
\end{equation}
Injecting (\ref{C4-FluxEstimate}) and the  expression for the velocity
into (\ref{C4-bilan}) gives
\begin{equation}
  R_c \approx  \left(\frac{Q_0\kappa_r^{1/2}}{\kappa_m k_s}\right)^{1/2}
  t^{1/4}.
  \label{C4-Bilan2}
\end{equation}
By non-dimensionalizing (\ref{C4-Bilan2}), we  obtain the evolution of
the   thermal   anomaly   when   it  has   detached   from   the   tip
$R_c(t)\sim \Omega^{-2}Pe^{1/4}t^{1/4}$ and hence
\begin{equation}
  \frac{R_c(t)}{R(t)} = 1.8\Omega^{-2}Pe^{1/4}\nu^{-1/8}t^{-1/4}
  \label{C4-Rc}
\end{equation}
where   we  have   used  the   scaling   law  for   $R(t)$  given   by
(\ref{C3-scaling-R-gravi-2})  and   the  numerical   prefactor,  which
depends on the  definition of the thermal anomaly, has  been chosen to
fit  the  simulations.  The  scaling  law,  which  is only  valid  for
$\Omega = O(1)$,  indeed closely fits the  simulations. In particular,
both  the dependence  with the  Peclet number  $Pe$ and  the viscosity
contrast  vanishes  when  rescaling  by  $Pe^{1/4}\nu^{-1/8}$  (Figure
\ref{C4-GRAV_RRc_RArrhenius_1-0} b).

\begin{figure}[h!]
  \begin{center}
    \graphicspath{ {/Users/thorey/Documents/These/Projet/Refroidissement/Skin_Model/Figure/Figure_Heating/} }
    \includegraphics[scale=0.4]{Visco_GRAV_Heating.eps}
    \caption{a)  Dimensionless effective  viscosity versus  time where
      the time has been rescaled by the time for the flow to enter the
      second phase $t_{g2}$. b) Same as  left but where we rescale the
      viscosity  by   $\nu$  and  the   time  by  $t_{g3}$.    In  all
      simulations, $\Omega=1.0$ and $\eta(\theta)=\eta_1$.}
    \label{C4-Visco_GRAV_Heating}
  \end{center}
\end{figure}

The time  $t_{g2}$ for the current  to enter the second  gravity phase
does  not change  much as  the  delay induced  by the  heating of  the
surrounding medium is offset by  the change in rheology.  Accordingly,
we  use the  time $t_{g2}$  (\ref{C3-tg2}) to  characterize the  first
gravity   transition  (Figure   \ref{C4-Visco_GRAV_Heating}  a).    In
contrast, the time  $t_{g3}$ for the current to enter  the third phase
of   the   flow   is   now   larger.    Processing   as   in   Section
\ref{C3-sec:effect-visc-blist-g}  but  using   (\ref{C4-Rc})  for  the
evolution  of  normalized  thermal anomaly  $R_c(t)/R(t)$  instead  of
(\ref{C3-Scaling-Rc-Gravy}), we find that $t_{g3}$ is given by
\begin{equation}
  t_{g3}= 80 \Omega^{-8}\nu^{-1/2}Pe St_m^{-1}
  \label{C4-tg3}
\end{equation}

\begin{table}[h!]
  \begin{center}
    \begin{tabular}{c|cc|cc}
      Name&From&To&Expression\\
      \hline
      $t_t$&Bending&Gravity&$6.5(\eta_e/\eta_h)^{2/7}h_f^{-1/7}$\\
      $t_t^h$&Bending&Gravity&$6.5h_f^{-1/7}$\\
      $t_t^c$&Bending&Gravity&$6.5\nu^{-2/7}h_f^{-1/7}$\\
      Bending regime&\multicolumn{2}{c}{}& \\
      $t_{b2}$&Phase 1& Phase 2&$0.1 Pe St_m^{-1} h_f^2$\\
      $t_{b3}$&Phase 2& Phase 3 &$0.4 h_f^{-4/17} St_m^{-11/17}Pe^{11/17}\nu^{-8/17}$\\
      Gravity regime&\multicolumn{3}{c}{} \\
      $t_{g2}$ &Phase 1& Phase 2 &$10^{-2}PeSt_m^{-1}$\\
      $t_{g3}$ &Phase 2& Phase 3 &$ 80Pe St_m^{-1}\nu^{-1/2}$\\
    \end{tabular}
    \caption{Summary of the different  transition times.  $t_t$ is the
      transition time  between bending and  gravity which is  bound by
      $t_t^h$,  when  the current  transitions  in  the first  bending
      thermal phase, and $t_t^c$, when  the current transitions in the
      third  bending   thermal  phase.   $t_{b2}$   (resp.   $t_{b3}$)
      represents  the time  to  transition  from phase  1  to phase  2
      (resp. from phase 2 to phase  3) in the bending regime. $t_{g2}$
      (resp. $t_{g3}$) represents the time  to transition from phase 1
      to  phase 2  (resp. from  phase  2 to  phase 3)  in the  gravity
      regime. }
    \label{tab:TimeTransition}
  \end{center}
\end{table}
\section{Evolution  with bending  and  gravity in  the more  realistic
  model}
\label{C4-sec:evol-with-bend}

In the  previous chapter,  we showed  that the  final evolution  of an
elastic-plated gravity  current depends on the  relative phase changes
within  each regime  and the  transition between  the bending  and the
gravity regime  itself.  The  Arrhenius rheology  tends to  offset the
delay caused by the heating of the surrounding medium and overall, the
phase  diagram  presented in  section  (\ref{C3-sec:diff-evol-with-1})
shows         only         minor         modifications         (Figure
\ref{C4-Phase_Diagram_Heating}).  Except for  the transitions from the
third bending phase to the second  and third gravity phases, which are
shifted  to  the  left,  the  phase diagram  is  indeed  not  modified
(Appendix  \ref{chap:A2}). Therefore,  in  the framework  of our  more
realistic model, the current is only  more likely to transition to the
gravity  regime  before reaching  the  third  bending phase.   In  the
following,  we   look  at   the  observations  discussed   in  chapter
\ref{chap2} in the light of our new model.

\begin{figure}[h!]
  \begin{center}
    \graphicspath{ {/Users/thorey/Documents/These/Projet/Refroidissement/Skin_Model/Figure/Figure_Heating/} }
    \includegraphics[scale=0.6]{PhaseDiagramJFM_Heating.eps}
    \caption{Phase diagram for the  evolution with bending and gravity
      for  the  more realistic  case  discussed  in this  chapter  for
      different  combinations  ($\nu$,$Pe_m$)  and a  given  value  of
      $h_f =  0.005$.  $B_iG_j$ refers  to the case where  the current
      transitions from the  $i$th bending thermal phase  to thew $j$th
      gravity thermal phase where $i$ and $j \in \{1,2,3\}$.}
    \label{C4-Phase_Diagram_Heating}
  \end{center}
\end{figure}

\section{Application to the spreading of shallow magmatic intrusions}
\label{C4-sec:appl-spre-shall}

\subsection{Elba Island christmas-tree laccolith complex}
\label{C4-sec:appl-arrest-terr}

The isoviscous elastic-plated  gravity current model has  been used in
Chapter   \ref{chap2}  to   study  the   laccoliths  at   Elba  Island
\citep{Michaut:2011kg}. It shows that,  while their final morphologies
are  consistent  with  their  arrest  in  the  bending  regime,  their
dimensions  require unreasonable  magma  viscosity to  agree with  the
isoviscous  model  (Chapter  \ref{chap2}).   In  addition,  given  the
fracture  toughness  of  rocks,  their  radii seem  too  small  to  be
fractured controlled  and their  arrest might  be better  explained by
their cooling  \citep{Michaut:2011kg}.  In  the following,  we compare
the  new model  predictions  to  the size  of  laccoliths provided  by
\citet{Rocchi:2002jy}.  In order to account for the intrinsic scale of
different settings for  each intrusion and compare them  to the model,
the  data have  first  to be  nondimensionalized using  characteristic
values for the model parameters.

\begin{table}[h!]
  \caption{Range of values for the model parameters}
  \begin{center}
    \scalebox{0.8}{
      \begin{tabular}{c|c|c|c|c}
        Parameters& Symbol & Earth & Moon&Unit\\
        \hline
                  &&&&\\
        Depth of intrusion & $d_c$ & $0.2-2.7$ &$0.5-1.5$ &km \\
        Young's Modulus & $E$ & $10$ &$10$ &GPa \\
        Poisson's ratio & $\nu^*$ & $0.25$ &$0.25$ &\\
        Gravity & $g$ & $9.81$ &1.62&m s$^{-2}$ \\
        Magma density & $\rho_{m}$ & $2500-2900$ &$2900$&kg m$^{-3}$ \\
        Liquidus magma viscosity & $\eta_h $ & $10^2-10^{6}$ &$1-10$&Pa s \\
        Solidus magma viscosity & $\eta_c $ & $10^6-10^{10}$ &$10^3-10^5$&Pa s \\
        Feeder dyke width & $a$ & $1-100$ &$10$&m \\
        Depth of the melt source & $Z_{c}$ & $ 1-10$&$ 500$& km \\ 
        Initial overpressure & $\Delta P$ & $20-50$ &$50$ &MPa \\
        Injection rate & $Q_{0}$ &$0.1-10^3$ &$1-10^4$&m$^{3}$ s$^{-1}$ \\
        Magma thermal conductivity &$k_m$& $2.5$& $2.5$ &
                                                          W
                                                          K$^{-1}$
                                                          m$^{-1}$\\
        Magma  thermal  diffusivity  &$\kappa_m$& $10^{-6}$  &$10^{-6}$  &
                                                                           m$^{2}$
                                                                           s$^{-1}$\\
        Magma  liquidus  temperature  &$T_L$   &  $900$-$1200$&  $1200$  &
                                                                           $\celsius$ \\
        Magma solidus temperature &$T_S$ & $700$-$1000$& $1000$& $\celsius$
        \\
        Magma heat  capacity &  $C_p$ & $4.18  \times 10^5$  &$4.18 \times
                                                               10^5$&        J
                                                                      kg$^{-1}$ K$^{-1}$\\
        Latent       heat       of      crystallization&       $L$       &
                                                                           $4.18\times10^5$&$4.18\times10^5$& J kg$^{-1}$\\
        Rock  thermal  diffusivity  &$\kappa_r$& $10^{-6}$  &$10^{-6}$  &
                                                                          m$^{2}$
                                                                          s$^{-1}$\\
                  &&&&\\
        \hline
        Characteristic scales & Symbol & Earth & Moon&Unit\\
        \hline
                  &&&&\\
        Height scale & $H$& $0.1-10$ &$0.1-1$ &m \\
        Length scale & $\Lambda$ & $1-7$&$2.2-12$& km \\
        Time scale & $\tau$ & $10^{-3}-100$&$10^{-3}-10$& years \\
                  &&&&\\
        \hline
        Dimensionless number & Symbol & Earth & Moon&\\
        \hline
                  &&&&\\
        Peclet number &$Pe$& $10^{-4}-500$&$10^{-3}-10^4$ &\\
        Viscosity contrast & $\nu$ & $10^{-4}-10^{-10}$& $10^{-3}-10^{-5}$&\\
        Modified Stefan number & $St_m$ & $0.1-0.5$ &  $0.1-0.5$ &\\
                  &$\Omega$ & $1$ & $1$&
                                         \label{C4-tab2}
      \end{tabular} 
    }
  \end{center}
  \label{C4-tab}
\end{table}

\subsubsection*{Range of values for the dimensionless numbers}
\label{C4-sec:range-valu-dimens}
 
The  different parameters  along  with a  discussion  on the  possible
values for  $h_f$ have  been provided in  Chapter \ref{chap2}  and are
summarized  in Table  \ref{C4-tab}.  We  refer the  reader to  Section
\ref{C2-sec:observ-vs-pred-Earth}   for  more   details  about   their
derivation.   In the  following, we  quantify  the values  of the  new
dimensionless numbers related to the cooling  of the flow for the case
of the Elba Island laccoliths.

For a latent heat of crystallization $L = 4.18\times10^5$ J kg$^{-1}$,
a   difference  between   solidus  temperature   $T_S$  and   liquidus
temperature  $T_L$ between  $100$ K  and  $300$ K,  the number  $St_m$
varies from $0.1$  to $0.5$.  For a thermal diffusivity  for the magma
equal  to $\kappa_m=  10^{-6}$  m s$^{-2}$,  an  injection rate  $Q_0$
between $0.1$ and $100$ m$^3$  s$^{-1}$ and an intrusion depth between
$0.2$ and $2.7$  km, the Peclet number varies from  $10^{-3}$ to $100$
and therefore,  $Pe_m$ varies from  $0.01$ to $1000$. The  increase in
viscosity upon cooling varies from $4$ to $6$ for mafic magmas and can
be   up   to   $10$   orders    of   magnitude   for   felsic   magmas
\citep{Anonymous:CZVBrBvv,Lejeune:1995fc,Giordano:2008em,Diniega:2013eh}.
We  thus  consider  that  the viscosity  contrast  $\nu$  ranges  from
$10^{-4}$ to $10^{-10}$.

It  is generally  assumed  that  the magma  stops  spreading when  its
crystal   content  becomes   close  to   its  maximum   packing,  i.e.
$\phi  \sim  60\%$   \citep{Pinkerton:1992fwa}.   Beyond  this  point,
crystal collisions  dominate and  the viscosity  jumps to  much higher
values \citep{Lejeune:1995fc,Giordano:2008em}.  We assume that this is
equivalent to  $\eta_e$ tending to  $\eta_c$ in our model.   With this
assumption, the  model thus predicts  that a magmatic  intrusion would
solidify as a laccolith upon reaching the third bending phase.

\subsubsection*{Do laccoliths stop in the bending regime ?}
\label{C4-sec:range-valu-dimens}

The dimensionless thickness  $h_0$ as a function  of its dimensionless
radius $R$  for a current  that solidifies in  the third phase  of the
bending   regime    can   be    derived   from   the    scaling   laws
(\ref{C3-ScalingH-Visco})  and  (\ref{C3-ScalingR-Visco})  and  should
follow
\begin{equation}
  h_0 = 0.3 h_f^{-1/7}\nu^{-2/7}R^{8/7}\label{C4-Hr}
\end{equation}
Using the parameters listed in Figure \ref{C4-Data}, the dimensionless
observations show a very good agreement with the model for a viscosity
contrast      close      to      $8$     orders      of      magnitude
($\nu =  6.9\pm 2.3 \times  10^{-8}$, $r^2=0.9$), which  is consistent
with  the  felsic composition  of  these  laccoliths, and  $h_f=0.001$
(Figure \ref{C4-Data} a) \citep{Marsh:1981dc,Diniega:2013eh}.  Varying
$h_f$ has only minor effect on  the best fit viscosity contrast and is
discussed  in Appendix  \ref{chap:A2}.  This  value for  the viscosity
contrast also  depends on the  chosen value  for the height  scale $H$
whose main uncertainty  is on the liquidus viscosity  $\eta_h$ and the
injection rate  $Q_0$.  Indeed,  the larger  this two  parameters, the
larger the height scale, the smaller the dimensionless thicknesses and
therefore, the smaller the best fit viscosity contrast.  Nevertheless,
introducing the  cooling in  the elastic-plated gravity  current model
allows to reconciliate  the model predictions and  the observations in
the  case  of laccoliths  (Chapter  \ref{chap2}).   The shape  and  in
particular, the  large thickness of  the laccoliths at Elba  Island is
now entirely consistent with the  model predictions and therefore with
their arrest in the bending regime. In the following, we use the phase
diagram  proposed  in  Section \ref{C4-sec:evol-with-bend}  to  better
constrain the viscosity contrast along with $\eta_h$ and $Q_0$.

\begin{figure}[h!]
  \begin{center}
    \graphicspath{ {/Users/thorey/Documents/These/Projet/Refroidissement/Skin_Model/Figure/Figure_Heating/} }
    \includegraphics[scale=0.4]{Dimensionlessdata_V9.eps}
    \caption{a)  Dimensionless maximum  thickness $h_0$  versus radius
      $R$ for laccoliths from Elba  Island and revised low-slope lunar
      domes.  Parameters  for calculating $\Lambda$  (\ref{C3-L1}) and
      $H$    (\ref{C3-H1})    are    $E=10^9$    GPa,    $\nu^*=0.25$,
      $\rho_m = 2500$ kg m$^{-3}$, $g=9.81$ m s$^{-2}$, $\eta_h =10^6$
      Pa s and $Q_0 = 10$ m$^3$ s$^{-1}$ on Earth and, everything else
      being equal, $g=1.62$ m s$^{-2}$, $\eta_h  =1$ Pa s on the Moon.
      Dotted  lines:   best  fit   scaling  laws   (\ref{C4-Hr})  with
      $h_f = 0.001$ for laccoliths  at Elba Island (red) and low-slope
      lunar  domes  (purple).   $\nu   =  6.9\pm  2.3  \cdot  10^{-8}$
      ($r^2=0.92$) and  $\nu = 1.8\pm 0.4  \cdot 10^{-8}$ ($r^2=0.88$)
      represent the linear least square best fit for the data on Earth
      and   the  Moon   respectively.    b)  Dimensionless   thickness
      $\hat{h_0}$ versus $\hat{R}$ where $\hat{h_0}$ and $\hat{R}$ are
      given by (\ref{C4-Scaling2}) with  $h_f=0.001$ for laccoliths at
      Elba Island.  Substituting  (\ref{C3-T1}) into (\ref{C3-Pe}), we
      obtain $Pe = Q_0 H  /(\pi \kappa \Lambda^2)$; the parameters for
      calculating $Pe$ for each laccolith are the same than those used
      for the  nondimensionalization, $\kappa=10^{-6}$ m  s$^{-2}$ and
      $St_m$ is considered  constant and set to  $0.1$.  The viscosity
      contrast is set  to $\nu =6.9\cdot 10^{-8}$  for all laccoliths.
      Dotted line: scaling law $ \hat{h_0} \sim 0.3\hat{R}^{8/7}$.}
    \label{C4-Data}
  \end{center}
\end{figure}

\begin{figure}[h!]
  \begin{center}
    \graphicspath{ {/Users/thorey/Documents/These/Projet/Refroidissement/Skin_Model/Figure/Figure_Heating/} }
    \includegraphics[scale=0.5]{PhaseDiagreRocchie.eps}
    \caption{Subset  of   the  phase   diagram  proposed   in  section
      \ref{C4-sec:evol-with-bend}   relevant   for    the   study   of
      terrestrial laccoliths. Red and purple crosses represent a range
      of  value for  $\nu$ and  $Pe$  for Elba  Island laccoliths  and
      low-slope lunar domes  respectively. The width of  each cross is
      defined by  the minimum and  the maximum value obtained  for the
      Peclet number given the range  of variation of parameters listed
      in  table \ref{C4-tab}  and  the injection  rate  $Q_0$ and  the
      viscosity at the liquidus  temperature $\eta_h$ indicated on the
      plot. The  height of  the cross corresponds  to the  minimum and
      maximum   values  for   the  viscosity   contrast  obtain   from
      (\ref{C4-Hr}) when $h_f=0.001$.}
    \label{C4-PhaseDiag}
  \end{center}
\end{figure}

\subsubsection*{What can we learn from the phase diagram ?}

Assuming that  the intrusion stops  when it reaches the  third bending
phase,     the      phase     diagram     proposed      in     section
\ref{C4-sec:evol-with-bend}  simplifies  (Figure  \ref{C4-PhaseDiag}).
It shows that sills and laccoliths are two specific end member regions
as a function of $Pe$ and  $\nu$. In particular, while the top portion
of the phase diagram corresponds  to magmatic intrusions more mafic in
composition,  the  bottom  should  be more  representative  of  felsic
magmatic intrusions.  The  boundary between the two  regions show that
felsic magmatic intrusions should solidify  as sills on a larger range
of number $Pe$. Indeed, felsic  magmatic intrusions tend to be thicker
than their  mafic counterparts.  In  the framework of our  model, they
then stay hot for a longer period of time and therefore, can reach the
gravity  regime more  easily.  Given  the felsic  composition of  Elba
island laccoliths,  this phase diagram  can then be used  to constrain
their Peclet  number, and  therefore $Q_0$ and  $\eta_h$, for  them to
fall in the  laccolith region. In the following, we  use this approach
to constrain the injection rate $Q_0$ for these laccoliths.

We first compute  a value for the Peclet number  for each laccolith at
Elba  Island. Injecting  the scales  (\ref{C3-L1}), (\ref{C3-H1})  and
(\ref{C3-T1}) in the expression of $Pe$ (\ref{C3-Pe}) gives
\begin{equation}
  Pe \approx 4.8 Q_{0}^{5/4}\eta_{h}^{1/4}\rho_m^{1/4}g^{1/4}E^{-1/2} d_c^{-3/2} \kappa_m^{-1}
\end{equation}
where we  set the  Poisson's ratio to  $\nu^*=0.25$. Except  for $Q_0$
,that we want  to constrain, and $d_c$, which is  given and depends on
the  laccolith,  we  take  the   parameter  values  listed  in  Figure
\ref{C4-Data}.  Therefore, $Pe$ reads
\begin{equation}
  Pe \approx 6\times10^4 d_c^{-3/2} Q_0^{5/4}
  \label{PeExpre}
\end{equation}
For $St_m  =0.1$, the intrusion depths  given by \citet{Rocchi:2002jy}
and  $Q_0 =10$  m$^{3}$ s$^{-1}$,  the modified  Peclet number  $Pe_m$
ranges from  $1$ to  $6$.  As  the best  fit range  of values  for the
viscosity  contrast  associated  with  $Q_0=10$  m$^{3}$  s$^{-1}$  is
$\nu  = 6.9\pm  2.3  \times  10^{-8}$, as  discussed  in the  previous
Section, the phase diagram thus  predicts that these laccoliths should
have stop in the gravity regime (Figure \ref{C4-PhaseDiag}). Hence, we
might have overestimate the injection  rate.  Indeed, taking a smaller
value for the injection rate of $Q_0=0.1$ m$^{3}$ s$^{-1}$, reasonable
for viscous felsic magmas  \citep{Harris:2000jd}, the height scale $H$
is smaller  and the dimensionless  thicknesses are larger.   The model
thus       predicts       a      larger       viscosity       contrast
$\nu=1.2\pm0.4\times  10^{-9}$, still  consistent  with  the range  of
expected values for felsic magmas,  and weaker Peclet numbers.  In the
end,  the range  of values  for  the dimensionless  numbers now  falls
within the laccolith  regions and is consistent  with the observations
(Figure \ref{C4-PhaseDiag}).

\subsubsection*{Is   conduction   cooling   enough   to   solifidy   a
  laccolith ?}

If the laccoliths stopped spreading as  soon as they reached the third
phase of the  bending regime, the variance in thickness  and radius in
between  the   different  intrusions  should  also   be  explained  by
variations  in the  Peclet number,  most likely  due to  variations in
intrusion  depths  in  this   example.   Indeed,  the  time  $t_{b3}$,
necessary  to reach  the third  bending phase,  the thickness  and the
radius  of the  current at  this time  all depend  on the  combination
($\nu$,$Pe_m$) considered (see Section \ref{C4-sec:evol-with-bend}).

\begin{figure}[h!]
  \begin{center}
    \graphicspath{ {/Users/thorey/Documents/These/Projet/Refroidissement/Skin_Model/Figure/Figure_Heating/} }
    \includegraphics[scale=0.45]{Visco_PE_Heating.eps}
    \caption{a)  Dimensionless thickness  at the  center $h_0$  versus
      dimensionless radius $R$ for different sets $(\nu,Pe)$ indicated
      on  the plot  ($\eta(\theta)=\eta_2$, $\Omega=1.0$).   Pentagons
      refer to the  size where the effective viscosity  of the current
      equal   $70\%$  of   the   maximum   viscosity  $\eta_c$,   i.e.
      $\eta_e=0.7\eta_c$.   b)   Dimensionless  thickness  $\hat{h_0}$
      versus  dimensionless  radius  $\hat{R}$ where  $\hat{h_0}$  and
      $\hat{R}$ are given by (\ref{C4-Scaling2}) with $h_f=0.001$.  As
      expected,  after  rescaling $h_0$  and  $R$,  the sizes  of  the
      solidified  laccoliths  should  collapse   almost  on  the  same
      point. }
    \label{C4-Visco_PE_Heating}
  \end{center}
\end{figure}


To test  this hypothesis,  we first  rescale the  time using  the time
$t_{b3}$ (\ref{C4-tb3}) as follow
\begin{equation}
  \hat{t}= h_f^{4/17} Pe_m^{-11/17}\nu^{8/17}t
  \label{C4-tRescale}
\end{equation}
where  $\hat{t}$ is  the new  variable. Injecting  (\ref{C4-tRescale})
into (\ref{C3-ScalingH-Visco}) and  (\ref{C3-ScalingR-Visco}), we find
the corresponding thickness $\hat{h_0}$ and radius $\hat{R}$ read
\begin{eqnarray}
  \hat{R}&=& h_f^{1/34}Pe_m^{-7/34}\nu^{1/17}R\label{HatR}\\
  \hat{h_0}&=& h_f^{3/17}Pe_m^{-4/17}\nu^{6/17}h_0\label{Hath0}
\end{eqnarray}
In terms of  $\hat{h_0}$ and $\hat{R}$, the  scaling law (\ref{C4-Hr})
rewrites $ \hat{h_0} \sim 0.3\hat{R}^{8/7}$ and does not depend on the
dimensionless numbers anymore  (Figure \ref{C4-Visco_PE_Heating}). For
laccoliths at Elba Island, we  use a constant viscosity contrast equal
to  $\nu =  8.2\cdot 10^{-9}$,  $h_f=0.001$  and we  compute a  Peclet
number for  each laccolith according to  (\ref{PeExpre}) with $Q_0=10$
m$^3$ $s^{-1}$. However,  the different laccoliths do  not collapse on
the same dot after rescaling (Figure \ref{C4-Data} b).  In particular,
the  dependence  in $Pe$  of  our  scaling, resulting  from  different
intrusion depths, is not enough to explain the variability in the size
of   terrestrial  laccoliths.    An   additional  cooling   mechanism,
amplifying the effect  of $Pe$, is thus required to  explain the exact
extent of laccoliths. This could  be extraction of heat by circulation
of fluid \citep{Senger:2014tt}.  To test this hypothesis, we look at
the low-slope  domes on the Moon  where conduction is most  likely the
only source of cooling.

\subsection{Low-slope lunar domes}
\label{sec:appl-arrest-terr-1}

Because the Moon is supposed  to be depleted of volatiles, circulation
of fluid in the lunar crust is likely to be very limited and the model
developed  in this  chapter  should be  appropriate  for studying  the
cooling of  low-slope lunar domes.   In this section, we  restrict our
analysis  to  some  specific  domes whose  characteristics  have  been
precisely   revisited  by   Mélanie   Thiriet   (Purple  dots   Figure
\ref{C2-Corry_Rocchie},   see  Section   \ref{C2-sec:observ-vs-pred}).
Their  shapes  and  characteristics  have already  been  discussed  in
Chapter \ref{chap2} and  hereafter, we look at their  dimension in the
light of the cooling elastic-plated gravity current model.

\subsubsection*{Range of values for the dimensionless numbers}
\label{C4-sec:range-valu-dimens}

Parameters  for  a  lunar  setting  have  been  discussed  in  Chapter
\ref{chap2} and are summarized  in Table \ref{C4-tab}.  In particular,
for the  same injection rate,  the smaller gravity, together  with the
higher  density and  the smaller  viscosity of  lunar magmas,  lead to
smaller Peclet numbers.  For instance, for an intrusion $1.5$ km deep,
using  $g=1.62$  m s$^{-2}$,  $\eta_h=1$  Pa  s and  $\rho_m=2900$  kg
m$^{-3}$  instead  of $g=9.81$  m  s$^{-2}$,  $\eta_h=10^6$ Pa  s  and
$\rho_m=2500$  kg m$^{-3}$  lead  to  a Peclet  number  two orders  of
magnitude  smaller on  the Moon  than  on Earth,  i.e.  $Pe=0.04$  and
$Pe=1.8$ respectively.  However, injection rates  on the Moon might be
larger  than   on  Earth   \citep{Crisp:1990gf,Zimbelman:1998ww}  (See
Section \ref{C2-sec:observ-vs-pred}).   For an  injection rate  one to
two orders of magnitude larger and $d$ between $0.5$ and $1.5$ km, the
range of Peclet  number is in fact very similar,  i.e.  from $10^{-3}$
to $10^4$.  Therefore,  taking $St_m=0.1$, we have  $Pe_m$ that varies
between $0.01$ and  $10^5$ for low-slope lunar  domes.  Finally, lunar
basalts  are mafic  in composition  and the  viscosity contrast  $\nu$
should vary between $10^{-3}$ and $10^{-5}$ \citep{Diniega:2013eh}.

\subsubsection*{Constraining the magma physical properties}
\label{sec:visc-contr-lunar}

For  an  injection  rate  of $Q_0=10$  m$^{3}$  s$^{-1}$,  a  liquidus
viscosity of  $\eta_h= 1$  Pa s  and the  parameters listed  in Figure
\ref{C4-Data},  the dimensionless  thickness of  these domes  are also
consistent with  a viscosity  contrast close  to $10^{-8}$  (best fit:
$\nu=1.8  \pm 0.4  \times  10^{-8}$) (Figure  \ref{C4-Data}).  On  the
Moon, (\ref{PeExpre}) becomes
\begin{equation}
  Pe \approx 125 d_c^{-3/2} Q_0^{5/4}\eta_h^{1/4}
  \label{PeExpreMoon}
\end{equation}
and assuming that  the intrusion depth ranges from $500$  m to $5$ km,
the Peclet number ranges from $0.1$ to $1$ and the range of values for
the dimensionless number falls at the boundary between the two domains
in the  phase diagram  (Figure \ref{C4-PhaseDiag}).  It  is consistent
with the radius of these lunar domes being close to $R=4$, i.e.  close
to   the   transition  radius   with   the   gravity  regime   (Figure
\ref{C4-Data}).  However,  the estimate for the  viscosity contrast is
much larger than the value  expected for their mafic composition.  For
the same  injection rate  and a  liquidus viscosity  for the  magma of
$\eta_h=10^3$ Pa s instead of $\eta_h=1$ Pa s, the height scale $H$ is
larger and the dimensionless thickness smaller. The best fit viscosity
contrast        is        now         close        to        $10^{-5}$
($\nu  = 6.9  \pm 1.8  \times10^{-6}$),  much closer  to the  expected
value,   and   the  Peclet   number   are   slightly  larger   (Figure
\ref{C4-PhaseDiag}).  A  similar value for the  viscosity contrast can
be  obtained  for  $\eta_h=1$  Pa s  and  $Q_0=1500$  m$^3$  s$^{-1}$.
However, in  that case,  the Peclet  numbers are  much larger  and the
range of  values for  the dimensionless numbers  fall within  the sill
region (Figure \ref{C4-PhaseDiag}).  In the end, it  suggests that the
injection  rates for  these  lunar  domes were  most  likely close  to
$Q_0\sim 10$ m$^{3}$ s$^{-1}$, hence a few orders of magnitude smaller
than the  effusion rates estimated  from the runout distances  of some
lava  flows in  the lunar  maria  i.e.  $Q_0\ge  10^6$ m$^3$  s$^{-1}$
\citep{TracyKPGregg:1996wp}.

\subsubsection*{Is conductive  cooling enough to solidify  a laccolith
  on the Moon ?}
\label{sec:visc-contr-lunar}

On the  Moon, the dimensionless sizes  of the domes vary  by less than
one order of  magnitude and might be explained only  by the conductive
cooling  of the  magmatic intrusion  (Figure \ref{C4-Data}).   Because
intrusion depths are not known for  these domes, we can not process as
for Elba Island laccoliths to test this hypothesis.
\begin{figure}[h!]
  \begin{center}
    \graphicspath{ {/Users/thorey/Documents/These/Projet/Refroidissement/Skin_Model/Figure/Figure_Heating/} }
    \includegraphics[scale=0.45]{DataMoon.eps}
    \caption{  a) Dimensionless  thickness $h_0$  versus dimensionless
      radius  $R$  for  some  lunar  low-slope  domes.   Purple  dots:
      characteristics   length  scale   $\Lambda$  (\ref{C3-L1})   and
      thickness $H$  (\ref{C3-H1}) are calculated  the same way  as in
      Figure \ref{C4-Data}.  Colored  diamonds: characteristics length
      scale  $\Lambda$ (\ref{C3-L1})  and thickness  $H$ (\ref{C3-H1})
      are calculated  the same way  as in Figure  \ref{C4-Data} except
      for  the   intrusion  depth,   taken  from   (\ref{C4-dd})  with
      $R^r  =  36.6$ km  and  $d_c^r=5$  km.  b)  Dimension  thickness
      $\hat{h_0}$  versus  dimensionless  radius  $\hat{R}$.   Colored
      polygons refers to the colors in a).}
    \label{C4-ArrestMoon}
  \end{center}
\end{figure}
Instead, we estimate  a range of intrusion depth that  would produce a
collapse of  the rescaled size of  the domes on a  reference dome size
given by $\hat{R^{\text{ref}}}$ and $\hat{h_0^{\text{ref}}}$.  Indeed,
using (\ref{HatR}), $\hat{R}=\hat{R}^{\text{ref}}$ implies that
\begin{equation}
  \Lambda^{-1}h_f^{1/34}Pe_m^{-7/34}\nu^{1/17}R =   \left(\Lambda^{\text{ref}}\right)^{-1}\left(h_f^{\text{ref}}\right)^{1/34}\left(Pe_m^{\text{ref}}\right)^{-7/34}\left(\nu^{\text{ref}}\right)^{1/17}R^{\text{ref}}\label{RefDome1}
\end{equation}
where the topscript ref denotes a reference dome and the radii $R$ and
$R^{\text{ref}}$ are with dimension. Assuming that the different lunar
domes  differ   only  by   their  intrusion   depth,  (\ref{RefDome1})
simplifies and reads
\begin{equation}
  d_c = \left(R/R^{\text{ref}}\right)^{34/15}d_c^{\text{ref}}
  \label{C4-dd}
\end{equation}
We take the  largest dome as a  reference and we set its  depth to the
largest reasonable value, i.e.  $d_c=5$  km, mainly to ensure that the
dimensionless  radius of  the other  domes remains  smaller than  $4$.
Injecting the dome  radii in (\ref{C4-dd}) then  give intrusion depths
between $0.5$ km  and $5$ km and Peclet numbers  between $10^{-2}$ and
$0.5$, consistent with  the expected values.  However,  while this new
parameters  result in  the  collapse of  $\hat{R}$  for the  different
domes,  the  variation  in  Peclet  number can  not  account  for  the
dispersion in  the dome thicknesses (Figure  \ref{C4-ArrestMoon}).  In
addition,   the  dimensionless   thickness  as   a  function   of  the
dimensionless  radius does  not follow  the scaling  law (\ref{C4-Hr})
anymore  (Figure  \ref{C4-ArrestMoon}).   The  same  observations  are
obtained using different  reference domes or by  setting the constrain
on   the   rescaled   thickness    instead   of   the   radius,   i.e.
$\hat{h_0}=\hat{h_0^{\text{ref}}}$.   Therefore,   conductive  cooling
does  not appear  to  be  responsible for  the  arrest of  terrestrial
laccoliths.

\subsection{Large mafic sills}
\label{sec:large-mafic-sill}

As we discussed in Chapter \ref{chap2},  the size of large mafic sills
reported  by  \citet{Cruden:tg}  show  an  increasing  thickness  with
diameter  apparently  in  contradiction with  the  constant  thickness
predicted  by   the  elastic-plated  gravity  current   model  (Figure
\ref{C2-Corry_Rocchie}).  One  possible explanation is  that different
sills  are  characterized  by  different  injection  rates,  i.e.   by
different  height scales.   Forcing the  dimensionless thicknesses  of
different sills to be constant imposes that
\begin{equation}
  Q_0 = (h_0/h_0^{\text{ref}})^4Q_0^{\text{ref}}
\end{equation}
where $h_0$ is the sill  thickness with dimension, $Q_0$ its injection
rate  and  $h_0^{\text{ref}}$  and  $Q_0^{\text{ref}}$  are  reference
values for this  parameters.  Taking the thickest sill  as a reference
with $Q_0^{\text{ref}} =  10^4$ m$^3$ s$^{-1}$, we find  that in order
to collapse all  the data on a constant thickness,  the injection rate
have  to  vary by  at  least  $7$  orders  of magnitudes,  i.e.   from
$Q_0=10^{-3}$ to $Q_0= 10^{4}$ m$^3$  s$^{-1}$. It is much larger than
the expected range  of variations for this parameter  and hence, these
mafic  sills do  not appear  to  have all  stop in  the third  gravity
regime. Another possible explanation is  that fracturation at the tip,
instead  of  cooling, have  triggered  the  arrest of  these  magmatic
intrusions in the second gravity phase.  Indeed, while fracturation is
not sufficient to stop a magmatic  intrusion in the bending regime, it
might   be  responsible   for  the   arrest  of   large  mafic   sills
\citep{Michaut:2011kg}.  The increasing  thickness with diameter would
thus be consistent with the  thickness increase induced by the cooling
of the  sill in the  second gravity phase.  However,  more information
about the intrusion  depth and the relationship  between the different
sill  units,  which  are  not given  by  \citet{Cruden:tg},  would  be
required to precisely test this hypothesis.

\subsection{Contact aureole}
\label{sec:thermal-aureol}

Contact  metamorphism  often  occurs   in  the  vicinity  of  magmatic
intrusion
\citep{Jaeger:1959du,SILLITOE:1998bs,Senger:2014tt}. Metamorphism is a
complex process and in the following, we discuss only the dimension of
the thermal aureole  in the vicinity of laccoliths  which have stopped
in the third bending regime.

\begin{figure}[h!]
  \begin{center}
    \graphicspath{ {/Users/thorey/Documents/These/Projet/Refroidissement/Skin_Model/Figure/Figure_Heating/} }
    \includegraphics[scale=0.42]{Contact_Aureole_Pe1.eps}
    \caption{a)  Snapshot  of  the  flow thermal  structure  with  its
      surrounding  thermal   aureole  at   $t=10$  for   $Pe=1.0$  and
      $\nu=10^{-3}$.  Isotherms  are indicated  on the plot.   b) Same
      plot but for $Pe=10.0$.}
    \label{Contact_Areuol2}
  \end{center}
\end{figure}

We define  the thickness of  the thermal  aureole by the  region where
$\Theta_r>0.1$  above the  center of  the flow.   Therefore, inverting
(\ref{C4-eq11}) gives  $L_h(t)$, the maximum thickness  of the thermal
aureole above the intrusion at $r=0$
\begin{equation}
  L_h(t)                                                           =
  \erf^{-1}\left(\frac{0.1}{\Theta_s(r=0,t)}\right)2Pe^{-1/2}t^{1/2}
  \label{Aureol}
\end{equation}
$L_h$ scales as  $Pe^{-1/2}$ and hence, is larger for  small values of
$Pe$ (Figure \ref{Contact_Areuol2}). Indeed, for large $Pe$, advection
dominates on  the emplacement  time scale and  the thermal  aureole is
restricted   to   a   small    zone   around   the   current   (Figure
\ref{Contact_Areuol2}).  For  instance, the  thickness of  the contact
aureole $L_h(t)$  at $t=10$ is  almost equal to the  current thickness
$h_0$  for $Pe=1$  whereas  it is  only  a few  percent  of $h_0$  for
$Pe=10.0$ (Figure \ref{C4-Grid_PeOmega_Heating}).

The dimension of  the thermal aureole also depends  on the emplacement
time $t$;  the longer the  injection, the  longer the heating  and the
larger the thermal aureole. An upper limit for the emplacement time is
given by the time $t_t$ for the intrusion to transition in the gravity
regime   while   it   is   in   the   third   bending   regime,   i.e.
$t_t=6.5(\eta_e/\eta_h)^{2/7}h_f^{-1/7}$          (See         Section
\ref{C3-sec:diff-evol-with-1}).   Injecting $t_t$  into (\ref{Aureol})
gives
\begin{equation}
  L_h^{\text{max}}                                                           =5.0  \erf^{-1}\left(\frac{0.1}{\Theta_s(r=0,t_{t})}\right)h_f^{-1/14}\nu^{-1/7}Pe^{-1/2}
  \label{Aureol2}
\end{equation}
and therefore, using (\ref{C3-ScalingH-Visco}), we get
\begin{equation}
  L_h^{\text{max}}/h_0(t_t)                                                           =3.7
  \erf^{-1}\left(\frac{0.1}{\Theta_s(r=0,t_{t})}\right)h_f^{1/14}\nu^{1/7}Pe^{-1/2}
  \label{Aureol3}
\end{equation}
While  the  absolute  size  of the  thermal  aureole  $L^{\text{max}}$
increases with  decreasing $Pe$  and $\nu$ (\ref{Aureol2}),  the ratio
$L_h^{\text{max}}/h_0$  is  smaller  for  larger  viscosity  contrasts
(Figure \ref{PhaseHeatingContact}).

\begin{figure}[h!]
  \begin{center}
    \graphicspath{ {/Users/thorey/Documents/These/Projet/Refroidissement/Skin_Model/Figure/Figure_Heating/} }
    \includegraphics[scale=0.35]{ThermalAureol_50Point_2figure_R4.eps}
    \caption{a)  Maximum  Size  of   the  thermal  aureole  above  the
      intrusion   $L_h^{\text{max}}$  normalized   by  the   intrusion
      thickness  at  the  center  $h_0$ ($\%$)  (\ref{Aureol3})  as  a
      function of the  number $Pe$ and $\nu$.  b) Same  plot but where
      the thermal  aureole is  defined with $\Theta_r=0.8$  instead of
      $0.1$ in (\ref{Aureol3}).  In both plot, we set $h_f=0.001$ and
      $\Theta_s(r=0,t_{t})=1$.}
    \label{PhaseHeatingContact}
  \end{center}
\end{figure}

For   the   felsic   laccoliths   at   Elba   Island,   we   estimated
$\nu     \sim     10^{-9}$     and    $Pe\sim     10^{-3}$     (Figure
\ref{PhaseHeatingContact}  a) and  thus,  the  thermal aureole  should
reach  $400\%$  of  the  intrusion  thickness at  the  center  of  the
flow. Important  temperature variations however,  i.e. $\Theta_r>0.8$,
should be restricted to the vicinity of the current, i.e.  $\sim 60\%$
of  the  intrusion  thickness  (Figure  \ref{PhaseHeatingContact}  b).
Theses  estimates represent  upper  bound as  the  temperature at  the
surface $\Theta_s(r=0,t)$ might be smaller  than $1$ for $Pe<0.1$.  In
addition,   the  model   neither  accounts   for  horizontal   thermal
conduction,  nor for  circulation of  fluid  in the  wall rocks,  both
effects that should also limit the size of the thermal aureole.


\section{Summary and discussion}
\label{sec:discussion}

In this chapter, we discuss a more realistic model for the emplacement
of magmatic intrusions in the shallow crust of terrestrial planets. In
particular, we  describe the dynamics  of a magma characterized  by an
Arrhenius rheology and heating the wall  rocks as it spreads.  We show
that relaxing  the thermal boundary  condition insulates the  flow and
therefore, allows  the intrusion to  stay hot  for a longer  period of
time.  In particular, the thermal anomaly detaches slower from the tip
of the  intrusion.  It also does  not reach a steady  state anymore in
the gravity regime as the heating of the surrounding medium constantly
decreases  the heat  loss in  the central  region.  Nevertheless,  the
Arrhenius  rheology   largely  compensates   for  the  delay   in  the
transitions induced by the heating  of the surrounding medium.  In the
end,  except for  the third  phase in  both regimes  which is  reached
slightly later, the dynamics shows only small variations in comparison
to the one described in chapter \ref{C3-JFM}.

Application of this  model to the christmas-tree  laccolith complex at
Elba   Island  shows   that   cooling  allows   to  reconciliate   the
elastic-plated gravity  current model  with the  observations. Indeed,
their   size    is   consistent   with   their    felsic   composition
($\nu  \approx  10^{-8}$)  and  their  arrest  in  the  third  bending
phase. However, the model does  not capture the second order variation
in the laccolith dimensions. Indeed, while  the mean trend in the data
is consistent with the model, conduction  alone is not able to explain
the variability  around this trend.  Further  application to low-slope
lunar domes,  where conduction is more  likely to be the  only cooling
mechanism, leads  to the  same conclusions.   The model  thus suggests
that other  mechanism than cooling  are responsible for the  arrest of
laccoliths.

In the end, while some progress  has been made in the understanding of
the  emplacement  of these  shallow  magmatic  intrusions, a  complete
picture of the solution, in regard  to the final size and thickness of
the  intrusion  as  a  function  of  the  parameters  of  the  problem
(elasticity and  toughness of the  host rock, viscosity of  the magma,
injection rate of the feeder dyke,  and depth of emplacement), has not
yet been obtained.  In general, more detailed observations, especially
at the tip  of these shallow magmatic intrusions,  may provide further
insights in the  arrest mechanism.  Precise information  on the extent
of  the thermal  aureole,  which  has not  yet  been  reported in  the
literature, could  also provide precious  constrain for the  model. We
discuss some possible extensions of this work in the last part of this
manuscript.


%%% Local Variables:
%%% mode: latex
%%% TeX-master: "../main"
%%% End:
