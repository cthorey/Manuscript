\chapter{Second  order  modelling  -  Relaxing  the  thermal  boundary
  condition}
\label{chap5}
\minitoc

The previous chapter  was first step toward the  understanding how the
cooling of the  laccolith interacts with its  dynamics.  Hereafter, we
examine  how the  heating of  the surrounding  layer changes  the flow
behavior.

\section{Introduction}
\label{sec:introduction}
 
Contact metamorphism around sill intrusion is a common process

\citep{Everett:2008tn}

\section{Theory}

\subsection{Formulation}
\label{sec:formulation-1}


The assumed  model of magma  intrusion is based on  the elastic-plated
gravity  current  with  temperature-dependent viscosity  described  in
Section \ref{sec:theory} where we relax the condition of a fixed
temperature $T_0$ at the edge of the flow.



In  particular, we  model  the  axisymmetric flow  of  fluid below  an
elastic layer  of constant thickness  $d_c$ and above a  semi infinite
rigid layer. The fluid is injected continuously at the base and center
of the current at a constant  rate $Q_0$ through a conduit of diameter
$a$.   Assuming  a  Poiseuille  flow within  the  cylindrical  feeding
conduit, the vertical injection velocity $w_i(r,t)$ and injection rate
$Q_0$ are given by
\begin{equation}
  w_i(r,t)=
  \begin{cases}
    \frac{ \Delta P}{4 \eta_h Z_{c}} (\frac{a^{2}}{4}-r^{2})& r \le \frac{a}{2}\\
    0 & r > \frac{a}{2}
  \end{cases}
  \label{C4-eq12}
\end{equation}
\begin{equation}
  Q_{0}=\frac{\pi \Delta P a^{4}}{128 \eta_h Z_c}
  \label{C4-eq11}
\end{equation}
where  $\Delta P$  is  the  initial overpressure  within  the melt  at
$z=Z_{c}$.

\vspace{.5cm} \textbf{Flow thickness equation} \vspace{.5cm}

The  equation  for  the  flow  thickness  evolution  follow  a  global
statement of mass and reads
\begin{equation}
  \frac{\partial h}{\partial t} = \frac{1}{r}
  \frac{\partial}{\partial r} \left( r\frac{\partial P}{\partial      r}\int_0^h\frac{1}{\eta(y)}\left(y-\frac{h}{2}\right)ydy\right)\right)
+ w_i\label{C4-Mass-2}
\end{equation}
where $P(r,z,t)$  is the pressure within  the flow.  It is  the sum of
three contributions:  the weight of the  magma and of the  upper layer
and the bending pressure
\begin{equation}
  P = \rho_m g (h-z)+\rho_rgd_c+D\nabla_r^4h
\end{equation}
where  $h(r,t)$ is  the flow  thickness, $\rho_r$  the density  of the
surrounding rocks and $D$ is the flexural rigidity of the thin elastic
layer, that  depends on Young's  modulus $E$, Poisson's  ratio $\nu^*$
and     on     the     elastic    layer     thickness     $d_c$     as
$D = Ed_c^3/\left(12(1-\nu^*)\right)$.

\vspace{.5cm} \textbf{Heat transport within the fluid} \vspace{.5cm}

The hot fluid  is intruded at temperature $T_i$ and  cools through the
top  and the  bottom by  conduction in  the surrounding  medium, whose
temperature $T_s$ is allowed to increase  with time and equal to $T_0$
far from  the intrusion. In  the laminar regime and  in axisymmetrical
coordinates  (r,z),  the  local energy  conservation  equation,  which
describes the evolution of the temperature field, reduces to
\begin{eqnarray}
  \frac{\partial T}{\partial t}+ u\frac{\partial T}{\partial r}
  + w\frac{\partial T}{\partial z}  &=& \frac{ \kappa_m}{1+St^{-1}}  \frac{\partial^2
                                        T}{\partial               z^2}
                                        \label{C4-EnergyCons2}
\end{eqnarray}
where  $u(r,z,t)$ and  $w(r,z,t)$ are  the radial  and vertical  fluid
velocities,  $St  =\left(C_{p,m}(T_L-T_S)\right)/L$   is  the  Stephan
number   and    $\kappa_m$   is   the   fluid    thermal   diffusivity
$\kappa_m  = k_m/\rho_m  C_{p,m}$ \citep{Thorey:2015vs}.   We use  the
following   approximation  for   the   vertical  temperature   profile
$T(r,z,t)$
\begin{equation}
  T=
  \begin{cases}
    T_b - (T_b-T_s)(1-\frac{z}{\delta})^2 & 0 \le z\le \delta \\
    T_b & \delta \le z\le h-\delta \\
    T_b - (T_b-T_s)(1-\frac{h-z}{\delta})^2 & h-\delta \le z\le h\\
  \end{cases}
  \label{C4-Temperature}
\end{equation}
where  $T_b(r,t)$  is the  temperature  at  the  center of  the  flow,
$T_s(r,t)$ the temperature  at the contact with  the surrounding rocks
and $\delta(r,t)$ is the thickness of the thermal boundary layer. This
profile assures  the continuity of  the temperature and the  heat flux
within the  flow as  well as  an evolving  temperature at  the surface
$T_s(r,z,t)$.   Integrating   (\ref{C4-EnergyCons2})  over   the  flow
thickness  and using  the vertical  temperature profile  approximation
(\ref{C4-Temperature}), the heat transport equation reduces to
\begin{eqnarray}
  \frac{\partial}{\partial
  t}\left( \delta( \bar{T}-T_b)\right)+\frac{1}{r}\frac{\partial}{\partial
  r} \left( r\delta(\overline{uT}-\bar{u}T_b)\right) = -2\frac{\kappa_m}{(1+St^{-1})}\frac{T_b-T_s}{\delta}
  \label{C4-LocalHeat3}
\end{eqnarray}
where the bars  indicate the vertical average over  a thermal boundary
layer
\begin{equation}
  \bar{f} = \frac{1}{\delta}\int_0^\delta f dz.
\end{equation}

\vspace{.5cm} \textbf{Rheology} \vspace{.5cm}

Equations (\ref{C4-Mass-2})  and (\ref{C4-LocalHeat3}) are  coupled by
the  flow  rheology   $\eta(T)$.   \citet{Thorey:2015vs}  considers  a
rheology  $\eta_1(T)$ where  the  viscosity depends  inversely on  the
temperature such that
\begin{equation}
  \eta_1(T)=\frac{\eta_h
    \eta_c(T_i-T_0)}{\eta_h(T_i-T_0)+(\eta_c-\eta_h)(T-T_0)}
  \label{rheology}
\end{equation}
where $\eta_h$  and $\eta_c$  are the viscosities  of the  hottest and
coldest  fluid  at  the   temperature  $T_i$  and  $T_0$  respectively
\citep{Bercovici:2007vc}.    While    this   model    possesses   some
simplification properties, it  restricts the change in  viscosity to a
narrow    range   of    temperature   close    to   $T=T_0$    (Figure
\ref{C4-Rheology}).     In     contrast,    the     Arrhenius    model
($\eta  \sim \exp(-k/T)$),  which is  a  more commonly  used model  to
relates  temperature and  viscosity  of lavas  \citep{Blatt:2ViMWPc0},
allow the  viscosity to increase  over a larger range  of temperature.
To get some insights into the effect of a more realistic viscosity, we
thus also use a first-order approximation  of the Arrhenius model as a
second rheology $\eta_2(T)$ \citep{Diniega:2013eh}
\begin{eqnarray}
  \eta_2(T) = \eta_h \exp\left(-\log\left(\frac{\eta_h}{\eta_c}\right)\left(1-\frac{T-T_0}{T_i-T_0}\right)\right)
\end{eqnarray}


\begin{figure}[htbp]
  \begin{center}
    \graphicspath{ {/Users/thorey/Documents/These/Projet/Refroidissement/Skin_Model/Figure/Figure_Heating/} }
    \includegraphics[scale=0.8]{Rheology.eps}
    \caption{Dimensionless viscosity  versus dimensionless temperature
      for  both rheology  $\eta_1$ and  $\eta_2$ described  in Section
      \ref{sec:formulation-1}.}
    \label{C4-Rheology}
  \end{center}
\end{figure}

\subsection{Thermal boundary conditions}
\label{C4-sec:thermal-boundary-condition}

At  the  contact with  the  surrounding  rock,  the  heat is  lost  by
conduction
\begin{eqnarray}
  k_m\left.\frac{\partial                                    T}{\partial
      z}\right|_{z=0}&=&k_r\left.\frac{\partial              T_r}{\partial
      z}\right|_{z=0}  \label{C4-Flux1}\\
  k_m\left.\frac{\partial                                  T}{\partial
      z}\right|_{z=h}&=&k_r\left.\frac{\partial            T_r}{\partial
      z}\right|_{z=h}
  \label{C4-Flux2}
\end{eqnarray}
where  $T_r(r,z)$ is  the temperature  in the  surrounding medium  and
$k_r$ its  thermal conductivity.  Assuming  a semi infinite  layer for
the rigid layer below  the intrusion, \citet{Carslaw:1959wf} show that
the temperature $T_r$ in the  surrounding rocks can be approximated to
a first order by
\begin{equation}
  T_r(r,z,t)-T_0=(T_{s}-T_0)\operatorname{erfc}{\left(\frac{-z}{2\sqrt{\kappa_r t}}\right)}.
  \label{eq22}
\end{equation}
The  thickness of  the upper  layer is  equal to  the intrusion  depth
$d_c$. However,  we assume that the  depth $d_c$ is large  compared to
the characteristic  length scale for  conduction $L_c$ and we  use the
same approximation to derive $T_r$ above the intrusion
\begin{equation}
  T_r(r,z,t)-T_0=(T_{s}-T_0)\operatorname{erfc}{\left(\frac{z-h}{2\sqrt{\kappa_r t}}\right)}.
  \label{eq11}
\end{equation}
Therefore, the  two thermal  boundary conditions  (\ref{C4-Flux1}) and
(\ref{C4-Flux2}) become
\begin{eqnarray}
  k_m\left.\frac{\partial                                    T}{\partial
      z}\right|_{z=0}&=& k_r
  \frac{T_{s}-T_{0}}{\sqrt{\pi \kappa_r t}}  \label{C4-2Flux2}\\
  k_m\left.\frac{\partial                                    T}{\partial
      z}\right|_{z=h}&=& -k_r
  \frac{T_{s}-T_{0}}{\sqrt{\pi \kappa_r t}}
  \label{C4-2Flux_2}
\end{eqnarray}

\subsection{Dimensionless equation}


We use the characteristic temperature interval $\Delta T = T_i-T_0$ to
nondimensionalize temperatures.  The dimensionless integral balance
approximation (\ref{C4-Temperature}) becomes
\begin{equation}
  \theta(z)=
  \begin{cases}
    \Theta_b\left(1 -(1-\frac{z}{\delta})^2\right)& 0 \le z\le \delta \\
    \Theta_b & \delta \le z\le h-\delta \\
    \Theta_b\left(1-(1-\frac{h-z}{\delta})^2\right)  &   h-\delta  \le
    z\le h
  \end{cases}
  \label{Temperature2}
\end{equation}
where   $\theta(r,z,t)$   is   the   dimensionless   temperature   and
$\Theta_b=\frac{T_b-T_0}{T_{i}-T_0}$.        Finally,        equations
(\ref{LocalHeat3}) and (\ref{C3-Mass-2})  are nondimensionalized using
a horizontal  scale $\Lambda$, a vertical  scale $H$ and a  time scale
$\tau$ given by
\begin{eqnarray}
  \Lambda &=& \left(\frac{D}{\rho_m g}\right)^{1/4}\label{L1}\\
  H&=&\left       (\frac{12\eta_h      Q_{0}}{\rho_{m}g       \pi}\right      )
       ^{1/4} \label{H1}\\
  \tau&=&\frac{\pi \Lambda^{2} H}{Q_{0}}\label{T1}
\end{eqnarray}
where  $\Lambda$  represents  the  flexural wavelength  of  the  upper
elastic layer \citep{Michaut:2011kg}, $H$ the characteristic thickness
of an isoviscous constant flux gravity current with viscosity $\eta_h$
\citep{Huppert:1982wr} and $\tau$ the characteristic time to fill up a
cylindrical flow of  radius $\Lambda$ and thickness $H$  at a constant
rate $Q_0$.   In addition, we  can define a horizontal  velocity scale
$U=\Lambda/\tau=\left(\rho_m           g           H^3\right)/\left(12
  \eta_h\Lambda\right)$.

The dimensionless equations are
\begin{eqnarray}
  \frac{\partial h}{\partial t}& =& \frac{12}{r}
                                    \frac{\partial}{\partial r} \left( r\left( \frac{\partial h}{\partial      r}+\frac{\partial}{\partial      r}\left(\nabla_r^4h\right)\right)I_1(h)\right)
                                    + w_i\label{EqFinal1}\\
  \frac{\partial}{\partial
  t}\left( \delta( \bar{\theta}-\Theta_b)\right)&=&-\frac{1}{r}\frac{\partial}{\partial
                                                    r}  \left(   r\delta(\overline{u\theta}-\bar{u}\Theta_b)\right)  -
                                                    \delta\left(      \frac{\partial       \Theta_b}{\partial      t}+
                                                    \overline{u}\frac{\partial     \Theta_b}{\partial    r}\right)\nonumber\\
                               &-&
                                   2Pe^{-1}St_m\frac{\Theta_b}{\delta}+\frac{w_{i}}{2}(1-\Theta_b)\label{HeatDimensionLess}\\
  w_{i}&=&
           \frac{32}{\gamma^{2}}\left(\frac{1}{4}-\frac{r^{2}}{\gamma^{2}}\right)\hspace{.2cm}
           \text{if} \hspace{.2cm} r < \gamma/2,\hspace{.2cm} w_i=0 \hspace{.2cm}
           \text{if} \hspace{.2cm} r \ge \gamma/2\\
  u(r,z,t)&   =&   12\left(   \frac{\partial   h}{\partial
                 r}+\frac{\partial}{\partial
                 r}\left(\nabla_r^4h\right)\right)I_0(z)\label{C3-Veloc}
\end{eqnarray}


\section{Evolution in the bending regime}
\label{sec:evol-bend-regime-1}


\section{Evolution in the gravity regime}
\label{sec:evol-grav-regime}



\subsection{Discussion}

\section{Heating of the surrounding layer}
\label{sec:heat-surr-layer}


\section{Applications}
\label{sec:applications}




%%% Local Variables:
%%% mode: latex
%%% TeX-master: "../main"
%%% End:

