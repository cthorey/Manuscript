Intrusive magmatism  is a major  process at  the scale of  a planetary
body and, most  likely, determinant in the evolution  of a terrestrial
crust. However,  it takes  place deep beneath  the surface  and remain
difficult  to study  without  a proper  model  for magmatic  intrusion
emplacement.  The objective of this thesis was two-fold: to characterize the
dynamics of  a cooling  magmatic intrusion  and to  shed light  on the
origin of floor-fractured craters. 

\section*{Dynamics of a magmatic intrusion in the upper crust on
  terrestrial planets.}

\subsection*{Summary}
\label{sec:conclusion}

In Chapter  \ref{C3-JFM}, we  propose a  model for  the cooling  of an
elastic-plated gravity current. 


and  the  flow  itself  results   in  important  deviations  from  the
isoviscous  case.    In  particular,   the  spreading  of   a  cooling
elastic-plated  gravity current  is characterized  by three  phases in
both the bending and the gravity regime
\begin{itemize}
\item A hot phase where the  current spreads as and isoviscous current
  with hot viscosity
\item  A  second  phase  where the  current  slows  down,  drastically
  thickens and its effective viscosity increases.
\item  A  cold  phase  where  the current  returns  in  an  isoviscous
  spreading but with a much larger viscosity.
\end{itemize}



Accordingly, the aspect ratio between the first and the third phase of
the spreading increases  by a factor $\nu^{-2/7}$. It is  thus able to
explain the  aspect ratio of  laccoliths observe  on Earth and  on the
Moon.


\subsection*{Laccoliths }
\label{sec:laccoliths-}




\newpage
\section*{Floor-fractured  craters and  their implication  in term  of
  lunar intrusive magmatism.}
%%% Local Variables:
%%% mode: latex
%%% TeX-master: "../main"
%%% End:
l
