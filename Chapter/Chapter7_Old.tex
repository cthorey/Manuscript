\chapter{Summary and research outlook}
\label{chap9}

\minitoc

Intrusive magmatism  is a major  process at  the scale of  a planetary
body and  plays a  fundamental role in  the accretionary  processes of
terrestrial crust.  However,  it takes place deep  beneath the surface
and remains  difficult to  study without a  proper model  for magmatic
intrusion emplacement.  The objective of  this thesis was two-fold: to
characterize the dynamics of a  cooling magmatic intrusion and to shed
light on the origin of floor-fractured craters.


\section{Dynamics of a cooling elastic-plated gravity current}

\subsection{Summary}
\label{sec:summary}

Intermediate-scaled  shallow  magmatic  intrusions  are  the  building
blocks    of    larger    plutons     intruded    into    the    crust
\citep{Petford:2000cc,Glazner:2004gv}.  We show in Chapter \ref{chap1}
that the  observations of planetary  surfaces can often  constrain the
topographic deformations  caused by  shallow intrusions; that  is, the
volume,  shape  and   other  dimensions  of  the   intrusions  can  be
quantified.  However, these  observations must be linked  to models of
magmatic intrusion  dynamics to  provide insights into  magma physical
properties and injection rate.

\citet{Michaut:2011kg} provides  a $2D$ model for  such elastic-plated
gravity   current  intrusion   which  directly   links  the   observed
deformation to  physical parameters of the  intrusion.  In particular,
depending mainly  on the injection  rate and the intrusion  depth, the
model predicts  two regimes  of propagation characterized  by specific
morphologies and  scaling laws  for intrusion thickness  versus length
and time.

In  Chapter  \ref{chap2}, we  develop  the  model in  an  axisymmetric
geometry and compare the model predictions to the morphologies of some
terrestrial shallow magmatic intrusions.   We show that laccoliths and
low-slope lunar  domes are consistent  with their arrest in  the early
bending  regime. In  addition,  the model  predictions are  consistent
across   different  planetary   settings.   Nevertheless,   the  model
underestimates the absolute dimension of these magmatic intrusions; in
particular, it  requires abnormally  high viscosity to  reconcile both
observations and predictions.

To get  some insights in the  effective flow viscosity, we  provide in
chapter \ref{C3-JFM}  and \ref{Heating} an  extension of the  model of
\citet{Michaut:2011kg}   that  accounts   for  the   cooling  of   the
current. We show  that the coupling between the  temperature field and
the flow  itself results  in important  deviation from  the isoviscous
flow.

During the bending  regime, the local thermal condition at  the tip of
the current  governs the  effective flow viscosity.   As the  fluid is
cooling by conduction,  the thermal anomaly detaches from  the tip and
the flow effective  viscosity rapidly increases to  stabilize when the
front  becomes entirely  ``cold''.  Nevertheless,  the formation  of a
highly  viscous plug  at the  front  of the  intrusion, and  therefore
cooling, is  probably not  responsible for  the arrest  of terrestrial
laccoliths.   Instead,  we propose  that  the  injection rate  is  the
limiting factor in the growth of these magmatic intrusions.

Available data for large mafic sills on Earth show less agreement with
the   model  prediction.    Indeed,   in   Chapter  \ref{C3-JFM}   and
\ref{Heating}, we show that sills should behave as ``cold'' isoviscous
gravity current when the thermal anomaly is small compared to the flow
itself, i.e.   in similar settings,  their thickness should tend  to a
constant.  The  increase in  thickness with  diameter recorded  in the
data might thus  suggest that they instead stop in  the second gravity
phase.  However,  the model  lacking of a  stopping criterion  for the
flow, we were not able to test this hypothesis.

\subsection{Toward an understanding of the small-scale dynamics in the
  tip}
\label{sec:perspectives}

In Chapter  \ref{chap2}, \ref{C3-JFM} and \ref{Heating},  we show that
the local condition at the tip of the current controls the dynamics in
the  bending  regime.  For  a  sake  of  simplicity,  we used  a  thin
prewetting film  at the tip to  avoid the requirement of  any boundary
condition at a genuine front. This approach allowed us to get insights
into  the  coupling  between  the   thermal  structure  and  the  flow
itself. In particular, it captures the leading order behavior of these
shallow intrusions.   Nevertheless, a more precise  description of the
mechanisms  operating  at the  scale  of  the  tip might  help  better
characterize the dynamics of both sills and laccoliths.

\subsubsection*{Fluid driven fracture}
\label{sec:caref-descr-tip}

A first step  would be to describe  the tip in term of  a fluid driven
fracture  instead of  the thin  prewetting  film. As  seen in  Section
\ref{C2-Toughness},  linear elastic  fracture mechanics  requires that
the  mode $I$  intensity  factor  $K_I$ equal  a  critical value,  the
fracture toughness  of the wall  rock $K_{c}$, for the  propagation to
occur.  This condition  is usually expressed in term  of an asymptotic
condition      on      the       crack      opening      at      $r=R$
\citep{Savitski:2002gy,Bunger:2005em,Bunger:2007vs,Detournay:2014fk}.

In  such  problem, the  lubrication  equation  is  thus coupled  to  a
description  of  the fracture  opening  based  on the  linear  elastic
fracture mechanics.  \citet{Bunger:2011cb} use  this approach to solve
the problem of  shallow magmatic intrusions and  found similar results
than  \citet{Michaut:2011kg}.  Interestingly,  they needed  values for
the fracture toughness $K_c$ two  or three orders of magnitudes larger
than  laboratory measurements  to agree  with the  observations, which
they attribute to  potentially crack blunting mechanism at  the tip of
laccoliths.  This  observation is consistent with  the rapid formation
of  a  highly viscous  plug  at  the  tip  of the  magmatic  intrusion
described in Chapter \ref{C3-JFM} and \ref{Heating}.

Nevertheless, this model also falls short to reproduce the geometry of
large  mafic sills.   In addition,  more realistic  model should  also
consider the process zone, i.e. the region of plastic rock deformation
near the leading edge of the fracture \citep{Bunger:2008cl}.

\subsubsection*{Gas filled region}
\label{sec:caref-descr-tip}

Furthermore, the large  negative pressure that developed  at the front
might  cause desolved  gasses to  exsolve  from the  magma.  With  the
formation and the evolution of a gap filled with gas at the tip of the
current, the  fluid and the  fracture front  do not coincide  with one
another, thus requiring the tracking of two moving boundaries.

Along      with      the     prewetting      film      regularization,
\citet{Anonymous:QWXp_4JV} propose  a second  regularization condition
where the tip of the elastic-gravity  current consists of a lag region
filled   with    gas   at    constant   negative    pressure   (Figure
\ref{C7-Sketch}).   They show  that the  solution depends  on the  gas
pressure  in the  tip  region  in similar  fashion  that the  solution
depends  on  the prewetting  film  thickness  in Chapter  \ref{chap2},
\ref{C3-JFM} and \ref{Heating}.
\begin{figure}[h!]
  \begin{center}
    \graphicspath{ {/Users/thorey/Documents/These/Manuscript/Figure/Chapter7/} }
    \includegraphics[scale=1.3]{Sketch.eps}
    \caption{Two different  regularization condition  at the  front of
      the current: a) thin prewetting film with thickness $h_f$ b) gas
      -filled region.}
    \label{C7-Sketch}
  \end{center}
\end{figure}
In particular, they show that
\begin{eqnarray}
  h_0&\propto& h_f^{-1/7}\nu^{-2/7}L^{10/7}~(\text{Thin film})\\
  h_0&\propto& \sigma^{1/9}\nu^{-2/9}L^{14/9}~(\text{Fluid lag})
\end{eqnarray}
where $L$  is the half  length of the  flow, $-\sigma$ is  the contant
negative  pressure  in  the  fluid   lag  and  we  have  rescaled  the
characteristic    thickness    and     time    by    $\nu^{1/4}$    in
\citet{Anonymous:QWXp_4JV}.    As   expected,    the   two   different
regularization condition leads  to only minor change  in the thickness
to   length   relationship   ($10/7\sim  1.4$,   $14/9\sim   1.5   $).
Nevertheless, a  complete description of  the dynamics of  the cooling
gas-filled region,  whose dynamics would depend  on temperature, might
also be required.

In the end, further works should  coupled the model derived in Chapter
\ref{C3-JFM} and  \ref{Heating} to a  more careful description  of the
different processes  involved at  the scale of  the current  tip. Such
approach should surely  give interesting insights and  a more complete
view of the dynamics of shallow magmatic intrusion.

\subsection{Further refinements for the model}

\subsubsection*{Viscous heating instabilities}

Viscous heating  is another mechanism  not taken into account  in this
study  that would  participate  to the  dynamics  of shallow  magmatic
intrusions. Indeed, especially for large  values of $Pe$ where we have
shown  that the  temperature  gradient within  the  flow are  stronger
(Section  \ref{C4-sec:infl-therm-bound-el}),  the  effect  of  viscous
heating could  be important.  \citet{Costa:2005bq} have  already shown
that viscous heating plays an important role in the dynamics of fluids
with strongly temperature-dependent viscosity. In particular, for lava
tube, they show that the heat generated by viscous friction produces a
local  temperature increase  near  the tube  walls  with a  consequent
decrease  of   the  viscosity   which  may  dramatically   change  the
temperature             and              velocity             profiles
\citep{Costa:2002cj,Costa:2003wk,Costa:2005bq}.      The     important
gradients near  the tip  region or within  the thermal  boundary layer
could  present  favorable  conditions  for  the  development  of  such
instabilities.

\subsubsection*{Stretching in the upper layer}

If the thickness of the intrusion  $h_0$ becomes large compared to the
intrusion depth $d_c$, the  analysis described in Chapter \ref{C3-JFM}
and  \ref{Heating}  is  not  valid  anymore. It  could  be  the  case,
especially for felsic intrusions characterized by large injection rate
intruding a layer such that $d_c\lessapprox 500$ m. In such situation,
the stretching  of the  upper layer  can no  longer be  neglected when
calculating the elastic stresses and  fluid pressure.  In that case, a
complete description  of the flow  in an axysimmetrical  and Cartesian
geometry for an isoviscous flow,  along with scaling laws for $h_0(t)$
and $R(t)$,  has already  been described by  \citet{Lister:2013ia} and
\citet{Anonymous:QWXp_4JV} respectively.  While the time dependence of
the scaling law are somewhat similar from those derived in the bending
dominated regime, the shape of the  flow is not bell-shaped anymore in
the early time solution and show a somewhat conical shape instead. For
instance, this  model could potentially  explain the conical  shape of
some     felsic      laccoliths     observed     in      Island     by
\citet{Anonymous:jHnLP36x}.
\begin{figure}[h!]
  \begin{center}
    \graphicspath{ {/Users/thorey/Documents/These/Manuscript/Figure/Chapter7/} }
    \includegraphics[scale=0.7]{Baula.eps}
    \caption{Felsic laccolith, named Baula,  in West Island.  Modified
      from \citet{Anonymous:jHnLP36x}.}
    \label{C7-Baula}
  \end{center}
\end{figure}

\subsubsection*{Contact aureole}

In chapter \ref{Heating},  we show that a  significant thermal aureole
should develop in the wall rocks above the central flow region.  Apart
from  plastic rock  deformation  that might  develop  in the  encasing
rocks,  studying  the  possible  thermal  erosion  could  also  be  an
interesting thing to look at. For instance, above the feeder dyke, the
temperature are  expected to be  maximum on  the roof and  might favor
subsequent dyke propagation. This could potentially explain the nested
structure of  several laccolith  complexes reported in  the literature
\citep{E:2015tl,Rocchi:2010dn}. Such  dyke formation would  also limit
the size of the terrestrial laccoliths.

\section{Lunar intrusive magmatism}

\subsection{Summary}
\label{sec:summary-1}

While it provides for important  constraints on the Moon's thermal and
petrogenetic evolution,  the total  volume of  melt produced  into the
Moon interior is  poorly known.  Indeed, although  the total extrusive
volume is quantified  through analyses of the lunar  maria, the volume
of intrusive  magma, which should be  large due to the  low density of
the lunar crust,  remains unknown. A first step  in the quantification
of the  intrusive activity  on the  Moon is  the detection  of shallow
magmatic systems.

Low-slope  lunar domes  and floor-fractured  craters are  two proposed
evidences for shallow  magmatic intrusion activity in  the upper lunar
crust. However, these observations must  be linked to shallow magmatic
intrusion models to asses their  intrusive origin.  In this thesis, we
first show that lunar-low slope domes morphology are indeed consistent
with a model  of a cooling magmatic intrusion.   In addition, adapting
the model  of elastic-plated  gravity current model  to account  for a
crater-centered intrusion, we show  in Chapter \ref{C5-chap6} that the
deformations observed  at floor-fractured craters are  also consistent
with  the  emplacement  of  magmatic  intrusions  below  their  floor.
Consistently, taking advantage of the  resolution of the lunar gravity
field  obtained   from  the  NASA’s  Gravity   Recovery  and  Interior
Laboratory  (GRAIL)  mission,  in combination  with  topographic  data
obtained from the Lunar Orbiter  Laser Altimeter (LOLA) instrument, we
show  in Chapter  \ref{chap7} that  their gravitational  signatures is
significantly larger that the one of normal impact craters.

\subsection{Perspectives}
\label{sec:perspectives-2}


Around  $10$ low-slope  lunar  domes and  about $200$  floor-fractured
craters have been detected at the  lunar surface, most of them located
close  or within  the lunar  maria. While  the total  volume of  these
magmatic intrusion should  not exceed $1\%$ of lunar  maria volume, it
advocates the presence of numerous  shallow magmatic intrusions in the
lunar crust. This  suggests that deeper and  probably larger intrusion
might stands at the base of the lunar crust.

\subsubsection*{Local stress field}
\label{sec:crust-magm-intr}

In Chapter  \ref{C5-chap6}, we argue  that the absence  of deformation
surrounding floor-fractured craters suggests  that the unload pressure
associated with the  crater might have drive magma  ascent below these
craters. However, the unload pressure, associated with the depression,
should  decrease with  depth on  a length  scale equal  to the  crater
diameter, some  tens of  kilometers.  Therefore,  for the  average FFC
diameter, $\sim 30$ km, the depression  caused by the crater can drive
magma flow only if the magma  is already present at depth smaller than
$30$ km, i.e.  within the crust. It thus raises the question of deeper
and larger  magmatic reservoir within  the lunar crust which  might be
observable in the lunar gravity field obtained from the GRAIL mission.

This idea is also supported by recent works on rift volcanism on Earth
that show that depression can play a crucial role in the trajectory of
magma on the local scale \citep{Maccaferri:2014ft}.  In particular, in
the extensive tectonic setting of rift, \citet{Maccaferri:2014ft} show
the  presence  of   a  stress  barrier,  consequence   of  the  graben
depression, where  $\sigma_3$ turns  to be vertical,  i.e.  preventing
vertical dyke  propagation.  Interestingly,  dyke nucleated  below the
stress  barrier  are  deviated  from  their  vertical  trajectory  and
off-rift  volcanism occur.   While  important  extensive stress  field
should not exist  at floor-fractured crater sites, this  work thus not
favor dykes  initiated deep in  the lunar interior, which  should have
been  deviated  from   the  crater  center,  to   feed  these  shallow
intrusions.

Meanwhile, a  detailed analysis of  the stress field below  the crater
depression  could  thus explain  why  these  craters, apart  from  the
underlying low  density breccia,  provide a favorable  environment for
magmatic  intrusions.   At  a  regional scale,  it  could  also  prove
fruitful to investigate  the link between the load of  the lunar maria
and the  distribution of floor-fractured  craters, which is  mainly at
the edge of the maria itself.

\subsubsection*{Thickness of the lunar maria}
\label{sec:crust-magm-intr}

Approximately $16\%$ of the Moon's surface is covered by basaltic lava
flows  that comprise  the lunar  maria. Although  the total  extent of
these lava  flows is  known, their thicknesses  are more  difficult to
constrain \citep{Thomson:2009eo}. Many  approaches, including indirect
techniques as gravity, seismic or  radar data, or direct measurements,
through analyses of impact that  have completely penetrated the maria,
have been proposed to  estimate their thicknesses.  The elastic-plated
gravity current  models provided in  this thesis  can also be  used to
provide  estimate useful  to constrain  thickness model  of the  lunar
maria.  Indeed,  if the intrusion  has stopped in the  bending regime,
the  inequality  $R<   4\Lambda$  in  the  case  of   lunar  domes  or
$R<4\Lambda/C$ in the case of  floor-fractured craters, provide for an
estimate for  the elastic thickness  and thus,  a lower bound  for the
maria thickness.

\subsubsection*{Floor-fractured craters on other terrestrial planets}

As proven of  the Moon, floor-fractured crater are a  good first basis
to probe the importance of intrusive magmatism on terrestrial planets.
While they  have first been observed  and described on the  Moon, many
evidences  show now  that  floor-fractured crater  might  be a  common
landscape on terrestrial planets.

\begin{figure}[htpb]
  \begin{center}
    \graphicspath{ {/Users/thorey/Documents/These/Manuscript/Figure/Chapter7/} }
    \includegraphics[scale=0.9]{FFCOther.eps}
    \caption{a),  b) and  c) Sample  from the  Marsian FFC  population
      located  respectively  at ($0.0^{\circ}$N,$337.3  ^{\circ}  $E),
      ($5.5^{\circ}$S,$322.6         ^{\circ}          $E)         and
      ($6.7^{\circ}$S,$333.4^{\circ}$E).   All are  TEHMIS daytime  IR
      image taken modified from \citet{Sato:2010ex}.  d) Potential FFC
      on Mercury reproduced from \citet{Schultz:1977ec}.  e) Barrymore
      crater, $50$ km diameter, located near Imdr Regio.  f) Mona lisa
      Crater,  $85$ km  in diameter,  located  on the  edge of  Eistla
      Regio.   Both  are potential  FFCs  on  Venus.  Reproduced  from
      \citet{Wichman:1995ju}.}
    \label{C7-FFCOther}
  \end{center}
\end{figure}

\begin{itemize}
\item  \textbf{Mars}: On  mars, almost  $200$ floor-fractured  craters
  have also been reported located mostly  along a narrow band south of
  the dichotomy boundary in  Arabia Terra \citep{Bamberg:2014hb}.  The
  observed deformations  within these craters  is very similar  to the
  one observed on the Moon, though Marsian floor-fracture craters tend
  to  exhibit a  more  extensive and  wider  fracture network  (Figure
  \ref{C7-FFCOther}  a,  b and  c).   This  is attributed  to  complex
  interaction of  the magmatic  intrusion with potential  ice/water in
  the  subsurface  \citep{Sato:2010ex,Bamberg:2014hb}. In  particular,
  the  melting  of the  water  (or  possibly  CO$_2$) trapped  in  the
  subsurface  would enhance  erosion of  the floor-fractured  which is
  consistent  with   some  small  and  medium   size  fluvial  outlets
  \citep{Sato:2010ex}.

  Interestingly,  deformations on  Martian floor-fractured  craters is
  not localized within the crater wall but can also extend further the
  crater rim (Figure \ref{C7-FFCOther} b,c).  In contrast to the Moon,
  the overpressure driving  the intrusion might have  been larger than
  the unloading pressure associated with the depression.  In addition,
  Martian magma,  at the  difference of  their lunar  counterpart, are
  most  likely to  be  buoyant  until the  surface  and the  mechanism
  favorable to intrusion below Martian crater is still debated.  Again
  on  Mars,  studying the  stress  field  associated with  the  crater
  depression  could  provide  a  viable  mechanism  to  trigger  magma
  spreading at depth below these craters.

\item   \textbf{Mercury}    \citet{Schultz:1977ec}   propose   several
  candidates  searching for  intra-crater dark  haloes or  other color
  variations  indicating post-impact  emplacement  of mafic  materials
  onto  the  floor.    They  did  find  several   crater  floors  with
  contrasting  deposits,  and  additionally  a  few  rimmed  moat-like
  depression (Figure \ref{C7-FFCOther} d).
  
\item \textbf{Venus}: Venus geologic record  have been largely cut off
  by  resurfacing events  constantly reworking  the Venusian  surface.
  Nevertheless, several candidates have also been proposed on Venus by
  \citet{Wichman:1995ju}.
\end{itemize}

Though most of these observations,  except for Marsian FFCs, have been
made in  the late nineties, they  provide an extremely good  basis for
work using  new data and  new methods at  hand today. In  addition, it
provides a  good opportunity to  study and constrain the  important of
shallow intrusive magmatism outside of the Moon.

\section{Modeling  the   ground  deformation  experienced   by  active
  volcanos}
\label{sec:caref-descr-tip}

The model proposed in  Chapter \ref{C3-JFM} and \ref{Heating} consider
a constant  injection rate.  In  real setting, as proposed  in Section
\ref{C4-sec:discussion}, the final intrusion might be better described
by multiple injections separated by  short rest periods. For instance,
\citet{Macedonio:2014et}   have   used   the   isoviscous   model   of
\citet{Michaut:2011kg}  and show  that the  dynamics of  such multiple
pulse injection fed magmatic intrusion  share similar pattern with the
unrest of volcanic  caldera observed in different  location across the
world.

In  other volcanic  contexts,  ground deformation  are  often used  as
precursors to eruption.  The model derived in  \ref{C5-chap6} could be
easily adapted to model the presence  of a volcanic edifice.  It could
then  be  used to  study  ground  deformations experienced  by  active
volcano as a response to  the dynamics of underground magmatic systems
\citep{Cayol:2014vo,Pedersen:2004kp,Patane:2006hn,Bonaccorso:2001iw,ChadwickJr:1995cz,Cannavo:2015fk}.


%%% Local Variables:
%%% mode: latex
%%% TeX-master: "../main"
%%% End:


%%% Local Variables:
%%% mode: latex
%%% TeX-master: "../main"
%%% End:
