\thispagestyle{fancy}
\begin{center}
 \Large
 \textbf{Résumé}
\end{center}

\textbf{Mots-clés} : Intrusion magmatique, Écoulement gravitaire sous
une plaque élastique, Refroidissement, Rhéologie, Viscosité dépendante
de la température, Fléchissement d'une plaque élastique, Laccolite,
Sill, Terre, Lune, Dômes à faible pente, Cratères d' impacte, Cratère
au sol fracturé, Anomalie gravitaire.\\

Le magmatisme intrusif est une source masquée, mais potentiellement
importante du magmatisme planétaire. En effet, les magmas, formaient au
sein du manteau, n'atteignent que rarement la surface. La grande
majorité se met en place et refroidit au sein de la croûte sous forme
d'intrusions magmatiques. Le volume ainsi que la morphologie de ces
intrusions peuvent être contraints par l’observation des surfaces
planétaires. Cependant, en l'absence d'un modèle capable de décrire la
mise en place de telles intrusions, il est difficile de se faire une
idée des propriétés physiques de l'écoulement et des magmas eux même.

Dans cette thèse, nous commençons par nous intéresser à la relation
qui existe entre la morphologie finale des intrusions de tailles
intermédiaires (sills et laccolites) et l'écoulement lui-même. Nous
proposons ainsi un modèle dynamique de la mise en place de l’intrusion
qui prend en compte une rhéologie réaliste pour le magma, l'énergie
libérée par sa cristallisation ainsi que le chauffage de l'encaissant.
Les conditions varient de la Terre à la Lune; nous examinons ainsi la
capacité du modèle à reproduire la morphologie de ces intrusions dans
ces deux différents contextes planétaires.

Sur la Lune,  la mise en place d'intrusions magmatiques  au sein de la
croûte a  aussi été proposée  pour expliquer les  déformations subites
par  certains  cratères  après  leurs formations.  Pour  tester  cette
hypothèse, nous proposons un modèle d'étalement d'intrusion magmatique
sous une dépression caractéristique de l'impact. Nous montrons que les
différentes déformations observées  au sein de ces  cratères sont bien
en accord  avec la mise  en place  d'importants volumes de  magma sous
leur sol. De plus, en utilisant la résolution sans précédente du champ
de gravité lunaire obtenue par la  mission GRAIL, nous montrons que la
plupart  de  ces cratères  montrent  bien  des anomalies  de  gravité;
anomalies impliquant notamment une  importante porosité dans la croûte
lunaire. Les implications en terme d'évolution lunaire sont finalement
évoquées.

%%% Local Variables:
%%% mode: latex
%%% TeX-master: "main"
%%% End:
