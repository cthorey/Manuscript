
\documentclass{article}           %% ceci est un commentaire (apres le caractere %)

 %% adapte le style article aux conventions francophones
\usepackage{amsmath}
\usepackage{graphicx}
\usepackage[T1]{fontenc} 

\begin{document}

\title{Intrusive magmatism  on terrestrial planets: what  can we learn
  form elastic-plated gravity current models ?}
\author{Cl�ment thorey\\
  IPGP\\
  \texttt{thorey@ipgp.fr}} \date{\today}
 
\maketitle

dezjdezj

dze


\textbf{Key  words}:   Magmatic  intrusions,   Elastic-plated  gravity
current, Thermal processes, Rheology, Temperature-dependent viscosity,
Elastic  sheet,   Laccolith,  Sill,  Earth,  Moon,   Low-slope  domes,
Floor-fractured craters, Elastic-sheet thickness, Crater depression,
Gravitational anomalies.\\

\textbf{Abstract}\\

When  magma is  forced  to the  surface,  only a  small  amount of  it
actually reaches  that level. Most of  the magma is intruded  into the
crust where  it solidifies  into a  wide range  of features,  from the
small scale  sills and laccoliths  to large scale  batholiths (several
hundred kilometers in  size). On Earth, these  magmatic intrusions are
often exposed at the surface by later erosion and geophysical sounding
can provide information on their  presence, morphology and growth even
at  depth.   On  terrestrial  planets,  our  only  way  to  probe  the
importance of intrusive  magmatism is to look  for surface deformation
produced by  eventual shallow  magmatic intrusions.   On the  Moon for
instance,  several  morphological  structures have  been  proposed  to
result from  the emplacement  of shallow  magmatic intrusions  such as
low-slope   domes  in   the  lunar   maria  or   the  atypical   floor
characteristics   of    floor-fractured   craters.     However,   such
observations must be linked to  dynamic models of of magma emplacement
at depth in order to  provide insights into magma physical properties,
injection rate, emplacement depth and the intrusion process itself.

In this  thesis, we first  investigate the relation between  the final
shape  of shallow  intermediate-scale magmatic  intrusions (sills  and
laccoliths) and their  cooling.  We propose a model  for the spreading
of  an elastic-plated  gravity  current  with a  temperature-dependent
viscosity  that   accounts  for  a  realistic   magma  rheology,  melt
crystallization and heating of the surrounding medium.  The mechanisms
that drive cooling  of the intrusions vary from the  Earth to the Moon
and  the ability  of the  model  to reproduce  the final  morphologies
(aspect ratio) of terrestrial laccoliths  and low-slope lunar domes is
examined.

On the  Moon, emplacement  of magmatic intrusions  into the  crust has
also  been proposed  as  a  possible mechanism  for  the formation  of
floor-fractured  craters.  We  propose a  model for  an elastic-plated
gravity current  spreading beneath  an elastic overburden  of variable
thickness.  We  find that  several characteristics  of floor-fractured
craters are indeed  consistent with the emplacement of  a large volume
of magma  beneath their floor.   In addition, using  the unprecedented
resolution  of the  NASA's  Gravity Recovery  and Interior  Laboratory
(GRAIL) mission,  in combination  with topographic data  obtained from
the  Lunar Orbiter  Laser Altimeter  (LOLA) instrument,  we show  that
lunar   floor-fractured   craters  present   gravitational   anomalies
consistent with magmatic intrusions intruding a crust characterized by
a $12\%$  porosity. The implications  in terms of lunar  evolution are
examined.

\end{document}
%%% Local Variables:
%%% mode: latex
%%% TeX-master: t
%%% End:
