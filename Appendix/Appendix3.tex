\chapter{Effect of the prewetting film}
\label{chap:A3}

To  mitigate the  problem at  the contact  line, we  introduce a  thin
prewetting     film,     with     thickness    $h_f$     such     that
$h(r,t)\rightarrow    h_f$   as    $r\rightarrow   \infty$    (Section
\ref{C2-sec:need-regularization}).  The  meaning of a  thin prewetting
film in  the application  to the spreading  of magmatic  intrusions is
unclear.  In  particular, the  model shows  no convergence  when $h_f$
tends to zero \citep{Lister:2013ia} and therefore, the thickness $h_f$
might be  linked to some structural  length scale at the  front of the
laccolith  or  to  the  natural imperfection  of  the  flow  geometry.
Reasonable values for $h_f$ are  values with physical significance for
this structural  length scale at the  tip and should range  from a few
centimeters    to    no    less   than    $0.1$    millimeter,    i.e.
$10^{-4}\le h_f \le 10^{-2}$ . In this appendix, we discuss the effect
of changing the prewetting film  thickness $h_f$ over this interval on
some results presented in Chapter \ref{C3-JFM} and \ref{Heating}.

\section{Scaling laws for the thickness and the radius}

The scaling laws for  the thickness $h_0(t)$ (\ref{C3-ScalingH-Visco})
as well as for the  radius $R(t)$ (\ref{C3-ScalingR-Visco}) derived in
Section  \ref{C3-sec:evol-bend-regime} depend  on the  film thickness
$h_f$.
\begin{figure}[h!]
  \begin{center}
    \graphicspath{ {/Users/thorey/Documents/These/Projet/Refroidissement/Skin_Model/Figure/JFM_V13/} }
    \includegraphics[scale=0.4]{Scaling_HR_ELAS_APPENDIX.eps}
    \caption{Left:    Dimensionless    thickness   at    the    center
      $h_0h_f^{1/11}$ versus dimensionless time $t$ for different sets
      $(\nu,h_f)$ indicated  on the plot.  Dashed-lines  represent the
      scaling   laws   $h_0h_f^{1/11}=   0.7\nu^{-2/11}t^{8/22}$   for
      $\nu = 1.0$ and $0.001$.  Right: Dimensionless radius $R$ versus
      dimensionless   time  $t$   for  the   same  sets   $(\nu,h_f)$.
      Dashed-lines       represent        the       scaling       laws
      $Rh_f^{-1/22}=  2.2\nu^{1/11}t^{7/22}$  for   $\nu  =  1.0$  and
      $0.001$.          Here,        $\Omega=         10^5$        and
      $\eta(\theta)=\eta_1(\theta)$.}
    \label{Scaling_HR_ELAS_APPENDIX}
  \end{center}
\end{figure}
Accordingly,  when rescaling  the thickness  by $h_f^{-1/11}$  and the
radius by $h_f^{1/22}$, the different simulations collapse on the same
curve (Figure \ref{Scaling_HR_ELAS_APPENDIX}).

Similarly,  when  rescaling  the  extent  of  the  cold  fluid  region
$R(t)-R_c(t)$ by $h_f^{7/66}$, the different simulations also collapse
on the same curve  (Figure \ref{R_Rc_ELAS_APPENDIX}).  Similar results
can  be  obtained for  $R(t)-R_c(t)$  in  the  framework of  the  more
realistic model described in Chapter \ref{Heating}. These scaling laws
are  thus able  to  account  for the  effect  of  the prewetting  film
thickness $h_f$ which is, in general, rather weak.

\begin{figure}[h!]
  \begin{center}
    \graphicspath{ {/Users/thorey/Documents/These/Projet/Refroidissement/Skin_Model/Figure/JFM_V13/} }
    \includegraphics[scale=0.4]{R_Rc_ELAS_APPENDIX.eps}
    \caption{Left:  Extent  of  the cold  fluid  region  $R(t)-R_c(t)$
      versus   dimensionless    time   for    different   combinations
      ($\nu$,$h_f$) indicated on the plot.  Right: Same plot but where
      we  have  rescaled  the  extent  of the  cold  fluid  region  by
      $h_f^{7/66}$.          Dashed-line:          scaling         law
      $(R(t)-R_c(t))h_f^{-7/66}= 2.1 Pe^{-1/3}\nu^{7/33}t^{9/22}$.}
    \label{R_Rc_ELAS_APPENDIX}
  \end{center}
\end{figure}

\section{Phase diagram}

The phase diagram presented in section \ref{C4-sec:evol-with-bend} and
its application  to the  spreading of laccoliths  also depends  on the
chosen value for $h_f$. However, as the dependence with $h_f$ is weak,
a variation of  $2$ orders of magnitude does  not change significantly
the results (Figure \ref{C3-PhaseDiagramJFM_Appendix}).

\begin{figure}[h!]
  \begin{center}
    \graphicspath{ {/Users/thorey/Documents/These/Projet/Refroidissement/Skin_Model/Figure/Figure_Heating/} }
    \includegraphics[scale=0.36]{PhaseDiagreRocchie_APPENDIX.eps}
    \caption{Phase diagram for the  evolution with bending and gravity
      for different  combinations ($\nu$,$Pe_m$) and  different values
      for the film thickness $h_f = 10^{-2}$ and $10^{-4}$.}
    \label{C3-PhaseDiagramJFM_Appendix}
  \end{center}
\end{figure}

The  same  result hold  when  we  look  at  the relation  between  the
thickness and the  radius of the laccolith  (\ref{C4-Hr}). Indeed, the
best fit value  for the viscosity contrast scales  as $h_f^{-1/2}$ and
therefore,  varying  $h_f$ by  two  orders  of magnitudes  change  the
viscosity contrast by  one order of magnitude which  is acceptable for
our application.   For instance, for  the parameters listed  in Figure
\ref{C4-Data}, the best fit value for the viscosity contrast scales is
$\nu_{\text{best}}\approx        h_f^{-1/2}2.59~10^{-10}$,        i.e.
$\nu_{\text{best}}\approx  2.6\times   10^{-9}$  for   $h_f=0.01$  and
$\nu_{\text{best}}\approx 2.6\times 10^{-8}$ for $h_f=10^{-4}$.


\section{Two stage growth in the second bending phase}
\label{C4-Heat:AppendixC}

In Chapter  \ref{Heating}, for some  simulations, the second  phase of
important thickening  in the  bending regime occurs  in two  stages: a
first stage  where the  thickness drastically  increases and  a second
stage  where   it  continues   increasing  but  much   slower  (Figure
\ref{C4-Scaling_HR_ELAS_Omega} and \ref{C4-Scaling_HR_ELAS_Rheology}).
To get some insights into this transition, we run some simulations for
$\Omega=1.0$  with  a  higher  spatial  resolution,  i.e.   $Dr=0.005$
instead               of               $Dr=0.01$               (Figure
\ref{C4-Appendix_Phase2_N11_0_Pe100_0_nu0_001}).

\begin{figure}[h!]
  \begin{center}
    \graphicspath{ {/Users/thorey/Documents/These/Projet/Refroidissement/Skin_Model/Figure/Figure_Heating/} }
    \includegraphics[scale=0.42]{Appendix_Phase2_N11_0_Pe100_0_nu0_001.eps}
    \caption{Dimensionless thickness  $h_0$ versus  dimensionless time
      $t$ for  $Pe=100.0$, $\nu=0.001$, $\Omega=1.0$ and  the rheology
      $\eta_1(\theta)$.  Colors  refer to the time  $t$.  Dotted line:
      Scaling  law $h_0=  0.7h_f^{-1/11}\nu^{-2/11}t^{8/22}$. Vertical
      dashed-lines:  initial,  intermediate  and final  times  of  the
      temperature profiles plotted in c).  b) Dimensionless radius $R$
      versus dimensionless  time $t$  for $Pe=100.0$,  $\nu=0.001$ and
      the rheology  $\eta_1(\theta)$.  Colors  refer to the  time $t$.
      Dotted line:  Scaling law  $R= 2.2h_f^{1/22}\nu^{1/11}t^{7/22}$.
      Vertical dashed-lines: same than in a). c) Dimensionless average
      temperature over  the flow thickness  $\overline{\theta}$ versus
      radial   axis  $r$   for  times   between  $t=3.8~10^{-3}$   and
      $t=7.6~10^{-2}$.   Dashed-line profiles:  profiles at  the three
      different times  underlined in a)  and b). Colors also  refer to
      the time on the same scale than  a) and b).  d), e) and f), same
      plots  than  a),  b)  and  c) but  for  the  Arrhenius  rheology
      $\eta_2$.}
    \label{C4-Appendix_Phase2_N11_0_Pe100_0_nu0_001}
  \end{center}
\end{figure}

The  simulations   show  that  this  transition   corresponds  to  the
detachment       of       the      thermal       anomaly       (Figure
\ref{C4-Appendix_Phase2_N11_0_Pe100_0_nu0_001}).     In    particular,
during the first  stage, the thermal anomaly is still  attached to the
tip  and  the  prewetting  film, located  beyond  $r=R(t)$,  is  still
cooling.  In contrast, during the second stage, which is characterized
by  a decrease  in the  thickening rate,  the prewetting  film located
beyond $r=R(t)$  is entirely cold  , i.e.  $\overline{\theta}  =0$ for
$r>R(t)$ and the thermal anomaly slowly gets away from the tip (Figure
\ref{C4-Appendix_Phase2_N11_0_Pe100_0_nu0_001}).   For  instance,  for
$\eta_1(\theta)$, $\nu=0.001$ and  $Pe=100.0$, the transitions between
the two  stages occurs at  $t=1.8~10^{-2}$ and indeed coincide  to the
film          becoming          entirely         cold          (Figure
\ref{C4-Appendix_Phase2_N11_0_Pe100_0_nu0_001}  a,  b,  c).   For  the
rheology  $\eta_2(\theta)$, the  transition  is  smoother because  the
viscosity   increases  on   a  wide   range  of   temperature  (Figure
\ref{C4-Appendix_Phase2_N11_0_Pe100_0_nu0_001} d, e,  f). Even if this
transition  should be  present for  all the  simulations, the  smaller
spatial resolution used in this Chapter \ref{C3-JFM} and \ref{Heating}
does not allow to resolve this  transition for all the combinations of
the dimensionless numbers.

%%% Local Variables:
%%% mode: latex
%%% TeX-master: "../main"
%%% End:
