\chapter{Core budget calculations}
\label{chap:A3}

\section{Entropy budget}

Entropy is a measure of the available energy in a system. When only reversible processess exist, this
value stays constant. However dissipation is an irreversible process and the available energy in a 
closed system decreases with time (i.e. entropy increases with time).
The total derivative of the entropy can be written as
%
\begin{equation}
\rho T \frac{ds}{dt} = \rho T \left(\frac{\partial s}{\partial t}\right)_{P,c} \frac{dT}{dt}
	+ \rho T \left(\frac{\partial s}{\partial c}\right)_{P,T} \frac{dc}{dt}
	+ \rho T \left(\frac{\partial s}{\partial P}\right)_{T,c} \frac{dP}{dt}
\end{equation}
where the right-hand side corresponds to the contribution from core cooling, chemical reactions and 
core contraction, respectively. The second and third terms have been shown to be small \cite[e.g.,][]{Nimmo:2007tw} 
and we will retain only the core cooling term here. We then use Maxwell's relation for the partial
derivative of the entropy and rewrite the total derivative of entropy as
%
\begin{equation}
\rho T \frac{ds}{dt} = \rho c_p \frac{dT}{dt}. 
\label{A3:eq2}
\end{equation}

Second, the local entropy budget can be written as \citep[e.g.,][]{Gubbins:2004ba}
%
\begin{equation}
\rho \frac{ds}{dt} = - \frac{\nabla \cdot q}{T} + \frac{\mu \nabla \cdot i}{T} + \frac{J^2}{\sigma T},
\end{equation}
%
where q and i are the heat and solute flux vector, respectively, J is the electric current density, $\sigma$ is 
the electrical conductivity and $\mu$ is the chemical potential. We can then use Eq (\ref{A3:eq2}) to obtain
%
\begin{equation}
\frac{\rho c_p}{T} \frac{dT}{dt} = 	- \frac{\nabla \cdot q}{T} 
									+ \frac{\mu \nabla \cdot i}{T} 
									+ \frac{J^2}{\sigma T}.
\label{A3:eq4}
\end{equation}

Now we can rewrite the first term on the right-hand side of Eq (\ref{A3:eq4}) as
%
\begin{eqnarray}
\frac{\nabla \cdot q}{T} &=& \nabla \cdot \left(\frac{q}{T}\right) - \frac{q \nabla T}{T^2} \nonumber \\
				 &=& \nabla \cdot \left(\frac{q}{T}\right) - k \left(\frac{\nabla T}{T}\right)^2.
\end{eqnarray}

Using this result in Eq (\ref{A3:eq4}), we obtain
%
\begin{equation}
\frac{\rho c_p}{T} \frac{dT}{dt} = - \nabla \cdot \left(\frac{q}{T}\right) + k \left(\frac{\nabla T}{T}\right)^2
								 + \frac{\mu \nabla \cdot i}{T} + \frac{J^2}{\sigma T},
\label{A3:eq6}
\end{equation}
%
which can then be integrated over the core volume to get
%
\begin{equation}
\int \frac{\rho c_p}{T} \frac{dT}{dt} dV = - \frac{Q_c}{T_c} + \int k \left(\frac{\nabla T}{T}\right)^2 dV
		 + \frac{Q_L}{T_i} + \frac{\Phi}{T_{\phi}},
\label{A3:eq7}
\end{equation}
%
where we used the divergence theorem on the first and last term on the right-hand side of Eq (\ref{A3:eq6}).

Replacing the energy budget $Q_c = Q_L + Q_g + Q_s$ in Eq (\ref{A3:eq7}) and regrouping similar terms, we obtain
%
\begin{equation}
\int \rho c_p \left(\frac{1}{T_c} - \frac{1}{T}\right) \frac{dT}{dt} dV + 
Q_L \left(\frac{1}{T_c}-\frac{1}{T_i}\right) +
\frac{Q_g}{T_c} =
\int k \left(\frac{\nabla T}{T}\right)^2 dV + \frac{\phi}{T_i}
\end{equation}
%
where we assumed that most of the dissipation occurs at the inner core boundary. This can finally be identified
with the entropy budget presented in Chapter 6 to yield 
%
\begin{eqnarray} 
E_S &=& \int \rho c_p \left( \frac{1}{T_c} - \frac{1}{T} \right) \frac{dT_c}{dt}dV, \\ 
E_L &=& Q_L \left( \frac{1}{T_c} - \frac{1}{T_i} \right), \\ 
E_g &=& \frac{Q_g}{T_c}, \\  
E_k &=& \int k \left(\frac{\nabla T}{T}\right)^2 dV. \end{eqnarray}

\section{Simplified equations}

In this section, we derive a simplified version of Equations (\ref{eq:qsa}) to (\ref{eq:qga}) to
second order in (r/D). For definition of the variables, please refer to Chapter \ref{chap6}.

\subsection*{Secular term}

The starting equation is the following
%
\begin{equation}
Q_s = \int \rho c_p \frac{dT_c}{dt} dV.
\end{equation}

Assuming that the outer core follows an adiabatic gradient and that the inner core is isothermal, the entire core thermal state is defined by one temperature. The temperature profile in the outer core is defined by

\begin{equation} T_a(r,T_c) = T_c \: exp\left(-\frac{r^2-r_c^2}{D^2}\right), \end{equation} 
%
as defined in Equation (\ref{eq:adiabat}). A Taylor expansion of second order in (r/D) gives
%
\begin{equation} T_a(r,T_c) = T_c \: \left(1 - \frac{r^2-r_c^2}{D^2}\right). \label{eqa:taylor} \end{equation} 
%
The ratio (r/D) is of order 0.1, thus the error on temperature is less than 0.01 K. The secular
cooling term can then be developed as
%
\begin{eqnarray}
Q_s &=& \int \rho c_p \frac{dT_c}{dt} \left(1 - \frac{r^2-r_c^2}{D^2}\right) dV \nonumber \\
	&=& M_c c_p \frac{dT_c}{dt} - 4\pi\rho c_p \frac{dT_c}{dt} \int r^2 \frac{r^2-r_c^2}{D^2} dr \nonumber \\
	&=& M_c c_p \frac{dT_c}{dt} - 4\pi\rho c_p \frac{dT_c}{dt} \left(\frac{r_c^5}{5D^2} - \frac{r_c^5}{3D^2}\right) \nonumber \\
	&=& M_c c_p \frac{dT_c}{dt} \left(1 + \frac{2}{5}\frac{r^2}{D^2}\right),
\end{eqnarray}
%
where $M_c = 4\pi \rho r_c^3/3$.

\subsection*{Latent heat term}

The starting equation is the following
%
\begin{equation}
Q_L = 4 \pi r_i^2 L_H \rho^{ic} \frac{dr_i}{dt},
\end{equation}
%
where 
%
\begin{equation}
\frac{dr_i}{dt} = \frac{1}{(\Delta - 1)dT_a/dP} \frac{1}{\rho^{ic} g} \frac{T_i}{T_c} \frac{dT_c}{dt}.
\label{eqa:inner}
\end{equation}

From Equation (\ref{eqa:taylor}), $dT_a/dT = -2rT/D^2$. In addition, assuming hydrostatic equilibrium $dT/dP = 1/\rho g$. Thus
\begin{eqnarray}
Q_L &=& 4 \pi r_i^2 L_H \rho^{ic} \frac{dr_i}{dt} \nonumber \\
	&=& \frac{4 \pi r_i^2 L_H}{\Delta - 1} \frac{T_i}{T_c} \frac{1}{\frac{2r_i T_i}{\rho D^2}} \frac{dT_c}{dt} \nonumber \\
	&=& \frac{4 \pi L_H}{\Delta - 1} \frac{\rho D^2 f r_c}{2 T_c} \frac{dT_c}{dt} \nonumber \\
	&=&\frac{3}{2} M_c \frac{f L_H}{T_c} \frac{D^2}{r_c^2} \frac{1}{\Delta - 1} \frac{dT_c}{dt},
\end{eqnarray}
%
where $M_c$ is the total mass of the core, $f = r_i/r_c$ and $T_i \simeq T_c$ has been assumed.

\subsection*{Gravity term}

The starting equation is the following
%
\begin{equation}
Q_g = \int \rho \phi \alpha_c \frac{Dc}{Dt} dV.
\end{equation}

It can be shown \citep[see][]{Nimmo:2007tw} that
%
\begin{eqnarray}
Q_g &=& \left[ \int_{oc} \rho \psi dV - M_{oc}\psi(r_i) \right] \alpha_c \frac{Dc}{Dt} \\
	&=& (A-B) \cdot C.
\end{eqnarray}

The several parts of this equation can be written as
%
\begin{eqnarray}
A	&=& 4\pi\rho \int_{oc} \frac{2}{3} \pi G \rho r^4 dr \nonumber \\
	&=& \frac{(4\pi)^2}{6} \rho^2 G \left( \frac{r_c^5 - r_i^5}{5} \right) \nonumber \\
	&=& \frac{(4\pi)^2}{6} \rho^2 G r_c^5 \left( \frac{1 - f^5}{5} \right)
\end{eqnarray}

In addition,
%
\begin{eqnarray}
B	&=& \frac{4}{3} \pi \rho (r_c^3 - r_i^3) \cdot \frac{2}{3} \pi G \rho ri^2 \nonumber \\
	&=& \frac{(4\pi)^2}{6} \rho^2 G \left( \frac{r_c^3-r_i^3}{3}\cdot r_i^2 \right) \nonumber \\
	&=& \frac{(4\pi)^2}{6} \rho^2 G r_c^5 \left( \frac{1 - f^3}{3}\cdot f^2 \right)
\end{eqnarray}
%
and using Equation (\ref{eqa:inner}) again,
%
\begin{eqnarray}
C	&=& \alpha_c \frac{4\pi r_i^2 \rho_i c}{M_{oc}} \frac{dr_i}{dt} \nonumber \\
	&=& \alpha_c \frac{4\pi r_i^2 \rho_i c}{M_{oc}} \frac{D^2}{2 T_c f r_c (\Delta - 1)} \frac{dT_c}{dt} \nonumber\\
	&=& \frac{\Delta \rho_c}{\rho_i} \frac{c}{\Delta c} \frac{3}{2 r_c^2} \frac{f}{1-f^3} \frac{D^2}{T_c} \frac{1}{\Delta - 1} \frac{dT_c}{dt}
\end{eqnarray}
%
where we have used the fact that $\alpha_c = \Delta \rho_c/(\rho_i \Delta c)$. Now assuming
$c/\Delta c = 1$, and merging A, B and C, one has
%
\begin{eqnarray}
Q_g &=& \frac{(4\pi)^2}{6} \rho^2 G r_c^5 \left( \frac{1-f^5}{5} - f^2 \cdot \frac{1-f^3}{3} \right) \cdot C \nonumber \\
	&=& 3 \pi \rho G M_c F \frac{\Delta \rho_c}{\rho_i}  \frac{D^2}{T_c} \frac{1}{\Delta - 1} \frac{dT_c}{dt}
\end{eqnarray}
%
where
%
\begin{equation}
F = \frac{f}{1-f^3}\left( \frac{1}{5} + \frac{2f^5}{15} - \frac{f^3}{3} \right).
\end{equation}
