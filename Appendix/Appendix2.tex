\chapter{Additional material}
\label{chap:A2}

\section{Additional figures}

In this appendix, we provide supplementary figures to illustrate thermal evolution scenarios that
were not directly shown in the main text.

\begin{figure}\begin{center}
\includegraphics[width=\textwidth]{Chapter4/A01.jpg}
\caption{Temperature cross sections of the lunar mantle for a complete thermal evolution for the
case with an intermediate initial temperature profile and the KREEP layer redistributed within crust
(model `T-0LW'). Numbers correspond to time before present in Ga. The black circle is the lunar core
and white corresponds to regions that are partially molten. The streamlines are shown as dashed
lines.} \label{fig:A1} \end{center} \end{figure}

\begin{figure}\begin{center}
\includegraphics[width=\textwidth]{Chapter4/A02.jpg}
\caption{Temperature cross sections of the lunar mantle for a complete thermal evolution for the
case with a cold initial temperature profile and the KREEP layer located below the crust
(model `A-0LB'). Numbers correspond to time before present in Ga. The black circle is the lunar core
and white corresponds to regions that are partially molten. The streamlines are shown as dashed
lines.} \label{fig:A2} \end{center} \end{figure}

\begin{figure}\begin{center}
\includegraphics[width=\textwidth]{Chapter4/A03.jpg}
\caption{Temperature cross sections of the lunar mantle for a complete thermal evolution for the
case with a hot initial temperature profile and the KREEP layer located below the crust
(model `D-0LB'). Numbers correspond to time before present in Ga. The black circle is the lunar core
and white corresponds to regions that are partially molten. The streamlines are shown as dashed
lines.}\label{fig:A3}  \end{center} \end{figure}

\begin{figure}\begin{center}
\includegraphics[width=0.49\textwidth]{Chapter4/A04a.eps}
\includegraphics[width=0.49\textwidth]{Chapter4/A04b.eps}
\caption{Cumulate melt volume on the nearside (left) and farside (right). The color indicates the
initial temperature profile (black is intermediate, blue is cold and red is hot). Solid lines
correspond to cases with the KREEP layer located below the crust whereas the dashed line corresponds
to KREEP initially redistributed within the crust. The nomenclature follows Table \ref{tab:summary}.} 
\label{fig:A4} \end{center} \end{figure} 


