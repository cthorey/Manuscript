\chapter{Details on the phase diagram}
\label{chap:A2}

A current in the $i$th thermal phase can transition in the $j$th phase
of the gravity regime where $i  \ge j$. An informal indication of this
result is that the effective viscosity being that of a small region at
the tip  in the bending regime  and the average flow  viscosity in the
gravity regime, it cannot increase during the transition.
\begin{figure}[h!]
  \begin{center}
    \graphicspath{ {/Users/thorey/Documents/These/Projet/Refroidissement/Skin_Model/Figure/Figure_Heating/} }
    \includegraphics[scale=0.4]{PhaseDiagramJFM_Appendix_Heating_2.eps}
    \caption{a)     Phase     transitions    reported     in     Table
      \ref{tab:ParameterAnalysis} for  the model described  in Chapter
      \ref{C3-JFM}. Each scenario $B_iG_j$ is  defined by two or three
      conditions, each of the transition defining a specific region in
      the phase  diagram.  The  intersection of the  different regions
      have  to be  non-zero  in the  range  of proposed  dimensionless
      number for a scenario to exist.  For instance, $B_1G_2$ does not
      exist                                                    because
      $\{(Pe_m,\nu):       Pe_m>\alpha_1      h_f^{-15/7}\}       \cap
      \{(Pe_m,\nu):Pe_m<\alpha_2           h_f^{-1/7}\}           \cap
      \{(Pe_m,\nu):\nu<\alpha_3     Pe_m^2h_f^{2/7}\}     =     \{\}$.
      Same plot but for the  more realistic model described in Chapter
      \ref{Heating}.}
    \label{PD_ALLpossible}
  \end{center}
\end{figure}
More  formally, each  evolution $B_iG_j$  is defined  by two  or three
conditions, each of  the transition defining a specific  region in the
phase   diagram   (Table    \ref{tab:ParameterAnalysis}   and   Figure
\ref{PD_ALLpossible}).  The intersection of  the different regions has
to be non-empty  in the range of dimensionless numbers  for a scenario
to exist. In particular, we have
\begin{itemize}
\item
  $B_1G_2          =         \{(Pe_m,\nu):t_t^h<t_{b2}\}          \cap
  \{(Pe_m,\nu):t_t^h>t_{g2}\} \cap \{(Pe_m,\nu):t_t^h<t_{g3}\}=\{\}$
\item
  $B_1G_3          =         \{(Pe_m,\nu):t_t^h<t_{b2}\}          \cap
  \{(Pe_m,\nu):t_t^h>t_{g3}\} =\{\}$
\item
  $B_2G_3          =         \{(Pe_m,\nu):t_t^h>t_{b2}\}          \cap
  \{(Pe_m,\nu):t_t^c<t_{b3}\} \cap \{(Pe_m,\nu):t_t^h>t_{g3}\}=\{\}$
\end{itemize}
Hence,  $B_1G_2$,   $B_1G_3$  and   $B_2G_3$  are   unfeasible  (Table
\ref{tab:ParameterAnalysis}  and Figure  \ref{PD_ALLpossible} a).   In
addition, the  transition from  the third bending  phase to  the first
gravity phase implies that $t_t^c>t_{b3}$ and $t_t^c<t_{g2}$, which is
not   possible    (Table   \ref{tab:ParameterAnalysis}    and   Figure
\ref{PD_ALLpossible} a).  Therefore, the  five possible sequences that
remain are $B_1G_1$, $B_2G_1$,  $B_2G_2$, $B_3G_2$ and $B_3G_3$ (Table
\ref{tab:ParameterAnalysis} and Figure \ref{PD_ALLpossible} a).

In the  more realistic model  described in Chapter  \ref{Heating}, the
time to  enter the  third flow  phase is delayed  in both  regimes. In
particular, for the current that  has reached the third bending phase,
$t_t^c>t_{b3}$  now  implies  $\nu>2.8\cdot10^{-7}  Pe^{7/2}hf^{-1/2}$
(Figure \ref{PD_ALLpossible}  b).  In addition, comparing  $t_t^c$ and
$t_{g3}$  now reads  $\nu<4.6\cdot 10^9  Pe_m^{14/3}h_f^{2/3}$ (Figure
\ref{PD_ALLpossible} b).

\begin{table}[h!]
  \begin{center}
    \scalebox{0.8}{
      \begin{tabular}{cccccc}
        Transition &Condition 1& Condition 2& Condition 3&Output\\
        \multicolumn{5}{c}{Transition  in  the   first  bending  thermal
        phase $B1$} \\
        $t_t=t_t^h$ & $t_t^h<t_{b2}$ & $t_t^h<t_{g2}$ &- &$B_1G_1$\\
                   &$Pe_m>\alpha_1h_f^{-15/7}$&$Pe_m>\alpha_2 h_f^{-1/7}$&- & Feasible\\
        $t_t=t_t^h$ & $t_t^h<t_{b2}$ & $t_t^h>t_{g2}$ &$t_t^h<t_{g3}$&$B_1G_2$ \\
                   &$Pe_m>\alpha_1h_f^{-15/7}$&$Pe_m<\alpha_2 h_f^{-1/7}$&$\nu<\alpha_3Pe_m^2h_f^{2/7}$& Unfeasible\\
        $t_t=t_t^h$ & $t_t^h<t_{b2}$ & $t_t^h>t_{g3}$ &-&$B_1G_3$ \\
                   &$Pe_m>\alpha_1h_f^{-15/7}$&$\nu>\alpha_3Pe_m^2h_f^{2/7}$&-&
                                                                                Unfeasible\\
        \multicolumn{5}{c}{Transition  in  the   second  bending  thermal
        phase $B2$} \\
        $t_t^h<t_t<t_t^c$ & $t_t^h>t_{b2}$ & $t_t^c<t_{b3}$ &$t_t^c<t_{g2}$& $B_2G_1$\\
                   &$Pe_m<\alpha_1h_f^{-15/7}$&$\nu<\alpha_4
                                                Pe_m^{7/2}h_f^{-1/2}$&$\nu>\alpha_5
                                                                       Pe_m^{-7/2}h_f^{-1/2}$&Feasible\\
        $t_t^h<t_t<t_t^c$ & $t_t^h>t_{b2}$ &$t_t^c<t_{b3}$& $t_t^c<t_{g3}$&$B_2G_2$ or $B_2G_1$\\
                   &$Pe_m<\alpha_1h_f^{-15/7}$&$\nu<\alpha_4                    Pe_m^{7/2}h_f^{-1/2}$
                                            &$\nu<\alpha_6Pe_m^{14/3}h_f^{2/3}$&
                                                                                 Feasible\\
        $t_t^h<t_t<t_t^c$ & $t_t^h>t_{b2}$ &$t_t^c<t_{b3}$& $t_t^h>t_{g2}$&$B_2G_2$\\
                   &$Pe_m<\alpha_1h_f^{-15/7}$&$\nu<\alpha_4      Pe_m^{7/2}h_f^{-1/2}$      &$Pe_m<\alpha_2
                                                                                               h_f^{-1/7}$&
                                                                                                            Feasible\\
        $t_t^h<t_t<t_t^c$ & $t_t^h>t_{b2}$ &$t_t^c<t_{b3}$& $t_t^h>t_{g3}$&$B_2G_3$\\
                   &$Pe_m<\alpha_1h_f^{-15/7}$&$\nu<\alpha_4
                                                Pe_m^{7/2}h_f^{-1/2}$
                                            &$\nu>\alpha_3Pe_m^2h_f^{2/7}$&
                                                                            Unfeasible\\
        \multicolumn{5}{c}{Transition  in  the   third  bending  thermal
        phase $B3$} \\
        $t_t=t_t^c$ & $t_t^c>t_{b3}$ & $t_t^c<t_{g2}$ &-&$B_3G_1$\\
                   &$\nu> \alpha_4 Pe_m^{7/2}h_f^{-1/2}$&$\nu>\alpha_5 Pe_m^{-7/2}h_f^{-1/2}$&-&Unfeasible\\
        $t_t=t_t^c$ & $t_t^c<t_{b2}$ & $t_t^c>t_{g2}$ &$t_t^c<t_{g3}$& $B_3G_2$\\
                   &$\nu> \alpha_4 Pe_m^{7/2}h_f^{-1/2}$&$\nu<\alpha_5 Pe_m^{-7/2}h_f^{-1/2}$&$\nu<\alpha_6Pe_m^{14/3}h_f^{2/3}$& Feasible\\
        $t_t=t_t^c$ & $t_t^c<t_{b2}$ & $t_t^c>t_{g3}$ &-&$B_3G_3$ \\
                   &$\nu>\alpha_4 Pe_m^{7/2}h_f^{-1/2}$&$\nu>\alpha_6Pe_m^{14/3}h_f^{2/3}$&-& Feasible
      \end{tabular}}
    \caption{Parameter  space analysis.   All  conditions  have to  be
      respected for a scenario to be possible. For the model described
      in  Chapter \ref{C3-JFM},  the coefficients  are: $\alpha_1=65$,
      $\alpha_2=650$,   $\alpha_3=151$,   $\alpha_4=8.3\cdot10^{-13}$,
      $\alpha_5=7.0  \cdot  10^{9}$,  $\alpha_6=0.3$.   For  the  more
      realistic   model   derived   in  Chapter   \ref{Heating},   the
      coefficients      are:       $\alpha_1=65$,      $\alpha_2=650$,
      $\alpha_3=40000$,          $\alpha_4=          2.8\cdot10^{-7}$,
      $\alpha_5=7.0 \cdot 10^{9}$, $\alpha_6=4.6 \cdot 10^9$.}
    \label{tab:ParameterAnalysis}
  \end{center}
\end{table}




%%% Local Variables:
%%% mode: latex
%%% TeX-master: "../main"
%%% End:
