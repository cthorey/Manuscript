\chapter{Numerical  scheme   for  a  cooling   elastic-plated  gravity
  current}
\label{chap:A1}

In this  appendix, we present the  numerical scheme used to  solve the
coupled  nonlinear partial  differential  equations (\ref{C4-HF})  and
(\ref{C4-TF}).    The  governing   equations   presented  in   Chapter
\ref{C3-JFM}     are     just      a     particular     case     where
$\Omega \rightarrow \infty$ and can  be solved using $\Omega=10^5$ for
instance.

\section{General procedure}
\label{C3-sec:general-procedure}

The coupled nonlinear partial differential equations (\ref{C4-HF}) and
(\ref{C4-TF}) are solved on a grid of size $M$ defined by the relation
$r_i =  (i-0.5)\Delta r$ for  $i=1,..,M$. The  grid is shifted  at the
center to avoid  problem arising from the  axisymmetrical geometry. We
index the grid point by the indice $i$ and denote the solution on this
grid  $h_i$ and  $\xi_i$ and  the secondary  variables $\Theta_{b,i}$,
$\Theta_{s,i}$ and $\delta_i$. Both equations  can be expressed on the
convenient form
\begin{equation}
  \frac{\partial u}{\partial t} - f = 0
\end{equation}
where $u$  is the function we  want to integrate and  $f$ a non-linear
function  that depends  on $u$.   We  solve these  equations by  first
discretizing all the spatial  derivatives using Finite Difference. The
accuracy of the  scheme is determined by the  higher order derivatives
since  their numerical  approximation requires  the largest  number of
sample points.  We  then get two systems of  $M$ ordinary differential
equations with the form
\begin{equation}
  \frac{\partial u_i}{\partial t} - f_i = 0 \hspace{1cm} i = 1,...,M
\end{equation}
The time derivatives are first  order and, since explicit schemes tend
to be  very sensitive and unstable,  we use a fully  implicit backward
Euler scheme to get
\begin{equation}
  \frac{u_i^{n+1}-u_i^n}{\Delta t} - f_i(u_i^{n+1}) = 0 \hspace{1cm} i
  = 1,...,M
  \label{C4-Num-1}
\end{equation}
Since  $f_i(u_i^{n+1})$ is  not a  linear function,  the system  above
cannot be re-arranged to solve $u_i^{n+1}$ in term of $u_i^{n}$ and an
iterative method  has to  be employed  instead. Fixed  point iteration
method have shown  poor results in converging toward  the solution and
we finally apply  second order Newton's method to  obtain the solution
at each time step.  In particular, we first linearize $u^{n+1}$ around
a guess  of the solution  by assuming $u^{n+1}=u^*+\delta  u^n$, where
$u^*$ is a guess and $\delta u^n$ is the error and we drop the $i$ for
clarity.   Then, we  expressed the  non-linear part  using a  Taylor's
expansion
\begin{equation}
  f^{n+1}=f(u^{n+1})=f(u^*+\delta
  u^n)=f(u^*)+J^h_{f}(u^*)\delta u^n\nonumber
\end{equation}
where  $J^u_{f}(u^*)$ is  the  jacobian matrix  for  the function  $f$
evaluated  in $h^*$.   Injecting the  expansion into  (\ref{C4-Num-1})
finally gives a  system of M linear equations for  the correction term
$\delta_h^n$ which can be expressed as
\begin{equation}
  (I-\Delta tJ^u_{f}(u^*))\delta u^n=u^n-u^*+\Delta t f(u^*)
\end{equation}
where $I$ is the identity matrix. Therefore, each iteration solves for
$\delta u^n$ and we use $u_n+\delta u^n$  as a new guess $u^*$ in each
iteration. This  is repeated  until $\delta u^n$  becomes sufficiently
small.  Finally, since the equations are coupled, we use a fixed-point
iteration method  to converge  toward the  solution $(h,\xi)$  at each
time step.   Therefore, the  algorithm is the  following at  each time
step
\begin{itemize}
\item Start with a guess for the values of all variables.
\item Solve  the thickness equation (\ref{C4-HF})  for $h^{n+1}$ using
  Newton-Rhapsod method.
\item  Solve the  heat  equation (\ref{C4-TF})  for $\xi^{n+1}$  using
  $h^{n+1}$ as a new guess for $h^*$ and Newton-Rhapsod method.
\item Repeat  step one until  further iterations cease to  produce any
  significant changes in the values of both $h^{n+1}$ and $\xi^{n+1}$.
\end{itemize}
The computational scheme is summarized in the following.

\section{Thickness equation}

The thickness equation (\ref{C4-HF}) is written as
\begin{eqnarray}
  \frac{\partial h}{\partial t}-f(h,\xi)&=&0
\end{eqnarray}
with
\begin{eqnarray}
  f& =& \frac{1}{r}
        \frac{\partial}{\partial      r}
        \left(      r  \phi\left(     \frac{\partial      }{\partial
        r}\left(h+P\right)\right)\right)+w_i\\
  \phi &=& 12I_1(h)
\end{eqnarray}
and where $P$ is the dimensionless bending pressure $P = \nabla^4h$.

\vspace{.5cm} \textbf{Spatial discretization of f} \vspace{.5cm}

The  spatial discretization  is  obtained using  a central  difference
scheme  over  a  sub-grid  shifted  by $0.5\Delta  r$  from  the  main
grid. Therefore, we have
\begin{eqnarray}
  f_i&=&\frac{1}{r_i \Delta_r}\left(r_{i+1/2}\phi_{i+1/2}\left.\left(\frac{\partial h}{\partial r}+\frac{\partial P}{\partial r}\right)\right|_{i+1/2}-r_{i-1/2}\phi_{i-1/2}\left.\left(\frac{\partial h}{\partial r}+\frac{\partial P}{\partial r}\right)\right|_{i-1/2}\right)\nonumber\\
     &=&A_i\phi_{i+1/2}\left(h_{i+1}-h_i\right)-B_i\phi_{i-1/2}\left(h_{i}-h_{i-1}\right)\nonumber\\
     &+&A_i\phi_{i+1/2}\left(P_{i+1}-P_i\right)-B_i\phi_{i-1/2}\left(P_{i}-P_{i-1}\right)\nonumber\\
     &+&w_i\label{C4-Num-3}
\end{eqnarray}
where                $A_i=r_{i+1/2}/(r_i\Delta_r^2)$               and
$B_i=r_{i-1/2}/(r_i\Delta_r^2)$.   The bending  pressure  term $P$  is
very stiff and  needs a careful treatment.  In  particular, the fourth
order derivative requires a fourth order central difference scheme and
therefore, $P_i$ is  expressed over a seven point stencil  on the main
grid such that
\begin{equation}
  P_{i}=   \alpha_{i}h_{i-3}  +   \beta_{i}h_{i-2}+\gamma_{i}  h_{i-1}
  +\lambda_{i}h_{i}+\kappa_{i}h_{i+1}+\delta_ih_{i+2}+\epsilon_ih_{i+3}
  \label{C4-Num-4}
\end{equation}
with
\begin{eqnarray}
  &\alpha_{i}&=\frac{1}{24\Delta r^{4}}\left(-4+3p_3\Delta_r \right)\nonumber \\
  &\beta_{i}&=\frac{1}{24\Delta r^{4}}\left(48-24p_3\Delta_r-2p_2\Delta_r^2+2p_1\Delta_r^3\right) \nonumber\\
  &\gamma_{i}&=\frac{1}{24\Delta r^{4}}\left(-156+39p_3\Delta_r+32p_2\Delta_r^2-16p_1\Delta_r^3\right)\nonumber\\
  &\lambda_{i}&=\frac{1}{24\Delta r^{4}}\left(224-60p_2\Delta r^{2}\right) \nonumber\\
  &\kappa_{i}&=\frac{1}{24\Delta r^{4}}\left( -156-39p_3\Delta_r+32p_2\Delta_r^2+16p_1\Delta_r^3\right)\nonumber\\
  &\delta_{i}&=\frac{1}{24\Delta r^{4}}\left( 48+24p_3\Delta_r-2p_2\Delta_r^2-2p_1\Delta_r^3\right) \nonumber\\
  &\epsilon_{i}&=\frac{1}{24\Delta r^{4}}\left(-4-3p_3\Delta_r \right)\nonumber
\end{eqnarray}
and where $p_1=1/r_i^3$, $p_2=1/r_i^2$ and $p_3 = 2/r_i$. Finally, the
term $\phi_{i-1/2}$  and $\phi_{i-1/2}$, which depend  on the variable
$\Theta_b$, $\delta$ as well as  different power of $h$, are evaluated
in $i-1/2$ and  $i+1/2$ respectively. Different choices  for the value
of the variable at the mid-cell grid point do not show any significant
difference  and a  simple  average  is taken  such  that the  variable
$u_{i+1/2}$ is taken as $0.5(u_i+u_{i+1})$.

\vspace{.5cm}    \textbf{Expression   of    the   jacobian    $J_f^h$}
\vspace{.5cm}

The discretized  function $f_i$ can be  break down in three  part, the
gravitational part $f_i^{g}$  which is expressed in term  of the value
of $h$ on three  grid points $\left\{{i-1,i,i+1}\right\}$, the bending
part $f_i^{b}$ which is expressed in term  of the value of $h$ on nine
grid points  $\left\{{i-4,i-3,...,i+3,i+4}\right\}$ and  the injection
term which depends only on the grid point $i$ such that
\begin{equation}
  f_i = f_i^g+f_i^b+w_i
\end{equation}
Therefore, the jacobian is  nona-diagonal and its coefficient $J_{il}$
are
\begin{equation}
  J_{il}=
  \begin{cases}
    \frac{\partial f^{b}_i}{\partial h_{l}} &
    l = \left\{{i-4,i-3,i-2,i+2,i+3,i+4}\right\}\\
    \frac{\partial       f^{g}_i}{\partial       h_{l}}+\frac{\partial
      f^{b}_i}{\partial h_{l}} & l =
    \left\{{i-1,i,i+1}\right\}\\
    0 & \text{otherwise}
  \end{cases}
  \label{C4-C2-eq12}
\end{equation}
The different  terms can be  easily derived from  (\ref{C4-Num-3}) and
(\ref{C4-Num-4}) with just slight  adjustment coming from the boundary
conditions.

\vspace{.5cm} \textbf{Boundary condition} \vspace{.5cm}

We begin with  $h_i=h_f$ for $i=1,..,M$.  Since the  flow is symmetric
in $r=0$, we require that
\begin{equation}
  \left.\frac{\partial h}{\partial r}\right|_{r=0} =\left.\frac{\partial P}{\partial r}\right|_{r=0} =0
\end{equation}
and therefore for $i=1$, we have
\begin{eqnarray}
  f_i     &=&A_1\phi_{i+1/2}\left(h_{i+1}-h_i\right)\nonumber\\
          &+&A_i\phi_{i+1/2}\left(P_{i+1}-P_i\right)\nonumber\\
          &+&w_i\label{C4-Num-5}
\end{eqnarray}
The expression  of the  bending pressure, evaluated  over a  $7$ point
stencils, is problematic close to the boundary and reflection formulae
will  be  used  in  order   to  accommodate  the  boundary  conditions
\citet{Patankar:1980vu}.   In   particular,  we  have  $h_0   =  h_1$,
$h_{-1}=h_2$ and  $h_{-2}=h_3$.  Similarly, boundary condition  at the
end of the mesh is accounted by using a grid much larger than the flow
itself and requiring
\begin{equation}
  \left.\frac{\partial h}{\partial r}\right|_{r=r_M} =\left.\frac{\partial P}{\partial r}\right|_{r=r_M} =0
\end{equation}
which gives for $i=M$
\begin{eqnarray}
  f_i     &=&B_i\phi_{i-1/2}\left(h_{i}-h_{i-1}\right)\nonumber\\
          &+&B_i\phi_{i-1/2}\left(P_{i}-P_{i-1}\right)\nonumber\\
          &+&w_i\label{C4-Num-5}
\end{eqnarray}
with $h_{i>=M}=h_f$.


\vspace{.5cm} \textbf{Newton-Rhapsod method} \vspace{.5cm}

The Newton-Rhapsod method reads
\begin{equation}
  (I-\Delta tJ^h_{f}(h_k^*))\delta h_k^n=h^n-h_k^*+\Delta t f(h_k^*)
\end{equation}
where the  $k$ refers  to the $k$  iterations, $I$ is  a $M  \times M$
diagonal  matrix and  $J_f^h(h^*)$  is a  $M  \times M$  nona-diagonal
matrix.   This  system of  linear  equations  can  be solved  using  a
nona-diagonal algorithm. At the first  iteration, we use $h^*_1 = h^n$
as     a    first     guess    and     then    we     iterate    using
$h^*_k  = h^n+\delta  h_{k-1}^n$ as  a new  guess for  each iterations
until $\delta h^n_{k}$ becomes  sufficiently small.  In particular, we
require that
\begin{equation}
  \delta h^n_k/h^*_{k}<\epsilon
\end{equation}
with $\epsilon = 10^{-4}$.

\section{Heat equation}

The heat equation (\ref{C4-TF}) is written as
\begin{eqnarray}
  \frac{\partial \xi}{\partial t}-g(h,\xi)&=&0
\end{eqnarray}
with
\begin{eqnarray}
  g& =& \frac{1}{r}\frac{\partial}{\partial                          r}
        \left( r\Gamma\xi\right) +\frac{1}{r}\frac{\partial}{\partial                          r}
        \left(r\Sigma\right)+2Pe^{-1}St_m\frac{\left(\Theta_b-\Theta_s\right)}{\delta}\\
  \overline{\theta}&=&\frac{1}{3}\left(2\Theta_b+\Theta_s\right)\\
  \Gamma&=&-\frac{12}{\delta}
            \frac{\partial
            P}{\partial
            r}\left(\delta
            I_0(\delta)-I_1(\delta)\right)\\
  \Sigma &=& \frac{12}{\delta} \frac{\partial P}{\partial r}\left(I_0(\delta)\left(G(\delta)-\delta\overline{\theta}\right)+\overline{\theta}I_1(\delta)-I_2(\delta)\right).
\end{eqnarray}

\vspace{.5cm} \textbf{Spatial discretization of g} \vspace{.5cm}

As for the thickness equation,  the spatial discretization is obtained
using  a  central  difference  scheme   over  a  sub-grid  shifted  by
$0.5\Delta r$ from the main grid. Therefore, we have
\begin{eqnarray}
  g_i &=& \left(C_i\Gamma_{i+1/2}\xi_{i+1/2}-D_i\Gamma_{i-1/2}\xi_{i-1/2}\right)\\
      &+&\left(C_i\Sigma_{i+1/2}-D_i\Sigma_{i-1/2}\right)\\
      &+&2Pe^{-1}St_m\frac{\Theta_{b,i}-\Theta_{s,i}}{\delta_i}
\end{eqnarray}
with         $C_i         =r_{i+1/2}/(r_i\Delta        r)$         and
$D_i =r_{i-1/2}/(r_i\Delta r)$.   We use the average  between the grid
point $i$ and $i-1$ (resp. $i+1$) to evaluate the quantity in $\Gamma$
and  $\Sigma$ at  $i-1/2$ (resp.   $i+1/2$).   In addition,  we use  a
classical upwind  scheme to handle $\xi$  at the mid grid  point which
requires
\begin{eqnarray}
  \xi_{i+1/2} &=& \xi_i\\
  \xi_{i-1/2} &=& \xi_{i-1}
\end{eqnarray}

\vspace{.5cm}  \textbf{Expression   of  the   Jacobian  $J_{g}^{\xi}$}
\vspace{.5cm}

The expression  of the Jacobian  is much straightforward in  that case
and its coefficient $J_{il}$ are
\begin{equation}
  J_{il}=
  \begin{cases}
    -D_i\Gamma_{i-1/2}&
    l = i-1\\
    C_i\Gamma_{i+1/2} & l = i \\
    0 & \text{otherwise}
  \end{cases}
  \label{C4-C2-eq12}
\end{equation}
with only slight adjustment coming from the boundary conditions.

\vspace{.5cm} \textbf{Boundary conditions} \vspace{.5cm}

We  consider $\Theta_b  =1$  and $\delta  = 10^{-4}$  in  the film  at
$t=0$. In this way, we ensure  that the average temperature across the
film at $t=0$ is close to $1$. By construction, $D_1=0$ and therefore,
for $i=1$ we have
\begin{eqnarray}
  g_i &=& C_i\Gamma_{i+1/2}\xi_{i}+ C_i\Sigma_{i+1/2} +2Pe^{-1}St_m\frac{\Theta_{b,i}-\Theta_{s,i}}{\delta_i}
\end{eqnarray}
For   $i=M$,   we    consider   that   $\Gamma_{i+1/2}=\Gamma_i$   and
$\Sigma_{i=1/2}=\Sigma_i$.   However,  the  choice  for  the  boundary
condition at the border of the grid $i=M$ is not important as we solve
the problem over a grid much larger than the flow itself.

\vspace{.5cm} \textbf{Newton-Rhapsod method} \vspace{.5cm}

The Newton-Rhapsod method reads
\begin{equation}
  (I-\Delta tJ^{\xi}_{g}(\xi_k^*))\delta \xi_k^n=\xi^n-\xi_k^*+\Delta t f(\xi_k^*)
\end{equation}
where the  $k$ refers  to the $k$  iterations, $I$ is  a $M  \times M$
diagonal  matrix and  $J_f^h(\xi^*)$ is  a $M  \times M$  tri-diagonal
matrix.   This  system of  linear  equations  can  be solved  using  a
tri-diagonal algorithm.  As  for the thickness equation,  at the first
iteration,  we use  $\xi^*_1 =  \xi^n$ as  a first  guess and  then we
iterate using $\xi^*_k = \xi^n+\delta  \xi_{k-1}^n$ as a new guess for
each iterations  until $\delta \xi^n_{k}$ becomes  sufficiently small.
In particular, we require that
\begin{equation}
  \delta \xi^n_k/\xi^*_{k}<\epsilon
\end{equation}
with  $\epsilon  = 10^{-4}$.   In  addition,  at each  iteration,  the
quantity $\Theta^*_{s,k}$, $\Theta^*_{b,k}$ and $\delta^*_k$, that are
needed to evaluate  $\Gamma$ and $\Sigma$, are derived  from the value
of $\xi^*_{k}$ using (\ref{C4-TS}), (\ref{C4-TB}) and (\ref{C4-DELTA})

\section{Integral expressions}
\label{sec:integral-expressions}

The model  developed in  Section \ref{C4-sec:theory-1} depends  on the
integrals
\begin{eqnarray}
  I_0(z)&=&\int_0^z\frac{1}{\eta(y)}\left(y-\frac{h}{2}\right)
            dy \\
  I_1(z) &=& \int_0^z\frac{1}{\eta(y)}\left(y-\frac{h}{2}\right)y dy\\
  I_2(z)&=&\int_0^y                         \frac{1}{\eta(y)}
            \left(y-\frac{h}{2}\right)G(y)dy
\end{eqnarray}
where $G(z)$  is a  primitive of $\theta(z)$  where $z<\delta$  and is
given by
\begin{equation}
  G(z) = \frac{z \left(3 \delta ^2 \Theta_s+3 \delta z (\Theta_b-\Theta_s)+z^2 (\Theta_s-\Theta_b)\right)}{3 \delta ^2}.
\end{equation}
In  particular, the  model requires  the expression  of $I_0(\delta)$,
$I_1(\delta)$, $I_1(h)$ and $I_2(\delta)$.

\vspace{.5cm}   \textbf{Rheology   1:   $\eta(\theta)=\eta_1(\theta)$}
\vspace{.5cm}

In that case, the four integrals can be easily derived and read
\begin{eqnarray}
  I_0(\delta)&=&\frac{\delta}{12} \left(6 \delta \nu + (1-\nu) \left(- \alpha_1 \delta + 2 \alpha_1 h + 6 \Theta_{b} \delta - 6 \Theta_{b} h\right) - 6 h \nu\right)\nonumber\\
  I_1(\delta)&=&\frac{\delta^{2}}{120} \left(40 \delta \nu + (1-\nu) \left(- 4 \alpha_1 \delta + 5 \alpha_1 h + 40 \Theta_{b} \delta - 30 \Theta_{b} h\right) - 30 h \nu\right)\nonumber\\
  I_1(h)&=&\frac{1}{60} \left((1-\nu) \left(- 4 \alpha_1 \delta^{3} + 10 \alpha_1 \delta^{2} h - 10 \alpha_1 \delta h^{2} + 5 \Theta_{b} h^{3}\right) + 5 h^{3} \nu\right)\nonumber\\
  I_2(\delta)&=&- \frac{\delta^{2}}{2520} \left(378  \alpha_1 \delta \nu -
                 315  \alpha_1 h  \nu -  840 \Theta_{b}  \delta \nu  + 630
                 \Theta_{b} h \nu \right)\nonumber\\
             &&-\frac{\delta^{2}}{2520}(1-\nu)  \left(- 50  \alpha_1^{2} \delta  + 70
                \alpha_1^{2}  h  +  462  \alpha_1   \Theta_{b}  \delta  -  420  \alpha_1
                \Theta_{b}  h -  840  \Theta_{b}^{2} \delta  + 630  \Theta_{b}^{2}
                h\right)\nonumber
\end{eqnarray}
where $\alpha_1=\Theta_b-\Theta_s$ has been introduced for clarity.

\vspace{.5cm}   \textbf{Rheology   2:   $\eta(\theta)=\eta_2(\theta)$}
\vspace{.5cm}

For cases where $\nu<1$, we have
\begin{eqnarray}
  I_0(\delta)&=&-\frac{\delta  \nu ^{1-\Theta_b} \left(\sqrt{\pi } \sqrt{\alpha_1} (2 \delta -h) \sqrt{-\alpha_2}
                 \text{erf}\left(\sqrt{\alpha_1} \sqrt{-\alpha_2}\right)+2 \delta  \left(\nu ^{\alpha_1}-1\right)\right)}{4 \alpha_1 \alpha_2}\nonumber\\
  I_1(\delta)&=&\frac{\delta ^2 \nu ^{1-\Theta_b} \left(\sqrt{\pi } \erf\left(\sqrt{\alpha_1} \sqrt{-\alpha_2}\right) (\alpha_1 (h-2
                 \delta ) \alpha_2+\delta )\right)}{4           \alpha_1^{3/2}          (-\alpha_2)^{3/2}}\nonumber\\
             &&+\frac{\delta ^2 \nu ^{1-\Theta_b} \left(\sqrt{\alpha_1} \sqrt{-\alpha_2} \left(2 \delta  \left(\nu ^{\alpha_1}-2\right)-h \nu
                ^{\alpha_1}+h\right)\right)}{4           \alpha_1^{3/2}          (-\alpha_2)^{3/2}}\nonumber\\
  I_1(h)&=&\frac{\nu ^{1-\Theta_b} \left(\sqrt{\alpha_1} \sqrt{-\alpha_2} \left(12 \delta ^2 \left(\delta  \left(\nu
            ^{\alpha_1}-2\right)-h \nu ^{\alpha_1}+h\right)+\alpha_1  (2 \delta -h)^3
            \log    (\nu    )\right)\rihgt)}{12   \alpha_1^{3/2}    (-\alpha_2)^{3/2}}\nonumber\\
             &&-\frac{\nu ^{1-\Theta_b} \left(3 \sqrt{\pi } \delta 
                \text{erf}\left(\sqrt{\alpha_1} \sqrt{-\alpha_2}\right) \left(\alpha_1 (h-2 \delta )^2 \alpha_2-2 \delta
                ^2\right)\right)}{12 \alpha_1^{3/2} (-\alpha_2)^{3/2}}\nonumber
\end{eqnarray}
\begin{eqnarray}
  I_2(\delta)&=&\frac{\delta ^2 \nu ^{1-\Theta_b} \left(\sqrt{\pi } \text{erf}\left(\sqrt{\alpha_1} \sqrt{-\alpha_2}\right) \left(-2
                 \alpha_1 (2  \delta -h)  (\alpha_1-3 \Theta_b)  \alpha_2^2-6 \delta
                 \Theta_b  \alpha_2-3   \delta  \right)\right)}{24
                 \alpha_1^{3/2} (-\alpha_2)^{5/2}}\nonumber\\
             &&+\frac{\delta ^2 \nu ^{1-\Theta_b} \left(2
                \sqrt{\alpha_1} \nu ^{\alpha_1} \sqrt{-\alpha_2} \left(\nu ^{-\alpha_1} (\alpha_2 (-2 \delta  (\alpha_1-6
                \Theta_b)-3 h \Theta_b)+2 \delta -h)\right)\right)}{24
                \alpha_1^{3/2} (-\alpha_2)^{5/2}}\nonumber\\
             &&+\frac{\delta ^2 \nu ^{1-\Theta_b} \left(2
                \sqrt{\alpha_1} \nu ^{\alpha_1} \sqrt{-\alpha_2} \left(2 \delta  \alpha_1 \alpha_2-6 \delta  \Theta_b \alpha_2+\delta
                -\alpha_1 h \alpha_2+3 h \Theta_b \alpha_2+h\right)\right)}{24 \alpha_1^{3/2} (-\alpha_2)^{5/2}}\nonumber
\end{eqnarray}
where    in   addition    to    $\alpha_1$,    we   also    introduced
$\alpha_2=\log(\nu)$  for clarity.   In  the case  where $\nu=1$,  the
expression above simplify and read
\begin{eqnarray}
  I_0(\delta)&=&\frac{1}{2} \delta  (\delta -h)\nonumber\\
  I_1(\delta)&=&\frac{1}{12} \delta ^2 (4 \delta -3 h)\nonumber\\
  I_1(h)&=&\frac{h^3}{12}\nonumber\\
  I_2(\delta)&=&-\frac{1}{120} \delta ^2 (18 \delta  \alpha_1-40 \delta  \Theta_b-15 \alpha_1 h+30 h \Theta_b)\nonumber
\end{eqnarray}

%%% Local Variables:
%%% mode: latex
%%% TeX-master: "../main"
%%% End:
