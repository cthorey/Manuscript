\chapter{Gravity field calculation}
\label{chap:A1}

\section{Gravity field}

%\subsection*{Solution to the Laplace equation}

The gravitational potential $U$ is derived from Newton's law of gravitation
%
\begin{equation} \label{A:pot} U(\mathbf{r}) = \int_V
\frac{G\rho(\mathbf{r'})}{|\mathbf{r}-\mathbf{r'}|} dV', \end{equation} 
%
where $\mathbf{r}$ is the vector position, $G$ the gravitational constant, $\rho$ the density
distribution and $V$ the volume. Exterior to a given mass distribution, the gravitational potential
satisfies the Laplace equation
%
\begin{equation} \nabla^2 U(\mathbf{r}) = 0. \end{equation}
%
In spherical coordinates, the solution of this equation can be expressed in spherical
harmonics as
%
\begin{equation} U(\mathbf{r}) = \frac{GM}{r} \sum_{l=0}^{\infty} \sum_{m=-l}^{l}
\left(\frac{R_0}{r}\right)^{l} C_{lm} Y_{lm}(\theta,\phi), \label{potential.sh} \end{equation}
%
where $M$ is the mass of the Moon, $R_0$ the reference radius, $Y_{lm}$ the spherical harmonic
functions of degree $l$ and order $m$, $C_{lm}$ the corresponding coefficients, and $\theta$ and
$\phi$ are the colatitude and longitude, respectively. The spherical harmonics are normalized as
follows
%
\begin{equation} \int_\Omega Y_{lm}(\Omega)Y_{l'm'}(\Omega)d\Omega = 4\pi
\delta_{ll'}\delta_{mm'}, \end{equation}
%
where $d\Omega = \sin\theta d\theta d\phi$. For $r > R_0$, the two following identities hold
%
\begin{align} \label{A:id} &\frac{1}{|\mathbf{r}-\mathbf{r'}|} = \frac{1}{r}
\sum_{l=0}^{\infty} \left( \frac{r'}{r} \right)^l P_l(cos \gamma), \\ &P_l(cos\gamma) =
\frac{1}{2l+1} \sum_{m=-l}^l Y_{lm}(\theta,\phi)Y_{lm}(\theta',\phi'), \end{align} 
%
where $P_l$ is the Legendre polynomial of order $l$ and $\gamma$ the angle between $\mathbf{r}$ and
$\mathbf{r'}$. Inserting Equation (\ref{A:id}) in Equation (\ref{A:pot}) leads to the definition
%
\begin{equation} \label{eq:clm} C_{lm} = \frac{1}{M(2l+1)} \int_{V}
\left(\frac{r'}{R_0}\right)^{l} \rho(\mathbf{r'}) Y_{lm}(\Omega') dV'. \end{equation}

In order to calculate the coefficients $C_{lm}$ numerically, we assume the Moon to be constituted of
$N$ thin shells of constant thickness with prescribed lateral density variations. For a given shell,
$C_{lm}(r)$ is given by 
%
\begin{equation} \label{eq:clm} C_{lm}(r) = \frac{4\pi r^2}{M(2l+1)} 
\left(\frac{r}{R_0}\right)^{l} \rho_{lm}(r) \Delta R, \end{equation} 
%
where $\Delta R$ is the shell thickness and $\rho_{lm}(r)$ the spherical harmonics expansion of
$\rho$ on that shell which can be computed as
%
\begin{equation} \label{A:rho} \rho_{lm}(r) = \frac{1}{4\pi} \int \rho(\theta,
\phi, r) Y_{lm}(\theta,\phi) d\Omega. \end{equation} 
%
Finally,
\begin{equation} \label{A:rho} C_{lm} = \sum_{n=1}^N C_{lm}(r_n). \end{equation}
%
The gravitational field is then calculated by taking the first derivative of (\ref{potential.sh})
with respect to $r$
%
\begin{equation} g(r) = \frac{GM}{r^2} \sum_{l=0}^{\infty} \sum_{m=-l}^{l} \left(
\frac{R_0}{r}\right)^l (l+1) C_{lm} Y_{lm}(\theta, \phi). \end{equation} 
%
where we use the convention that g is positiive when directed downward. This allows us to calculate
the effect on gravity from changes in density within the mantle, either due to thermal expansion or
compositional changes. However, induced surface topography plays a role as well that needs to be
estimated. Two approaches can be considered, as described in Appendix B: either surface topography
is maintained dynamically or statically. Regardless how we choose to compute this topography, its
influence on the potential can be calculated according to Equation (A8):
%
\begin{equation} C_{lm}^{surf} = \frac{4\pi}{(2l+1)M} R_0^2 (\rho h)_{lm}.
\end{equation} 
%
where the topography $h_{lm}$ is the spherical harmonics expansion of the topography $h$, referenced
to radius $R_0$.

The Laplacian operator being linear, the gravitational influence of these mass anomalies can be
considered independently and directly added to the coefficients of Eq. (A10)
%
 \begin{equation} C_{lm}^{tot} = C_{lm} + C_{lm}^{surf}. \end{equation}



%\subsection*{Static approach}

\section{Calculation of static and dynamic topography}

Thermal expansion leads to a volume increase of the material. Because our thermochemical models are
incompressible and have fixed boundaries, this effect is only taken into account within the
Boussinesq approximation: the density decreases locally, but the volume stays constant. However, for
gravity calculations, volume changes need to be considered. On the one hand, assuming a strengthless
lithosphere and the direct transmission of stresses vertically, this increase in volume directly
translates to a surface uplift. On the other hand, if one considers an infinite elastic lithosphere,
the induced topography is zero. Our calculations should therefore be seen as an upper bound on the
contribution. There are two ways to compute topography. The static approach assumes that once topography
is emplaced, it stays in place whereas the dynamic topography is the topography that can be sustained
by radial stresses at the surface.

\subsection*{Static topography}

Our numerical grid is regular, therefore the sum of the volume expansion of each cell $\delta V_N$
along one vertical profile is equal to the volume increase at the surface $V_{uplift}$
%
\begin{equation} V_{uplift} = \sum_N \delta V_N, \end{equation} 
%
where $N$ is the number of shells of the model. Thermal expansion $\alpha$ dictates that
%
\begin{equation} \delta V_N = \alpha V_N \delta T_N, \end{equation}
%
where $\delta T_N$ is the local increase in temperature with respect to a constant reference.
Writing $S_i$ the surface area of the cell at shell $i$, we can compute the surface uplift as
%
\begin{equation} h=\frac{1}{S_0} \sum_N \alpha V_N \delta T_N, \end{equation}
%
where $V_N = S_N \Delta R$, $\Delta R$ being the thickness of the shell.


\subsection*{Dynamic topography}

The other approach assumes that the current topography is sustained dynamically. Our models
have fixed boundaries, but the radial stresses at these boundaries can be used \emph{a
posteriori} to compute the induced dynamic topography that would be present with a free surface
using
%
\begin{equation} h = \frac{\tau_{rr}}{\Delta \rho g_0}, \end{equation}
%
where $\Delta \rho$ is the density jump across the interface and $g_0$ the gravitational
acceleration at this same boundary \citep{Peltier:1989ft}. The radial stress is defined by
%
\begin{equation} \tau_{rr} = P - 2\eta \frac{dv_r}{dr}, \end{equation}
%
with $P$ being the dynamic pressure, $\eta$ the viscosity and $v_r$ the radial viscosity at the
interface \citep{Landau:1959vb}. The radial stresses are the direct result of the mantle dynamics
and therefore includes the effect of thermal expansion. We found that in our case, the assumption of
either static of dynamic topgraphy give similar results. Therefore the static approach is used here
for simplicity.


