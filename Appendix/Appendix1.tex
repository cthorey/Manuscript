\chapter{Numerical  schemes}
\label{chap:A1}

\section{  Numerical  scheme  for  a  cooling  elastic-plated  gravity
  current}
\label{A1-sec:-cool-elast-plat}

In this  section, we present  the numerical  scheme used to  solve the
coupled  nonlinear partial  differential  equations (\ref{C4-HF})  and
(\ref{C4-TF}).    The  governing   equations   presented  in   Chapter
\ref{C3-JFM}     are     just      a     particular     case     where
$\Omega \rightarrow \infty$ and can  be solved using $\Omega=10^5$ for
instance.

\subsection{General procedure}
\label{C3-sec:general-procedure}

The coupled nonlinear partial differential equations (\ref{C4-HF}) and
(\ref{C4-TF}) are solved on a grid of size $M$ defined by the relation
$r_i =  (i-0.5)\Delta r$ for  $i=1,..,M$. The  grid is shifted  at the
center to avoid  problem arising from the  axisymmetrical geometry. We
index the grid point by the indice $i$ and denote the solution on this
grid  $h_i$ and  $\xi_i$ and  the secondary  variables $\Theta_{b,i}$,
$\Theta_{s,i}$ and $\delta_i$. Both equations  can be expressed on the
convenient form
\begin{equation}
  \frac{\partial \psi}{\partial t} - f = 0,
\end{equation}
where $\psi$  is the function we  want to integrate and  $f$ a non-linear
function  that depends  on $\psi$.   We  solve these  equations by  first
discretizing all the spatial  derivatives using Finite Difference. The
accuracy of the  scheme is determined by the  higher order derivatives
since  their numerical  approximation requires  the largest  number of
sample points.  We  then get two systems of  $M$ ordinary differential
equations with the form
\begin{equation}
  \frac{\partial \psi_i}{\partial t} - f_i = 0 \hspace{1cm} i = 1,...,M.
\end{equation}
The time derivatives are first  order and, since explicit schemes tend
to be  very sensitive and unstable,  we use a fully  implicit backward
Euler scheme to get
\begin{equation}
  \frac{\psi_i^{n+1}-\psi_i^n}{\Delta t} - f_i(\psi_i^{n+1}) = 0 \hspace{1cm} i
  = 1,...,M.
  \label{C4-Num-1}
\end{equation}
Since  $f_i(\psi_i^{n+1})$ is  not a  linear function,  the system  above
cannot be re-arranged to solve $\psi_i^{n+1}$ in term of $\psi_i^{n}$ and an
iterative method  has to  be employed  instead. Fixed  point iteration
method have shown  poor results in converging toward  the solution and
we finally apply  second order Newton's method to  obtain the solution
at each time step.  In particular, we first linearize $\psi^{n+1}$ around
a guess  of the solution  by assuming $\psi^{n+1}=\psi^*+\delta  \psi^n$, where
$\psi^*$ is a  guess, $\delta \psi^n$ is  the error and we drop  the $i$ for
clarity.   Then, we  expressed the  non-linear part  using a  Taylor's
expansion
\begin{equation}
  f^{n+1}=f(\psi^{n+1})=f(\psi^*+\delta
  \psi^n)=f(\psi^*)+J^\psi_{f}(\psi^*)\delta \psi^n,
  \label{C4-Fn1}
\end{equation}
where  $J^\psi_{f}(\psi^*)$ is  the  Jacobian matrix  for  the function  $f$
evaluated  in $\psi^*$.   Injecting (\ref{C4-Fn1})  into (\ref{C4-Num-1})
finally gives a  system of M linear equations for  the correction term
$\delta \psi^n$ which can be expressed as
\begin{equation}
  (I-\Delta tJ^\psi_{f}(\psi^*))\delta \psi^n=\psi^n-\psi^*+\Delta t f(\psi^*),
\end{equation}
where $I$ is the identity matrix. Therefore, each iteration solves for
$\delta \psi^n$ and we use $\psi_n+\delta \psi^n$  as a new guess $\psi^*$ in each
iteration. This  is repeated  until $\delta \psi^n$  becomes sufficiently
small.  Finally,  since the equations (\ref{C4-HF})  and (\ref{C4-TF})
are coupled, we use a  fixed-point iteration method to converge toward
the solution $(h,\xi)$  at each time step.  In the  end, the algorithm
is the following at each time step
\begin{itemize}
\item Start with a guess for the values of all variables.
\item Solve  the thickness equation (\ref{C4-HF})  for $h^{n+1}$ using
  Newton-Rhapsod method.
\item  Solve the  heat  equation (\ref{C4-TF})  for $\xi^{n+1}$  using
  $h^{n+1}$ as a new guess for $h^*$ and Newton-Rhapsod method.
\item Repeat  step one until  further iterations cease to  produce any
  significant changes in the values of both $h^{n+1}$ and $\xi^{n+1}$.
\end{itemize}
Unless otherwise  specified, we use  $D_r= 0.01$ and  $D_t=10^{-6}$ in
the  simulations.  The  time  step  could appear  very  small but  the
stiffness of  the equations required  such low value at  the beginning
when  $\nu  < 1$.   The  computational  scheme  is summarized  in  the
following.

\subsection{Thickness equation}

The thickness equation (\ref{C4-HF}) is written as
\begin{eqnarray}
  \frac{\partial h}{\partial t}-f(h,\xi)&=&0,
\end{eqnarray}
with
\begin{eqnarray}
  f& =& \frac{1}{r}
        \frac{\partial}{\partial      r}
        \left(      r  \phi\left(     \frac{\partial      }{\partial
        r}\left(h+P_e\right)\right)\right)+w_i,\\
  \phi &=& 12I_1(h),
\end{eqnarray}
and where $P$ is the dimensionless bending pressure $P = \nabla^4h$.

\vspace{.5cm} \textbf{Spatial discretization of f} \vspace{.5cm}

The  spatial discretization  is  obtained using  a central  difference
scheme  over  a  sub-grid  shifted  by $0.5\Delta  r$  from  the  main
grid. Therefore, we have
\begin{eqnarray}
  f_i&=&\frac{1}{r_i \Delta_r}\left(r_{i+1/2}\phi_{i+1/2}\left.\left(\frac{\partial h}{\partial r}+\frac{\partial P}{\partial r}\right)\right|_{i+1/2}-r_{i-1/2}\phi_{i-1/2}\left.\left(\frac{\partial h}{\partial r}+\frac{\partial P}{\partial r}\right)\right|_{i-1/2}\right)\nonumber\\
     &=&A_i\phi_{i+1/2}\left(h_{i+1}-h_i\right)-B_i\phi_{i-1/2}\left(h_{i}-h_{i-1}\right)\nonumber\\
     &+&A_i\phi_{i+1/2}\left(P_{i+1}-P_i\right)-B_i\phi_{i-1/2}\left(P_{i}-P_{i-1}\right)\nonumber\\
     &+&w_i,\label{C4-Num-3}
\end{eqnarray}
where                $A_i=r_{i+1/2}/(r_i\Delta_r^2)$               and
$B_i=r_{i-1/2}/(r_i\Delta_r^2)$.   The bending  pressure  term $P$  is
very stiff and  needs a careful treatment.  In  particular, the fourth
order derivative requires a fourth order central difference scheme and
therefore, $P_i$ is  expressed over a seven point stencil  on the main
grid such that
\begin{equation}
  P_{i}=   \alpha_{i}h_{i-3}  +   \beta_{i}h_{i-2}+\gamma_{i}  h_{i-1}
  +\lambda_{i}h_{i}+\kappa_{i}h_{i+1}+\delta_ih_{i+2}+\epsilon_ih_{i+3},
  \label{C4-Num-4}
\end{equation}
with
\begin{eqnarray}
  &\alpha_{i}&=\frac{1}{24\Delta r^{4}}\left(-4+3p_3\Delta_r \right)\nonumber ,\\
  &\beta_{i}&=\frac{1}{24\Delta r^{4}}\left(48-24p_3\Delta_r-2p_2\Delta_r^2+2p_1\Delta_r^3\right) \nonumber,\\
  &\gamma_{i}&=\frac{1}{24\Delta r^{4}}\left(-156+39p_3\Delta_r+32p_2\Delta_r^2-16p_1\Delta_r^3\right)\nonumber,\\
  &\lambda_{i}&=\frac{1}{24\Delta r^{4}}\left(224-60p_2\Delta r^{2}\right) \nonumber,\\
  &\kappa_{i}&=\frac{1}{24\Delta r^{4}}\left( -156-39p_3\Delta_r+32p_2\Delta_r^2+16p_1\Delta_r^3\right)\nonumber,\\
  &\delta_{i}&=\frac{1}{24\Delta r^{4}}\left( 48+24p_3\Delta_r-2p_2\Delta_r^2-2p_1\Delta_r^3\right) \nonumber,\\
  &\epsilon_{i}&=\frac{1}{24\Delta r^{4}}\left(-4-3p_3\Delta_r \right),\nonumber
\end{eqnarray}
and where $p_1=1/r_i^3$, $p_2=1/r_i^2$ and $p_3 = 2/r_i$. Finally, the
term $\phi_{i-1/2}$ and $\phi_{i-1/2}$,  which depend on the variables
$\Theta_b$, $\delta$ as well as different powers of $h$, are evaluated
in $i-1/2$ and  $i+1/2$ respectively. Different choices  for the value
of  the  variables  at  the  mid-cell  grid  point  do  not  show  any
significant differences  and a simple  average is taken such  that the
variable $x_{i+1/2}$ is taken as $0.5(x_i+x_{i+1})$.

\vspace{.5cm}    \textbf{Expression   of    the   Jacobian    $J_f^h$}
\vspace{.5cm}

The discretized  function $f_i$ can be  break down in three  part, the
gravitational part $f_i^{g}$  which is expressed in term  of the value
of $h$ on three  grid points $\left\{{i-1,i,i+1}\right\}$, the bending
part $f_i^{b}$ which is expressed in term  of the value of $h$ on nine
grid points  $\left\{{i-4,i-3,...,i+3,i+4}\right\}$ and  the injection
term which depends only on the grid point $i$ such that
\begin{equation}
  f_i = f_i^g+f_i^b+w_i.
\end{equation}
Therefore, the Jacobian is  nona-diagonal and its coefficient $J_{il}$
are
\begin{equation}
  J_{il}=
  \begin{cases}
    \frac{\partial f^{b}_i}{\partial h_{l}} &
    l = \left\{{i-4,i-3,i-2,i+2,i+3,i+4}\right\}\\
    \frac{\partial       f^{g}_i}{\partial       h_{l}}+\frac{\partial
      f^{b}_i}{\partial h_{l}} & l =
    \left\{{i-1,i,i+1}\right\}\\
    0 & \text{otherwise}
  \end{cases}.
  \label{C4-C2-eq12}
\end{equation}
The different  terms can be  easily derived from  (\ref{C4-Num-3}) and
(\ref{C4-Num-4}) with just slight  adjustment coming from the boundary
conditions.

\vspace{.5cm} \textbf{Boundary condition} \vspace{.5cm}

We begin with  $h_i=h_f$ for $i=1,..,M$.  Since the  flow is symmetric
in $r=0$, we require that
\begin{equation}
  \left.\frac{\partial h}{\partial r}\right|_{r=0} =\left.\frac{\partial P}{\partial r}\right|_{r=0} =0,
\end{equation}
and therefore for $i=1$, we have
\begin{eqnarray}
  f_i     &=&A_1\phi_{i+1/2}\left(h_{i+1}-h_i\right)\nonumber\\
          &+&A_i\phi_{i+1/2}\left(P_{i+1}-P_i\right)\nonumber\\
          &+&w_i\label{C4-Num-5}.
\end{eqnarray}
The expression  of the  bending pressure, evaluated  over a  $7$ point
stencils, is problematic close to  the boundary and reflection formula
will  be  used  in  order   to  accommodate  the  boundary  conditions
\citet{Patankar:1980vu}.   In   particular,  we  have  $h_0   =  h_1$,
$h_{-1}=h_2$ and  $h_{-2}=h_3$.  Similarly, boundary condition  at the
end of the mesh is accounted by using a grid much larger than the flow
itself and requiring
\begin{equation}
  \left.\frac{\partial h}{\partial r}\right|_{r=r_M} =\left.\frac{\partial P}{\partial r}\right|_{r=r_M} =0,
\end{equation}
which gives for $i=M$
\begin{eqnarray}
  f_i     &=&B_i\phi_{i-1/2}\left(h_{i}-h_{i-1}\right)\nonumber\\
          &+&B_i\phi_{i-1/2}\left(P_{i}-P_{i-1}\right)\nonumber\\
          &+&w_i\label{C4-Num-5},
\end{eqnarray}
with $h_{i>=M}=h_f$.


\vspace{.5cm} \textbf{Newton-Rhapsod method} \vspace{.5cm}

The Newton-Rhapsod method reads
\begin{equation}
  (I-\Delta tJ^h_{f}(h_k^*))\delta h_k^n=h^n-h_k^*+\Delta t f(h_k^*),
\end{equation}
where the  $k$ refers  to the $k$  iterations, $I$ is  a $M  \times M$
diagonal  matrix and  $J_f^h(h^*)$  is a  $M  \times M$  nona-diagonal
matrix.   This  system of  linear  equations  can  be solved  using  a
nona-diagonal algorithm. At the first  iteration, we use $h^*_1 = h^n$
as     a    first     guess    and     then    we     iterate    using
$h^*_k  = h^n+\delta  h_{k-1}^n$ as  a new  guess for  each iterations
until $\delta h^n_{k}$ becomes  sufficiently small.  In particular, we
require that
\begin{equation}
  \delta h^n_k/h^*_{k}<\epsilon,
\end{equation}
with $\epsilon = 10^{-4}$.

\subsection{Heat equation}

The heat equation (\ref{C4-TF}) is written as
\begin{eqnarray}
  \frac{\partial \xi}{\partial t}-g(h,\xi)&=&0,
\end{eqnarray}
with
\begin{eqnarray}
  g& =& \frac{1}{r}\frac{\partial}{\partial                          r}
        \left( r\Gamma\xi\right) +\frac{1}{r}\frac{\partial}{\partial                          r}
        \left(r\Sigma\right)+2Pe^{-1}St_m\frac{\left(\Theta_b-\Theta_s\right)}{\delta},\\
  \overline{\theta}&=&\frac{1}{3}\left(2\Theta_b+\Theta_s\right),\\
  \Gamma&=&-\frac{12}{\delta}
            \frac{\partial
            P}{\partial
            r}\left(\delta
            I_0(\delta)-I_1(\delta)\right),\\
  \Sigma &=& \frac{12}{\delta} \frac{\partial P}{\partial r}\left(I_0(\delta)\left(G(\delta)-\delta\overline{\theta}\right)+\overline{\theta}I_1(\delta)-I_2(\delta)\right).
\end{eqnarray}

\vspace{.5cm} \textbf{Spatial discretization of g} \vspace{.5cm}

As for the thickness equation,  the spatial discretization is obtained
using  a  central  difference  scheme   over  a  sub-grid  shifted  by
$0.5\Delta r$ from the main grid. Therefore, we have
\begin{eqnarray}
  g_i &=& \left(C_i\Gamma_{i+1/2}\xi_{i+1/2}-D_i\Gamma_{i-1/2}\xi_{i-1/2}\right)\\
      &+&\left(C_i\Sigma_{i+1/2}-D_i\Sigma_{i-1/2}\right)\\
      &+&2Pe^{-1}St_m\frac{\Theta_{b,i}-\Theta_{s,i}}{\delta_i},
\end{eqnarray}
with         $C_i         =r_{i+1/2}/(r_i\Delta        r)$         and
$D_i =r_{i-1/2}/(r_i\Delta r)$.   We use the average  between the grid
point $i$ and $i-1$ (resp. $i+1$) to evaluate the quantity in $\Gamma$
and  $\Sigma$ at  $i-1/2$ (resp.   $i+1/2$).   In addition,  we use  a
classical upwind  scheme to handle $\xi$  at the mid grid  point which
requires
\begin{eqnarray}
  \xi_{i+1/2} &=& \xi_i,\\
  \xi_{i-1/2} &=& \xi_{i-1}.
\end{eqnarray}

\vspace{.5cm}  \textbf{Expression   of  the   Jacobian  $J_{g}^{\xi}$}
\vspace{.5cm}

The expression of the Jacobian is straightforward in that case and its
coefficient $J_{il}$ are
\begin{equation}
  J_{il}=
  \begin{cases}
    -D_i\Gamma_{i-1/2}&
    l = i-1\\
    C_i\Gamma_{i+1/2} & l = i \\
    0 & \text{otherwise}
  \end{cases},
  \label{C4-C2-eq12}
\end{equation}
with only slight adjustment coming from the boundary conditions.

\vspace{.5cm} \textbf{Boundary conditions} \vspace{.5cm}

We consider $\Theta_b =1$, $\Theta_s=0$  and $\delta = 10^{-4}$ in the
film at  $t=0$. In this  way, we  ensure that the  average temperature
across the film at $t=0$ is close to $1$. By construction, $D_1=0$ and
therefore, for $i=1$ we have
\begin{eqnarray}
  g_i &=& C_i\Gamma_{i+1/2}\xi_{i}+ C_i\Sigma_{i+1/2} +2Pe^{-1}St_m\frac{\Theta_{b,i}-\Theta_{s,i}}{\delta_i}.
\end{eqnarray}
For   $i=M$,   we    consider   that   $\Gamma_{i+1/2}=\Gamma_i$   and
$\Sigma_{i+1/2}=\Sigma_i$.   However,  the  choice  for  the  boundary
condition at the border of the grid $i=M$ is not important as we solve
the problem over a grid much larger than the flow itself.

\vspace{.5cm} \textbf{Newton-Rhapsod method} \vspace{.5cm}

The Newton-Rhapsod method reads
\begin{equation}
  (I-\Delta tJ^{\xi}_{g}(\xi_k^*))\delta \xi_k^n=\xi^n-\xi_k^*+\Delta t f(\xi_k^*),
\end{equation}
where the  $k$ refers  to the $k$  iterations, $I$ is  a $M  \times M$
diagonal  matrix and  $J_f^h(\xi^*)$ is  a $M  \times M$  tri-diagonal
matrix.   This  system of  linear  equations  can  be solved  using  a
tri-diagonal algorithm.  As  for the thickness equation,  at the first
iteration,  we use  $\xi^*_1 =  \xi^n$ as  a first  guess and  then we
iterate using $\xi^*_k = \xi^n+\delta  \xi_{k-1}^n$ as a new guess for
each iterations  until $\delta \xi^n_{k}$ becomes  sufficiently small.
In particular, we require that
\begin{equation}
  \delta \xi^n_k/\xi^*_{k}<\epsilon,
\end{equation}
with  $\epsilon  = 10^{-4}$.   In  addition,  at each  iteration,  the
quantity $\Theta^*_{s,k}$, $\Theta^*_{b,k}$ and $\delta^*_k$, that are
needed to evaluate  $\Gamma$ and $\Sigma$, are derived  from the value
of $\xi^*_{k}$ using (\ref{C4-TS}), (\ref{C4-TB}) and (\ref{C4-DELTA}).

\subsection{Integral expressions}
\label{sec:integral-expressions}

The model  developed in  Section \ref{C4-sec:theory-1} depends  on the
integrals
\begin{eqnarray}
  I_0(z)&=&\int_0^z\frac{1}{\eta(y)}\left(y-\frac{h}{2}\right)
            dy, \\
  I_1(z) &=& \int_0^z\frac{1}{\eta(y)}\left(y-\frac{h}{2}\right)y dy,\\
  I_2(z)&=&\int_0^y                         \frac{1}{\eta(y)}
            \left(y-\frac{h}{2}\right)G(y)dy,
\end{eqnarray}
where $G(z)$  is a  primitive of $\theta(z)$  where $z<\delta$  and is
given by
\begin{equation}
  G(z) = \frac{z \left(3 \delta ^2 \Theta_s+3 \delta z (\Theta_b-\Theta_s)+z^2 (\Theta_s-\Theta_b)\right)}{3 \delta ^2}.
\end{equation}
In  particular, the  model requires  the expression  of $I_0(\delta)$,
$I_1(\delta)$, $I_1(h)$ and $I_2(\delta)$.

\vspace{.5cm}   \textbf{Rheology   1:   $\eta(\theta)=\eta_1(\theta)$}
\vspace{.5cm}

In that case, the four integrals can be easily derived and read
\begin{eqnarray}
  I_0(\delta)&=&\frac{\delta}{12} \left(\left(6 \delta-6h\right) \nu +
                 (1-\nu) \left(\alpha_1\left(-  \delta + 2  h\right) +
                 \Theta_{b}\left(         6\delta          -         6
                 h\right))\right)\right),\nonumber\\
  I_1(h)&=&\frac{1}{60} \left(5 h^{3} \nu +(1-\nu) \left(\alpha_1\left(- 4  \delta^{3} + 10 \delta^{2} h - 10 \delta h^{2}\right) + 5 \Theta_{b} h^{3}\right) \right)\nonumber,\\
  I_1(\delta)&=&\frac{\delta^{2}}{120} \left(\left(40 \delta-30h\right) \nu + (1-\nu) \left(\alpha_1\left(- 4  \delta + 5 h\right) + \Theta_{b}\left(40  \delta - 30  h\right)\right) \right)\nonumber,\\
  I_2(\delta)&=&- \frac{\delta^{2}}{2520} \nu\left(\alpha_1\left(378   \delta  -
                 315 h\right) + \Theta_{b}\left(-  840  \delta  + 630
                 h\right) \right)\nonumber\\
             &&-\frac{\delta^{2}}{2520}(1-\nu)  \left(\alpha_1^{2}\left(- 50   \delta  + 70
                h\right)  +   \alpha_1\Theta_{b}\left(462      \delta  -  420 
                h\right) -  \Theta_{b}^{2}\left(840   \delta  + 630  
                h\right)\right),\nonumber
\end{eqnarray}
where $\alpha_1=\Theta_b-\Theta_s$ has been introduced for clarity.

\vspace{.5cm}   \textbf{Rheology   2:   $\eta(\theta)=\eta_2(\theta)$}
\vspace{.5cm}

For cases where $\nu<1$, we have
\begin{eqnarray}
  I_0(\delta)&=&-\frac{\delta  \nu ^{1-\Theta_b} \left(\sqrt{\pi } \sqrt{\alpha_1} (2 \delta -h) \sqrt{-\alpha_2}
                 \text{erf}\left(\sqrt{\alpha_1} \sqrt{-\alpha_2}\right)+2 \delta  \left(\nu ^{\alpha_1}-1\right)\right)}{4 \alpha_1 \alpha_2}\nonumber,\\
  I_1(\delta)&=&\frac{\delta ^2 \nu ^{1-\Theta_b} \left(\sqrt{\pi } \erf\left(\sqrt{\alpha_1} \sqrt{-\alpha_2}\right) (\alpha_1 (h-2
                 \delta ) \alpha_2+\delta )\right)}{4           \alpha_1^{3/2}          (-\alpha_2)^{3/2}}\nonumber\\
             &&+\frac{\delta ^2 \nu ^{1-\Theta_b} \left(\sqrt{\alpha_1} \sqrt{-\alpha_2} \left(2 \delta  \left(\nu ^{\alpha_1}-2\right)-h \nu
                ^{\alpha_1}+h\right)\right)}{4           \alpha_1^{3/2}          (-\alpha_2)^{3/2}}\nonumber,\\
  I_1(h)&=&\frac{\nu ^{1-\Theta_b} \left(\sqrt{\alpha_1} \sqrt{-\alpha_2} \left(12 \delta ^2 \left(\delta  \left(\nu
            ^{\alpha_1}-2\right)-h \nu ^{\alpha_1}+h\right)+\alpha_1  (2 \delta -h)^3
            \log    (\nu    )\right)\rihgt)}{12   \alpha_1^{3/2}    (-\alpha_2)^{3/2}}\nonumber\\
             &&-\frac{\nu ^{1-\Theta_b} \left(3 \sqrt{\pi } \delta 
                \text{erf}\left(\sqrt{\alpha_1} \sqrt{-\alpha_2}\right) \left(\alpha_1 (h-2 \delta )^2 \alpha_2-2 \delta
                ^2\right)\right)}{12 \alpha_1^{3/2} (-\alpha_2)^{3/2}},\nonumber
\end{eqnarray}
\begin{eqnarray}
  I_2(\delta)&=&\frac{\delta ^2 \nu ^{1-\Theta_b} \left(\sqrt{\pi } \text{erf}\left(\sqrt{\alpha_1} \sqrt{-\alpha_2}\right) \left(-2
                 \alpha_1 (2  \delta -h)  (\alpha_1-3 \Theta_b)  \alpha_2^2-6 \delta
                 \Theta_b  \alpha_2-3   \delta  \right)\right)}{24
                 \alpha_1^{3/2} (-\alpha_2)^{5/2}}\nonumber\\
             &&+\frac{\delta ^2 \nu ^{1-\Theta_b} \left(2
                \sqrt{\alpha_1} \nu ^{\alpha_1} \sqrt{-\alpha_2} \left(\nu ^{-\alpha_1} (\alpha_2 (-2 \delta  (\alpha_1-6
                \Theta_b)-3 h \Theta_b)+2 \delta -h)\right)\right)}{24
                \alpha_1^{3/2} (-\alpha_2)^{5/2}}\nonumber\\
             &&+\frac{\delta ^2 \nu ^{1-\Theta_b} \left(2
                \sqrt{\alpha_1} \nu ^{\alpha_1} \sqrt{-\alpha_2} \left(2 \delta  \alpha_1 \alpha_2-6 \delta  \Theta_b \alpha_2+\delta
                -\alpha_1 h \alpha_2+3 h \Theta_b \alpha_2+h\right)\right)}{24 \alpha_1^{3/2} (-\alpha_2)^{5/2}},\nonumber
\end{eqnarray}
where    in    addition    to   $\alpha_1$,    we    also    introduce
$\alpha_2=\log(\nu)$  for clarity.   In  the case  where $\nu=1$,  the
expression above simplify and read
\begin{eqnarray}
  I_0(\delta)&=&\frac{1}{2} \delta (\delta -h),\nonumber\\
  I_1(\delta)&=&\frac{1}{12} \delta ^2 (4 \delta -3 h),\nonumber\\
  I_1(h)&=&\frac{h^3}{12},\nonumber\\
  I_2(\delta)&=&-\frac{1}{120} \delta ^2 (18 \delta  \alpha_1-40 \delta  \Theta_b-15 \alpha_1 h+30 h \Theta_b)\nonumber.
\end{eqnarray}

\section{Numerical scheme for a crater-centered intrusion}
\label{A1-sec:numer-scheme-crat}

We   use   a   fully    implicit   finite-volume   method   to   solve
(\ref{C5-eq21}). The discretization is  obtained by integrating over a
finite number of non overlapping  control volumes, each control volume
surrounding  one  grid  point \citep{Patankar:1980vu}.   The  grid  is
defined  by the  relation $r_{i}=(i-0.5)\Delta  r$ in  order to  avoid
problems at the center.  The point $b$  and $a$ define the face of the
control volume surrounding $i$  such that $r_{a}=r_{i}+\Delta r/2$ and
$r_{b}=r_{i}-\Delta  r/2$.  Because  we  are  using an  axisymmetrical
geometry, the  control volume is  an annulus of interior  radius $r_b$
and     exterior    radius     $r_a$     and     its    surface     is
$S=\pi(r_{a}^{2}-r_{b}^{2})$. Integration of  (\ref{C5-eq21}) over the
control volume surrounding $i$ during a time $\Delta t$ gives
\begin{equation}
  \int_{t}^{t+\Delta t} \int_{r_b}^{r_a} \frac{\partial h^{*}}{\partial t} 2\pi r dr dt=\int_{t}^{t+\Delta t} \int_{b}^{a} \Phi(r,t) 2\pi r, dr dt
  \label{C5-eq27}
\end{equation}
where $\Phi(r,t)$ stands for the right hand side of (\ref{C5-eq21}).

The  classical  second-order  $\left   (\propto  \Delta  r^2  \right)$
approximations  is taken  to derive  the successive  space derivatives
(i.e.
$\frac{\partial                                      \Phi(r)}{\partial
  r}\mid_{r_a}=\frac{\Phi(i+1)-\Phi(i)}{\Delta                   r}$).
In   this   way,   we   ensure   that   our   final   scheme   is   of
second-order. Moreover,  for more  precision, the elastic  pressure is
calculated using  a fourth-order  scheme (see \ref{C5-Fourth}.3)  . In
the  following,  we  derive  each  term of  the  right  hand  side  of
(\ref{C5-eq21}) separately, $h$  refers to the value  of the thickness
at a time $t+\Delta t$ and $h^{n}$ to the value at a time $ t$.
\subsection{Discretization}
\label{C5-Fourth}
\begin{enumerate}
\item \textbf{Time  derivative} To discretize the  time derivative, we
  shall consider  that the  value of the  grid point  $h_{i}$ prevails
  throughout the control volume such that
  \begin{equation}
    \label{C5-num1}
    \int_{t}^{t+\Delta t} \int_{r_b}^{r_a} \frac{\partial h^{*}}{\partial t} 2\pi r dr dt=(h_{i}-h^{n}_{i})S.
  \end{equation}
\item \textbf{Gravitational term}
  \begin{eqnarray}
    \label{C5-num2}
    \int_{t}^{t+\Delta t} \int_{r_b}^{r_a} \frac{1}{r^{*}} \frac{\partial}{\partial r^{*}}\left (r^{*} h^{3} \frac{\partial h}{\partial r^{*}} \right) 2\pi r dr dt\\
    =A^{g}_{i}h_{i+1}+B^{g}_{i}h_{i-1} -(A^{g}_{i}+B^{g}_{i})h_{i} \nonumber,
  \end{eqnarray}
  with  $A^{g}_{i}= A=(2\pi  \Delta  t  r_{a}h_{a}^{3})/\Delta r$  and
  $B^{g}_{i}=B=(2\pi  \Delta  t  r_{b}h_{b}^{3})/\Delta r$  where  the
  value  of  $h_{a}^{3}$  (resp.    $h_{b}^{3}$)  is  approximated  by
  $(h_{i+1}^{3}+h_{i}^{3})/2$ (resp. $(h_{i}^{3}+h_{i-1}^{3})/2$).
\item \textbf{Elastic term}
  \begin{eqnarray}
    \label{C5-num3}
    \int_{t}^{t+\Delta t} \int_{r_b}^{r_a} \Theta \frac{1}{r^{*}} \frac{\partial}{\partial r^{*}}\left (r^{*} h^{3} \frac{\partial P_{e}}{\partial r^{*}} \right) 2\pi r dr dt \\
    =A^{e}_{i}P_{e,i+1} +B^{e}_{i}P_{e,i-1 } -(A^{e}_{i}+B^{e}_{i})P_{e,i} \nonumber,
  \end{eqnarray}
  where    $A^{e}_{i}=   \Theta    A$,   $B^{e}_{i}=\Theta    B$   and
  $P_{e}=\nabla^{2}_{r}\left( \Pi(r)\nabla^{2}_{r}h(r)  \right)$, with
  $\Pi(r)=(1+\Psi\xi(r))^3$,  is  the dimensionless  elastic  pressure
  which is discretized using a fourth order finite difference scheme
  \begin{equation}
    \label{C5-num6}
    P_{e,i}= \alpha_{i}h_{i-3} + \beta_{i}h_{i-2}+\gamma_{i} h_{i-1} +\lambda_{i}h_{i}+\kappa_{i}h_{i+1}+\delta_ih_{i+2}+\epsilon_ih_{i+3},
  \end{equation}
  with
  \begin{eqnarray}
    &\alpha_{i}&=\frac{1}{24\Delta r^{4}}\left(-4p_4+3p_3\Delta_r \right)\nonumber ,\\
    &\beta_{i}&=\frac{1}{24\Delta r^{4}}\left(48p_4-24p_3\Delta_r-2p_2\Delta_r^2+2p_1\Delta_r^3\right) \nonumber,\\
    &\gamma_{i}&=\frac{1}{24\Delta r^{4}}\left(-156p_4+39p_3\Delta_r+32p_2\Delta_r^2-16p_1\Delta_r^3\right)\nonumber,\\
    &\lambda_{i}&=\frac{1}{24\Delta r^{4}}\left(224p_4-60p_2\Delta r^{2}\right) \nonumber,\\
    &\kappa_{i}&=\frac{1}{24\Delta r^{4}}\left( -156p_4-39p_3\Delta_r+32p_2\Delta_r^2+16p_1\Delta_r^3\right)\nonumber,\\
    &\delta_{i}&=\frac{1}{24\Delta r^{4}}\left( 48p_4+24p_3\Delta_r-2p_2\Delta_r^2-2p_1\Delta_r^3\right) \nonumber,\\
    &\epsilon_{i}&=\frac{1}{24\Delta r^{4}}\left(-4p_4-3p_3\Delta_r \right)\nonumber,
  \end{eqnarray}
  where
  \begin{eqnarray}
    p_1&=&\frac{ \Pi^{\prime\prime}_{i}}{r_i}-\frac{\Pi^{\prime}_{i}}{r_i^2} +\frac{\Pi}{r_i^3}\nonumber,\\
    p_2&=&\Pi^{\prime\prime}_{i}+\frac{3\Pi^{\prime}_{i}}{r_i} +\frac{\Pi}{r_i^2}\nonumber,\\
    p_3&=&2\Pi^{\prime}_{i}+\frac{2\Pi_i}{r_i}\nonumber,\\
    p_4&=&\Pi_i\nonumber,
  \end{eqnarray}
  and  where $  \Pi_{i}=(1+\Psi\xi_i)^{3}$ and  $\Pi^{\prime}_{i}$ and
  $ \Pi^{\prime\prime}_{i}$ are its  first and second derivatives with
  respect to the radial coordinate.
\item \textbf{Hydrostatic term}
  \begin{eqnarray}
    \label{C5-num4}
    S^{h}_{i}= \int_{t}^{t+\Delta t} \int_{r_b}^{r_a} \Xi \frac{1}{r^{*}} \frac{\partial}{\partial r^{*}}\left (r^{*}h^{3}\frac{\partial \xi}{\partial r}\right )2\pi r dr\\
    =U^{h}\left ( r_{a}h_{a}^{3} \left. \frac{\partial \xi}{\partial r}\right|_{a}-r_{b}h_{b}^{3}\left. \frac{\partial \xi}{\partial r}\right|_{b} \right )\nonumber,
  \end{eqnarray}
  where $U^{h}= 2 \pi \Xi \Delta t$.
\item \textbf{Injection term}
  \begin{eqnarray}
    \label{C5-num5}
    S^{i}_{i}&=&\int_{t}^{t+\Delta t} \int_{r_b}^{r_a} \frac{32}{\gamma^{2}} (\frac{1}{4}-\frac{r^{2}}{\gamma^{2}})  2 \pi r dr dt \\
             &=&U^{i}(\gamma^{2}-2(r_{a}^{2}+r_{b}^{2}))\nonumber,
  \end{eqnarray}
  where $U^{i}=\frac{8 S \Delta t}{\gamma^{4}}$.
 
 
\item \textbf{Implicit scheme}
 
  Substituting   (\ref{C5-num1}),  (\ref{C5-num2}),   (\ref{C5-num3}),
  (\ref{C5-num4}) and (\ref{C5-num5}) in (\ref{C5-eq27}) and injecting
  (\ref{C5-num6}),  we get  the final  scheme given  by the  following
  equation
  \begin{equation}
    a_ih_{i-4}+b_ih_{i-3}+c_ih_{i-2}+d_ih_{i-1}+e_ih_i+f_ih_{i+1}+g_ih_{i+2}+k_ih_{i+3}+l_ih_{i+4}=J_i,
    \label{C5-eqd3}
  \end{equation}
  where the different coefficients are defined by
  \begin{eqnarray}
    a_i&=&-B_i^{e}\alpha_{i-1} ,\\
    b_i&=&(B_i^{e}+A_i^{e})\alpha_{i}-B_i^{e}\beta_{i-1} ,\\
    c_i&=&(B_i^{e}+A_i^{e})\beta_{i}-B_i^{e}\gamma_{i-1}-A_i^{e}\alpha_{i+1} ,\\
    d_i&=&(B_i^{e}+A_i^{e})\gamma_{i}-B_i^{e}\lambda_{i-1}-A_i^{e}\beta_{i+1} -B^{g},\\
    e_i&=&S+(B_i^{e}+A_i^{e})\lambda_{i}-B_i^{e}\kappa_{i-1}-A_i^{e}\gamma_{i+1} +B^{g}+A^{g},\\
    f_i&=&(B_i^{e}+A_i^{e})\kappa_{i}-B_i^{e}\delta_{i-1}-A_i^{e}\lambda_{i+1} - A^{g},\\
    g_i&=&(B_i^{e}+A_i^{e})\delta_{i}-B_i^{e}\epsilon_{i-1}-A_i^{e}\kappa_{i+1} ,\\
    k_i&=&(B_i^{e}+A_i^{e})\epsilon_{i}-A_i^{e}\delta_{i+1} ,\\
    l_i&=&-A_i^{e}\epsilon_{i+1} ,\\
    J_i&=&(Sh_i^n+ S^{i}_{i}+S^{h}_{i}).
  \end{eqnarray}	
  \subsection{Boundary conditions}
	
  Since the flow is symmetric in $r=0$, we require that
  \begin{eqnarray}
    \left.\frac{\partial h}{\partial r}\right|_{r=0}=0,
    \left.\frac{\partial P_{e}}{\partial r}\right|_{r=0}=0.
  \end{eqnarray}
  Boundary conditions at the front  of the intrusion are accounted for
  by using  a grid much larger  than the intrusion where  $h=0$ beyond
  the flow.

  \subsection{Algorithm}
  The fully  implicit discretization (\ref{C5-eqd3}) can  be rewritten
  as a linear system $\Omega(h^{3})\bar{h}=\bar{J}$ where $\bar{h}$ is
  a vector  containing the value of  $h$ a $t+\Delta t$  and $\bar{J}$
  containing  the  right hand  side  of  (\ref{C5-eqd3}).  The  matrix
  $\Omega(h^{3})$ is a  septadiagonal matrix and is solved  by using a
  septadiagonal algorithm.   However, due to the  non-linearity of the
  problem (i.e.   the coefficients $A_{e}, B_{e},  A_{g}$, $B_{g}$ and
  $S^{h}$ within  the matrix $\Omega(h_{i}^{3})$ and  $\bar{J}$ depend
  on $h_{i}^{3}$), we first have to  assume values for $h_{i}$ at each
  grid point  to inverse for  the matrix and get  the value of  $h$ at
  $t+\Delta t$.

  We use the following iterative scheme
  \begin{enumerate}
  \item Start with a guess at all grid-point for $h_{i}=h_i^n$.
  \item Calculate  tentative values for the  different coefficients of
    the system (non linear terms).
  \item   Apply   the   septadiagonal  matrix   algorithm   to   solve
    (\ref{C5-eqd3}) and get a new value of $h_{i}$.
  \item With this  new $h_{i}$, we return to step  $2$ and repeat step
    $2$  to  $4$  until  further  repetitions  cease  to  produce  any
    significant         change         in        $h_{i}$         (i.e.
    $\mid h^{new}_{i}-h_{i}\mid < \epsilon$ where $\epsilon=10^{-4}$).
  \end{enumerate}
  The final  unchanging state  is considered as  the solution  for the
  thickness of the flow at $t+\Delta t$.


%%% Local Variables:
%%% mode: latex
%%% TeX-master: "../main"
%%% End:
