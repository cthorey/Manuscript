\chapter{Floor-fractured craters}
\label{chap:A4}

\section{Elastic stresses in the upper elastic layer}
\label{C5-AppendixB}
The  stress conditions  within the  crater floor  can be  approximated
using the small displacement theory.  In an axisymmetric geometry, the
small strain-displacement equations at the surface are:

\begin{eqnarray}
  \epsilon_{rr}&=&\frac{\partial u_{r}}{\partial r}=-\frac{T_{e}(r)}{2}\frac{\partial^{2} h}{\partial r^{2}}\\
  \epsilon_{\theta\theta}&=&\frac{u_{r}}{r}=-\frac{T_{e}(r)}{2r}\frac{\partial h}{\partial r} 
\end{eqnarray}

Hence, the stress conditions at the  surface are given by Hooke's laws
for a material under plane stress:
\begin{eqnarray}
  \sigma_{rr}&=&-\frac{E T_{e}(r)}{2(1-\nu^2)}\left (\frac{\partial^{2} h}{\partial r^{2}} +\frac{\nu}{r}\frac{\partial h}{\partial r} \right)\\
  \sigma_{\theta\theta}&=&-\frac{E T_{e}(r)}{2(1-\nu^2)}\left (\frac{1}{r}\frac{\partial h}{\partial r}+\nu \frac{\partial^{2}h}{\partial r^2}\right)
\end{eqnarray}

These equations  are made dimensionless  using the scaling  of Section
3.5 where the pressure scale is $\rho_m g H$. Dimensionless radial and
tangential stresses become:

\begin{eqnarray}
  \sigma_{rr}&=&- \Theta \Phi \left(1+\Psi \xi(r)\right) \left ( \frac{\partial^{2} h}{\partial r^{2}} +\frac{\nu}{r}\frac{\partial h}{\partial r} \right) \\
  \sigma_{\theta\theta}&=&- \Theta \Phi \left(1+\Psi \xi(r)\right) \left (\frac{1}{r}\frac{\partial h}{\partial r}+\nu \frac{\partial^{2}h}{\partial r^2}\right)
\end{eqnarray}

where  $\xi(r)$  is   given  by  (\ref{C5-eqqqq})  and   $\Phi$  is  a
dimensionless number given by:

\begin{equation}
  \Phi= \frac{6 }{(1-\nu^2)}\left( \frac{C}{T_{e}^0} \right)^{2}
\end{equation}

The locations  of the  maximum stresses, where  the fractures  are the
most  likely  to  initiate,  depend on  the  number  $\Theta$  (Figure
\ref{C5-fig7-1} a). If the number $\Theta$ is such that $4\Lambda>>C$,
i.e.   $\Theta>10^{-3}$ the  intrusion  reaches the  wall  zone in  an
elastic  regime and  the  maximum  stresses are  at  the center.   For
$\Theta\sim10^{-3}$, $4\Lambda \sim C$ and the transition to a gravity
current regime occurs at the crater wall zone. In that case, the floor
is still  convex but the  area of maximum  stress is located  within a
crown at a  given coordinate, intermediate between the  center and the
wall  zone, i.e.   $0<r_{\sigma_{max}}<1$ (Figure  \ref{C5-fig7-1} b).
Radial and tangential stresses are of the same order of magnitude. For
a large  crater or  a shallow  intrusion, i.e.  a  small value  of the
number $\Theta<10^{-3}$, the maximum  stresses are concentrated within
a crown  adjacent to the wall  zone upon the intrusion  edge where the
elastic  deformation is  important (Figure  \ref{C5-fig7-1} c  ).  The
radial stresses that develop at  the surface are generally larger than
the tangential stresses favoring a circular mode of fracturing.


\begin{figure}[h!]
  \begin{center}
    \graphicspath{{/Users/thorey/Documents/These/Submission/Article/FFC_JGR_2013/Paper_APRES_2nd_REVIEW/}}
    \includegraphics[scale=0.65]{fig7-1.eps}
    \caption{Solid  lines: Dimensionless  radial stress  $\sigma_{rr}$
      (top) and tangential  stress $\sigma_{\theta \theta}$(bottom) at
      the crater floor in the case  of an intrusion spreading below an
      overlying  elastic layer  with a  complex crater  topography for
      $\Theta=10^{-2}$  and $\Phi=1100$  (left), $\Theta=10^{-3}$  and
      $\Phi=2500$  (center)   and  $\Theta=10^{-5}$   and  $\Phi=4500$
      (right) at $t=2$. For all  plots: the dotted lines represent the
      initial  dimensionless topography  $T_p(r)$ (\ref{C5-TopoFinal})
      and  the  dash-dotted  lines   represent  the  floor  appearance
      $T_p(r)+h(r)$  at  $t=2$.   Stress  is  considered  positive  in
      extension.   We use  $\gamma=0.02$,  $\Xi=20$, $\zeta=0.13$  and
      $\Psi=1$.}
    \label{C5-fig7-1}
  \end{center}
\end{figure}

\section{Central peak}
\label{C5-AppendixC}

Central peaks induce  an increase in the lithostatic  pressure as well
as an increase in the overlying layer elastic thickness directly above
the intrusion  center. Herein, we  consider an extreme case  where the
central peak height is one third of  the crater depth and its width is
one  fourth  of the  crater  size  by  introducing an  extra  gaussian
function into the elastic thickness expression:

\begin{equation}
  T_e(r)=T_e^0(1+\Psi(\xi(r)+C_p(r)))
\end{equation}
with
\begin{equation}
  C_p(r)=50 \left(\frac{0.07}{4}\right)^2\exp\left(-\frac{r^2}{2(\frac{0.07}{4})^2}\right)
\end{equation}

For   a  strengthless   overlying   layer   and  $\Theta=0$   (Section
\ref{C5-Strengthless_Layer1}  equation  (\ref{C5-eq22})), the  central
peak only adds an excess in  lithostatic pressure at the center of the
crater  floor.  In  response, the  intrusion  preferentially  develops
around the central  peak and then spreads until it  reaches the crater
wall (Figure \ref{C5-fig7-2}  a). At the crater  wall, the lithostatic
pressure increase  induces the  thickening of the  intrusion. However,
due to the excess of lithostatic pressure at the center, the center of
the  intrusion  below the  central  peak  does  not thickens  and  the
thickening only  concerns an  annulus located  in between  the central
peak and the  crater wall (Figure \ref{C5-fig7-2} a).  At the surface,
the central  peak height decreases  until the thickening  is important
enough to compensate for the initial excess in lithostatic pressure. A
balance  between the  two  pressures  gives the  final  height of  the
central    peak,     equal    to    the    initial     height    times
$(\rho_m-\rho_c)/\rho_m$   (Figure  \ref{C5-fig7-2}   a).  Next,   the
resulting central peak is just leveled up with the whole crater floor.

For an elastic  overlying layer such that  $\Theta=10^{-5}$, the inner
part of  the intrusion  adjacent to  the central peak  is bent  by the
weight of  the central peak.  As a consequence, during  the thickening
stage, a  second circular moat,  whose size is $4\Lambda$,  arises and
borders  the central  peak.  As previously,  the  central peak  height
decreases  until  the sum  of  the  elastic and  hydrostatic  pressure
compensate for the  initial excess of lithostatic pressure  due to the
central peak and is then leveled up during floor uplift.

Finally, in the case of a  thick elastic overlying layer, i.e. a large
value of $\Theta$,  the flexural wavelength is almost  not affected by
the presence  of the central peak  and the central peak  is leveled up
with   the   convex  floor   during   crater   floor  uplift   (Figure
\ref{C5-fig7-2} c).


\begin{figure}[h!]
  \begin{center}
    \graphicspath{{/Users/thorey/Documents/These/Submission/Article/FFC_JGR_2013/Paper_APRES_2nd_REVIEW/}}
    \includegraphics[scale=0.55]{fig7-2.eps}
    \caption{   \textbf{a)}  Dimensionless   intrusion  profiles   for
      different dimensionless times indicated on the plot for $\Psi=4$
      and for  an intrusion  spreading below a  strengthless overlying
      layer  with  a complex  crater  topography  and a  central  peak
      i.e. $\Theta=0$ and $\zeta=0.13$. For each time, a corresponding
      3D  graph, showing  the  dimensionless  crater floor  appearance
      given        by         $T_p(r)+h(r)$        where,        here,
      $T_p(r)=\Xi  \Omega(\xi(r)+C_p(r))$,  is represented.  For  each
      plot,      the       initial      topography       given      by
      $T_p(r)=\Xi  \Omega(\xi(r)+C_p(r))$   is  superimposed   in  low
      opacity.  \textbf{b)} Same  plot  but for  an overlying  elastic
      layer such that $\Theta=10^{-5}$.  \textbf{c)} Same plot but for
      an elastic  overlying layer such that  $\Theta=10^{-2}$. Here we
      use, $\gamma=0.02$, $\Xi=20$ and $\Psi=4$.}
    \label{C5-fig7-2}
  \end{center}
\end{figure}


%%% Local Variables:
%%% mode: latex
%%% TeX-master: "../main"
%%% End:
