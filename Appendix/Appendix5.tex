\chapter[Gravitational  signature of  lunar FFCs]{Gravitational  signature of  lunar floor-fractured  craters:
  Supplementary material}
\label{C6-chap:A5}


\section{Synthetic gravity anomaly}
\label{sec:synth-grav-anom-1}

The  spherical harmonic  coefficients  associated  with the  intrusion
thickness profile have the form
\begin{equation}
  C_{lm}=\frac{4\pi \Delta \rho}{M}\frac{R_i^3}{2l +1}\sum^{n_{\tex{max}}}_{n=1}\frac{^nh_{lm}}{R_i^n n!}\frac{\prod^n_{j=1}(l+4-j)}{l+3},
\end{equation}
where  $^nh_{lm}$  are  the  spherical harmonic  coefficients  of  the
expansion    of    the    powers     of    the    thickness    profile
$h_{topo}^n(\theta,\phi)$
\begin{equation}
  h^n_{topo}(\theta,\phi)= \sum_{l=0}^{L_{\text{max}}}\sum_{m=-l}^{l}{^n}h_{lm}Y_{lm}(\theta,\phi).
\end{equation}

These  calculations were  performed on  grids that  resolved spherical
harmonics up to degree $1000$, which  corresponds to a grid spacing of
$2.7$  km.   When  calculating  the  spherical  harmonic  coefficients
$C_{lm}$, $n_{\text{max}}$  was set equal  to $9$, which is  more than
sufficient  given  the small  amplitudes  of  the magmatic  intrusions
considered here.  Gravity anomalies  are presented  in mGal,  which is
$10^{-5}$ m s$^{-2}$ and calculated at a radius $r=R_0$ where $R_0$ is
the mean lunar radius.
\section{Effect of the downward continuation filter $\lambda$}
\label{sec:effect-downw-cont}

We  present the  crustal  gravity anomaly  around two  floor-fractured
craters using  different values  for the downward  continuation filter
parameter $\lambda$.  For the FFC Beals,  that is $48$ km in diameter,
the  amplitude of  the  gravity anomaly  within  the crater  increases
somewhat with  increasing $\lambda$, due  to the removing  of regional
negative  trends, to  reach a  maximum around  $\lambda =  80$ (Figure
\ref{Figure_Supp_1}). For Taruntius, a  floor-fractured crater that is
a $56$ km  in diameter, models characterized by a  $\lambda<80$ do not
remove enough of the positive  regional trend to clearly delineate the
central anomaly,  while the  models characterized by  $\lambda>80$ are
too  restrictive and  would remove  too much  signal at  the intrusion
scale (Figure \ref{Figure_Supp_1}).

\begin{figure}[h!]
  \graphicspath{ {/Users/thorey/Documents/These/Projet/FFC/Gravi_GRAIL/Article/Papier/SOUMISSION_2_EPSL/} }
  \begin{center}
    \includegraphics[scale=0.6]{Figure_Supp_1.pdf}
    \caption{Top: Floor-fractured  crater Beals, $48$ km  in diameter.
      Top  left:   topography  (km)  obtained  from   LOLA  (64  ppd).
      Following  plots: The  crustal gravity  anomaly using  different
      values for the downward continuation filter parameter $\lambda$.
      The gravity anomaly within the crater becomes more apparent with
      increasing $\lambda$  due to  the removing of  regional negative
      trends.  Bottom:  same plots but for  the floor-fractured crater
      Taruntius, $56$ km in diameter.}
    \label{Figure_Supp_1}
  \end{center}
\end{figure}
\clearpage
\section{Definition of the gravity anomaly}

\begin{figure}[h!]
  \graphicspath{ {/Users/thorey/Documents/These/Projet/FFC/Gravi_GRAIL/Article/Papier/SOUMISSION_2_EPSL/} }
  \begin{center}
    \includegraphics[scale=0.5]{Figure_Supp_2.pdf}
    \caption{Definition of the  gravity anomaly at a  crater site. The
      gravity anomaly  $\delta_g$ associated to  a crater is  equal to
      the mean value  of the gravity anomaly measured  interior to the
      crater rim (left) minus the mean gravity anomaly measured within
      an annulus surrounding  the crater and extending  from the outer
      flank of the rim to a circle of diameter 2D.}
    \label{Figure_Supp_2}
  \end{center}
\end{figure}

\begin{table}[h!]
    \caption{Gravity  anomaly distribution  characteristics for  the two
      normal crater populations and the  FFC population in the lunar
      in the highlands, maria and
      South Pole Aitken basin}
    \scalebox{0.7}{
    \begin{tabular}{>{\centering\arraybackslash}m{4cm}| | >{\centering\arraybackslash}m{1.5cm}|>{\centering\arraybackslash}m{1.5cm}|>{\centering\arraybackslash}m{1.5cm}|>{\centering\arraybackslash}m{1.5cm}|>{\centering\arraybackslash}m{1cm}|>{\centering\arraybackslash}m{1.5cm}|>{\centering\arraybackslash}m{1cm}|>{\centering\arraybackslash}m{1.5cm}| }
      \textbf{}&\multicolumn{4}{>{\centering\arraybackslash}m{2cm}|}{}&\multicolumn{2}{c|}{t-test}&\multicolumn{2}{c|}{KS test} \\ 
               & N & $\mu_{\delta_g}$ &  SEM$_{\delta_g}$ & SD$_{\delta_g}$ & $t$
                                      & $p$& $D$ & $p$\\ 
      \hline 
      \textbf{Lunar highlands}&&&&&\multicolumn{2}{c|}{}&\multicolumn{2}{c|}{} \\ 
      Unmod. Craters & 4054 & -0.71 & 0.12 & 7.44 & 3.24 & 0.00119 & 0.00 & 1.00000 \\ 
      Unmod. Crat. FFC & 584 & -0.39 & 0.33 & 7.91 & 2.49 & 0.01284 & 0.04 & 0.34231 \\ 
      FFC & 80 & 2.03 & 1.07 & 9.57 & 0.00 & 1.00000 & 0.16 & 0.02923 \\ 
      \hline
      \hline
      \textbf{Lunar maria}&&&&&\multicolumn{2}{c|}{}&\multicolumn{2}{c|}{} \\ 
      Unmod. Craters & 306 & 1.51 & 0.68 & 11.85 & 1.08 & 0.27961 & 0.00 & 1.00000 \\ 
      Unmod. Crat. FFC & 70 & 1.94 & 1.32 & 11.01 & 0.81 & 0.41745 & 0.08 & 0.89097 \\ 
      FFC & 22 & 4.43 & 3.52 & 16.49 & 0.00 & 1.00000 & 0.17 & 0.58770 \\ 
      \hline
      \hline
      \textbf{Lunar SPA}&&&&&\multicolumn{2}{c|}{}&\multicolumn{2}{c|}{}\\ 
      Unmod. Craters & 603 & -3.54 & 0.46 & 11.27 & 1.08 & 0.27939 & 0.00 & 1.00000 \\ 
      Unmod. Crat. FFC & 148 & -3.86 & 0.94 & 11.49 & 1.14 & 0.25632 & 0.05 & 0.89320 \\ 
      FFC & 14 & -0.25 & 2.52 & 9.42 & 0.00 & 1.00000 & 0.21 & 0.52995 \\ 
      \multicolumn{9}{l}{
      \begin{minipage}{18.5cm}
        \vspace{.1cm}  \footnotesize  $^*$  $N$  is the  size  of  the
        population,  $\mu_{\delta_g}$  is  the  mean  of  the  gravity
        anomalies (mGal),  $SEM_{\delta_g}$ is  the standard  error of
        the  mean (mGal),  $SD_{\delta_g}$ is  the standard  deviation
        (mGal).  $t$ and $p$ are the values of a Student's t-test that
        compared   the  gravity   anomaly  means   of  the   different
        populations with  the FFC gravity  anomaly mean.  $D$  and $p$
        are the  values of  a two-sample Kolmogorov-Smirnov  test that
        compared  the gravity  anomaly distribution  of the  different
        populations  with  the  gravity anomaly  distribution  of  the
        normal  crater  population.   Unmod.  Craters  refers  to  the
        normal crater population  and Unmod. Crat.  FFC  refers to the
        normal crater population that  shares the spatial distribution
        of FFCs.
      \end{minipage}
      }\\
    \end{tabular}}
    \label{Table3-1}
  \end{table}

\section{Crater depth}
\label{sec:crater-depth}


The crater  depth, $d_c$, is  the difference in elevation  between the
crater rim and the  crater floor.  Following \citet{Kalynn:2013fg}, we
use the  gridded topographic data  within a circular region  of radius
$D$ to  derive the floor  elevation and  within an annulus  bounded by
$0.98$D and $1.05$D to derive the  rim elevation.  We use the diameter
reported by \citet{Head:2010fy}  and \citet{Jozwiak:2012dq} and assume
these to be error-free. We produce histograms of elevations, binned in
$50$  m intervals,  and  examine the  distributions.   For the  crater
floor,  the minimum  elevation $h_{\text{min}}$  is affected  by later
crater deformation such as fractures,  subsequent cratering or in some
cases, wall slump and moats.  For this reason, rather than considering
the minimum  $h_{\text{min}}$ and the mode  elevation $h_{\text{mod}}$
to characterize the crater floor depth as in \citet{Kalynn:2013fg}, we
consider only  the mode  of the  distribution $h_{\text{mod}}$  and we
take its value  as the crater floor  elevation $h_{\text{floor}}$.  We
assign to  this value an uncertainty  $\sigma_{\text{floor}}$ equal to
the width of the distribution  mode.  Concerning the rim elevation, we
follow \citet{Kalynn:2013fg} and take the rim elevation to be equal to
the average of  the modal $h_{\text{mode}}$ elevation  and the maximum
$h_{\text{max}}$ elevation within the crater rim region.  We assign an
uncertainty        to        the        rim        elevation        of
$\sigma_{\text{rim}}=(h_{\text{mode}}-h_{\text{max}})/2$
\citep{Kalynn:2013fg}.  The crater depth,  $d_c$, is the difference in
elevation  between the  floor  and  the rim,  to  which  we assign  an
uncertainty                          equal                          to
$\sigma_{d}=(\sigma_{floor}^2+\sigma_{rim}^2)^{1/2}$.

 \begin{table}[h!]
    \caption{  Topographic  analyses  for   the  normal  crater
      populations and the FFC population  in the highlands, maria and
      South Pole Aitken basin.}
\scalebox{0.7}{
    \begin{tabular}{>{\centering\arraybackslash}m{4cm}| | >{\centering\arraybackslash}m{1.5cm}|>{\centering\arraybackslash}m{1.5cm}|>{\centering\arraybackslash}m{1.5cm}|>{\centering\arraybackslash}m{1.5cm}|>{\centering\arraybackslash}m{1.5cm}|>{\centering\arraybackslash}m{1.2cm}|>{\centering\arraybackslash}m{1.2cm}|>{\centering\arraybackslash}m{1.2cm}| }
      &\multicolumn{3}{>{\centering\arraybackslash}m{1.5cm}|}{}&\multicolumn{2}{>{\centering\arraybackslash}m{1.2cm}|}{}&\multicolumn{3}{c|}{$d_c=AD^B$} \\ 
      & N & $\mu_{\text{depth}}$ & $\sigma_{\text{depth}}$ & $t$&$p$&$A$ & $B$ & $\sigma_{fit}$ \\ 
      \hline 
      \textbf{Lunar highlands}&&&&&&&& \\ 
      FFC&80 & 1.87 & 0.94 & 0.00000 & 1.00000 & 0.70 & 0.33 & 0.55 \\ 
      Unmod. Crat. FFC&584 & 1.93 & 1.07 & -0.48457 & 0.62814 & 0.54 & 0.44 & 0.66 \\ 
      \hline 
      \hline
      \textbf{Lunar maria}&&&&&&&& \\ 
      FFC & 22 & 1.16 & 0.59 & 0.00000 & 1.00000 & 0.18 & 0.62 & 0.67 \\ 
      Unmod. Crat. FFC&  70 & 1.14 & 0.73 & 0.07832 & 0.93775 & 0.05 & 0.93 & 0.76 \\ 
      \hline 
      \hline
      \textbf{Lunar SPA basin}&&&&&&&& \\ 
      FFC & 14 & 1.88 & 0.81 & 0.00000 & 1.00000 & 0.05 & 0.93 & 0.30 \\ 
      Unmod. Crat. FFC & 148 & 2.09 & 1.10 & -0.72867 & 0.46727 & 0.24 & 0.64 & 0.88 \\ 
      \multicolumn{9}{l}{
      \begin{minipage}{18.5cm}
        \vspace{.1cm}  \footnotesize  $^*$  $N$  is the  size  of  the
        population, $\mu_{\text{depth}}$ is the mean of the population
        depth  (km),  $\sigma_{\text{depth}}$  is   the  mean  of  the
        uncertainties in the  depth estimation (km).  $t$  and $p$ are
        the values of a Student's  t-test that compared the mean depth
        of the normal  craters that share the  spatial distribution of
        FFC  to the  FFC  mean  depth itself.   $A$  and  $B$ are  the
        coefficients  for  the  power law  relationship  $d_c=AD^B$  and
        $\sigma_{fit}$ is the dispersion around the power law best fit
        (km).   Unmod.   Crat.   FFC   refers  to  the  normal  crater
        population that share the spatial distribution of FFCs.
      \end{minipage}
      }\\

    \end{tabular}}
    \label{Table4-1}
  \end{table}
\begin{table}[h!]
  \caption{Derived  FFC   intrusion  thickness   ($H_0$)  distribution
    characteristics in the highlands, maria and
    South Pole Aitken basin.} \begin{tabular}{>{\centering\arraybackslash}m{4cm}||>{\centering\arraybackslash}m{2cm}|>{\centering\arraybackslash}m{2cm}|>{\centering\arraybackslash}m{2cm}| }
                                & N & $\mu_{H_0}$ & $\sigma_{H_0}$   \\ 
                                \hline 
                                Highlands & 80.00 & 0.49 & 0.85 \\ 
                                Maria &      22.00 &       0.09 &       1.01 \\ 
                                SPA &     14.00 &       1.11 &       0.93 \\
                                \multicolumn{4}{l}{
                                \begin{minipage}{11.5cm}
                                  \vspace{.1cm} \footnotesize $^*$ $N$
                                  is  the  size   of  the  population,
                                  $\mu_{H_0}$  is  the   mean  of  the
                                  population  thickness at  the center
                                  (km) and $\sigma_{H_0}$  is the mean
                                  of   the    uncertainties   in   the
                                  thickness estimation (km).
                                \end{minipage}
                                }\\

                              \end{tabular}
                              \label{Table4-2}
                            \end{table}

\begin{table}[h!]
                                \caption{Forward   modeling   for
                                  the  density  contrasts  between
                                  the magmatic  intrusions and the
                                  crust    at   the    sites   of
                                  floor-fractured craters in the highlands, maria and
                                  South Pole Aitken basin.}
\scalebox{0.7}{
                                \begin{tabular}{>{\centering\arraybackslash}m{1.7cm}|
                                  |
                                  >{\centering\arraybackslash}m{0.5cm}|>{\centering\arraybackslash}m{1.2cm}|>{\centering\arraybackslash}m{1.2cm}|>{\centering\arraybackslash}m{1.5cm}|>{\centering\arraybackslash}m{1.5cm}|>{\centering\arraybackslash}m{1.2cm}|>{\centering\arraybackslash}m{1.2cm}|>{\centering\arraybackslash}m{2cm}|>{\centering\arraybackslash}m{2cm}|>{\centering\arraybackslash}m{2cm}|}
                                  &\multicolumn{1}{>{\centering\arraybackslash}m{0.5cm}|}{}&\multicolumn{2}{c|}{Observed
                                                                                             gravity}&\multicolumn{2}{c|}{Synthetic
                                                                                                       gravity}&\multicolumn{2}{c|}{Density
                                                                                                                 contrast}&\multicolumn{2}{c|}{
                                                                                                                            t-test}\\
                                  & N & $\mu_{\delta_g^*}$ & SEM$_{\delta_g^*}$ & $\mu^S_{\delta_g}$ & SEM$_{\delta_g^S}$ & $\mu_{\Delta \rho}$ & SEM$_{\Delta \rho}$ &  $t$ & $p$ \\  
                                  \hline 
                                  Highlands & 80 &  3.48 & 0.98 &4.71e-03 & 2.19e-04 & 913 & 269 &  0.00000 & 1.00000 \\ 
                                  Maria  & 22  &  2.48  & 2.93  &4.68e-03 & 3.41e-04 & 484 & 669 &  0.69086 & 0.49125 \\ 
                                  SPA &14 & 3.22 & 1.65 & 1.44e-02 & 1.99e-03 & 974 & 846 &  -0.08425 & 0.93304 \\ 
                                  \multicolumn{11}{l}{
                                  \begin{minipage}{18cm}
                                    \vspace{.1cm}  \footnotesize  $^*$
                                    $N$ is the size of the population,
                                    $\mu_{\delta_g^*}$ is  the mean of
                                    the  corrected gravity  anomalies,
                                    SEM$_{\delta_g^*}$ is the standard
                                    error of the mean of the corrected
                                    gravity                 anomalies,
                                    $\mu_{\delta_g^S}$ is  the mean of
                                    the  synthetic  gravity  anomalies
                                    obtained for unit density contrast
                                    $\Delta  \rho  = 1$  kg  m$^{-3}$,
                                    $SEM_{\delta_g^S}$ is the standard
                                    error of the mean of the synthetic
                                    gravity    anomalies   for    unit
                                    density,  $\mu_{\Delta  \rho}$  is
                                    the  mean  value  of  the  density
                                    contrast between the magma and the
                                    crust
                                    $\mu_{\Delta                \rho}=
                                    \sum_{i=0}^{i=N}
                                    \frac{\delta_{g;i}^{c}}{\delta_{g;i}^{s}}/N$,
                                    SEM$_{\Delta   {\rho}}$   is   the
                                    standard error of the mean density
                                    contrast.   $t$  and $p$  are  the
                                    values of a  Student's t-test that
                                    compared   the  density   contrast
                                    distribution   of  the   different
                                    populations   with   the   density
                                    contrast of the  FFC population in
                                    the highlands.
                                  \end{minipage}
                                  }\\

                                \end{tabular}}
                                \label{Table4-3}
                              \end{table}
                              
%%% Local Variables:
%%% mode: latex
%%% TeX-master: "../main"
%%% End:
